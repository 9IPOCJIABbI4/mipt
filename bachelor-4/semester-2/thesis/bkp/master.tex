\documentclass[a4paper, 14pt]{extarticle}
\usepackage{fullpage}
% Этот шаблон документа разработан в 2014 году
% Данилом Фёдоровых (danil@fedorovykh.ru) 
% для использования в курсе 
% <<Документы и презентации в \LaTeX>>, записанном НИУ ВШЭ
% для Coursera.org: http://coursera.org/course/latex .
% Исходная версия шаблона --- 
% https://www.writelatex.com/coursera/latex/5.3

% В этом документе преамбула

\usepackage{siunitx}
%%% Работа с русским языком
\usepackage{cmap}					% поиск в PDF
\usepackage{mathtext} 				% русские буквы в формулах
\usepackage[T2A]{fontenc}			% кодировка
\usepackage[utf8]{inputenc}			% кодировка исходного текста
\usepackage[english,russian]{babel}	% локализация и переносы
\usepackage{indentfirst}
\frenchspacing

\renewcommand{\epsilon}{\ensuremath{\varepsilon}}
\renewcommand{\phi}{\ensuremath{\varphi}}
\renewcommand{\kappa}{\ensuremath{\varkappa}}
\renewcommand{\le}{\ensuremath{\leqslant}}
\renewcommand{\leq}{\ensuremath{\leqslant}}
\renewcommand{\ge}{\ensuremath{\geqslant}}
\renewcommand{\geq}{\ensuremath{\geqslant}}
\renewcommand{\emptyset}{\varnothing}
\renewcommand{\Im}{\operatorname{Im}}
\renewcommand{\Re}{\operatorname{Re}}


%%% Дополнительная работа с математикой
\usepackage{amsmath,amsfonts,amssymb,amsthm,mathtools} % AMS
\usepackage{icomma} % "Умная" запятая: $0,2$ --- число, $0, 2$ --- перечисление

%% Номера формул
%\mathtoolsset{showonlyrefs=true} % Показывать номера только у тех формул, на которые есть \eqref{} в тексте.
%\usepackage{leqno} % Нумереация формул слева

%% Свои команды
\DeclareMathOperator{\sgn}{\mathop{sgn}}
\DeclareMathOperator{\sign}{\mathop{sign}}
\DeclareMathOperator*{\res}{\mathop{res}}
\DeclareMathOperator*{\tr}{\mathop{tr}}

%% Перенос знаков в формулах (по Львовскому)
\newcommand*{\hm}[1]{#1\nobreak\discretionary{}
{\hbox{$\mathsurround=0pt #1$}}{}}

%%% Работа с картинками
\usepackage{graphicx}  % Для вставки рисунков
\graphicspath{{figures/}}  % папки с картинками
\setlength\fboxsep{3pt} % Отступ рамки \fbox{} от рисунка
\setlength\fboxrule{1pt} % Толщина линий рамки \fbox{}
\usepackage{wrapfig} % Обтекание рисунков текстом

%%% Работа с таблицами
\usepackage{array,tabularx,tabulary,booktabs} % Дополнительная работа с таблицами
\usepackage{longtable}  % Длинные таблицы
\usepackage{multirow} % Слияние строк в таблице

%%% Теоремы
\theoremstyle{plain} % Это стиль по умолчанию, его можно не переопределять.
\newtheorem{theorem}{Теорема}
\newtheorem*{thm}{Теорема}
\newtheorem{prop}{Утверждение}
 
\theoremstyle{definition} % "Определение"
%\newtheorem{corollary}{Следствие}[theorem]
\newtheorem*{dfn}{Определение}
\newtheorem{problem}{Задача}
\newtheorem*{problem*}{Задача}

 
\theoremstyle{remark} % "Примечание"
\newtheorem*{sol}{Решение}
\newtheorem*{rem}{Замечание}

%%% Программирование
\usepackage{etoolbox} % логические операторы

%%% Страница
%\usepackage{extsizes} % Возможность сделать 14-й шрифт
%\usepackage{geometry} % Простой способ задавать поля
%	\geometry{top=25mm}
%	\geometry{bottom=35mm}
%	\geometry{left=35mm}
%	\geometry{right=20mm}
 
\usepackage{fancyhdr} % Колонтитулы
%	\pagestyle{fancy}
 %	\renewcommand{\headrulewidth}{0pt}  % Толщина линейки, отчеркивающей верхний колонтитул
	%\lfoot{Нижний левый}
	%\rfoot{Нижний правый}
	%\rhead{Верхний правый}
	%\chead{Верхний в центре}
	%\lhead{Верхний левый}
	%\cfoot{Нижний в центре} % По умолчанию здесь номер страницы

\usepackage{setspace} % Интерлиньяж
%\onehalfspacing % Интерлиньяж 1.5
%\doublespacing % Интерлиньяж 2
%\singlespacing % Интерлиньяж 1

\usepackage{lastpage} % Узнать, сколько всего страниц в документе.

\usepackage{soul} % Модификаторы начертания

\usepackage{hyperref}
%\usepackage[usenames,dvipsnames,svgnames,table,rgb]{xcolor}
\hypersetup{				% Гиперссылки
    unicode=true,           % русские буквы в раздела PDF
    pdftitle={Заголовок},   % Заголовок
    pdfauthor={Автор},      % Автор
    pdfsubject={Тема},      % Тема
    pdfcreator={Создатель}, % Создатель
    pdfproducer={Производитель}, % Производитель
    pdfkeywords={keyword1} {key2} {key3}, % Ключевые слова
    colorlinks=true,       	% false: ссылки в рамках; true: цветные ссылки
    linkcolor=red,          % внутренние ссылки
    citecolor=black,        % на библиографию
    filecolor=magenta,      % на файлы
    urlcolor=cyan           % на URL
}

\usepackage{csquotes} % Еще инструменты для ссылок

%\usepackage[style=apa,maxcitenames=2,backend=biber,sorting=nty]{biblatex}

\usepackage{multicol} % Несколько колонок

\usepackage{tikz} % Работа с графикой
\usepackage{pgfplots}
\usepackage{pgfplotstable}
%\usepackage{coloremoji}
\usepackage{floatrow}
\usepackage{subcaption}
\newcommand*{\N}{\mathbb{N}}
\newcommand*{\R}{\mathbb{R}}
\newcommand*{\K}{\mathbb{K}}
\newcommand*{\V}{\mathcal{V}}
\newcommand*{\A}{\mathcal{A}}
\newcommand*{\ii}{\mathbf{1}}
\newcommand*{\oo}{\mathbf{0}}
\newcommand*{\ba}{\mathbf{a}}
\newcommand*{\bb}{\mathbf{b}}
\newcommand*{\Q}{\mathbb{Q}}
\graphicspath{{figures/}}
%\usepackage{breqn}

\renewcommand\thesubfigure{\asbuk{subfigure}}
%\addbibresource{master.bib}

\usepackage{import}
\usepackage{pdfpages}
\usepackage{transparent}
\usepackage{xcolor}
\usepackage{xifthen}

%\newcommand{\incfig}[1]{%
%    \def\svgwidth{\columnwidth}
%    \import{./figures/}{#1.pdf_tex}
%}


\newcommand{\incfig}[2][1]{%
    \def\svgwidth{#1\columnwidth}
    \import{./figures/}{#2.pdf_tex}
}
\usepackage{titlesec}
%\titleformat{\section}{\normalfont\Large\bfseries}{}{0pt}{}
%----------------------STANDART:
%\titleformat{\chapter}[display]
%  {\normalfont\huge\bfseries}{\chaptertitlename\ \thechapter}{20pt}{\Huge}
%\titleformat{\section}{\normalfont\Large\bfseries}{\thesection}{1em}{}
%\titleformat{\subsection}
%  {\normalfont\large\bfseries}{\thesubsection}{1em}{}
%\titleformat{\subsubsection}
%  {\normalfont\normalsize\bfseries}{\thesubsubsection}{1em}{}
%\titleformat{\paragraph}[runin]
%  {\normalfont\normalsize\bfseries}{\theparagraph}{1em}{}
%\titleformat{\subparagraph}[runin]
%  {\normalfont\normalsize\bfseries}{\thesubparagraph}{1em}{}

\pdfsuppresswarningpagegroup=1
\pgfplotsset{compat=1.16}

\usepackage{xifthen}
\makeatother
%\def\@lecture{}%
%\newcommand{\lecture}[3]{
%    \ifthenelse{\isempty{#3}}{%
%        \def\@lecture{Неделя #1}%
%    }{%
%        \def\@lecture{Неделя #1: #3}%
%    }%
%    \section*{\@lecture}
%    \marginpar{\small\textsf{\mbox{#2}}}
%}
\makeatletter

%\newcommand{\lec}{\subsection{Лекция}}
%\newcommand{\sem}{\subsection{Семинар}}
%\newcommand{\hw}{\subsection{Домашняя работа}}
%\newcommand{\prob}[1]{\textbf{#1}}
%\renewcommand{\thesubsection}{}
%\renewcommand{\thesubsubsection}{}

%\setcounter{tocdepth}{1} % only parts,chapters,sections
%\titleformat{\subsection}{\normalfont\large\bfseries}{}{0em}{}
%\titleformat{\subsubsection}{\normalfont\normalsize\bfseries}{}{0em}{}

%\newcommand{\textover}[2]{\stackrel{\mathclap{\normalfont\mbox{#2}}}{#1}}

\author{Драчов Ярослав\\
Факультет общей и прикладной физики МФТИ}
\newcommand{\veq}{\mathrel{\rotatebox{90}{$=$}}}
%\newcommand{\teto}[1]{\stackrel{\mathclap{\normalfont\tiny\mbox{#1}}}{\to}}
%\renewcommand{\thesubsection}{\arabic{subsection}}

%%\setcounter{secnumdepth}{0}

\definecolor{tabblue}{RGB}{30, 119, 180}
\definecolor{taborange}{RGB}{255, 127, 15}
\definecolor{tabgreen}{RGB}{45, 160, 43}
\definecolor{tabred}{RGB}{214, 38, 40}
\definecolor{tabpurple}{RGB}{148, 103, 189}
\definecolor{tabbrown}{RGB}{140, 86, 76}
\definecolor{tabpink}{RGB}{227, 119, 193}
\definecolor{tabgray}{RGB}{127, 127, 127}
\definecolor{tabolive}{RGB}{188, 189, 33}
\definecolor{tabcyan}{RGB}{22, 190, 207}
\pgfplotscreateplotcyclelist{colorbrewer-tab}{
{tabblue},
{taborange},
{tabgreen},
{tabred},
{tabpurple},
{tabbrown},
{tabpink},
{tabgray},
{tabolive},
{tabcyan},
}
\usepackage{csvsimple}
\usepackage{extarrows}
%\renewcommand{\labelenumii}{\asbuk{enumii})}
%\renewcommand{\labelenumiv}{\Asbuk{enumiv}}
\newcommand{\prob}[1]{\subsubsection*{#1}}
\sisetup{output-decimal-marker = {,},separate-uncertainty = true,exponent-product = \cdot}

\usepackage{braket}
\usepackage{enumerate}
\usepackage{chngcntr}
%\counterwithin*{equation}{problem}
%\usepackage{bbold}

\newtheoremstyle{hiProb}% ⟨name ⟩ 
{3pt}% ⟨Space above ⟩1 
{3pt}% ⟨Space below ⟩1
{}% ⟨Body font ⟩
{}% ⟨Indent amount ⟩2
{\bfseries}% ⟨Theorem head font⟩
{.}% ⟨Punctuation after theorem head ⟩
{.5em}% ⟨Space after theorem head ⟩3
%{\thmname{#1} \thmnote{#3}}% ⟨Theorem head spec (can be left empty, meaning ‘normal’)⟩
{\thmnote{#3}}% ⟨Theorem head spec (can be left empty, meaning ‘normal’)⟩
\theoremstyle{hiProb} % "Определение"
%\newtheorem{hiProb}{Задача}
\newtheorem{hiProb}{}
\usepackage{mmacells}
\newcommand{\textover}[2]{\stackrel{\mathclap{\normalfont\scriptsize\mbox{#2}}}{#1}}
\usepackage{units}
\usepackage[math]{cellspace}%
\setlength\cellspacetoplimit{2pt}
\setlength\cellspacebottomlimit{2pt}

%%% Работа с русским языком
\usepackage{cmap}					% поиск в PDF
\usepackage{mathtext} 				% русские буквы в формулах
\usepackage[T2A]{fontenc}			% кодировка
\usepackage[utf8]{inputenc}			% кодировка исходного текста
\usepackage[english,russian]{babel}	% локализация и переносы
\usepackage{indentfirst}
\frenchspacing

\renewcommand{\epsilon}{\ensuremath{\varepsilon}}
\newcommand{\phibackup}{\ensuremath{\phi}}
\renewcommand{\phi}{\ensuremath{\varphi}}
\renewcommand{\varphi}{\ensuremath{\phibackup}}
\renewcommand{\kappa}{\ensuremath{\varkappa}}
\renewcommand{\le}{\ensuremath{\leqslant}}
\renewcommand{\leq}{\ensuremath{\leqslant}}
\renewcommand{\ge}{\ensuremath{\geqslant}}
\renewcommand{\geq}{\ensuremath{\geqslant}}
\renewcommand{\emptyset}{\varnothing}
\renewcommand{\Im}{\operatorname{Im}}
\renewcommand{\Re}{\operatorname{Re}}

\author{Драчов Ярослав\\
Факультет общей и прикладной физики МФТИ}


\usepackage{ytableau}
\ytableausetup{boxsize=0.2em}
\title{Квантовая деформация иерархии $B$КП}
\begin{document}
\maketitle
\tableofcontents
\section{Иерархия КП}
\emph{Иерархия КП} --- бесконечный набор нелинейных дифференциальных
уравнений, первое из которых даётся как
\begin{equation}
\frac{1}{4} \frac{\partial ^2F}{\partial t_2^2} =
\frac{1}{3} \frac{\partial ^2 F}{\partial t_1 \partial t_3} 
-\frac{1}{2} \left( \frac{\partial ^2 F}{\partial t_1^2}  \right) ^2 -\frac{1}{12} \frac{\partial ^4F}{\partial t_1^4} 
\label{eq:FKP}
.\end{equation} 
Далее будут обсуждаться экспоненциированные решения
данной иерархии ($\tau$\emph{-функции}) $\tau\left( \mathbf{t} \right) =\exp \left( F(\mathbf{t}) \right) $.
В разделах \ref{subsec:lp} и \ref{subsec:bh}
опишем методы получения уравнений данной
иерархии. \subsection{Пары Лакса}
\label{subsec:lp}
Для псевдодифференциального оператора
	\begin{equation}
L=\partial + \sum_{j=1}^{\infty} f_j \partial^{-j},\qquad
\text{где}\qquad \partial = \frac{\partial }{\partial t_1},
\end{equation} 
условия совместности системы линейных уравнений
\begin{equation}
\left\{
\begin{aligned}
Lw&= kw, \\
\frac{\partial w}{\partial t_j} &=  B_j w,
\end{aligned}
\right.\qquad\text{где }
B_j=\left( L^j \right) _+.
\end{equation} 
могут быть записаны в Лаксовой форме
\begin{equation}
	\frac{\partial L}{\partial t_j} =\left[ 
	B_j,\,L\right] 
.\end{equation} 
Из второго уравнения данной иерархии получаем
\begin{equation}
\frac{\partial f_1}{\partial t_2} =
\frac{\partial ^2 f_1}{\partial t_1^2} +
2 \frac{\partial f_2}{\partial t_1} ,\qquad
\frac{\partial f_2}{\partial t_2} =\frac{\partial^2
f_2}{\partial t_1^2}+ 2 \frac{\partial f_3}{\partial t_1} 
+2 \frac{\partial f_1}{\partial t_1} f_1
,\qquad\ldots\end{equation} 
Из третьего --- 
\begin{equation}
\frac{\partial f_1}{\partial t_3} =
\frac{\partial ^3f_1}{\partial t_1^3} +3 \frac{\partial ^2f_2}{\partial t_1^2} +3 \frac{\partial f_3}{\partial t_1} 
+6 \frac{\partial f_1}{\partial t_1} f_1
,\qquad\ldots\end{equation} 
Устраняя $f_2$ и $f_3$ из данных уравнений и
пользуясь обозначениями  $u=-2f_1$, получаем
\emph{уравнение КП}
\begin{equation}
3 \frac{\partial ^2 u}{\partial t_2^2} =
\frac{\partial }{\partial t_1} \left( 
4 \frac{\partial u}{\partial t_3} +6 u \frac{\partial u}{\partial t_1} -\frac{\partial ^3u}{\partial t_1^3} \right) 
\label{eq:ukp}
.\end{equation} 
Требует пояснений введённое обозначение для $u$.
Решение первоначальной задачи на собственные значения
можно искать в виде
\begin{equation}
	w=\mathrm{e} ^{\xi\left( \mathbf{t},\,k \right) }\left( 1+ \frac{w_1}{k}+\frac{w_2}{k^2}+\ldots \right) ,\qquad \text{где}\qquad \xi\left( \mathbf{t},\,k \right) =
	\sum_{j=1}^{\infty} t_j k^j
.\end{equation} 
Тогда связь между $w_i$ и $f_i$ может быть найдена, например, из уравнений
\begin{equation}
\frac{\partial w}{\partial t_j} =B_j w
.\end{equation} 
Из уравнения на $t_1$ получаем
\begin{equation}
\frac{\partial w_1}{\partial t_1} =-f_1,\qquad\ldots
\label{eq:wf}
.\end{equation} 
Оказывается, что все функции $w_1,\,w_2,\ldots$ могут
быть выражены через одну функцию $\tau$ по формуле
\begin{equation}
	w=\frac{\displaystyle \tau\left( t_1- \frac{1}{k},\,t_2-\frac{1}{2k^2},\,t_3 -\frac{1}{3k^3},\ldots \right) }{\tau\left( t_1,\,t_2,\,t_3,\ldots \right) }
	\mathrm{e} ^{\xi\left( \mathbf{t},\,k \right) }
.\end{equation} 
Откуда, например,
\begin{equation}
w_1=-\frac{\partial \tau}{\partial t_1} \cdot \tau^{-1}
\label{eq:wtau}
.\end{equation} 
Иерархия КдФ получается из иерархии КП условием $L^2=\partial^2+u$ и бесконечный набор функций  $f_i$ выражается через $u$. Таким же свойством обладает
$\tau$-функция, между ними имеется связь
\begin{equation}
u= 2\frac{\partial ^2}{\partial t_1^2} \ln \tau
\label{eq:utau}
.\end{equation} 
Обозначение $u=-2f_1$ теперь поясняется формулами
\eqref{eq:wf}, \eqref{eq:wtau}, \eqref{eq:utau}.
Также, интегрируя два раза по $t_1$ уравнение  \eqref{eq:ukp} и принимая во внимание обозначение для $F$, получаем
в точности уравнение \eqref{eq:FKP}.
\subsection{Билинейное тождество Хироты}
\label{subsec:bh}
\begin{dfn*}
	\emph{Производные хироты} $\mathrm{D}_1^{n_1} \cdots \mathrm{D}_m^{n_m}$ 
задаются из соотношения
\begin{equation}
	\mathrm{e} ^{y_1 \mathrm{D}_1+y_2 \mathrm{D}_2+ \ldots}f\cdot g=
f\left( x_1+y_1,\,x_2+y_2,\ldots \right) 
g\left( x_1-y_1,\,x_2-y_2,\ldots \right) 
.\end{equation} 
\end{dfn*}
\begin{thm*}[билинейное тождество]
Для любых $x$ и $x'$ положим
\begin{equation}
	\xi=\xi\left( \mathbf{t},\,k \right) ,\qquad
	\xi'=\xi\left( \mathbf{t}',\,k \right) 
.\end{equation} 
Тогда справедливо следующее тождество:
\begin{equation}
0= \oint \frac{\mathrm{d} k}{2\pi \mathrm{i} }
\mathrm{e} ^{\xi-\xi'} \tau\left( t_1-\frac{1}{k},\,
t_2- \frac{1}{2k^2},\ldots\right) \tau\left( 
t_1'+\frac{1}{k},\,t_2'+ \frac{1}{2k^2},\ldots\right) 
.\end{equation} 
\end{thm*}
Уравнения КП получаются из билинейного
тождества после замены $t_j=x_j+y_j$, $t_j'=x_j-y_j$:
 \begin{multline}
\oint \frac{\mathrm{d} k}{2\pi \mathrm{i} }
\exp \left( 2 \sum_{j=1}^{\infty} k^j y_j \right) 
\tau \left( x_1+y_1-\frac{1}{k},\,x_2+y_2-\frac{1}{2k^2},\ldots \right)\times\\\times
\tau\left( x_1-y_1+\frac{1}{k},\,x_2-y_2+\frac{1}{2k^2},\ldots \right) =\\= \oint \frac{\mathrm{d} k}{2\pi \mathrm{i}}
\exp \left( 2 \sum_{j=1}^{\infty} k^j y_j \right) 
\exp \left( \sum_{j=1}^{\infty} \left( y_l-\frac{1}{l k^l} \right) \mathrm{D}_l \right) \tau\cdot \tau
.\end{multline} 
Раскладывая подынтегральную функцию в ряд по степеням
$y_j$ и вычисляя коэффициент при $k^{-1}$, получаем
уравнения КП. Например
\begin{equation}
	(\mathrm{D}_1^4 +3\mathrm{D}_2^2-4\mathrm{D}_1 \mathrm{D}_3)\tau\cdot \tau=0,\qquad
	(\mathrm{D}_1^3 \mathrm{D}_2+2\mathrm{D}_2 \mathrm{D}_3-3\mathrm{D}_1 \mathrm{D}_4)\tau\cdot \tau=0
.\end{equation} 
Первое из них --- буквально \eqref{eq:FKP} с учётом
определения $F$.
\subsection{Полиномы Шура, соотношения Плюккера, гипергеометрические $\tau$-функции}
\label{subsec:schur}
Решения данной иерархии
могут быть разложены по базису полиномов Шура
\begin{equation}
	\tau\left( \mathbf{t} \right) =\sum_{\lambda}^{} C_\lambda s_\lambda\left( \mathbf{t} \right) 
\label{eq:2}
.\end{equation} 
Можно показать, что $\tau(\mathbf{t})$ --- решение
иерархии КП тогда и только тогда, когда коэффициенты
$C_\lambda$ удовлетворяют соотношениям Плюкера, первое
из которых
\begin{equation}
C_{\left[ 2,\,2 \right] }C_{\left[ \emptyset \right] }
-C_{\left[ 2,\,1 \right] }C_{\left[ 1 \right] }
+C_{\left[ 2 \right] }C_{\left[ 1,\,1 \right] }=0
.\end{equation} 
Имеется важное подмножество решений иерархии КП --- 
\emph{гипергеометрические} $\tau$\emph{-функции}.
Они определяются как
\begin{equation}
	\tau\left( \mathbf{t} \right) 
	=\sum_{\lambda}^{} r_\lambda
	s_\lambda (\beta) s_\lambda \left( \mathbf{t} \right) 
,\end{equation} 
где $s_\lambda(\beta)$ --- полином Шура от переменных
$\beta_k$,
\begin{equation}
	r_\lambda=\prod_{w \in \lambda}^{} r\left( c\left( w \right)  \right)  
,\end{equation} 
\begin{equation}
	c(w)=j-i,\qquad 1\le  i\le l(\lambda),\qquad
	1\le j\le \lambda_i
.\end{equation} 
\subsection{$\tau$-функция чисел Гурвица}
Производящая функция простых чисел Гурвица
\begin{equation}
	\tau_H(\mathbf{t})=
	\sum_{m=0}^{\infty} \sum_{\mu}^{} h_{m;\mu}^{\circ}
	t_{\mu_1} t_{\mu_2} \cdots t_{\mu_{l(\mu)}}
	\frac{u^m}{m!}
,\end{equation} 
где
\begin{equation}
h^{\circ}_{m;\mu}= \frac{1}{|\mu|!}
\left| \left\{ \left( \eta_1,\ldots,\,\eta_m \right) ,\ 
\eta_i \in C_2 \left( S_{|\mu|} \right) \colon
\eta_m \circ \cdots \circ \eta_1 \in C_\mu\left( S_{|\mu|} \right) \right\}  \right| 
,\end{equation} 
$S_{|\mu|}$ --- симметрическая группа перестановок $\mu$ 
элементов, $C_2\left( S_{|\mu|} \right) $ --- множество
всех транспозиций в $S_{|\mu|}$ и $C_\mu \left( S_{|\mu|} \right) $ --- множество всех перестановок циклического типа $\mu$.

Можно показать (TODO), что
\begin{equation}
	\tau_H(\mathbf{t})=\sum_{\lambda}^{} \mathrm{e}^{u c(\lambda)}s_\lambda \left( \beta_k=\delta_{k,\,1} \right) 
	s_\lambda(\mathbf{t})
\label{eq:4}
,\end{equation} 
где $c(\lambda)= \sum_{w \in \lambda}^{} c(w)$. 
То есть данная производящая функция --- гипергеометрическая
$\tau$-функция с параметрами
\begin{equation}
	r(n)=\mathrm{e} ^{un},\qquad
	\beta_1=1,\qquad \beta_k=0,\ k\ge 2
.\end{equation} 
\subsection{Фермионный формализм и бозон-фермионное соответствие}
Будем рассматривать бесконечномерную алгебру Клиффорда
с генераторами $\left\{ \psi_n,\,\psi^*_m \mid n,\,m \in \mathbb{Z} \right\} $, обладающими коммутационными соотношениями
\begin{equation}
\left\{ \psi_n,\,\psi_m \right\} =0,\qquad
\left\{ \psi_n^*,\,\psi^*_{m} \right\}=0,\qquad
\left\{ \psi_n,\,\psi_{m}^* \right\}=\delta_{n,\,m}
.\end{equation} 
Пространство Фока для фермионов определим дествием
алгебры Клиффорда на вакуумный вектор $\ket{0}$ <<моря Дирака>>. Действие генераторов алгебры на вакуумный вектор
даётся как
\begin{equation}
\psi_k\ket{0}=0,\quad k<0;\qquad\psi^* \ket{0}=0,\quad
k\ge 0
.\end{equation} 
То есть операторы $\psi_k$ при $k\ge 0$ и $\psi^*_k$ при
$k<0$ --- это операторы рождения, а $\psi_k$ при $k<0$ 
и $\psi_k^*$ при  $k\ge 0$ --- операторы уничтожения.

Введём также производящие функции для фермионов
\[
\psi(z)=\sum _{k\in \mathbb{Z}}^{} \psi_k z^k,\qquad
\psi^* (z)=\sum_{k \in \mathbb{Z}}^{} \psi^*_k z^{-k-1}
.\] 
Будем обозначать нормальное упорядочение фермионных
операторов как $\normord{\left( \ldots \right) }$.
Результатом нормального упорядочения будет перемещение
всех операторов уничтожения направо, а операторов
рождения --- налево, где каждая транспозиция двух
фермионов будет производиться в согласии с антикоммутационным
соотношением и, следовательно, давать множитель $(-1)$. 

Рассмотрим так же матричную алгебру Ли $\bar{\mathfrak{a}}_\infty$.
Элементами данной алгебры будут бесконечномерные матрицы,
у которых конечное количество диагоналей отличны от нулевых.
Базисом данной алгебры можно выбрать матрицы  $E_{ij}$,
у которых единственный, отличный от нуля, элемент равен 1 и находится
на пересечении $i$-й строки и $j$-го столбца.
 \begin{equation}
	(E_{ij})_{kl}=\delta_{ik}\delta_{jl}
.\end{equation} 
\begin{multline}
\left[ E_{ij},\,E_{kl} \right] =
E_{ij}E_{kl}-E_{kl}E_{ij}=
\delta_{in}\delta_{jm}\delta_{km}\delta_{lp}-
\delta_{kn}\delta_{lm}\delta_{im}\delta_{jp}=\\=
\delta_{jk}E_{il}-\delta_{il}E_{kj}
.\end{multline} 


Представлением данной алгебры
на введённом пространстве Фока будет
\begin{equation}
	r(E_{ij})=\psi_i \psi_j^{*}
\end{equation} 
т.\:к.
\begin{multline}
	\left[ \psi_i \psi^*_j,\,\psi_k \psi^*_l \right] 
	=\psi_i \psi_j^* \psi_k \psi^*_l -\psi_k \psi^*_l
	\psi_i \psi^*_j=\\=\psi_i \left( \delta_{jk} -\psi_k \psi_j^* \right) \psi^*_l-\psi_k \psi_l^* \psi_i \psi^*_j=\\=\delta_{jk}\psi_i \psi^*_l-\psi_k \psi_i  \psi_l^*
	\psi_j^*-\psi_k \psi_l^* \psi_i \psi_j^*=\\=
	\delta_{jk} \psi_i \psi_{l}^*-
	\delta_{il}\psi_k \psi_j^*
.\end{multline}

Далее рассмотрим центральное расширение алгебры Ли
$\bar{\mathfrak{a}}_\infty$, а именно алгебру Ли $\mathfrak{a}_\infty$, определённую как
\begin{equation}
	\mathfrak{a}_\infty= \bar{\mathfrak{a}}_\infty
	\oplus  \mathbb{C}c
\end{equation} 
с центром $\mathbb{C}c$ и скобкой
\begin{equation}
	\left[ a,\,b \right] =ab-ba+ \alpha\left( a,\,b \right) c
,\end{equation} 
где два-коцикл $\alpha(a,\,b)$ линеен по каждому
переменному и определён на матрицах $E_{ij}$ как
\begin{equation}
	\begin{aligned}
		&\alpha\left( E_{ij},\,E_{ji} \right) =-\alpha
	\left( E_{ji},\,E_{ij} \right) =1\quad
	\text{при } i< 0,\ j\ge 0,\\
	&\alpha\left( E_{ij},\,E_{kl} \right) =0
		\quad \text{во всех остальных случаях.}
	\end{aligned}
	\label{}
\end{equation}
%Коммутационными соотношениями данной алгебры будут
%\begin{align}
%	\left[ E_{ij},\,E_{kl} \right] &=0 &&
%	\text{при } j \neq k ,\ l \neq i,\\
%	\left[ E_{ij},\,E_{jl} \right] &=
%	E_{il} && \text{при } l\neq i,\\
%	\left[ E_{ij},\,E_{ki} \right] &=-E_{kj}
%				       && \text{при } j
%				       \neq k,
%	\left[ E_{ij},\,E_{ji} \right] =
%.\end{align}

Покажем, что представлением данной
алгебры на пространстве Фока будет
\[
	\hat{r}(E_{ij})= \normord{\psi_i \psi_j^*}
.\] 
%\[
%	\braket{0 | \left[ \normord{\psi_i\psi_j^*},\,\normord{\psi_k \psi^*_l} \right] |0}=
%.\] 
%\[
%	\braket{\mathcal{S} | \normord{\psi_i \psi_j^*} |\ldots,\,n_{-1},\,n_0,\,n_1,\ldots}
%.\] 
%\begin{multline*}
%	\left[ \normord{\psi_i\psi_j^*},\,\normord{\psi_k \psi^*_l} \right]=
%\begin{cases}
%	\left[ \psi_i\psi_j^*,\,\psi_k \psi^*_l \right],&
% j\ge 0,\, l\ge 0,\\
%	\left[ \psi_i\psi_j^*,\,\psi_k \psi^*_l \right],&
% j\ge 0,\, l< 0,\,k\ge 0,\\
%	\left[ \psi_i\psi_j^*,\,\psi^*_l \psi_k \right],&
% j\ge 0,\, l< 0,\,k< 0,\\
%		\left[ \psi_i\psi_j^*,\,\psi_k \psi^*_l \right],&
%j<0,\, l\ge 0,\,i\ge 0,\\
%		\left[ \psi_j^*\psi_i,\,\psi_k \psi^*_l \right],&
%j<0,\, l\ge 0,\,i< 0,\\
%		\left[ \psi_i\psi_j^*,\,\psi_k \psi^*_l \right],&
%j<0,\, l< 0,\,i\ge  0,\,k\ge 0,\\
%		\left[ \psi_i\psi_j^*,\,\psi^*_l\psi_k  \right],&
%j<0,\, l< 0,\,i\ge  0,\,k< 0,\\
%		\left[ \psi_j^*\psi_i,\,\psi_k \psi^*_l \right],&
%j<0,\, l< 0,\,i<  0,\,k\ge  0,\\
%		\left[ \psi_j^*\psi_i,\,\psi^*_l\psi_k  \right],&
%j<0,\, l< 0,\,i<  0,\,k<0,  \\
%\end{cases}
%\end{multline*}
По определению свёртки
\begin{equation}
	\normord{\psi_i \psi_j^*}=\psi_i \psi_j^*-
	\wick{\c\psi_i \c\psi_j^*}
	\label{}
.\end{equation}
Откуда
\begin{align}
	\wick{\c\psi_i \c\psi_i^*}&=1 && \text{при }
	i<0,\\
	\wick{\c \psi_i \c \psi_j^*}&=0 && \text{во всех остальных случаях}
.\end{align}
Тогда
\begin{multline}
	\left[ \normord{\psi_i \psi_j^*},\,\normord{
	\psi_k \psi_l^*} \right] =
	\wick{\left[ \psi_i\psi_j^*-\c\psi_i \c\psi_j^* ,\,\psi_k \psi_l^*- \c1 \psi_k \c1 \psi_l^* \right]}=\\=
	\left[ \psi_i \psi_j^*,\,\psi_{k}\psi_l^* \right]
	=\delta_{jk} \psi_i \psi_l^* -\delta_{il}\psi_k \psi_j^*=\\=\delta_{jk} \wick{\left( \normord{\psi_i \psi_l^*}+
	\c\psi_i \c\psi_l^*\right) }-\delta_{il} \wick{\left( \normord{\psi_k \psi_j^*}+
	\c\psi_k \c\psi_j^*\right) }=\\=
	\delta_{jk} \normord{\psi_i\psi_l^*}
	-\delta_{il} \normord{\psi_k \psi_j^*}+
	\delta_{jk} \wick{\c \psi_i \c\psi_l^*}-
	\delta_{il} \wick{\c \psi_k \c \psi_j^*}=\\=
	\delta_{jk} \normord{\psi_i\psi_l^*}
	-\delta_{il} \normord{\psi_k \psi_j^*}+
	\alpha\left( E_{ij},\,E_{kl} \right) 
	\label{}
.\end{multline}
%\begin{multline*}
%	\left[ \normord{\psi_i\psi_j^*},\,\normord{\psi_k \psi^*_l} \right]=\normord{\psi_i \psi_j^*}\normord{\psi_k \psi_l^*}-\normord{\psi_k \psi_l^*}\normord{\psi_i \psi_j^*}=\\=
%\begin{cases}
%	\psi_i \psi^*_j \psi_k \psi_l^*-
%	\psi_k \psi_l^* \psi_i \psi_j^*,&
% j\ge 0,\, l\ge 0,\\
%	\psi_i \psi^*_j \psi_k \psi_l^*-
%	\psi_k \psi_l^* \psi_i \psi_j^*,&
% j\ge 0,\, l< 0,\,k\ge 0,\\
%	\psi_i \psi^*_j \psi_k \psi_l^*-
%	\psi_k \psi_l^* \psi_i \psi_j^*,&
% i,\,l\ge 0,\ j< 0,\\
%\end{cases}
%\end{multline*}

Бозон-фермионное соответствие $\Phi: \mathcal{F} \to \mathcal{B}$:
\ytableausetup{boxsize=0.5em}
\begin{equation}
	\Phi\left( \ket{0,\,\lambda} \right) =s_\lambda(\mathbf{t})
.\end{equation} 
\begin{equation}
	\Phi\left( G \ket{0} \right) = \braket{0 | \mathrm{e} ^{H(\mathbf{t})}G | 0},\qquad H(\mathbf{t})= \sum_{k=1}^{\infty} t_k H_k,\qquad
	H_n= \sum_{k \in \mathbb{Z}}^{} \normord{\psi_k \psi_{k+n}^{*}}
.\end{equation} 
В случае иерархии КП $\tau$-функция в фермионном представлении имеет вид
 \begin{equation}
	 \tau(\mathbf{t})= \braket{0 | \mathrm{e}^{H(\mathbf{t})}G | 0}
.\end{equation} 
Сравнивая с $\tau$-функцией в бозонном представлении \eqref{eq:2}
можно получить, что
\begin{equation}
	C_\lambda= \det_{l(\lambda)\times l(\lambda)}
	\left( G^{-1,-2,-3,\ldots}_{i_{-1},i_{-2},i_{-3},\ldots} \right) 
,\end{equation} 
где $i_{-1},\,i_{-2},\ldots$ определяются из $\lambda= \left[ i_{-1}+1,\,i_{-2}+2,\ldots \right] $.
Для гипергеометрических $\tau$-функций выполняется
\begin{equation}
	G=\mathrm{e} ^{A(\beta)},\qquad
	A(\beta)=\sum_{k=1}^{\infty} \beta_k A_k
.\end{equation} 
\begin{equation}
A_k= \oint \frac{dz}{2\pi \mathrm{i} } \normord{
\left[ \left( \frac{1}{z} r(D) \right) ^k \psi(z) \right] \cdot \psi^*(z)}
,\end{equation} 
где $D=z \frac{\mathrm{d} }{\mathrm{d} z}$.

\section{Квантовая деформация иерархии КП}


В $\hbar $\emph{-деформированной иерархии КП} связь между $F$ и $\tau$ будет следующей
\begin{equation}
	F^\hbar \left( \mathbf{t} \right) =
	\hbar ^2 \ln \left( \tau^\hbar \left( \mathbf{t} \right)  \right) 
.\end{equation} 
Уравнения иерархии $\hbar$-КП получаются  из
иерархии КП заменой  $\mathbf{t}\to \mathbf{t} /\hbar $.
Первое уравнение деформированной иерархии
\begin{equation}
	\frac{1}{4} \frac{\partial ^2 F}{\partial t_2^2} =\frac{1}{3} \frac{\partial ^2 F}{\partial t_1 \partial t_3} -\frac{1}{2} \left( \frac{\partial^2 F}{\partial t_1^2} \right) ^2-\frac{\hbar ^2}{12} \frac{\partial^4 F}{
	\partial t_1^4}
.\end{equation} 
$\tau$-функции  иерархии $\hbar $-КП
\begin{equation}
	\tau^\hbar \left( \mathbf{t} \right) =
	\sum_{\lambda}^{} C_\lambda^\hbar 
	s_\lambda \left( \frac{\mathbf{t}}{\hbar } \right) 
	\label{eq:3}
,\end{equation} 
где $C^\hbar _{\lambda}$ удовлетворяют соотношениям Плюкера.
В разделах далее будем добавлять $\hbar $
в $\tau$-функции так, чтобы полученные $\tau^\hbar $-функции
удовлетворяли $\hbar$-деформированной иерархии, а также
имели  <<хорошее>> разложение по степеням $\hbar $.
\subsection{Фермионный формализм}
Рецепт деформации $\tau$-функций в фермионном формализме
\begin{equation}
	\tau^\hbar (\mathbf{t})= \braket{0 | \mathrm{e} ^{H\left( \mathbf{t} /\hbar  \right) } \exp \left( \frac{1}{\hbar }A^\hbar  \right) | 0}
,\end{equation} 
где
\begin{equation}
A^\hbar = \oint \frac{dz}{2\pi \mathrm{i} }
\normord{\left[ \hat{A}\left( z,\,\hbar \frac{\mathrm{d} }{\mathrm{d} z}\right)\psi(z)   \right] \cdot \psi^*(z)}
\end{equation} 
и
\begin{equation}
	\hat{A}\left( z,\,\hbar  \frac{\mathrm{d} }{\mathrm{d} z} \right) = \sum_{i \in \mathbb{Z},j\ge 0}^{} a_{ij}
	z^i \left( \hbar  \partial_z \right) ^j
.\end{equation} 
\subsection{$\tau$-функция чисел Гурвица}
Формула Римана-Гурвица
\begin{equation}
	2g-2=m-\left| \mu \right| -l(\mu)
,\end{equation} 
позволяет нам разделить вклады различных родов $g$ 
в производящую функцию. Каждая точка простого ветвления
даёт вклад $+1$ к степени $\hbar $, каждый цикл
длины $\mu_i$ даёт вклад $-\mu_i -1$ к степени $\hbar $.
Получаем замену переменных
\begin{equation}
t_{\mu_i}\to  \hbar ^{-\mu_i -1} t_{\mu_i},\qquad
u\to \hbar u
\label{eq:1}
.\end{equation} 
Умножая производящую функцию на $\hbar ^2$, чтобы
избавиться от отрицательных степеней $\hbar $ 
в разложении, получаем
\begin{equation}
	F^\hbar _H(\mathbf{t})=\sum_{g=0}^{\infty} \hbar ^{2g}F_H^g(\mathbf{t})
.\end{equation}
Покажем, что топологической деформацией \eqref{eq:1}
из $\tau$-функции иерархии КП \eqref{eq:4} действительно можно
получить $\tau$-функцию иерархии  $\hbar $-КП \eqref{eq:3}.
Имеем
\begin{equation}
	\tau^\hbar _H(\mathbf{t})=
	\sum_{\lambda}^{} \mathrm{e} ^{u\hbar c(\lambda)}
	s_\lambda\left( \beta_k =\delta_{k,1} \right) 
	s_\lambda\left( \frac{t_1}{\hbar ^2},\,
	\frac{t_2}{\hbar ^3},\ldots\right) 
.\end{equation}
Можно показать (это вопросов не вызывает), что
\begin{equation}
	\tau^\hbar _H(\mathbf{t})=
	\sum_{\lambda}^{} \mathrm{e} ^{u\hbar c(\lambda)}
	s_\lambda\left( \frac{1}{\hbar },\,0,\,0,\ldots \right) s_\lambda\left( \frac{\mathbf{t}}{\hbar } \right) 
\end{equation}
и что коэффициенты
\begin{equation}
	C_\lambda^\hbar = \mathrm{e} ^{u\hbar  c(\lambda)}
	s_\lambda \left( \beta_k =\frac{\delta_{k,1}}{\hbar } \right) 
\end{equation}
удовлетворяют соотношениям Плюкера. Это показывает, что $\hbar $-деформацией
мы получаем $\tau$-функцию иерархии $\hbar $-КП. Для найденной функции было явно посчитано разложение по $\mathbf{t}$ для первых порядков и проверено, что в неё действительно входят только чётные степени $\hbar $.
\section{Иерархия $B$КП}
\subsection{Билинейное тождество $B$КП}
\emph{Билинейное тождество иерархии $B$КП}
\begin{multline}
\frac{1}{2\pi \mathrm{i} } \oint
\mathrm{e}^{\xi^{\text{н}}\left( \mathbf{t}-\mathbf{t}',\,k \right) }\tau_{B\text{КП}}\left( \mathbf{t}-2 \left[ k^{-1} \right]  \right) \tau_{B\text{КП}}\left( \mathbf{t}'+2 \left[ k^{-1} \right]  \right) \frac{\mathrm{d} k}{k}=\\=
\tau_{B\text{КП}}\left( \mathbf{t} \right) \tau_{B\text{КП}}
(\mathbf{t}')
,\end{multline} 
где
\begin{equation}
	\mathbf{t}\pm \left[ k^{-1} \right] 
	\xlongequal[]{\text{опр}}
	\left\{ t_1\pm k^{-1},\,t_2 \pm \frac{1}{2}
	k^{-2},\,t_3\pm \frac{1}{3}k^{-3},\ldots\right\} 
\end{equation} 
и
\begin{equation}
	\xi^\text{н}\left( \mathbf{t},\,k \right)=
	\sum_{j \in \mathbb{Z}_{\text{неч}}^+}^{} t_j k^j
.\end{equation} 
Первое уравнение иерархии $B$КП в терминах производных
Хироты
\begin{equation}
	\left( \mathrm{D}_1^6-5\mathrm{D}_1^3 \mathrm{D}_3
	-5 \mathrm{D}_3^2+9 \mathrm{D}_1 \mathrm{D}_5\right) \tau_{B\text{КП}}\cdot \tau_{B \text{КП}}
.\end{equation} 
Что можно переписать как
\begin{multline}
-60 \left(\frac{\partial ^2F}{\partial
   t_1^2}\right)^3-30 \frac{\partial ^4F}{\partial
   t_1^4} \frac{\partial ^2F}{\partial
   t_1^2}+30 \frac{\partial ^2F}{\partial t_1\, \partial t_3}
   \frac{\partial ^2F}{\partial t_1^2}-\frac{\partial
   ^6F}{\partial t_1^6}+\\+5 \frac{\partial ^2F}{\partial
   t_3^2}-9 \frac{\partial ^2F}{\partial t_1\, \partial t_5}+5
   \frac{\partial ^4F}{\partial t_1^3\, \partial t_3}=0	
	\label{}
.\end{multline}
\subsection{$Q$-полиномы Шура, соотношения Плюккера  $B$КП,
гипергеометрические  $\tau^{B\text{КП}}$-функции}
\emph{Соотношения Плюккера} для $B$КП
\begin{multline}
c_{\left[ \alpha_1,\ldots,\,\alpha_k \right] }c_{\left[ 
\alpha_1,\ldots,\,\alpha_k,\,\beta_1,\,\beta_2,\,\beta_3,\,\beta_4\right] }-c_{\left[ \alpha_1,\ldots,\,\alpha_k,\,
\beta_1,\,\beta_2\right] }c_{\left[ \alpha_1,\ldots,\,
\alpha_k,\,\beta_3,\,\beta_4\right] }+\\+c_{\left[ \alpha_1,\ldots,\,\alpha_k,\,
\beta_1,\,\beta_3\right] }c_{\left[ \alpha_1,\ldots,\,
\alpha_k,\,\beta_2,\,\beta_4\right] }-c_{\left[ \alpha_1,\ldots,\,\alpha_k,\,
\beta_1,\,\beta_4\right] }c_{\left[ \alpha_1,\ldots,\,
\alpha_k,\,\beta_2,\,\beta_3\right] }=0
.\end{multline} 
\ytableausetup{boxsize=0.2em}
Простейшее соотношение Плюккера \begin{equation}
	c_{\emptyset}c_{\ydiagram{3,2,1}}-c_{\ydiagram{1}}c_{\ydiagram{3,2}}+c_{\ydiagram{2}}c_{\ydiagram{3,1}}-c_{\ydiagram{3}}c_{\ydiagram{2,1}}=0
.\end{equation} 
Проверено, что \emph{$Q$-полиномы Шура} действительно удовлетворяют
простейшему соотношению Плюккера.

Для определения гипергеометрических $\tau$-функций зададим функцию
\begin{equation}
	r_\lambda= \prod_{w \in \lambda}^{} r\left( c(w) \right)  
,\end{equation} 
где
\begin{equation}
	c(w)=j,\qquad  1\le i\le l(\lambda),\qquad
	1\le j\le \lambda_i
.\end{equation} 
Визуализация функции $c(w)$ на таблице Юнга:
\ytableausetup{boxsize=1.5em}
\begin{center}
\begin{ytableau}
	1&2&3&4\\
	1&2&3\\
	1&2
\end{ytableau}
\end{center}
\emph{Гипергеометрическими $\tau$-функциями $B$КП} будем называть
функции вида
\begin{equation}
	\tau(\mathbf{t})=\sum_{\lambda}^{} r_\lambda
	Q_\lambda(\boldsymbol{\beta}) Q_\lambda(\mathbf{t})
.\end{equation} 
Данные функции действительно решают иерархию $B$КП,  т.\:к.
$Q$-полиномы Шура удовлетворяют соотношениям Плюккера $B$КП,
а множители $r(c(w))$ выносятся как общие. Например,
для простейшего соотношения Плюккера общим множителем
будет
\begin{equation}
	r(1)^3r(2)^2 r(3)
.\end{equation} 
\subsection{$\tau$-функция спиновых чисел Гурвица}
Следующая $\tau$-функция является решением иерархии $B$КП:
\begin{equation}
	\tau \left( \mathbf{p},\,\bar{\mathbf{p}} \right) =
	\sum_{R \in \mathrm{SP}}^{} \left( \mathrm{e} ^{u \left[ 
	\Phi_R\left( \left[ 3 \right]  \right) +\frac{1}{2}\Phi_R \left( \left[ 1,\,1 \right]  \right) \right] } \right) Q_R\left(\frac{\mathbf{p}}{2}\right)Q_R \left(\frac{\bar{\mathbf{p}}}{2}\right)
	\label{}
.\end{equation}
\section{Квантовая деформация иерархии $B$КП}
Аналогично деформации иерархии КП получаем первое уравнение $\hbar $-$B$КП
\begin{multline}
-60 \left(\frac{\partial ^2F}{\partial
   t_1^2}\right)^3-30\hbar ^2 \frac{\partial ^4F}{\partial
   t_1^4} \frac{\partial ^2F}{\partial
   t_1^2}+30 \frac{\partial ^2F}{\partial t_1\, \partial t_3}
   \frac{\partial ^2F}{\partial t_1^2}-\hbar ^4\frac{\partial
   ^6F}{\partial t_1^6}+\\+5 \frac{\partial ^2F}{\partial
   t_3^2}-9 \frac{\partial ^2F}{\partial t_1\, \partial t_5}+5
   \hbar ^2\frac{\partial ^4F}{\partial t_1^3\, \partial t_3}=0	
	\label{}
.\end{multline}

\end{document}
