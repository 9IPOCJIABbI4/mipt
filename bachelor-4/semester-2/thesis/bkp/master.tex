\documentclass[a5paper,twoside]{extarticle}
%\usepackage{fullpage}
% Этот шаблон документа разработан в 2014 году
% Данилом Фёдоровых (danil@fedorovykh.ru) 
% для использования в курсе 
% <<Документы и презентации в \LaTeX>>, записанном НИУ ВШЭ
% для Coursera.org: http://coursera.org/course/latex .
% Исходная версия шаблона --- 
% https://www.writelatex.com/coursera/latex/5.3

% В этом документе преамбула

\usepackage{siunitx}
%%% Работа с русским языком
\usepackage{cmap}					% поиск в PDF
\usepackage{mathtext} 				% русские буквы в формулах
\usepackage[T2A]{fontenc}			% кодировка
\usepackage[utf8]{inputenc}			% кодировка исходного текста
\usepackage[english,russian]{babel}	% локализация и переносы
\usepackage{indentfirst}
\frenchspacing

\renewcommand{\epsilon}{\ensuremath{\varepsilon}}
\renewcommand{\phi}{\ensuremath{\varphi}}
\renewcommand{\kappa}{\ensuremath{\varkappa}}
\renewcommand{\le}{\ensuremath{\leqslant}}
\renewcommand{\leq}{\ensuremath{\leqslant}}
\renewcommand{\ge}{\ensuremath{\geqslant}}
\renewcommand{\geq}{\ensuremath{\geqslant}}
\renewcommand{\emptyset}{\varnothing}
\renewcommand{\Im}{\operatorname{Im}}
\renewcommand{\Re}{\operatorname{Re}}


%%% Дополнительная работа с математикой
\usepackage{amsmath,amsfonts,amssymb,amsthm,mathtools} % AMS
\usepackage{icomma} % "Умная" запятая: $0,2$ --- число, $0, 2$ --- перечисление

%% Номера формул
%\mathtoolsset{showonlyrefs=true} % Показывать номера только у тех формул, на которые есть \eqref{} в тексте.
%\usepackage{leqno} % Нумереация формул слева

%% Свои команды
\DeclareMathOperator{\sgn}{\mathop{sgn}}
\DeclareMathOperator{\sign}{\mathop{sign}}
\DeclareMathOperator*{\res}{\mathop{res}}
\DeclareMathOperator*{\tr}{\mathop{tr}}

%% Перенос знаков в формулах (по Львовскому)
\newcommand*{\hm}[1]{#1\nobreak\discretionary{}
{\hbox{$\mathsurround=0pt #1$}}{}}

%%% Работа с картинками
\usepackage{graphicx}  % Для вставки рисунков
\graphicspath{{figures/}}  % папки с картинками
\setlength\fboxsep{3pt} % Отступ рамки \fbox{} от рисунка
\setlength\fboxrule{1pt} % Толщина линий рамки \fbox{}
\usepackage{wrapfig} % Обтекание рисунков текстом

%%% Работа с таблицами
\usepackage{array,tabularx,tabulary,booktabs} % Дополнительная работа с таблицами
\usepackage{longtable}  % Длинные таблицы
\usepackage{multirow} % Слияние строк в таблице

%%% Теоремы
\theoremstyle{plain} % Это стиль по умолчанию, его можно не переопределять.
\newtheorem{theorem}{Теорема}
\newtheorem*{thm}{Теорема}
\newtheorem{prop}{Утверждение}
 
\theoremstyle{definition} % "Определение"
%\newtheorem{corollary}{Следствие}[theorem]
\newtheorem*{dfn}{Определение}
\newtheorem{problem}{Задача}
\newtheorem*{problem*}{Задача}

 
\theoremstyle{remark} % "Примечание"
\newtheorem*{sol}{Решение}
\newtheorem*{rem}{Замечание}

%%% Программирование
\usepackage{etoolbox} % логические операторы

%%% Страница
%\usepackage{extsizes} % Возможность сделать 14-й шрифт
%\usepackage{geometry} % Простой способ задавать поля
%	\geometry{top=25mm}
%	\geometry{bottom=35mm}
%	\geometry{left=35mm}
%	\geometry{right=20mm}
 
\usepackage{fancyhdr} % Колонтитулы
%	\pagestyle{fancy}
 %	\renewcommand{\headrulewidth}{0pt}  % Толщина линейки, отчеркивающей верхний колонтитул
	%\lfoot{Нижний левый}
	%\rfoot{Нижний правый}
	%\rhead{Верхний правый}
	%\chead{Верхний в центре}
	%\lhead{Верхний левый}
	%\cfoot{Нижний в центре} % По умолчанию здесь номер страницы

\usepackage{setspace} % Интерлиньяж
%\onehalfspacing % Интерлиньяж 1.5
%\doublespacing % Интерлиньяж 2
%\singlespacing % Интерлиньяж 1

\usepackage{lastpage} % Узнать, сколько всего страниц в документе.

\usepackage{soul} % Модификаторы начертания

\usepackage{hyperref}
%\usepackage[usenames,dvipsnames,svgnames,table,rgb]{xcolor}
\hypersetup{				% Гиперссылки
    unicode=true,           % русские буквы в раздела PDF
    pdftitle={Заголовок},   % Заголовок
    pdfauthor={Автор},      % Автор
    pdfsubject={Тема},      % Тема
    pdfcreator={Создатель}, % Создатель
    pdfproducer={Производитель}, % Производитель
    pdfkeywords={keyword1} {key2} {key3}, % Ключевые слова
    colorlinks=true,       	% false: ссылки в рамках; true: цветные ссылки
    linkcolor=red,          % внутренние ссылки
    citecolor=black,        % на библиографию
    filecolor=magenta,      % на файлы
    urlcolor=cyan           % на URL
}

\usepackage{csquotes} % Еще инструменты для ссылок

%\usepackage[style=apa,maxcitenames=2,backend=biber,sorting=nty]{biblatex}

\usepackage{multicol} % Несколько колонок

\usepackage{tikz} % Работа с графикой
\usepackage{pgfplots}
\usepackage{pgfplotstable}
%\usepackage{coloremoji}
\usepackage{floatrow}
\usepackage{subcaption}
\newcommand*{\N}{\mathbb{N}}
\newcommand*{\R}{\mathbb{R}}
\newcommand*{\K}{\mathbb{K}}
\newcommand*{\V}{\mathcal{V}}
\newcommand*{\A}{\mathcal{A}}
\newcommand*{\ii}{\mathbf{1}}
\newcommand*{\oo}{\mathbf{0}}
\newcommand*{\ba}{\mathbf{a}}
\newcommand*{\bb}{\mathbf{b}}
\newcommand*{\Q}{\mathbb{Q}}
\graphicspath{{figures/}}
%\usepackage{breqn}

\renewcommand\thesubfigure{\asbuk{subfigure}}
%\addbibresource{master.bib}

\usepackage{import}
\usepackage{pdfpages}
\usepackage{transparent}
\usepackage{xcolor}
\usepackage{xifthen}

%\newcommand{\incfig}[1]{%
%    \def\svgwidth{\columnwidth}
%    \import{./figures/}{#1.pdf_tex}
%}


\newcommand{\incfig}[2][1]{%
    \def\svgwidth{#1\columnwidth}
    \import{./figures/}{#2.pdf_tex}
}
\usepackage{titlesec}
%\titleformat{\section}{\normalfont\Large\bfseries}{}{0pt}{}
%----------------------STANDART:
%\titleformat{\chapter}[display]
%  {\normalfont\huge\bfseries}{\chaptertitlename\ \thechapter}{20pt}{\Huge}
%\titleformat{\section}{\normalfont\Large\bfseries}{\thesection}{1em}{}
%\titleformat{\subsection}
%  {\normalfont\large\bfseries}{\thesubsection}{1em}{}
%\titleformat{\subsubsection}
%  {\normalfont\normalsize\bfseries}{\thesubsubsection}{1em}{}
%\titleformat{\paragraph}[runin]
%  {\normalfont\normalsize\bfseries}{\theparagraph}{1em}{}
%\titleformat{\subparagraph}[runin]
%  {\normalfont\normalsize\bfseries}{\thesubparagraph}{1em}{}

\pdfsuppresswarningpagegroup=1
\pgfplotsset{compat=1.16}

\usepackage{xifthen}
\makeatother
%\def\@lecture{}%
%\newcommand{\lecture}[3]{
%    \ifthenelse{\isempty{#3}}{%
%        \def\@lecture{Неделя #1}%
%    }{%
%        \def\@lecture{Неделя #1: #3}%
%    }%
%    \section*{\@lecture}
%    \marginpar{\small\textsf{\mbox{#2}}}
%}
\makeatletter

%\newcommand{\lec}{\subsection{Лекция}}
%\newcommand{\sem}{\subsection{Семинар}}
%\newcommand{\hw}{\subsection{Домашняя работа}}
%\newcommand{\prob}[1]{\textbf{#1}}
%\renewcommand{\thesubsection}{}
%\renewcommand{\thesubsubsection}{}

%\setcounter{tocdepth}{1} % only parts,chapters,sections
%\titleformat{\subsection}{\normalfont\large\bfseries}{}{0em}{}
%\titleformat{\subsubsection}{\normalfont\normalsize\bfseries}{}{0em}{}

%\newcommand{\textover}[2]{\stackrel{\mathclap{\normalfont\mbox{#2}}}{#1}}

\author{Драчов Ярослав\\
Факультет общей и прикладной физики МФТИ}
\newcommand{\veq}{\mathrel{\rotatebox{90}{$=$}}}
%\newcommand{\teto}[1]{\stackrel{\mathclap{\normalfont\tiny\mbox{#1}}}{\to}}
%\renewcommand{\thesubsection}{\arabic{subsection}}

%%\setcounter{secnumdepth}{0}

\definecolor{tabblue}{RGB}{30, 119, 180}
\definecolor{taborange}{RGB}{255, 127, 15}
\definecolor{tabgreen}{RGB}{45, 160, 43}
\definecolor{tabred}{RGB}{214, 38, 40}
\definecolor{tabpurple}{RGB}{148, 103, 189}
\definecolor{tabbrown}{RGB}{140, 86, 76}
\definecolor{tabpink}{RGB}{227, 119, 193}
\definecolor{tabgray}{RGB}{127, 127, 127}
\definecolor{tabolive}{RGB}{188, 189, 33}
\definecolor{tabcyan}{RGB}{22, 190, 207}
\pgfplotscreateplotcyclelist{colorbrewer-tab}{
{tabblue},
{taborange},
{tabgreen},
{tabred},
{tabpurple},
{tabbrown},
{tabpink},
{tabgray},
{tabolive},
{tabcyan},
}
\usepackage{csvsimple}
\usepackage{extarrows}
%\renewcommand{\labelenumii}{\asbuk{enumii})}
%\renewcommand{\labelenumiv}{\Asbuk{enumiv}}
\newcommand{\prob}[1]{\subsubsection*{#1}}
\sisetup{output-decimal-marker = {,},separate-uncertainty = true,exponent-product = \cdot}

\usepackage{braket}
\usepackage{enumerate}
\usepackage{chngcntr}
%\counterwithin*{equation}{problem}
%\usepackage{bbold}

\newtheoremstyle{hiProb}% ⟨name ⟩ 
{3pt}% ⟨Space above ⟩1 
{3pt}% ⟨Space below ⟩1
{}% ⟨Body font ⟩
{}% ⟨Indent amount ⟩2
{\bfseries}% ⟨Theorem head font⟩
{.}% ⟨Punctuation after theorem head ⟩
{.5em}% ⟨Space after theorem head ⟩3
%{\thmname{#1} \thmnote{#3}}% ⟨Theorem head spec (can be left empty, meaning ‘normal’)⟩
{\thmnote{#3}}% ⟨Theorem head spec (can be left empty, meaning ‘normal’)⟩
\theoremstyle{hiProb} % "Определение"
%\newtheorem{hiProb}{Задача}
\newtheorem{hiProb}{}
\usepackage{mmacells}
\newcommand{\textover}[2]{\stackrel{\mathclap{\normalfont\scriptsize\mbox{#2}}}{#1}}
\usepackage{units}
\usepackage[math]{cellspace}%
\setlength\cellspacetoplimit{2pt}
\setlength\cellspacebottomlimit{2pt}

%%% Работа с русским языком
\usepackage{cmap}					% поиск в PDF
\usepackage{mathtext} 				% русские буквы в формулах
\usepackage[T2A]{fontenc}			% кодировка
\usepackage[utf8]{inputenc}			% кодировка исходного текста
\usepackage[english,russian]{babel}	% локализация и переносы
\usepackage{indentfirst}
\frenchspacing

\renewcommand{\epsilon}{\ensuremath{\varepsilon}}
\newcommand{\phibackup}{\ensuremath{\phi}}
\renewcommand{\phi}{\ensuremath{\varphi}}
\renewcommand{\varphi}{\ensuremath{\phibackup}}
\renewcommand{\kappa}{\ensuremath{\varkappa}}
\renewcommand{\le}{\ensuremath{\leqslant}}
\renewcommand{\leq}{\ensuremath{\leqslant}}
\renewcommand{\ge}{\ensuremath{\geqslant}}
\renewcommand{\geq}{\ensuremath{\geqslant}}
\renewcommand{\emptyset}{\varnothing}
\renewcommand{\Im}{\operatorname{Im}}
\renewcommand{\Re}{\operatorname{Re}}

\author{Драчов Ярослав\\
Факультет общей и прикладной физики МФТИ}


\usepackage[top=1cm,bottom=1cm,left=1cm,right=1.5cm,includeheadfoot]{geometry}
\usepackage{ytableau}
\ytableausetup{boxsize=0.2em}
%\pagestyle{headings}
\usepackage{fancyhdr}
\pagestyle{fancy}
\fancyhf{}
\fancyhead[LO]{\sl Драчов Ярослав}
\fancyhead[RE]{\sl КВАНТОВАЯ ДЕФОРМАЦИЯ ИЕРАРХИИ $B$КП}
\fancyhead[LE,RO]{\thepage}
\renewcommand{\headrulewidth}{0pt}
%\usepackage[sc]{titlesec}
%\usepackage[titles]{tocloft}
%\renewcommand{\cftsecfont}{\scshape}
\title{Квантовая деформация иерархии $B$КП}
\begin{document}
\maketitle
\tableofcontents
\section{Иерархия КП}
\emph{Иерархия КП} --- бесконечный набор нелинейных дифференциальных
уравнений, первое из которых даётся как
\begin{equation}
\frac{1}{4} \frac{\partial ^2F}{\partial t_2^2} =
\frac{1}{3} \frac{\partial ^2 F}{\partial t_1 \partial t_3} 
-\frac{1}{2} \left( \frac{\partial ^2 F}{\partial t_1^2}  \right) ^2 -\frac{1}{12} \frac{\partial ^4F}{\partial t_1^4} 
\label{eq:FKP}
.\end{equation} 
Далее будут обсуждаться экспоненциированные решения
данной иерархии ($\tau$\emph{-функции}) $\tau\left( \mathbf{t} \right) =\exp \left( F(\mathbf{t}) \right) $.
В разделах \ref{subsec:lp} и \ref{subsec:bh}
опишем методы получения уравнений данной
иерархии. \subsection{Пары Лакса}
\label{subsec:lp}
Для псевдодифференциального оператора
	\begin{equation}
L=\partial + \sum_{j=1}^{\infty} f_j \partial^{-j},\qquad
\text{где}\qquad \partial = \frac{\partial }{\partial t_1},
\end{equation} 
условия совместности системы линейных уравнений
\begin{equation}
\left\{
\begin{aligned}
Lw&= kw, \\
\frac{\partial w}{\partial t_j} &=  B_j w,
\end{aligned}
\right.\qquad\text{где }
B_j=\left( L^j \right) _+.
\end{equation} 
могут быть записаны в Лаксовой форме
\begin{equation}
	\frac{\partial L}{\partial t_j} =\left[ 
	B_j,\,L\right] 
.\end{equation} 
Из второго уравнения данной иерархии получаем
\begin{equation}
\frac{\partial f_1}{\partial t_2} =
\frac{\partial ^2 f_1}{\partial t_1^2} +
2 \frac{\partial f_2}{\partial t_1} ,\qquad
\frac{\partial f_2}{\partial t_2} =\frac{\partial^2
f_2}{\partial t_1^2}+ 2 \frac{\partial f_3}{\partial t_1} 
+2 \frac{\partial f_1}{\partial t_1} f_1
,\qquad\ldots\end{equation} 
Из третьего --- 
\begin{equation}
\frac{\partial f_1}{\partial t_3} =
\frac{\partial ^3f_1}{\partial t_1^3} +3 \frac{\partial ^2f_2}{\partial t_1^2} +3 \frac{\partial f_3}{\partial t_1} 
+6 \frac{\partial f_1}{\partial t_1} f_1
,\qquad\ldots\end{equation} 
Устраняя $f_2$ и $f_3$ из данных уравнений и
пользуясь обозначениями  $u=-2f_1$, получаем
\emph{уравнение КП}
\begin{equation}
3 \frac{\partial ^2 u}{\partial t_2^2} =
\frac{\partial }{\partial t_1} \left( 
4 \frac{\partial u}{\partial t_3} +6 u \frac{\partial u}{\partial t_1} -\frac{\partial ^3u}{\partial t_1^3} \right) 
\label{eq:ukp}
.\end{equation} 
Требует пояснений введённое обозначение для $u$.
Решение первоначальной задачи на собственные значения
можно искать в виде
\begin{equation}
	w=\mathrm{e} ^{\xi\left( \mathbf{t},\,k \right) }\left( 1+ \frac{w_1}{k}+\frac{w_2}{k^2}+\ldots \right) ,\qquad \text{где}\qquad \xi\left( \mathbf{t},\,k \right) =
	\sum_{j=1}^{\infty} t_j k^j
.\end{equation} 
Тогда связь между $w_i$ и $f_i$ может быть найдена, например, из уравнений
\begin{equation}
\frac{\partial w}{\partial t_j} =B_j w
.\end{equation} 
Из уравнения на $t_1$ получаем
\begin{equation}
\frac{\partial w_1}{\partial t_1} =-f_1,\qquad\ldots
\label{eq:wf}
.\end{equation} 
Оказывается, что все функции $w_1,\,w_2,\ldots$ могут
быть выражены через одну функцию $\tau$ по формуле
\begin{equation}
	w=\frac{\displaystyle \tau\left( t_1- \frac{1}{k},\,t_2-\frac{1}{2k^2},\,t_3 -\frac{1}{3k^3},\ldots \right) }{\tau\left( t_1,\,t_2,\,t_3,\ldots \right) }
	\mathrm{e} ^{\xi\left( \mathbf{t},\,k \right) }
.\end{equation} 
Откуда, например,
\begin{equation}
w_1=-\frac{\partial \tau}{\partial t_1} \cdot \tau^{-1}
\label{eq:wtau}
.\end{equation} 
Иерархия КдФ получается из иерархии КП условием $L^2=\partial^2+u$ и бесконечный набор функций  $f_i$ выражается через $u$. Таким же свойством обладает
$\tau$-функция, между ними имеется связь
\begin{equation}
u= 2\frac{\partial ^2}{\partial t_1^2} \ln \tau
\label{eq:utau}
.\end{equation} 
Обозначение $u=-2f_1$ теперь поясняется формулами
\eqref{eq:wf}, \eqref{eq:wtau}, \eqref{eq:utau}.
Также, интегрируя два раза по $t_1$ уравнение  \eqref{eq:ukp} и принимая во внимание обозначение для $F$, получаем
в точности уравнение \eqref{eq:FKP}.
\subsection{Билинейное тождество Хироты}
\label{subsec:bh}
\begin{dfn*}
	\emph{Производные хироты} $\mathrm{D}_1^{n_1} \cdots \mathrm{D}_m^{n_m}$ 
задаются из соотношения
\begin{equation}
	\mathrm{e} ^{y_1 \mathrm{D}_1+y_2 \mathrm{D}_2+ \ldots}f\cdot g=
f\left( x_1+y_1,\,x_2+y_2,\ldots \right) 
g\left( x_1-y_1,\,x_2-y_2,\ldots \right) 
.\end{equation} 
\end{dfn*}
\begin{thm*}[билинейное тождество]
Для любых $x$ и $x'$ положим
\begin{equation}
	\xi=\xi\left( \mathbf{t},\,k \right) ,\qquad
	\xi'=\xi\left( \mathbf{t}',\,k \right) 
.\end{equation} 
Тогда справедливо следующее тождество:
\begin{equation}
0= \oint \frac{\mathrm{d} k}{2\pi \mathrm{i} }
\mathrm{e} ^{\xi-\xi'} \tau\left( t_1-\frac{1}{k},\,
t_2- \frac{1}{2k^2},\ldots\right) \tau\left( 
t_1'+\frac{1}{k},\,t_2'+ \frac{1}{2k^2},\ldots\right) 
.\end{equation} 
\end{thm*}
Уравнения КП получаются из билинейного
тождества после замены $t_j=x_j+y_j$, $t_j'=x_j-y_j$:
 \begin{multline}
\oint \frac{\mathrm{d} k}{2\pi \mathrm{i} }
\exp \left( 2 \sum_{j=1}^{\infty} k^j y_j \right) 
\tau \left( x_1+y_1-\frac{1}{k},\,x_2+y_2-\frac{1}{2k^2},\ldots \right)\times\\\times
\tau\left( x_1-y_1+\frac{1}{k},\,x_2-y_2+\frac{1}{2k^2},\ldots \right) =\\= \oint \frac{\mathrm{d} k}{2\pi \mathrm{i}}
\exp \left( 2 \sum_{j=1}^{\infty} k^j y_j \right) 
\exp \left( \sum_{j=1}^{\infty} \left( y_l-\frac{1}{l k^l} \right) \mathrm{D}_l \right) \tau\cdot \tau
.\end{multline} 
Раскладывая подынтегральную функцию в ряд по степеням
$y_j$ и вычисляя коэффициент при $k^{-1}$, получаем
уравнения КП. Например
\begin{equation}
	(\mathrm{D}_1^4 +3\mathrm{D}_2^2-4\mathrm{D}_1 \mathrm{D}_3)\tau\cdot \tau=0,\qquad
	(\mathrm{D}_1^3 \mathrm{D}_2+2\mathrm{D}_2 \mathrm{D}_3-3\mathrm{D}_1 \mathrm{D}_4)\tau\cdot \tau=0
.\end{equation} 
Первое из них --- буквально \eqref{eq:FKP} с учётом
определения $F$.
\subsection{Полиномы Шура, соотношения Плюккера, гипергеометрические $\tau$-функции}
\label{subsec:schur}
Решения данной иерархии
могут быть разложены по базису полиномов Шура
\begin{equation}
	\tau\left( \mathbf{t} \right) =\sum_{\lambda}^{} C_\lambda s_\lambda\left( \mathbf{t} \right) 
\label{eq:2}
.\end{equation} 
Можно показать, что $\tau(\mathbf{t})$ --- решение
иерархии КП тогда и только тогда, когда коэффициенты
$C_\lambda$ удовлетворяют соотношениям Плюкера, первое
из которых
\begin{equation}
C_{\left[ 2,\,2 \right] }C_{\left[ \emptyset \right] }
-C_{\left[ 2,\,1 \right] }C_{\left[ 1 \right] }
+C_{\left[ 2 \right] }C_{\left[ 1,\,1 \right] }=0
.\end{equation} 
Имеется важное подмножество решений иерархии КП --- 
\emph{гипергеометрические} $\tau$\emph{-функции}.
Они определяются как
\begin{equation}
	\tau\left( \mathbf{t} \right) 
	=\sum_{\lambda}^{} r_\lambda
	s_\lambda (\beta) s_\lambda \left( \mathbf{t} \right) 
,\end{equation} 
где $s_\lambda(\beta)$ --- полином Шура от переменных
$\beta_k$,
\begin{equation}
	r_\lambda=\prod_{w \in \lambda}^{} r\left( c\left( w \right)  \right)  
,\end{equation} 
\begin{equation}
	c(w)=j-i,\qquad 1\le  i\le l(\lambda),\qquad
	1\le j\le \lambda_i
.\end{equation} 
\subsection{$\tau$-функция чисел Гурвица}
Производящая функция простых чисел Гурвица
\begin{equation}
	\tau_H(\mathbf{t})=
	\sum_{m=0}^{\infty} \sum_{\mu}^{} h_{m;\mu}^{\circ}
	t_{\mu_1} t_{\mu_2} \cdots t_{\mu_{l(\mu)}}
	\frac{u^m}{m!}
,\end{equation} 
где
\begin{equation}
h^{\circ}_{m;\mu}= \frac{1}{|\mu|!}
\left| \left\{ \left( \eta_1,\ldots,\,\eta_m \right) ,\ 
\eta_i \in C_2 \left( S_{|\mu|} \right) \colon
\eta_m \circ \cdots \circ \eta_1 \in C_\mu\left( S_{|\mu|} \right) \right\}  \right| 
,\end{equation} 
$S_{|\mu|}$ --- симметрическая группа перестановок $\mu$ 
элементов, $C_2\left( S_{|\mu|} \right) $ --- множество
всех транспозиций в $S_{|\mu|}$ и $C_\mu \left( S_{|\mu|} \right) $ --- множество всех перестановок циклического типа $\mu$.

Можно показать (TODO), что
\begin{equation}
	\tau_H(\mathbf{t})=\sum_{\lambda}^{} \mathrm{e}^{u c(\lambda)}s_\lambda \left( \beta_k=\delta_{k,\,1} \right) 
	s_\lambda(\mathbf{t})
\label{eq:4}
,\end{equation} 
где $c(\lambda)= \sum_{w \in \lambda}^{} c(w)$. 
То есть данная производящая функция --- гипергеометрическая
$\tau$-функция с параметрами
\begin{equation}
	r(n)=\mathrm{e} ^{un},\qquad
	\beta_1=1,\qquad \beta_k=0,\ k\ge 2
.\end{equation} 
\subsection{Фермионный формализм и бозонно-фермионное соответствие}
Будем рассматривать бесконечномерную алгебру Клиффорда
с генераторами $\left\{ \psi_n,\,\psi^*_m \mid n,\,m \in \mathbb{Z} \right\} $, обладающими коммутационными соотношениями
\begin{equation}
\left\{ \psi_n,\,\psi_m \right\} =0,\qquad
\left\{ \psi_n^*,\,\psi^*_{m} \right\}=0,\qquad
\left\{ \psi_n,\,\psi_{m}^* \right\}=\delta_{n,\,m}
.\end{equation} 

Пространство Фока для фермионов определим дествием
алгебры Клиффорда на вакуумный вектор $\ket{0}$ <<моря Дирака>>. Действие генераторов алгебры на вакуумный вектор
даётся как
\begin{equation}
\psi_k\ket{0}=0,\quad k< 0;\qquad\psi^* \ket{0}=0,\quad
k\ge  0
.\end{equation} 
То есть операторы $\psi_k$ при $k>0$ и $\psi^*_k$ при
$k\le 0$ --- это операторы рождения, а $\psi_k$ при $k\le 0$ 
и $\psi_k^*$ при  $k> 0$ --- операторы уничтожения.

Далее определим $m$-е вакуумы
\begin{equation}
\psi_k\ket{m}=0,\quad k< m;\qquad\psi^* \ket{m}=0,\quad
k\ge  m
,\end{equation}
и состояния
\[
	\ket{m,\,\lambda}= \psi_{\lambda_1+m-1}\psi_{\lambda_2+m-2}\cdots\psi_{\lambda_{l(\lambda)}+m-l(\lambda)}\psi^*_{m-l(\lambda)}\cdots\psi^*_{m-2}\psi^*_{m-1}\ket{m}
.\] 

Введём также производящие функции для фермионов
\begin{equation}
\psi(z)=\sum _{k\in \mathbb{Z}}^{} \psi_k z^k,\qquad
\psi^* (z)=\sum_{k \in \mathbb{Z}}^{} \psi^*_k z^{-k-1}
.\end{equation} 

Будем обозначать нормальное упорядочение фермионных
операторов как $\normord{\left( \ldots \right) }$.
Результатом нормального упорядочения будет перемещение
всех операторов уничтожения направо, а операторов
рождения --- налево, где каждая транспозиция двух
фермионов будет производиться в согласии с антикоммутационным
соотношением и, следовательно, давать множитель $(-1)$. 

Алгебра Ли $gl_\infty$ определяется как пространство
\begin{equation}
	gl_\infty= \left\{ \left( a_{ij} \right) _{i,j \in \mathbb{Z}}\mid\text{все }a_{ij}\text{, кроме конечного
	числа, равны 0} \right\} 
.\end{equation} 
Базисом данной алгебры можно выбрать матрицы  $E_{ij}$,
у которых единственный, отличный от нуля, элемент равен 1 и находится
на пересечении $i$-й строки и $j$-го столбца.
 \begin{equation}
	(E_{ij})_{kl}=\delta_{ik}\delta_{jl}
.\end{equation} 
\begin{multline}
\left[ E_{ij},\,E_{kl} \right] =
E_{ij}E_{kl}-E_{kl}E_{ij}=
\delta_{in}\delta_{jm}\delta_{km}\delta_{lp}-
\delta_{kn}\delta_{lm}\delta_{im}\delta_{jp}=\\=
\delta_{jk}E_{il}-\delta_{il}E_{kj}
.\end{multline} 


Данную алгебру можно рассматривать как алгебру Ли группы
$\mathrm{GL}_\infty$, определённой следующим образом
\begin{equation}
	\mathrm{GL}_\infty= \left\{ 
	A=\left( a_{ij} \right) _{i,j \in \mathbb{Z}} \mid \substack{A\text{ обратима и все числа } a_{ij} -\delta_{ij},\\ \text{кроме конечного числа, равны 0}}\right\} 
.\end{equation} 
Представление $R$ группы $\mathrm{GL}_\infty$ и представление
$\rho$ алгебры Ли $gl_\infty$ в пространстве $\mathcal{F}$ 
задаётся формулами
\begin{align}
	R\left( A \right) \left( \psi_{i_1} \wedge 
	\psi_{i_2} \wedge \ldots\right) &=
	A \psi_{i_1} \wedge A \psi_{i_2}\wedge 
	\ldots,
	\label{eq:GLr}\\
	\rho(a)\left( \psi_{i_1} \wedge \psi_{i_2}\wedge 
\ldots\right) &=a \psi_{i_1} \wedge \psi_{i_2}
	\wedge \ldots+ \psi_{i_1} \wedge 
	a \psi_{i_2}\wedge \ldots+\ldots
.\end{align} 
Последние две формулы связаны соотношением
\begin{equation}
	\mathrm{e} ^{\rho(a)}=R\left( \mathrm{e} ^{a} \right) ,\qquad a \in gl_{\infty}
.\end{equation} 
Представлением данной алгебры
на введённом пространстве Фока будет
\begin{equation}
	\rho(E_{ij})=\psi_i \psi_j^{*}
\end{equation} 
т.\:к.
\begin{multline}
	\left[ \psi_i \psi^*_j,\,\psi_k \psi^*_l \right] 
	=\psi_i \psi_j^* \psi_k \psi^*_l -\psi_k \psi^*_l
	\psi_i \psi^*_j=\\=\psi_i \left( \delta_{jk} -\psi_k \psi_j^* \right) \psi^*_l-\psi_k \psi_l^* \psi_i \psi^*_j=\\=\delta_{jk}\psi_i \psi^*_l-\psi_k \psi_i  \psi_l^*
	\psi_j^*-\psi_k \psi_l^* \psi_i \psi_j^*=\\=
	\delta_{jk} \psi_i \psi_{l}^*-
	\delta_{il}\psi_k \psi_j^*
.\end{multline}
Пользуясь формулой \eqref{eq:GLr} и стандартным исчислнием внешних степней найдём оператор представления $R$ элемента $A \in \mathrm{GL}_\infty$ в пространстве $\mathcal{F}$
\begin{multline}
	R(A)\left( \psi_{i_m}\wedge \psi_{i_{m-1}}\wedge \ldots	 \right) =\sum_{j_m,j_{m-1},\ldots \in \mathbb{Z}}^{}
	A_{j_m,i_{m}}\psi_{j_m}\wedge A_{j_{m-1},i_{m-1}} \psi_{j_{m-1}}\wedge  \ldots	=\\= \sum_{j_m >j_{m-1}> \ldots}^{} \left( 
		\det A_{j_m,j_{m-1},\ldots}^{i_{m},\,i_{m-1},\ldots}\right) \psi_{j_m} \wedge \psi_{j_{m-1}}\wedge \ldots
	\label{}
.\end{multline}
где $A^{i_m,i_{m-1},\ldots}_{j_m,j_{m-1},\ldots}$ означает
матрицу, состоящую из элементов, стоящих
на пересечении строк $j_m,\,j_{m-1},\ldots$ и столбцов
$i_m,\,i_{m-1},\ldots$ матрицы $A$.

В альтернативных обозначениях
\begin{equation}
	R(A) \ket{m,\,\lambda} =\sum_{\mu \in \mathrm{Par}}^{} \left( 
	\det A_{\mu_1+m,\mu_2+m-1,\ldots}^{\lambda_1+m,\lambda_2+m-1,\ldots}\right) \ket{m,\,\mu}.\end{equation} 

Определим также бóльшую алгебру Ли $\bar{\mathfrak{a}}_\infty$.
Элементами данной алгебры будут бесконечномерные матрицы,
у которых конечное количество диагоналей отличны от нулевых:
\begin{equation}
	\bar{\mathfrak{a}}_\infty=\left\{ 
	\left( a_{ij} \right) \mid i,\,j \in \mathbb{Z},\ a_{ij}=0 \text{ при } \left| i-j \right| \gg 0\right\} 
.\end{equation} 
$\bar{\mathfrak{a}}_\infty$ является алгеброй Ли
с матричным коммутатором в качестве операции, содержащей
алгебру Ли $gl_{\infty}$ как подалгебру.




Далее рассмотрим центральное расширение алгебры Ли
$\bar{\mathfrak{a}}_\infty$, а именно алгебру Ли $\mathfrak{a}_\infty$, определённую как
\begin{equation}
	\mathfrak{a}_\infty= \bar{\mathfrak{a}}_\infty
	\oplus  \mathbb{C}c
\end{equation} 
с центром $\mathbb{C}c$ и скобкой
\begin{equation}
	\left[ a,\,b \right] =ab-ba+ \alpha\left( a,\,b \right) c
.\end{equation} 
Функцию $\alpha(a,\,b)$ называют \emph{два-коциклом}, для
корректности определения скобки Ли на неё накладываются
дополнительные ограничения:
\begin{itemize}
	\item Антисимметричность скобки влечёт антисимметричность $\alpha(a,\,b)$.
	\item Линейность скобки по обоим аргументам влечёт аналогичную линейность $\alpha(a,\,b)$.
	\item Из тождества Якоби скобки
\begin{equation}
\left[ \left[ x,\,y \right],\,z \right] +\left[ \left[ y,\,z \right] ,\,x \right] +\left[ \left[ z,\,x \right] ,\,y \right] =0
\end{equation} 
следует, что
\begin{multline}
	\left[ xy -yx+\alpha\left( x,\,y \right) c,\,z \right] +\left[ yz-zy+\alpha\left( y,\,z \right) c,\,x \right] +\\+\left[ zx-xz+\alpha\left( z,\,x \right) c,\,y \right] =0
\end{multline} 
и, как итог,
\begin{equation}
	\alpha\left( xy-yx,\,z \right) +\alpha\left( yz-zy,\,x \right) +\alpha\left( zx-xz,\,y \right) =0
.\end{equation} 
\end{itemize}
Два-коцикл $\alpha(a,\,b)$  определим на матрицах $E_{ij}$ как
\begin{equation}
	\begin{aligned}
		&\alpha\left( E_{ij},\,E_{ji} \right) =-\alpha
	\left( E_{ji},\,E_{ij} \right) =1\quad
	\text{при } i<  0,\ j\ge 0,\\
	&\alpha\left( E_{ij},\,E_{kl} \right) =0
		\quad \text{во всех остальных случаях.}
	\end{aligned}
	\label{alph}
\end{equation}
Антисимметричность, линейность по каждому
аргументу данного два-коцикла очевидны.
%Коммутационными соотношениями данной алгебры будут
%\begin{align}
%	\left[ E_{ij},\,E_{kl} \right] &=0 &&
%	\text{при } j \neq k ,\ l \neq i,\\
%	\left[ E_{ij},\,E_{jl} \right] &=
%	E_{il} && \text{при } l\neq i,\\
%	\left[ E_{ij},\,E_{ki} \right] &=-E_{kj}
%				       && \text{при } j
%				       \neq k,
%	\left[ E_{ij},\,E_{ji} \right] =
%.\end{align}

Покажем, что представлением данной
алгебры на пространстве Фока будет
\begin{equation}
	\hat{\rho}(E_{ij})= \normord{\psi_i \psi_j^*}
.\end{equation} 
%\[
%	\braket{0 | \left[ \normord{\psi_i\psi_j^*},\,\normord{\psi_k \psi^*_l} \right] |0}=
%.\] 
%\[
%	\braket{\mathcal{S} | \normord{\psi_i \psi_j^*} |\ldots,\,n_{-1},\,n_0,\,n_1,\ldots}
%.\] 
%\begin{multline*}
%	\left[ \normord{\psi_i\psi_j^*},\,\normord{\psi_k \psi^*_l} \right]=
%\begin{cases}
%	\left[ \psi_i\psi_j^*,\,\psi_k \psi^*_l \right],&
% j\ge 0,\, l\ge 0,\\
%	\left[ \psi_i\psi_j^*,\,\psi_k \psi^*_l \right],&
% j\ge 0,\, l< 0,\,k\ge 0,\\
%	\left[ \psi_i\psi_j^*,\,\psi^*_l \psi_k \right],&
% j\ge 0,\, l< 0,\,k< 0,\\
%		\left[ \psi_i\psi_j^*,\,\psi_k \psi^*_l \right],&
%j<0,\, l\ge 0,\,i\ge 0,\\
%		\left[ \psi_j^*\psi_i,\,\psi_k \psi^*_l \right],&
%j<0,\, l\ge 0,\,i< 0,\\
%		\left[ \psi_i\psi_j^*,\,\psi_k \psi^*_l \right],&
%j<0,\, l< 0,\,i\ge  0,\,k\ge 0,\\
%		\left[ \psi_i\psi_j^*,\,\psi^*_l\psi_k  \right],&
%j<0,\, l< 0,\,i\ge  0,\,k< 0,\\
%		\left[ \psi_j^*\psi_i,\,\psi_k \psi^*_l \right],&
%j<0,\, l< 0,\,i<  0,\,k\ge  0,\\
%		\left[ \psi_j^*\psi_i,\,\psi^*_l\psi_k  \right],&
%j<0,\, l< 0,\,i<  0,\,k<0,  \\
%\end{cases}
%\end{multline*}
По определению свёртки
\begin{equation}
	\normord{\psi_i \psi_j^*}=\psi_i \psi_j^*-
	\wick{\c\psi_i \c\psi_j^*}
	\label{}
.\end{equation}
Для неё выполняется
\begin{equation}
	\wick{\c\psi_i \c\psi_j^*}=\wick{\c\psi_i \c\psi_j^*}\braket{0  | 0}=\wick{\braket{0 | \c\psi_i \c\psi_j^* | 0}}
	=\braket{0 | \psi_i \psi_j^*-\normord{\psi_i \psi_j^*} | 0}=\braket{0 | \psi_i \psi_j^*| 0}
.\end{equation} 
Откуда
\begin{align}
	\wick{\c\psi_i \c\psi_i^*}&=1 && \text{при }
	i< 0,
	\label{eq:contr1}\\
	\wick{\c \psi_i \c \psi_j^*}&=0 && \text{во всех остальных случаях}
\label{eq:contr2}
.\end{align}
Тогда
\begin{multline}
	\left[ \normord{\psi_i \psi_j^*},\,\normord{
	\psi_k \psi_l^*} \right] =
	\wick{\left[ \psi_i\psi_j^*-\c\psi_i \c\psi_j^* ,\,\psi_k \psi_l^*- \c1 \psi_k \c1 \psi_l^* \right]}=\\=
	\left[ \psi_i \psi_j^*,\,\psi_{k}\psi_l^* \right]
	=\delta_{jk} \psi_i \psi_l^* -\delta_{il}\psi_k \psi_j^*=\\=\delta_{jk} \wick{\left( \normord{\psi_i \psi_l^*}+
	\c\psi_i \c\psi_l^*\right) }-\delta_{il} \wick{\left( \normord{\psi_k \psi_j^*}+
	\c\psi_k \c\psi_j^*\right) }=\\=
	\delta_{jk} \normord{\psi_i\psi_l^*}
	-\delta_{il} \normord{\psi_k \psi_j^*}+
	\delta_{jk} \wick{\c \psi_i \c\psi_l^*}-
	\delta_{il} \wick{\c \psi_k \c \psi_j^*}=\\=
	\delta_{jk} \normord{\psi_i\psi_l^*}
	-\delta_{il} \normord{\psi_k \psi_j^*}+
	\alpha\left( E_{ij},\,E_{kl} \right) 
	\label{}
.\end{multline}
В последнем переходе мы воспользовались соотношением
\begin{equation}
\alpha\left( E_{ij},\,E_{kl} \right)= \delta_{jk} \wick{\c \psi_i \c\psi_l^*}-
	\delta_{il} \wick{\c \psi_k \c \psi_j^*}
,\end{equation} 
верность которого следует из \eqref{alph}, \eqref{eq:contr1} и \eqref{eq:contr2}. Далее проверим, что скобка с определённым выше два-коциклом удовлетворяет тождеству Якоби
\begin{multline}
	\alpha\left( E_{ij}E_{kl}-E_{kl}E_{ij},\,E_{mn}  \right) +
	\alpha\left( E_{kl}E_{mn}-E_{mn}E_{kl},\,E_{ij} \right) +\\+
	\alpha\left( E_{mn}E_{ij}-E_{ij}E_{mn},\,E_{kl} \right) =0
,\end{multline} 
\begin{multline}
	\alpha\left( \delta_{jk}E_{il}-\delta_{il}E_{kj},\,E_{mn}  \right) +	\alpha\left( \delta_{lm}E_{kn}-\delta_{kn}E_{ml},\,E_{ij} \right) +\\+
	\alpha\left( \delta_{ni}E_{mj}-\delta_{mj}E_{in},\,E_{kl} \right) =0
,\end{multline}
\begin{multline}
	\delta_{jk}\alpha \left(E_{il},\,E_{mn}\right)-\delta_{il}\alpha\left(E_{kj},\,E_{mn}\right)  +\\+
	\delta_{lm}\alpha \left(E_{kn},\,E_{ij}\right)-\delta_{kn}\alpha\left(E_{ml},\,E_{ij}\right) +\\+
	\delta_{ni}\alpha \left(E_{mj},\,E_{kl}\right)-\delta_{mj}\alpha\left(E_{in},\,E_{kl}\right) =0
,\end{multline}
\begin{multline}
	\delta_{jk}\wick{\left( \delta_{lm} \c \psi_i \c\psi_n^*-
	\delta_{in} \c1 \psi_m \c1 \psi_l^* \right)} -\delta_{il}\wick{\left( \delta_{jm} \c \psi_k \c\psi_n^*-
	\delta_{kn} \c1 \psi_m \c1 \psi_j^* \right)} +\\+	\delta_{lm}\wick{\left( \delta_{in} \c \psi_k \c\psi_j^*-
	\delta_{kj} \c1 \psi_i \c1 \psi_n^* \right)}-\delta_{kn}\wick{\left( \delta_{il} \c \psi_m \c\psi_j^*-
	\delta_{mj} \c1 \psi_i \c1 \psi_l^* \right)}+\\+	\delta_{ni}\wick{\left( \delta_{jk} \c \psi_m \c\psi_l^*-
	\delta_{lm} \c1 \psi_k \c1 \psi_j^* \right)}-\delta_{mj}\wick{\left( \delta_{kn} \c \psi_i \c\psi_l^*-
	\delta_{il} \c1 \psi_k \c1 \psi_n^* \right)}=0
,\end{multline}
\begin{equation}
0=0
.\end{equation} 



%\begin{multline*}
%	\left[ \normord{\psi_i\psi_j^*},\,\normord{\psi_k \psi^*_l} \right]=\normord{\psi_i \psi_j^*}\normord{\psi_k \psi_l^*}-\normord{\psi_k \psi_l^*}\normord{\psi_i \psi_j^*}=\\=
%\begin{cases}
%	\psi_i \psi^*_j \psi_k \psi_l^*-
%	\psi_k \psi_l^* \psi_i \psi_j^*,&
% j\ge 0,\, l\ge 0,\\
%	\psi_i \psi^*_j \psi_k \psi_l^*-
%	\psi_k \psi_l^* \psi_i \psi_j^*,&
% j\ge 0,\, l< 0,\,k\ge 0,\\
%	\psi_i \psi^*_j \psi_k \psi_l^*-
%	\psi_k \psi_l^* \psi_i \psi_j^*,&
% i,\,l\ge 0,\ j< 0,\\
%\end{cases}
%\end{multline*}
Оказывается, можно построить изоморфизм между описанным
фермионным пространством Фока $\mathcal{F}$ и бозонным пространством $\mathcal{B}$ полиномов
от $x_1,\,x_3,\ldots,\,z$ и $z^{-1}$:
\begin{equation}
	\Phi: \mathcal{F}\to \mathcal{B}= \mathbb{C}\left[ x_1,\,x_2,\ldots;\,z,\,z^{-1} \right] 
.\end{equation} 
Отображение $\Phi$ буквально задаёт бозонно-фермионное соответствие. Как следствие, можно построить бозонные представления
$\rho^B=\Phi \rho \Phi^{-1}$ и $\hat{\rho}^B= \Phi \hat{\rho} \Phi^{-1}$ на данном пространстве.
\ytableausetup{boxsize=0.5em}

Определим
\begin{equation}
	 H(\mathbf{t})= \sum_{k=1}^{\infty} t_k H_k,\qquad
	H_n= \sum_{k \in \mathbb{Z}}^{} \normord{\psi_k \psi_{k+n}^{*}}
.\end{equation} 
Тогда
\begin{multline}
	H_n= \sum_{k \in \mathbb{Z}}^{} \normord{\psi_k \psi_{k+n}^*}= \sum_{k \in \mathbb{Z}}^{} \wick{\left( 
	\psi_{k}\psi_{k+n}^*-\c \psi_k \c \psi_{k+n}^*\right) }=\\=\begin{cases}
		\displaystyle \sum_{k \in \mathbb{Z}}^{} \psi_k \psi_{k+n}^*, & n\neq 0,\\
		\displaystyle \sum_{i>0}^{} \psi_k \psi_k^*-
		\sum_{i\le 0}^{} \psi^*_k \psi_k,&
		n=0.
	\end{cases}
	\label{}
\end{multline}
\begin{equation}
	\mathrm{e} ^{H(\mathbf{t})}=
	\exp  \left( \sum_{k\ge 1}^{} t_k H_k  \right)=
	\exp \left( \sum_{k\ge 1}^{} t_k H_1^k  \right) =
	\sum_{k\ge 0}^{} s_k(\mathbf{t}) H_k
.\end{equation} 
\begin{equation}
	\left(	\mathrm{e} ^{H(\mathbf{t})}\right)_{mn}=
	s_{n-m}(\mathbf{t})
.\end{equation} 
Получаем явный вид бозонно-фермионного соответствия (TODO привести мотивацию)
\begin{multline}
	\Phi\left( \ket{m,\,\lambda} \right) =z^m
	\braket{m | \mathrm{e} ^{H(\mathbf{t})} | m,\,\lambda}=\\=z^m\braket{m | \sum_{\mu \in \mathrm{Par}}^{} \det \left[\left(\mathrm{e} ^{H(\mathbf{t})}\right)_{\mu_1+m,\mu_2+m-1,\ldots}^{\lambda_1+m,\lambda_2+m-1,\ldots}\right] | m,\,\mu}=\\=z^m
	\det\left[ \left( \mathrm{e} ^{H(\mathbf{t})} \right)_{m,m-1,\ldots}^{\lambda_1+m,\lambda_2+m-1,\ldots}
	\right] =z^m\det \left[ \left( \mathrm{e} ^{H(\mathbf{t})} \right) _{m,\,m-1,\ldots}^{\lambda_1+m,\,\lambda_{2}+m-1,\ldots} \right]=\\=z^m\det_{i,\,j}  
	s_{\lambda_i-i+j}(\mathbf{t})
 = z^m s_{\lambda}(\mathbf{t})
	\label{}
.\end{multline}
Рассмотрим выражение
\begin{multline}
\braket{0 | \mathrm{e}^{H(\mathbf{t})}G | 0}=
\braket{0 | \mathrm{e} ^{H(\mathbf{t})} \sum_{\lambda \in \mathrm{Par}}^{} \det \left(G_{\lambda_1,\lambda_2-1,\ldots}^{0,-1,\ldots}\right) |0,\,\mu }=\\=
\sum_{\lambda \in \mathrm{Par}}^{} \det \left(G_{\lambda_1,\lambda_2-1,\ldots}^{0,-1,\ldots}\right)\braket{0 | \mathrm{e} ^{H(\mathbf{t})}  |0,\,\lambda }=\sum_{\lambda \in \mathrm{Par}}^{} \det \left(G_{\lambda_1,\lambda_2-1,\ldots}^{0,-1,\ldots}\right)s_\lambda(\mathbf{t})
.\end{multline} 
Сравнивая с $\tau$-функцией в бозонном представлении \eqref{eq:2}
можно получить, что
\begin{equation}
	\tau(\mathbf{t})=\braket{0 | \mathrm{e}^{H(\mathbf{t})}G | 0}\quad \text{при} \quad C_\lambda= \det
	\left( G^{0,-1,\ldots}_{\lambda_1,\lambda_2-1,\ldots} \right) 
.\end{equation} 
Для гипергеометрических $\tau$-функций выполняется
\begin{equation}
	G=\mathrm{e} ^{A(\beta)},\qquad
	A(\beta)=\sum_{k=1}^{\infty} \beta_k A_k
.\end{equation} 
\begin{equation}
A_k= \oint \frac{dz}{2\pi \mathrm{i} } \normord{
\left[ \left( \frac{1}{z} r(D) \right) ^k \psi(z) \right] \cdot \psi^*(z)}
,\end{equation} 
где $D=z \frac{\mathrm{d} }{\mathrm{d} z}$.
\[
	D^n z^k=D^{n-1}z \frac{\partial }{\partial z} z^k=kD^{n-1}z^k=k^n z^k
.\] 
\[
	\left( \hbar D- j \right) ^n z^k=
	\left( \hbar D-j \right) ^{n-1}\left( \hbar k -j\right) z^k=\left( \hbar k-j \right) ^n z^k 
.\] 

\begin{multline*}
	 \frac{1}{z}r(D) \psi(z)  =\frac{1}{z} \sum_{n=0}^{\infty} \frac{r^{(n)}(0)}{n!}D^n \sum_{k \in \mathbb{Z}}^{} \psi_k z^k  =
	 \sum_{n=0}^{\infty} \sum_{k \in \mathbb{Z}}^{} 
	 \frac{r^{(n)}(0)}{n!}\psi_k k^n z^{k-1}=\\=
	 \sum_{k \in \mathbb{Z}}^{} r(k)\psi_k z^{k-1}
.\end{multline*} 
\begin{multline*}
	\frac{1}{z}r(\hbar D) \psi(z)  =\frac{1}{z} \sum_{n=0}^{\infty} \frac{r^{(n)}(0)}{n!}(\hbar D)^n \sum_{k \in \mathbb{Z}}^{} \psi_k z^k  =\\=
	 \sum_{n=0}^{\infty} \sum_{k \in \mathbb{Z}}^{} 
	 \frac{r^{(n)}(0)}{n!}\psi_k (\hbar k)^n z^{k-1}=
	 \sum_{k \in \mathbb{Z}}^{} r(\hbar k)\psi_k z^{k-1}
.\end{multline*}
\begin{multline}
	\frac{1}{z} r(\hbar D) \sum_{k \in \mathbb{Z}}^{} r(\hbar k) \psi_k z^{k-1}=\frac{1}{z}  \sum_{n=0}^{\infty} \frac{r^{(n)}(0)}{n!} \left( \hbar D \right) ^n
	\sum_{k \in \mathbb{Z}}^{} r\left( \hbar k \right) 
	\psi_k z^{k-1}=\\=
	\frac{1}{z}\sum_{n=0}^{\infty} \sum_{k \in \mathbb{Z}}^{} \frac{r^{(n)}(0)}{n!}\hbar ^n
	r\left( \hbar k \right) \psi_k \left( k-1 \right) ^nz^{k-1}=\\=\sum_{k \in \mathbb{Z}}^{} r\left( \hbar (k-1) \right) r(\hbar k)\psi_k z^{k-2}
	\label{}
.\end{multline}
\begin{multline*}
	\frac{1}{z}r(D)\frac{1}{z}r(D)=
	\frac{1}{z} \sum_{n=0}^{\infty} \frac{r^{(n)}(0)}{n!}
	D^n \frac{1}{z}\sum_{k=0}^{\infty} \frac{r^{(k)}(0)}{k!} D^k=\\=\frac{1}{z} \sum_{n=0}^{\infty} \frac{r^{(n)}(0)}{n!}
	\sum_{i=0}^{n} \binom{n}{i}\left(D^{n-i} \frac{1}{z}\right)\sum_{k=0}^{\infty} \frac{r^{(k)}(0)}{k!} D^{k+i}=\\=\frac{1}{z^2} \sum_{n=0}^{\infty} \frac{r^{(n)}(0)}{n!}
	\sum_{i=0}^{n} \binom{n}{i}(-1)^{n-i} \sum_{k=0}^{\infty} \frac{r^{(k)}(0)}{k!} D^{k+i}=
\\=\frac{1}{z^2} \sum_{n=0}^{\infty} \frac{r^{(n)}(0)}{n!}
 (D-1)^n \sum_{k=0}^{\infty} \frac{r^{(k)}(0)}{k!} D^{k}=
 \frac{1}{z^2}r(D-1)r(D)
.\end{multline*}
\begin{multline*}
	\frac{1}{z}r(\hbar D)\frac{1}{z}r(\hbar D)=
	\frac{1}{z} \sum_{n=0}^{\infty} \frac{r^{(n)}(0)}{n!}
	(\hbar D)^n \frac{1}{z}\sum_{k=0}^{\infty} \frac{r^{(k)}(0)}{k!} (\hbar D)^k=\\=\frac{1}{z} \sum_{n=0}^{\infty}\frac{r^{(n)}(0)}{n!} \hbar ^n
	\sum_{i=0}^{n} \binom{n}{i}\left(D^{n-i} \frac{1}{z}\right)\sum_{k=0}^{\infty} \frac{r^{(k)}(0)}{k!} \hbar ^kD^{k+i}=\\=\frac{1}{z^2} \sum_{n=0}^{\infty} \frac{r^{(n)}(0)}{n!}\hbar ^n
	\sum_{i=0}^{n} \binom{n}{i}(-1)^{n-i} \sum_{k=0}^{\infty} \frac{r^{(k)}(0)}{k!} \hbar ^kD^{k+i}=
\\=\frac{1}{z^2} \sum_{n=0}^{\infty} \frac{r^{(n)}(0)}{n!}
\left(\hbar (D-1)\right)^n \sum_{k=0}^{\infty} \frac{r^{(k)}(0)}{k!} (\hbar D)^{k}=\\=
\frac{1}{z^2}r\left(\hbar (D-1)\right)r(\hbar D)
.\end{multline*}
\begin{multline*}
	\left[ \left( \frac{1}{z}r(D)  \right) ^k
	\psi(z)\right] \cdot \psi^*(z)=\\=
	\sum_{n \in \mathbb{Z}}^{} r(n)r(n-1)\cdots
	r\left( n-k+1 \right) \psi_n z^{n-k}\sum_{j \in \mathbb{Z}}^{} \psi_j^* z^{-j-1}
.\end{multline*}
\begin{equation}
	A_k=\sum_{n\in \mathbb{Z}}^{} \prod_{i=n-k+1}^{n} 
	r(i)\normord{\psi_n \psi_{n-k}}	\label{}
.\end{equation}
\section{Квантовая деформация иерархии КП}


В $\hbar $\emph{-деформированной иерархии КП} связь между $F$ и $\tau$ будет следующей
\begin{equation}
	F^\hbar \left( \mathbf{t} \right) =
	\hbar ^2 \ln \left( \tau^\hbar \left( \mathbf{t} \right)  \right) 
.\end{equation} 
Уравнения иерархии $\hbar$-КП получаются  из
иерархии КП заменой  $\mathbf{t}\to \mathbf{t} /\hbar $.
Первое уравнение деформированной иерархии
\begin{equation}
	\frac{1}{4} \frac{\partial ^2 F}{\partial t_2^2} =\frac{1}{3} \frac{\partial ^2 F}{\partial t_1 \partial t_3} -\frac{1}{2} \left( \frac{\partial^2 F}{\partial t_1^2} \right) ^2-\frac{\hbar ^2}{12} \frac{\partial^4 F}{
	\partial t_1^4}
.\end{equation} 
$\tau$-функции  иерархии $\hbar $-КП
\begin{equation}
	\tau^\hbar \left( \mathbf{t} \right) =
	\sum_{\lambda}^{} C_\lambda^\hbar 
	s_\lambda \left( \frac{\mathbf{t}}{\hbar } \right) 
	\label{eq:3}
,\end{equation} 
где $C^\hbar _{\lambda}$ удовлетворяют соотношениям Плюкера.
В разделах далее будем добавлять $\hbar $
в $\tau$-функции так, чтобы полученные $\tau^\hbar $-функции
удовлетворяли $\hbar$-деформированной иерархии, а также
имели  <<хорошее>> разложение по степеням $\hbar $.
\subsection{Фермионный формализм}
Рецепт деформации $\tau$-функций в фермионном формализме
\begin{equation}
	\tau^\hbar (\mathbf{t})= \braket{0 | \mathrm{e} ^{H\left( \mathbf{t} /\hbar  \right) } \exp \left( \frac{1}{\hbar }A^\hbar  \right) | 0}
,\end{equation} 
где
\begin{equation}
A^\hbar = \oint \frac{dz}{2\pi \mathrm{i} }
\normord{\left[ \hat{A}\left( z,\,\hbar \frac{\mathrm{d} }{\mathrm{d} z}\right)\psi(z)   \right] \cdot \psi^*(z)}
\end{equation} 
и
\begin{equation}
	\hat{A}\left( z,\,\hbar  \frac{\mathrm{d} }{\mathrm{d} z} \right) = \sum_{i \in \mathbb{Z},j\ge 0}^{} a_{ij}
	z^i \left( \hbar  \partial_z \right) ^j
.\end{equation} 
\subsection{$\tau$-функция чисел Гурвица}
Формула Римана-Гурвица
\begin{equation}
	2g-2=m-\left| \mu \right| -l(\mu)
,\end{equation} 
позволяет нам разделить вклады различных родов $g$ 
в производящую функцию. Каждая точка простого ветвления
даёт вклад $+1$ к степени $\hbar $, каждый цикл
длины $\mu_i$ даёт вклад $-\mu_i -1$ к степени $\hbar $.
Получаем замену переменных
\begin{equation}
t_{\mu_i}\to  \hbar ^{-\mu_i -1} t_{\mu_i},\qquad
u\to \hbar u
\label{eq:1}
.\end{equation} 
Умножая производящую функцию на $\hbar ^2$, чтобы
избавиться от отрицательных степеней $\hbar $ 
в разложении, получаем
\begin{equation}
	F^\hbar _H(\mathbf{t})=\sum_{g=0}^{\infty} \hbar ^{2g}F_H^g(\mathbf{t})
.\end{equation}
Покажем, что топологической деформацией \eqref{eq:1}
из $\tau$-функции иерархии КП \eqref{eq:4} действительно можно
получить $\tau$-функцию иерархии  $\hbar $-КП \eqref{eq:3}.
Имеем
\begin{equation}
	\tau^\hbar _H(\mathbf{t})=
	\sum_{\lambda}^{} \mathrm{e} ^{u\hbar c(\lambda)}
	s_\lambda\left( \beta_k =\delta_{k,1} \right) 
	s_\lambda\left( \frac{t_1}{\hbar ^2},\,
	\frac{t_2}{\hbar ^3},\ldots\right) 
.\end{equation}
Можно показать (это вопросов не вызывает), что
\begin{equation}
	\tau^\hbar _H(\mathbf{t})=
	\sum_{\lambda}^{} \mathrm{e} ^{u\hbar c(\lambda)}
	s_\lambda\left( \frac{1}{\hbar },\,0,\,0,\ldots \right) s_\lambda\left( \frac{\mathbf{t}}{\hbar } \right) 
\end{equation}
и что коэффициенты
\begin{equation}
	C_\lambda^\hbar = \mathrm{e} ^{u\hbar  c(\lambda)}
	s_\lambda \left( \beta_k =\frac{\delta_{k,1}}{\hbar } \right) 
\end{equation}
удовлетворяют соотношениям Плюкера. Это показывает, что $\hbar $-деформацией
мы получаем $\tau$-функцию иерархии $\hbar $-КП. Для найденной функции было явно посчитано разложение по $\mathbf{t}$ для первых порядков и проверено, что в неё действительно входят только чётные степени $\hbar $.
\section{Иерархия $B$КП}
\subsection{Билинейное тождество $B$КП}
\emph{Билинейное тождество иерархии $B$КП}
\begin{multline}
\frac{1}{2\pi \mathrm{i} } \oint
\mathrm{e}^{\xi^{\text{н}}\left( \mathbf{t}-\mathbf{t}',\,k \right) }\tau_{B\text{КП}}\left( \mathbf{t}-2 \left[ k^{-1} \right]  \right) \tau_{B\text{КП}}\left( \mathbf{t}'+2 \left[ k^{-1} \right]  \right) \frac{\mathrm{d} k}{k}=\\=
\tau_{B\text{КП}}\left( \mathbf{t} \right) \tau_{B\text{КП}}
(\mathbf{t}')
,\end{multline} 
где
\begin{equation}
	\mathbf{t}\pm \left[ k^{-1} \right] 
	\xlongequal[]{\text{опр}}
	\left\{ t_1\pm k^{-1},\,t_2 \pm \frac{1}{2}
	k^{-2},\,t_3\pm \frac{1}{3}k^{-3},\ldots\right\} 
\end{equation} 
и
\begin{equation}
	\xi^\text{н}\left( \mathbf{t},\,k \right)=
	\sum_{j \in \mathbb{Z}_{\text{неч}}^+}^{} t_j k^j
.\end{equation} 
Первое уравнение иерархии $B$КП в терминах производных
Хироты
\begin{equation}
	\left( \mathrm{D}_1^6-5\mathrm{D}_1^3 \mathrm{D}_3
	-5 \mathrm{D}_3^2+9 \mathrm{D}_1 \mathrm{D}_5\right) \tau_{B\text{КП}}\cdot \tau_{B \text{КП}}
.\end{equation} 
Что можно переписать как
\begin{multline}
-60 \left(\frac{\partial ^2F}{\partial
   t_1^2}\right)^3-30 \frac{\partial ^4F}{\partial
   t_1^4} \frac{\partial ^2F}{\partial
   t_1^2}+30 \frac{\partial ^2F}{\partial t_1\, \partial t_3}
   \frac{\partial ^2F}{\partial t_1^2}-\frac{\partial
   ^6F}{\partial t_1^6}+\\+5 \frac{\partial ^2F}{\partial
   t_3^2}-9 \frac{\partial ^2F}{\partial t_1\, \partial t_5}+5
   \frac{\partial ^4F}{\partial t_1^3\, \partial t_3}=0	
	\label{}
.\end{multline}
\subsection{$Q$-полиномы Шура, соотношения Плюккера  $B$КП,
гипергеометрические  $\tau^{B\text{КП}}$-функции}
\emph{Соотношения Плюккера} для $B$КП
\begin{multline}
c_{\left[ \alpha_1,\ldots,\,\alpha_k \right] }c_{\left[ 
\alpha_1,\ldots,\,\alpha_k,\,\beta_1,\,\beta_2,\,\beta_3,\,\beta_4\right] }-c_{\left[ \alpha_1,\ldots,\,\alpha_k,\,
\beta_1,\,\beta_2\right] }c_{\left[ \alpha_1,\ldots,\,
\alpha_k,\,\beta_3,\,\beta_4\right] }+\\+c_{\left[ \alpha_1,\ldots,\,\alpha_k,\,
\beta_1,\,\beta_3\right] }c_{\left[ \alpha_1,\ldots,\,
\alpha_k,\,\beta_2,\,\beta_4\right] }-c_{\left[ \alpha_1,\ldots,\,\alpha_k,\,
\beta_1,\,\beta_4\right] }c_{\left[ \alpha_1,\ldots,\,
\alpha_k,\,\beta_2,\,\beta_3\right] }=0
.\end{multline} 
\ytableausetup{boxsize=0.2em}
Простейшее соотношение Плюккера \begin{equation}
	c_{\emptyset}c_{\ydiagram{3,2,1}}-c_{\ydiagram{1}}c_{\ydiagram{3,2}}+c_{\ydiagram{2}}c_{\ydiagram{3,1}}-c_{\ydiagram{3}}c_{\ydiagram{2,1}}=0
.\end{equation} 
Проверено, что \emph{$Q$-полиномы Шура} действительно удовлетворяют
простейшему соотношению Плюккера.

Для определения гипергеометрических $\tau$-функций зададим функцию
\begin{equation}
	r_\lambda= \prod_{w \in \lambda}^{} r\left( c(w) \right)  
,\end{equation} 
где
\begin{equation}
	c(w)=j,\qquad  1\le i\le l(\lambda),\qquad
	1\le j\le \lambda_i
.\end{equation} 
Визуализация функции $c(w)$ на таблице Юнга:
\ytableausetup{boxsize=1.5em}
\begin{center}
\begin{ytableau}
	1&2&3&4\\
	1&2&3\\
	1&2
\end{ytableau}
\end{center}
\emph{Гипергеометрическими $\tau$-функциями $B$КП} будем называть
функции вида
\begin{equation}
	\tau(\mathbf{t})=\sum_{\lambda}^{} r_\lambda
	Q_\lambda(\boldsymbol{\beta}) Q_\lambda(\mathbf{t})
.\end{equation} 
Данные функции действительно решают иерархию $B$КП,  т.\:к.
$Q$-полиномы Шура удовлетворяют соотношениям Плюккера $B$КП,
а множители $r(c(w))$ выносятся как общие. Например,
для простейшего соотношения Плюккера общим множителем
будет
\begin{equation}
	r(1)^3r(2)^2 r(3)
.\end{equation} 
\subsection{$\tau$-функция спиновых чисел Гурвица}
Следующая $\tau$-функция является решением иерархии $B$КП:
\begin{equation}
	\tau \left( \mathbf{p},\,\bar{\mathbf{p}} \right) =
	\sum_{R \in \mathrm{SP}}^{} \left( \mathrm{e} ^{u \left[ 
	\Phi_R\left( \left[ 3 \right]  \right) +\frac{1}{2}\Phi_R \left( \left[ 1,\,1 \right]  \right) \right] } \right) Q_R\left(\frac{\mathbf{p}}{2}\right)Q_R \left(\frac{\bar{\mathbf{p}}}{2}\right)
	\label{}
.\end{equation}
\subsection{Фермионный формализм и бозонно-фермионное соответствие}
Нейтральные фермионы определяются как
\begin{equation}
	\phi_m= \frac{\psi_m +\left( -1 \right) ^m \psi^*_{-m}}{\sqrt{2} }
,\end{equation} 
\begin{equation}
	\phi_m^*= \frac{\psi_m^* +\left( -1 \right) ^m \psi_{-m}}{\sqrt{2} }
.\end{equation}
Откуда сразу можно заметить что 
\begin{equation}
	\phi_m^*= (-1)^m \phi_{-m}
.\end{equation} 
Благодаря этому свойству далее мы можем ограничиться
рассмотрением лишь фермионов $ \phi_m  $.
Прямой подстановкой можно получить каноническое коммутационное соотношение нейтральных
фермионов
\begin{equation}
	\left\{ \phi_k,\,\phi_m \right\} =\left( -1 \right) ^k
	\delta_{k+m,0}
,\end{equation} 
а также их действие на вакуум <<моря Дирака>>
\begin{equation}
\phi_m \ket{0}=0,\quad
\bra{0}\phi_{-m} =0,\quad m<0.\end{equation} 

Коммутационные соотношениям
билинейных комбинаций нейтральных фермионов $\phi_k \phi_m$
\begin{multline}
\left[ \phi_{a}\phi_b,\,\phi_{c}\phi_d \right] =
\phi_a \phi_b \phi_c \phi_d -\phi_c \phi_d \phi_a \phi_b=
\phi_a \left( \left( -1 \right) ^b \delta_{b+c,0}-\phi_c \phi_b
 \right) 	\phi_d-\\-
 \phi_c \left( \left( -1 \right) ^a \delta_{a+d,0}-\phi_a \phi_d \right) \phi_b=
\left( -1 \right) ^b \delta_{b+c,0} \phi_a \phi_d-
(\left( -1 \right)^a \delta_{a+c,0}- \phi_c \phi_a) \phi_b \phi_d-\\-\left( -1 \right) ^a
\delta_{a+d,0} \phi_c \phi_b+\phi_c \phi_a \left(\left( -1 \right) ^b \delta_{b+d,0}-\phi_b \phi_d
\right) =\left( -1 \right) ^b \delta_{b+c,0} \phi_a \phi_d-\\-
\left( -1 \right)^a \delta_{a+c,0} \phi_b \phi_d+ \left( -1 \right) ^b \delta_{b+d,0} \phi_c \phi_a-\left( -1 \right) ^a
\delta_{a+d,0} \phi_c \phi_b
%\phi_a \phi_b \phi_c \phi_d -\\-
%\phi_c \left( \left(-1\right)^a \delta_{a+d,0}-\phi_a \phi_d \right) \phi_b=
%\phi_a \phi_b \phi_c \phi_d -
%\left( -1 \right) ^a \delta_{a+d,0}\phi_c \phi_b
%+\\+\left( \left( -1 \right) ^a \delta_{a+c,0}- \phi_a \phi_c \right) \left( \left( -1 \right) ^b\delta_{b+d,0} -\phi_b \phi_d \right) =\\=
%\phi_a \phi_b \phi_c \phi_d -\left( -1 \right) ^a
%\delta_{a+d,0} \phi_c \phi_b
%\left( -1 \right) ^b \delta_{b+c,0}
%	\phi_a \phi_d-\left( -1 \right) ^a
%	\delta_{a+c,0}
%	\phi_b \phi_d+\left( -1 \right) ^b
%	\delta_{b+d,0}\phi_c \phi_a-\\-
%	\left( -1 \right) ^a \delta_{a+d,0}
%	\phi_c \phi_b
.\end{multline} 
Элементами матричной алгебры Ли $\mathfrak{go}(\infty)$
являются матрицы
\begin{equation}
	F_{k,m}=(-1)^m E_{k,-m}-(-1)^k E_{m,-k}
.\end{equation} 
Их коммутационные соотношения
\begin{multline}
	\left[ F_{a,b},\,F_{c,d} \right] =(-1)^{b+d}\left[ E_{a,-b},\, E_{c,-d} \right] -(-1)^{b+c} \left[ 
	E_{a,-b},\,E_{d,-c}\right] -\\-\left( -1 \right) ^{
a+d}\left[ E_{b,-a},\,E_{c,-d} \right] +
\left( -1 \right) ^{a+c}\left[ E_{b,-a},\,E_{d,-c} \right] =
\\=(-1)^{b+d}\left( \delta_{b+c,0} E_{a,-d}-
\delta_{a+d,0} E_{c,-b}\right) -
\left( -1 \right) ^{b+c}\left( \delta_{b+d,0}E_{a,-c}-
\delta_{a+c,0}E_{d,-b}\right)-\\-
\left( -1 \right) ^{a+d} \left( \delta_{a+c,0} E_{b,-d}
-\delta_{b+d,0}E_{c,-a}\right) +\left( -1 \right) ^{a+c}
\left( \delta_{a+d,0} E_{b,-c} -\delta_{b+c,0} E_{d,-a} \right) =\\=
(-1)^b \delta_{b+c,0} F_{a,d}-(-1)^a \delta_{a+c}F_{b,d}+
\left( -1 \right) ^b\delta_{b+d,0} F_{c,a}-\left( -1 \right) ^a \delta_{a+d,0} F_{c,b}
\end{multline} 
совпадают с коммутационными соотношениями билинейных
комбинаций нейтральных фермионов. Значит представление
алгебры $\mathfrak{go}(\infty)$ может быть реализовано
на введённом пространстве Фока элементами $\phi_k \phi_m$.
\begin{equation}
	\phi_k \phi_m=
	\frac{1}{2}\left(\psi_k +\left( -1 \right) ^k \psi^*_{-k}
	\right)\left( \psi_m+\left( -1 \right) ^m \psi_{-m}^* \right) =???
.\end{equation} 
Нормальное упорядочение на этот раз определим как
\begin{equation}
\normord{\phi_k \phi_m}= \phi_k \phi_m - \braket{0 | \phi_k
\phi_m| 0}
.\end{equation} 
Легко видеть, что 
\begin{equation}
	\braket{0 | \phi_k \phi_m | 0}=\delta_{k+m,0} H[m]
,\end{equation}
где
\begin{equation}
	H[m]= \begin{cases}
		0,& m<0,\\
		\frac{1}{2},& m=0,\\
		(-1)^m,& m>0.
	\end{cases}
\end{equation} 
Коммутационное соотношение алгебры нормально упорядоченных пар нейтральных
фермионов 
\begin{multline}
	\left[ \normord{\phi_a \phi_b},\,\normord{\phi_c \phi_d} \right] =\left( -1 \right) ^b \delta_{b+c,0}
	\normord{\phi_a \phi_d}-\left( -1 \right) ^a
	\delta_{a+c,0}
	\normord{\phi_b \phi_d}+\left( -1 \right) ^b
	\delta_{b+d,0}\normord{\phi_c \phi_a}-\\-
	\left( -1 \right) ^a \delta_{a+d,0}
	\normord{\phi_c \phi_b}+
	\left(\delta_{c,b}\delta_{a,d}-\delta_{a-c,0}
	\delta_{b-d,0}\right) \left( 
\left( -1 \right) ^a H[b]-\left( -1 \right) ^b H[a]\right) 
.\end{multline} 
\section{Квантовая деформация иерархии $B$КП}
Аналогично деформации иерархии КП получаем первое уравнение $\hbar $-$B$КП
\begin{multline}
-60 \left(\frac{\partial ^2F}{\partial
   t_1^2}\right)^3-30\hbar ^2 \frac{\partial ^4F}{\partial
   t_1^4} \frac{\partial ^2F}{\partial
   t_1^2}+30 \frac{\partial ^2F}{\partial t_1\, \partial t_3}
   \frac{\partial ^2F}{\partial t_1^2}-\hbar ^4\frac{\partial
   ^6F}{\partial t_1^6}+\\+5 \frac{\partial ^2F}{\partial
   t_3^2}-9 \frac{\partial ^2F}{\partial t_1\, \partial t_5}+5
   \hbar ^2\frac{\partial ^4F}{\partial t_1^3\, \partial t_3}=0	
	\label{}
.\end{multline}

\end{document}
