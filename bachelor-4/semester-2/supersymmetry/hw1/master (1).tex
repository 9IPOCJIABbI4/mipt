\documentclass[a4paper, 14pt]{extarticle}
\usepackage{fullpage}
% Этот шаблон документа разработан в 2014 году
% Данилом Фёдоровых (danil@fedorovykh.ru) 
% для использования в курсе 
% <<Документы и презентации в \LaTeX>>, записанном НИУ ВШЭ
% для Coursera.org: http://coursera.org/course/latex .
% Исходная версия шаблона --- 
% https://www.writelatex.com/coursera/latex/5.3

% В этом документе преамбула

\usepackage{siunitx}
%%% Работа с русским языком
\usepackage{cmap}					% поиск в PDF
\usepackage{mathtext} 				% русские буквы в формулах
\usepackage[T2A]{fontenc}			% кодировка
\usepackage[utf8]{inputenc}			% кодировка исходного текста
\usepackage[english,russian]{babel}	% локализация и переносы
\usepackage{indentfirst}
\frenchspacing

\renewcommand{\epsilon}{\ensuremath{\varepsilon}}
\renewcommand{\phi}{\ensuremath{\varphi}}
\renewcommand{\kappa}{\ensuremath{\varkappa}}
\renewcommand{\le}{\ensuremath{\leqslant}}
\renewcommand{\leq}{\ensuremath{\leqslant}}
\renewcommand{\ge}{\ensuremath{\geqslant}}
\renewcommand{\geq}{\ensuremath{\geqslant}}
\renewcommand{\emptyset}{\varnothing}
\renewcommand{\Im}{\operatorname{Im}}
\renewcommand{\Re}{\operatorname{Re}}


%%% Дополнительная работа с математикой
\usepackage{amsmath,amsfonts,amssymb,amsthm,mathtools} % AMS
\usepackage{icomma} % "Умная" запятая: $0,2$ --- число, $0, 2$ --- перечисление

%% Номера формул
%\mathtoolsset{showonlyrefs=true} % Показывать номера только у тех формул, на которые есть \eqref{} в тексте.
%\usepackage{leqno} % Нумереация формул слева

%% Свои команды
\DeclareMathOperator{\sgn}{\mathop{sgn}}
\DeclareMathOperator{\sign}{\mathop{sign}}
\DeclareMathOperator*{\res}{\mathop{res}}
\DeclareMathOperator*{\tr}{\mathop{tr}}

%% Перенос знаков в формулах (по Львовскому)
\newcommand*{\hm}[1]{#1\nobreak\discretionary{}
{\hbox{$\mathsurround=0pt #1$}}{}}

%%% Работа с картинками
\usepackage{graphicx}  % Для вставки рисунков
\graphicspath{{figures/}}  % папки с картинками
\setlength\fboxsep{3pt} % Отступ рамки \fbox{} от рисунка
\setlength\fboxrule{1pt} % Толщина линий рамки \fbox{}
\usepackage{wrapfig} % Обтекание рисунков текстом

%%% Работа с таблицами
\usepackage{array,tabularx,tabulary,booktabs} % Дополнительная работа с таблицами
\usepackage{longtable}  % Длинные таблицы
\usepackage{multirow} % Слияние строк в таблице

%%% Теоремы
\theoremstyle{plain} % Это стиль по умолчанию, его можно не переопределять.
\newtheorem{theorem}{Теорема}
\newtheorem*{thm}{Теорема}
\newtheorem{prop}{Утверждение}
 
\theoremstyle{definition} % "Определение"
%\newtheorem{corollary}{Следствие}[theorem]
\newtheorem*{dfn}{Определение}
\newtheorem{problem}{Задача}
\newtheorem*{problem*}{Задача}

 
\theoremstyle{remark} % "Примечание"
\newtheorem*{sol}{Решение}
\newtheorem*{rem}{Замечание}

%%% Программирование
\usepackage{etoolbox} % логические операторы

%%% Страница
%\usepackage{extsizes} % Возможность сделать 14-й шрифт
%\usepackage{geometry} % Простой способ задавать поля
%	\geometry{top=25mm}
%	\geometry{bottom=35mm}
%	\geometry{left=35mm}
%	\geometry{right=20mm}
 
\usepackage{fancyhdr} % Колонтитулы
%	\pagestyle{fancy}
 %	\renewcommand{\headrulewidth}{0pt}  % Толщина линейки, отчеркивающей верхний колонтитул
	%\lfoot{Нижний левый}
	%\rfoot{Нижний правый}
	%\rhead{Верхний правый}
	%\chead{Верхний в центре}
	%\lhead{Верхний левый}
	%\cfoot{Нижний в центре} % По умолчанию здесь номер страницы

\usepackage{setspace} % Интерлиньяж
%\onehalfspacing % Интерлиньяж 1.5
%\doublespacing % Интерлиньяж 2
%\singlespacing % Интерлиньяж 1

\usepackage{lastpage} % Узнать, сколько всего страниц в документе.

\usepackage{soul} % Модификаторы начертания

\usepackage{hyperref}
%\usepackage[usenames,dvipsnames,svgnames,table,rgb]{xcolor}
\hypersetup{				% Гиперссылки
    unicode=true,           % русские буквы в раздела PDF
    pdftitle={Заголовок},   % Заголовок
    pdfauthor={Автор},      % Автор
    pdfsubject={Тема},      % Тема
    pdfcreator={Создатель}, % Создатель
    pdfproducer={Производитель}, % Производитель
    pdfkeywords={keyword1} {key2} {key3}, % Ключевые слова
    colorlinks=true,       	% false: ссылки в рамках; true: цветные ссылки
    linkcolor=red,          % внутренние ссылки
    citecolor=black,        % на библиографию
    filecolor=magenta,      % на файлы
    urlcolor=cyan           % на URL
}

\usepackage{csquotes} % Еще инструменты для ссылок

%\usepackage[style=apa,maxcitenames=2,backend=biber,sorting=nty]{biblatex}

\usepackage{multicol} % Несколько колонок

\usepackage{tikz} % Работа с графикой
\usepackage{pgfplots}
\usepackage{pgfplotstable}
%\usepackage{coloremoji}
\usepackage{floatrow}
\usepackage{subcaption}
\newcommand*{\N}{\mathbb{N}}
\newcommand*{\R}{\mathbb{R}}
\newcommand*{\K}{\mathbb{K}}
\newcommand*{\V}{\mathcal{V}}
\newcommand*{\A}{\mathcal{A}}
\newcommand*{\ii}{\mathbf{1}}
\newcommand*{\oo}{\mathbf{0}}
\newcommand*{\ba}{\mathbf{a}}
\newcommand*{\bb}{\mathbf{b}}
\newcommand*{\Q}{\mathbb{Q}}
\graphicspath{{figures/}}
%\usepackage{breqn}

\renewcommand\thesubfigure{\asbuk{subfigure}}
%\addbibresource{master.bib}

\usepackage{import}
\usepackage{pdfpages}
\usepackage{transparent}
\usepackage{xcolor}
\usepackage{xifthen}

%\newcommand{\incfig}[1]{%
%    \def\svgwidth{\columnwidth}
%    \import{./figures/}{#1.pdf_tex}
%}


\newcommand{\incfig}[2][1]{%
    \def\svgwidth{#1\columnwidth}
    \import{./figures/}{#2.pdf_tex}
}
\usepackage{titlesec}
%\titleformat{\section}{\normalfont\Large\bfseries}{}{0pt}{}
%----------------------STANDART:
%\titleformat{\chapter}[display]
%  {\normalfont\huge\bfseries}{\chaptertitlename\ \thechapter}{20pt}{\Huge}
%\titleformat{\section}{\normalfont\Large\bfseries}{\thesection}{1em}{}
%\titleformat{\subsection}
%  {\normalfont\large\bfseries}{\thesubsection}{1em}{}
%\titleformat{\subsubsection}
%  {\normalfont\normalsize\bfseries}{\thesubsubsection}{1em}{}
%\titleformat{\paragraph}[runin]
%  {\normalfont\normalsize\bfseries}{\theparagraph}{1em}{}
%\titleformat{\subparagraph}[runin]
%  {\normalfont\normalsize\bfseries}{\thesubparagraph}{1em}{}

\pdfsuppresswarningpagegroup=1
\pgfplotsset{compat=1.16}

\usepackage{xifthen}
\makeatother
%\def\@lecture{}%
%\newcommand{\lecture}[3]{
%    \ifthenelse{\isempty{#3}}{%
%        \def\@lecture{Неделя #1}%
%    }{%
%        \def\@lecture{Неделя #1: #3}%
%    }%
%    \section*{\@lecture}
%    \marginpar{\small\textsf{\mbox{#2}}}
%}
\makeatletter

%\newcommand{\lec}{\subsection{Лекция}}
%\newcommand{\sem}{\subsection{Семинар}}
%\newcommand{\hw}{\subsection{Домашняя работа}}
%\newcommand{\prob}[1]{\textbf{#1}}
%\renewcommand{\thesubsection}{}
%\renewcommand{\thesubsubsection}{}

%\setcounter{tocdepth}{1} % only parts,chapters,sections
%\titleformat{\subsection}{\normalfont\large\bfseries}{}{0em}{}
%\titleformat{\subsubsection}{\normalfont\normalsize\bfseries}{}{0em}{}

%\newcommand{\textover}[2]{\stackrel{\mathclap{\normalfont\mbox{#2}}}{#1}}

\author{Драчов Ярослав\\
Факультет общей и прикладной физики МФТИ}
\newcommand{\veq}{\mathrel{\rotatebox{90}{$=$}}}
%\newcommand{\teto}[1]{\stackrel{\mathclap{\normalfont\tiny\mbox{#1}}}{\to}}
%\renewcommand{\thesubsection}{\arabic{subsection}}

%%\setcounter{secnumdepth}{0}

\definecolor{tabblue}{RGB}{30, 119, 180}
\definecolor{taborange}{RGB}{255, 127, 15}
\definecolor{tabgreen}{RGB}{45, 160, 43}
\definecolor{tabred}{RGB}{214, 38, 40}
\definecolor{tabpurple}{RGB}{148, 103, 189}
\definecolor{tabbrown}{RGB}{140, 86, 76}
\definecolor{tabpink}{RGB}{227, 119, 193}
\definecolor{tabgray}{RGB}{127, 127, 127}
\definecolor{tabolive}{RGB}{188, 189, 33}
\definecolor{tabcyan}{RGB}{22, 190, 207}
\pgfplotscreateplotcyclelist{colorbrewer-tab}{
{tabblue},
{taborange},
{tabgreen},
{tabred},
{tabpurple},
{tabbrown},
{tabpink},
{tabgray},
{tabolive},
{tabcyan},
}
\usepackage{csvsimple}
\usepackage{extarrows}
%\renewcommand{\labelenumii}{\asbuk{enumii})}
%\renewcommand{\labelenumiv}{\Asbuk{enumiv}}
\newcommand{\prob}[1]{\subsubsection*{#1}}
\sisetup{output-decimal-marker = {,},separate-uncertainty = true,exponent-product = \cdot}

\usepackage{braket}
\usepackage{enumerate}
\usepackage{chngcntr}
%\counterwithin*{equation}{problem}
%\usepackage{bbold}

\newtheoremstyle{hiProb}% ⟨name ⟩ 
{3pt}% ⟨Space above ⟩1 
{3pt}% ⟨Space below ⟩1
{}% ⟨Body font ⟩
{}% ⟨Indent amount ⟩2
{\bfseries}% ⟨Theorem head font⟩
{.}% ⟨Punctuation after theorem head ⟩
{.5em}% ⟨Space after theorem head ⟩3
%{\thmname{#1} \thmnote{#3}}% ⟨Theorem head spec (can be left empty, meaning ‘normal’)⟩
{\thmnote{#3}}% ⟨Theorem head spec (can be left empty, meaning ‘normal’)⟩
\theoremstyle{hiProb} % "Определение"
%\newtheorem{hiProb}{Задача}
\newtheorem{hiProb}{}
\usepackage{mmacells}
\newcommand{\textover}[2]{\stackrel{\mathclap{\normalfont\scriptsize\mbox{#2}}}{#1}}
\usepackage{units}
\usepackage[math]{cellspace}%
\setlength\cellspacetoplimit{2pt}
\setlength\cellspacebottomlimit{2pt}

%%% Работа с русским языком
\usepackage{cmap}					% поиск в PDF
\usepackage{mathtext} 				% русские буквы в формулах
\usepackage[T2A]{fontenc}			% кодировка
\usepackage[utf8]{inputenc}			% кодировка исходного текста
\usepackage[english,russian]{babel}	% локализация и переносы
\usepackage{indentfirst}
\frenchspacing

\renewcommand{\epsilon}{\ensuremath{\varepsilon}}
\newcommand{\phibackup}{\ensuremath{\phi}}
\renewcommand{\phi}{\ensuremath{\varphi}}
\renewcommand{\varphi}{\ensuremath{\phibackup}}
\renewcommand{\kappa}{\ensuremath{\varkappa}}
\renewcommand{\le}{\ensuremath{\leqslant}}
\renewcommand{\leq}{\ensuremath{\leqslant}}
\renewcommand{\ge}{\ensuremath{\geqslant}}
\renewcommand{\geq}{\ensuremath{\geqslant}}
\renewcommand{\emptyset}{\varnothing}
\renewcommand{\Im}{\operatorname{Im}}
\renewcommand{\Re}{\operatorname{Re}}

\author{Драчов Ярослав\\
Факультет общей и прикладной физики МФТИ}


\title{Д/з по суперсимметрии}
\begin{document}
	\maketitle
\section*{Задание 1.1}
\begin{hiProb}[Задача 1]
\end{hiProb}
\begin{sol}
%Уравнения Эйлера-Лагранжа
%\[
%	\frac{\partial L}{\partial x}-\frac{\mathrm{d} }{\mathrm{d} t} \frac{\partial L}{\partial \dot{x}}=0,\qquad
%\frac{\partial L}{\partial \psi}-\frac{\mathrm{d} }{\mathrm{d} t} \frac{\partial L}{\partial \dot{\psi}}=0,
%\qquad \frac{\partial L}{\partial \psi^\dagger}-\frac{\mathrm{d} }{\mathrm{d} t} \frac{\partial L}{\partial \dot{\psi}^\dagger}=0
%.\] 
%Откуда
%\[
%\ddot{x}=0,\qquad -\mathrm{i}  \dot{\psi}^\dagger+
%\psi^\dagger \ddot{h}=0,\qquad
%\mathrm{i} \dot{\psi}+\psi \ddot{h}=0
%.\] 
Найдём вариацию действия
\begin{multline*}
	\delta S= \int \mathrm{d} t \left[ 
	\dot{x} \delta \dot{x} + \mathrm{i} 
\delta \psi^\dagger \dot{\psi} + \mathrm{i} \psi^\dagger
\delta \dot{\psi} -h' h''
\delta x \right. + \\ + \left. h''' \delta x \psi^\dagger
\psi + h''\left( \delta \psi^\dagger \psi+ \psi^\dagger
\delta \psi\right) \right] =\\=
\int \mathrm{d} t \left[ 
\left(-\ddot{x}-h'h''+h''' \psi^\dagger \psi\right)\delta x\right. + \\ + \left.\left( -\mathrm{i} \dot{\psi}^\dagger+ h'' \psi^\dagger \right) \delta \psi+\left( -\mathrm{i} \dot{\psi}-h''\psi  \right)\delta\psi^\dagger  \right] 
.\end{multline*} 
Откуда
\[
\ddot{x}= h''' \psi^\dagger \psi-h' h'' ,\qquad
\dot{\psi}=\mathrm{i} h'' \psi,\qquad
\dot{\psi}^\dagger=-\mathrm{i} h'' \psi^\dagger
.\] 
Далее
\[
h''= \frac{h'''\psi^\dagger \psi-\ddot{x}}{h'}
.\] 
\[
\dot{\psi}=-\mathrm{i} \frac{\ddot{x}}{h'}\psi,\qquad
\dot{\psi}^\dagger=  \mathrm{i} \frac{\ddot{x}}{h'} \psi^\dagger
.\] 
%\begin{multline*}
%	0=\int \mathrm{d} t \left( 
%\ddot{x} \psi +\mathrm{i} h'\dot{\psi} \right) =C+
%\int \mathrm{d} t \left(-\dot{x}\dot{\psi} +\mathrm{i} h' \dot{\psi} \right) =\\=
%C- \int \mathrm{d} \psi \left(\dot{x}- \mathrm{i} h'  \right) =
%C- \left( \dot{x}-\mathrm{i} h' \right) \int \mathrm{d} \psi=C
%.\end{multline*} 
%\[
%	\frac{\mathrm{d} \psi / \mathrm{d} t}{\mathrm{d} \dot{x} / \mathrm{d} t}=\mathrm{i} \frac{\psi}{h'}, \implies
%h' \mathrm{d} \psi- \mathrm{i} \psi \mathrm{d} \dot{x}=0
%.\] 
%\[
%	0= \int \mathrm{d} t \ \left( 
%	h' \dot{\psi}- \mathrm{i}  \ddot{x} \psi\right) 
%.\]  
Дифференцируя суперзаряды по времени, получаем
\begin{multline*}
	0=\dot{Q}=\ddot{x}\psi+ \dot{x}\dot{\psi} - \mathrm{i} \left( h''\dot{x}\psi + h' \dot{\psi} \right) =\\=
	\left( \ddot{x} \psi-\mathrm{i} h'\dot{\psi} \right)+
	\dot{x}\left(\dot{\psi}-\mathrm{i} h''\psi \right)=0 
.\end{multline*} 
\begin{multline*}
	0=\dot{Q}^\dagger=\ddot{x}\psi^\dagger+ \dot{x}\dot{\psi}^\dagger + \mathrm{i} \left( h''\dot{x}\psi^\dagger + h' \dot{\psi} ^\dagger\right) =\\=
	\left( \ddot{x} \psi^\dagger+\mathrm{i} h'\dot{\psi} ^\dagger\right)+
	\dot{x}\left(\dot{\psi}^\dagger+\mathrm{i} h''\psi ^\dagger\right)=0 .\end{multline*} 
%\[
%\dot{Q}=0\implies
%.\] 
%\[
%	0=\delta Q= \psi\delta \dot{x} + \dot{x} \delta \psi- \mathrm{i} \left( h''\delta x \psi +h' \delta \psi \right),\qquad
%	\] 
%\[
%0=\delta Q^\dagger= \psi^\dagger \delta \dot{x}+\dot{x}\delta \psi^\dagger+\mathrm{i} 
%	\left( h''\delta x \psi^\dagger +h' \delta \psi^\dagger \right) 
%.\] 
%\[
%	0=\psi \delta Q= \left( \dot{x}-\mathrm{i} h' \right) \psi \delta\psi=Q \delta \psi\implies
%	\delta 	\psi\sim  Q
%.\] 
%\begin{multline*}
%	0=\psi^\dagger \left(  \psi\delta \dot{x} + \dot{x} \delta \psi- \mathrm{i} \left( h''\delta x \psi +h' \delta \psi \right)\right) +\\+\psi \left( 
%	\psi^\dagger \delta \dot{x}+\dot{x}\delta \psi^\dagger+\mathrm{i} 
%	\left( h''\delta x \psi^\dagger +h' \delta \psi^\dagger \right) 
%\right) =\\=\left(\dot{x}-\mathrm{i}  h' \right)\psi^\dagger \delta \psi +
%\left( \dot{x}+\mathrm{i} h' \right) \psi \delta \psi^\dagger+\mathrm{i} \left( \psi\psi^\dagger-\psi^\dagger \psi  \right) h''\delta x
%.\end{multline*} 
%Из сохранения суперзарядов получим суперсимметрию
%\[
%\delta x= \mathrm{i} \left[ \epsilon ^\dagger
%Q-\epsilon Q^\dagger,\,x\right] =\epsilon ^\dagger \psi
%-\epsilon \psi^\dagger
%,\] 
%\[
%	\delta \psi= \mathrm{i} \left\{ \epsilon ^\dagger Q-\epsilon Q^\dagger,\,\psi \right\} =\epsilon \left( -\mathrm{i} \dot{x}+h' \right) 
%,\] 
%\[
%	\delta \psi^\dagger= \mathrm{i} \left\{ \epsilon ^\dagger Q-\epsilon Q^\dagger,\,\psi^\dagger \right\} =\epsilon^\dagger \left( \mathrm{i} \dot{x}+h' \right) 
%.\]
%Подставим в полученную вариацию действия найденные преобразования симметрии
%\begin{multline*}
%\delta S= \int \mathrm{d} t \ \epsilon \left[ 
%-\dot{x} \dot{\psi}^\dagger - \mathrm{i} \psi^\dagger \frac{\mathrm{d} }{\mathrm{d} t} \left( -\mathrm{i} \dot{x}
%+h'\right) +h'h'' \psi^\dagger-h''' \psi^\dagger\psi
%^\dagger \psi\right. - \\ - \left.h'' \psi^\dagger \left( -\mathrm{i} \dot{x}+h' \right) \right] 
%+\epsilon ^\dagger \left[ 
%\dot{x}\dot{\psi}+\mathrm{i} \left( \mathrm{i} \dot{x}
%+h'\right) \dot{\psi}-h'h''\psi\right. + \\ + \left.h'''\psi\psi^\dagger
%\psi+h'' \left( \mathrm{i} \dot{x}+h' \right) \psi\right]=\\=
%\int \mathrm{d} t \ \epsilon  \left[ 
%\mathrm{i} \dot{\psi}^\dagger h'+ h'h''\psi^\dagger
%-h'' \psi^\dagger \left( -\mathrm{i} \dot{x}+h' \right) \right]+\\+\epsilon ^\dagger \left[ 
%\mathrm{i} h' \dot{\psi}-h'h'' \psi+h''\left( \mathrm{i} \dot{x}+h' \right) \psi\right] =\\=
%\int \mathrm{d} t \ \mathrm{i} \epsilon  \frac{\mathrm{d} }{\mathrm{d} t} \left( h'\psi^\dagger \right) +\mathrm{i} \epsilon ^\dagger \frac{\mathrm{d} }{\mathrm{d} t}
%\left( h' \psi \right) =0
%.\end{multline*} 
Квантовомеханическое следствие теоремы Нётер
\[
	\delta \left( \text{бозонное поле} \right) =
	\mathrm{i} \epsilon \left[\text{заряд},\,\text{поле}  \right] 
,\] 
\[
	\delta \left( \text{фермионное поле} \right) =
	\mathrm{i} \epsilon \left\{\text{заряд},\,\text{поле}  \right\} 
.\] 
В нашем случае из сохранения заряда $Q$ следует симметрия
\[
\delta x=\epsilon \psi,\qquad \delta \psi=0,\qquad
\delta \psi^\dagger=\epsilon \left(\mathrm{i} \dot{x}+h'\right)
.\] 
Проверим, что вариация действия на данной симметрии
действительно равна нулю
\begin{multline*}
\delta S=\int \mathrm{d}  t \ \epsilon  \left[ 
\dot{x} \dot{\psi}^\dagger+ \mathrm{i} 
 \left( \mathrm{i} \dot{x}+h' \right)\dot{\psi} -h'h''\psi^\dagger\right. + \\ + \left.h'''\psi^\dagger \psi^\dagger \psi +h''
\left( \mathrm{i} \dot{x}+h' \right)\psi \right] =
\\=\int \mathrm{d} t\ \epsilon  \left[ \mathrm{i} h'\dot{\psi}  -h'h''\psi+
h''\left(\mathrm{i} \dot{x}+h'  \right)\psi \right] 
=\\= \int \mathrm{d} t\ \mathrm{i} \epsilon  \frac{\mathrm{d} }{\mathrm{d} t}\left( h' \psi\right) =0
.\end{multline*} 
Из сохранения заряда $Q^\dagger$ следует симметрия
\[
	\delta x=\epsilon \psi^\dagger,\qquad \delta \psi=\epsilon \left( \mathrm{i} \dot{x}-h' \right) ,\qquad
\delta \psi^\dagger=0
.\] 
Аналогично
\begin{multline*}
\delta S=\int \mathrm{d}  t \ \epsilon  \left[ 
\dot{x} \dot{\psi}^\dagger- \mathrm{i} 
\psi^\dagger \frac{\mathrm{d} }{\mathrm{d} t}
\left( \mathrm{i} \dot{x}-h' \right) -h'h''\psi^\dagger\right. + \\ + \left.h'''\psi^\dagger \psi^\dagger \psi -h''\psi^\dagger
\left( \mathrm{i} \dot{x}-h' \right) \right] =
\\=\int \mathrm{d} t\ \epsilon  \left[ -\mathrm{i} \dot{\psi}^\dagger h' -h'h''\psi^\dagger-
h''\psi^\dagger\left(\mathrm{i} \dot{x}-h'  \right) \right] 
=\\=- \int \mathrm{d} t\ \mathrm{i} \epsilon  \frac{\mathrm{d} }{\mathrm{d} t}\left( h' \psi^\dagger \right) =0
.\end{multline*} 
Найдём канонические сопряжённые импульсы
\[
	p(t)= \frac{\delta S}{\delta \dot{x}(t)}=
	\dot{x}(t),\qquad
	\pi(t)= \frac{\delta S}{\delta \dot{\psi}(t)}=\mathrm{i} 
	\psi^\dagger(t)
.\] 
Гамильтониан
\[
H=p \dot{x}+\pi \dot{\psi}-L
.\] 
Его <<суперсимметризация>>
\[
	H=\frac{1}{2}\left( p^2 +h'^2 \right) -\frac{1}{2}
	h''\left( \psi^\dagger\psi-\psi \psi^\dagger \right) 
.\] 
%Для $Q$ имеем симметрии
%\[
%\delta x= \epsilon  \frac{\partial Q}{\partial p} =
%\epsilon \psi,\qquad
%\delta \psi= \epsilon \frac{\partial Q}{\partial \pi} 
%=-\epsilon h',
%.\] 
%\[
%\delta L= \frac{\partial L}{\partial x} \delta x+
%\frac{\partial L}{\partial \psi}\delta \psi +\frac{\partial L}{\partial \psi^\dagger} \delta \psi^\dagger+
%.\] 
\end{sol}
\begin{hiProb}[Задача 2]
\end{hiProb}
\begin{sol}
Можем сконструировать гамильтониан вида
\[
H= \frac{1}{2} \left\{ c^\dagger,\,c \right\} =\frac{1}{2}
\mathbbold{1} 
,\] 
однако, кажется, что система с одним энергетическим
уровнем плохо подходит под определение фермионного осциллятора.
Системой с двумя энергетическими уровнями будет, например,
\[
	H=\frac{1}{2} \left[ c^\dagger,\,c \right] =F-\frac{1}{2}\mathbbold{1} 
.\] 
Откуда
\[
H \ket{0}= -\frac{1}{2}\ket{0},\qquad
H\ket{1}=\frac{1}{2} \ket{1}
.\] 
Это уже похоже на то, что должно называться фермионным осциллятором и совпадает с последним членом в гамильтониане
суперсимметричной частицы при $h''=1$.

Рассмотрим нестационарное равнение Шрёдингера
\[
\mathrm{i} \frac{\partial }{\partial t} \ket{\psi}=
H \ket{\psi}
\] 
с начальным условием
\[
\ket{\psi}= a \ket{0}+b \ket{1}
.\] 
Его решением будет
\[
	\ket{\psi}=\mathrm{e} ^{-\mathrm{i} Ht }
	\left( a \ket{0}+b\ket{1} \right) =
	a \mathrm{e} ^{\mathrm{i} t /2}\ket{0}+
	b \mathrm{e} ^{-\mathrm{i} t /2}\ket{1}
.\] 
\end{sol}
\begin{hiProb}[Задача 3]
\end{hiProb}
\begin{sol}
Представлением алгебры нескольких фермионных осцилляторов
может быть, например,
\[
c^i=\underbrace{\sigma_3 \otimes \sigma_3 \otimes \cdots \otimes
\sigma_3 }_{i-1 \text{ раз}}\otimes \begin{pmatrix} 0&0\\1&0 \end{pmatrix} 
\otimes \mathbbold{1} \otimes \cdots \otimes \mathbbold{1} 
.\] 
\[
c^{i\dagger}=\underbrace{\sigma_3 \otimes \sigma_3 \otimes \cdots \otimes
\sigma_3 }_{i-1 \text{ раз}}\otimes \begin{pmatrix} 0&1\\0&0 \end{pmatrix} 
\otimes \mathbbold{1} \otimes \cdots \otimes \mathbbold{1} 
.\] 
Т.\:к.
\[
\left\{ \sigma_3,\,\begin{pmatrix} 0&0\\1&0 \end{pmatrix} 
 \right\} =0,\qquad
\left\{ \sigma_3,\,\begin{pmatrix} 0&1\\0&0 \end{pmatrix} 
 \right\} =0
,\] 
(остальные коммутационные соотношения известны), то выполнено 
\[
\left\{ c^i,\,c^j \right\} =\left\{ c^{i\dagger},\,c^{j\dagger} \right\} =0,\qquad \left\{ c^{i\dagger},\,c^{j} \right\} =\delta^{ij}
.\] 
Гамильтониан системы фермионных осцилляторов
может иметь вид
\[
H=\frac{1}{2} \sum_{i=1}^{n} \left[ c^{i\dagger},\,c^i \right]+ \sum_{i,j,k,l}^{} c^{i\dagger}c^{j\dagger}c^kc^l 
.\] 
\end{sol}
\begin{hiProb}[Задача 4]
\end{hiProb}
\begin{sol}
%Тогда
%\begin{multline*}
%H= \frac{1}{2}\left\{ Q^\dagger,\,Q \right\} =\\=
%\frac{1}{2} \sum_{i=1}^{2} \left( p_i^2+\left( \partial
%_i h\right) ^2 \right) \otimes \mathbbold{1} \otimes \mathbbold{1} -\frac{1}{2}\partial_1^2 h \otimes \sigma_3 \otimes \mathbbold{1}-\frac{1}{2}
%\partial_2^2 h \otimes \mathbbold{1} \otimes \sigma_3
%.\end{multline*} 
Пусть
\[
	Q=\left( p_i-\mathrm{i} \partial_i h \right) \psi^i,
	\qquad Q^\dagger=
	\left( p_i+\mathrm{i} \partial_i h \right) \psi^{\dagger i}
.\] 
Тогда
%\begin{multline*}
%\left\{ Q,\,Q^\dagger \right\} =
%\left( p_i- \mathrm{i} \partial_i h \right) \psi^i
%\left( p_j +\mathrm{i} \partial_j h \right)
%\psi^{\dagger j}+
%\left( p_i+\mathrm{i} \partial_i h \right) \psi^{\dagger i}
%\left( p_j-\mathrm{i} \partial_{j} h \right) \psi^{j}=
%\\=
%\sum_{i=1}^{2} \left( p_i^2+ \left( \partial_i h \right) ^2 \right) -\mathrm{i} \partial_i h p_j \psi^i\psi^{\dagger
%j}+\mathrm{i}p_i\partial_j h \psi^i   \psi^{\dagger j}
%-\mathrm{i} p_i \partial_j h \psi^{\dagger i}\psi^j+
%\mathrm{i} \partial_i h p_j \psi^{\dagger i}\psi^{j}=\\=
%\sum_{i=1}^{2} \left( p_i^2+ \left( \partial_i h \right) ^2 \right) -\mathrm{i} \partial_j h p_i \psi^j\psi^{\dagger
%i}+\mathrm{i}p_j\partial_i h \psi^j   \psi^{\dagger i}
%-\mathrm{i} p_i \partial_j h \psi^{\dagger i}\psi^j+
%\mathrm{i} \partial_i h p_j \psi^{\dagger i}\psi^{j}=
%.\end{multline*} 
%Тогда
%\begin{multline*}
%\left\{ Q,\,Q^\dagger \right\} =
%\left( p_i- \mathrm{i} \partial_i h \right) \psi^i
%\left( p_j +\mathrm{i} \partial_j h \right)
%\psi^{\dagger j}+
%\left( p_j+\mathrm{i} \partial_j h \right) \psi^{\dagger j}
%\left( p_i-\mathrm{i} \partial_{i} h \right) \psi^{i}=
%\end{multline*}
%\begin{multline*}
%	\left( p_1\psi^1-\mathrm{i} \partial_1 h\psi^1 
%	+p_1 \psi^2 -\mathrm{i}  \partial_2 h \psi^2\right) 
%	\left( p_1 \psi^{\dagger 1}+\mathrm{i} \partial_1
%	h \psi^{\dagger 1}
%+p_2 \psi^{\dagger 2} +\mathrm{i} \partial_2 h \psi^{\dagger 2}\right) +\\+\left( p_1 \psi^{\dagger 1}+\mathrm{i} \partial_1
%	h \psi^{\dagger 1}
%+p_2 \psi^{\dagger 2} +\mathrm{i} \partial_2 h \psi^{\dagger 2}\right)\left( p_1\psi^1-\mathrm{i} \partial_1 h\psi^1 
%	+p_1 \psi^2 -\mathrm{i}  \partial_2 h \psi^2\right)=
%p_1^2+ p_2^2
%.\end{multline*} 
\begin{multline*}
\left\{ Q,\,Q^\dagger \right\} =
\left(p_i p_j +\partial_i h \partial_j h\right)\left\{ \psi^i,\,\psi^{\dagger j} \right\}  +\left( 
\mathrm{i} p_i \partial_j h-\mathrm{i} \partial_i h p_j\right) \psi^i \psi^{\dagger j}+\\+
\left( \mathrm{i} \partial_j h p_i - \mathrm{i} p_j
\partial_i h\right) \psi^{\dagger j}\psi^{i}=
\sum_{i=1}^{n} \left( p_i^2 +\left( \partial_i h \right) ^2 \right) +\\+
\left( -\partial_i \partial_j h+\mathrm{i} \partial_j h
p_i-\mathrm{i} \partial_i h p_j\right) \psi^i \psi^{\dagger j}+ \left( \mathrm{i} \partial_j h p_i + \partial_j \partial_i h-\mathrm{i} \partial_ih p_j \right) \psi^{\dagger
j }\psi^i=\\=
\sum_{i=1}^{n} \left( p_i^2 +\left( \partial_i h \right) ^2 \right) +\partial_j \partial_i h \left[ \psi^{\dagger j},\,
	\psi^i\right] +\mathrm{i} \left(\partial_j h p_i -
	\partial_i h p_j\right)\left\{ 
\psi^i,\,\psi^{\dagger j}\right\} =\\=
\sum_{i=1}^{n} \left( p_i^2 +\left( \partial_i h \right) ^2 \right) + \partial_i \partial_j h  \left[ 
\psi^{\dagger i},\,\psi^j\right] 
.\end{multline*} 
И
\[
H=\frac{1}{2} \left\{ Q,\,Q^\dagger\right\} 
.\] 
Будем работать в представлении
\[
	Q= \left( 
	p_1 -\mathrm{i}  \partial_1 h \right) \otimes 
	\begin{pmatrix} 0&0\\1&0 \end{pmatrix}\otimes \mathbbold{1} +
	\left(p_2 -\mathrm{i}  \partial_2 h \right)\otimes  \sigma_3
	\otimes \begin{pmatrix} 0&0\\1&0 \end{pmatrix}
,\] 
\[
	Q^\dagger= \left( 
	p_1 +\mathrm{i}  \partial_1 h \right) \otimes 
	\begin{pmatrix} 0&1\\0&0 \end{pmatrix}\otimes \mathbbold{1} +
	\left(p_2 +\mathrm{i}  \partial_2 h \right)\otimes  \sigma_3
	\otimes \begin{pmatrix} 0&1\\0&0 \end{pmatrix}
.\]
В котором
\[
	\Psi(x)= \begin{pmatrix} \psi_1(x)\\ \psi_2(x)\\ \psi_3(x)\\ \psi_4(x)\\ \end{pmatrix} 
.\] 
Из условия
$
Q\Psi=0
$ получаем уравнения
\[
-\psi_1 \partial_2 h +\partial_2 \psi_1=0,\qquad
-\psi_1 \partial_1 h+\partial_1 \psi_1=0
,\] 
\[
\psi_3 \partial_2 h-\partial_2 \psi_3 -
\psi_2 \partial_1  h+\partial_1 \psi_2=0
.\] 
Аналогично для $Q^\dagger \Psi =0$ 
\[
\psi_2 \partial_2 h+ \partial_2 \psi_2 +\psi_3 \partial_1 h+\partial_1 \psi_3=0
.\] 
\[
\psi_4 \partial_1 h+ \partial_1 \psi_4=0,\qquad
\psi_4 \partial_2 h+\partial_2 \psi_4=0
.\] 
\[
\left\{
\begin{aligned}
\psi_3 \frac{\partial h}{\partial x_2} -\frac{\partial \psi_3}{\partial x_2} -\psi_2 \frac{\partial h}{\partial x_1} + \frac{\partial \psi_2}{\partial x_1} &= 0, \\
\psi_2 \frac{\partial h}{\partial x_2} +
\frac{\partial \psi_2}{\partial x_2} +
\psi_3 \frac{\partial h}{\partial x_1} +
\frac{\partial \psi_3}{\partial x_1} &=0.
\end{aligned}
\right.
\] 
\[
\left\{
\begin{aligned}
	\left( \frac{\partial h}{\partial x_2} -\frac{\partial }{\partial x_2}  \right) \psi_3+
\left( \frac{\partial }{\partial x_1} -\frac{\partial h}{\partial x_1}  \right) \psi_2&= 0, \\
\left( \frac{\partial h}{\partial x_2} +\frac{\partial }{\partial x_2}  \right) \psi_2+
\left( \frac{\partial h}{\partial x_1} +\frac{\partial }{\partial x_1}  \right) \psi_3&=0.
\end{aligned}
\right.
\] 
\[
	\begin{pmatrix}\frac{\partial }{\partial x_1} -\frac{\partial h}{\partial x_1} &
	\frac{\partial h}{\partial x_2} -\frac{\partial }{\partial x_2} \\
\frac{\partial h}{\partial x_2} +\frac{\partial }{\partial x_2} & \frac{\partial h}{\partial x_1} +\frac{\partial }{\partial x_1} \end{pmatrix} \begin{pmatrix} 
\psi_2 \\ \psi_3\end{pmatrix} =\begin{pmatrix} 0\\0 \end{pmatrix} 
.\] 
\end{sol}
\begin{hiProb}[Доп. задача]
\end{hiProb}
\begin{sol}
\[
Q_+= p_i \Psi^i- \frac{\mathrm{i} }{2} \bar{\partial}_{\bar{i}}\bar{W}\bar{\tilde{\Psi}}^{\bar{i}},\qquad
Q_+^\dagger= \bar{p}_{\bar{i}}\bar{\Psi}^{\bar{i}}
+\frac{\mathrm{i} }{2} \partial_i W \tilde{\Psi}^i
.\] 
\[
	Q_-=\bar{p}_{\bar{i}}\bar{\tilde{\Psi}}^{\bar{i}}
	- \frac{\mathrm{i} }{2} \partial_i W \Psi^i,\qquad
	Q_-^\dagger=p_i \tilde{\Psi}^i+\frac{\mathrm{i} }{2}
	\partial_{\bar{i}} \bar{W} \bar{\Psi}^{\bar{i}}
.\] 
\begin{multline*}
\left\{ Q_+,\,Q_+^\dagger \right\} =
\left\{ p_i \Psi^i- \frac{\mathrm{i} }{2} \bar{\partial}_{\bar{i}}\bar{W}\bar{\tilde{\Psi}}^{\bar{i}},\,\bar{p}_{\bar{j}}\bar{\Psi}^{\bar{j}}
+\frac{\mathrm{i} }{2} \partial_j W \tilde{\Psi}^j \right\} 
=  p_i \bar{p}_{\bar{j}} \left\{ 
\Psi^i,\,\bar{\Psi}^{\bar{j}}\right\} +\\+
\frac{1}{4} \bar{\partial}_{\bar{i}} \bar{W}  \partial_j W
\left\{ \bar{ \tilde{\Psi}}^{\bar{i}},\,\tilde{\Psi}^i \right\} - \frac{\mathrm{i} }{2}\left( 
	\bar{\partial}_{\bar{i}} \bar{W}\bar{p}_{\bar{j}}-\bar{\partial}_{\bar{i}} \bar{W}\bar{p}_{\bar{j}}+\mathrm{i} \bar{\partial}_{\bar{i}}\bar{\partial}_{\bar{j}}\bar{W}\right) 
\bar{\tilde{\Psi}}^{\bar{i}}\bar{\Psi}^{\bar{j}}+\\+
\frac{\mathrm{i} }{2} \left( \partial_jWp_i- \partial_jWp_i -\mathrm{i} \partial_i \partial_j W\right)\Psi^i \tilde{\Psi}^j=
\sum_{i=1}^{n} \left(\left| p_i \right| ^2+\frac{1}{4}
\left| \partial_iW \right| ^2\right)+\\+
\frac{1}{2} \sum_{i,j}^{} \left( 
\partial_i \partial_j W \Psi^i \tilde{\Psi}^j+\bar{\partial}_{\bar{i}}\bar{\partial}_{\bar{j}}\bar{W}\bar{\tilde{\Psi}}^{\bar{i}}\bar{\Psi}^{\bar{j}}
\right) =H
.\end{multline*} 
\end{sol}
\section*{Задание 1.2}
\begin{hiProb}[Задача 1]
\end{hiProb}
\begin{sol}
Найдём канонически сопряжённые импульсы
\[
	\pi_M= \frac{\delta S_M}{\delta\dot{\psi}}=
	\frac{\mathrm{i} }{2}\psi,\qquad
	\pi_F= \frac{\delta S}{\delta \dot{\psi}}=
	\frac{\mathrm{i} }{2} \psi^\dagger
.\] 
Канонические коммутационные соотношения
\[
\left\{ \psi,\,\pi_M \right\} =0,\qquad
\left\{ \psi,\,\pi_F \right\} =\frac{\mathrm{i} }{2}
.\] 
Найдём гамильтонианы
\[
H_M= \pi_M \dot{\psi}-L_M=0,\qquad
H_F=\pi_F \dot{\psi}-L_F=0
.\] 
Спектром будет $\lambda=0$ $\forall \psi$. Статсумма
\[
Z= \Tr \mathrm{e} ^{-\beta H}= \int \mathrm{d} \psi \braket{\psi|\psi} = \int \mathrm{d} \psi=0
.\] 
\end{sol}
\begin{hiProb}[Задача 2]
\end{hiProb}
\begin{sol}
Пусть $D$ обозначает дифференцирование по грассманновой
переменной, а $I$ --- интегрирование по грассманновой
переменной, где интегрирование понимается как определённый
интеграл. Пусть они удовлетворяют соотношениям
\renewcommand{\labelenumi}{(\arabic{enumi})}
\begin{enumerate}
\item $ID=0$,
\item $DI=0$,
\item $D(A)=0 \implies I(BA)=I(B)A$,
\end{enumerate}
где  $A$ и $B$ --- произвольные функции  грассманновых
переменных. Первое соотношение говорит о том, что
интеграл от производной любой функции является разностью
значений функции на границе, которую мы полагаем
равной нулю. Второе соотношение говорит о том, что
производная определённого интеграла равна нулю. Третье
соотношение говорит о том, что из выражения $D(A)=0$ 
следует, что $ A$ --- константа и поэтому её можно 
выносить за знак интеграла. Эти соотношения будут
выполнены, если взять $I \propto D$. Используя нормировку
$I=D$ и, полагая
\[
	\int \mathrm{d} \theta\  f(\theta)= \frac{\partial f(\theta)}{\partial(\theta)}
,\] 
находим из определений производной функции грассманновых
переменных, что
\[
\int \mathrm{d} \theta=\frac{\partial \,1}{\partial \theta} 
=0,\qquad
\int d\theta\ \theta=\frac{\partial \theta}{\partial \theta} =1
.\] 
\end{sol}
\begin{hiProb}[Задача 3]
\end{hiProb}
\begin{sol}
Можно показать, что для грассманновых векторов $\boldsymbol{\theta}$ и $\boldsymbol{\theta}^\dagger$ выполняется
\[
\int \mathrm{d} \theta_1^\dagger \mathrm{d} \theta_1
\cdots \mathrm{d} \theta_n^\dagger \mathrm{d} \theta_n
\exp \left( \boldsymbol{\theta}^\dagger \boldsymbol{M\theta} \right) =\det \boldsymbol{M}
.\] 
Тогда
\begin{multline*}
	\int D\psi^\dagger D\psi \exp \left( 
	-\int \mathrm{d} \tau \left[ 
\psi^\dagger \frac{\mathrm{d} \psi}{\mathrm{d} \tau}
+\omega \psi^\dagger \psi\right] \right)=\\=
\int D\psi^\dagger D\psi \exp \left( 
	-\int \mathrm{d} \tau \left[ 
		\psi^\dagger \left(\frac{\mathrm{d} }{\mathrm{d} \tau}
+\omega  \right)\psi\right] \right)\sim \\\sim 
\lim_{\Delta \tau \to 0} 
\int \prod_{i}^{}  \mathrm{d} \psi^\dagger_i \mathrm{d} \psi_i \exp \left( 
	-\Delta \tau \sum_{j}^{} \left[ 
		\psi^\dagger_j \left(\frac{\mathrm{d} }{\mathrm{d} \tau}
+\omega  \right)_j\psi_j\right] \right)\sim \\\sim 
\det \left[ \frac{\mathrm{d} }{\mathrm{d} \tau}+\omega \right] 
.\end{multline*} 
Для антипериодических граничных условий собственные функции
будут иметь вид
\[
	\psi(\tau)= \eta_0 \mathrm{e} ^{\mathrm{i}  k\tau}
	\implies \left( \frac{\mathrm{d} }{\mathrm{d} \tau}
	+\omega\right) \psi=\left( \mathrm{i} k+\omega \right) \psi
 \] 
для некого грассманного параметра $\eta_0$.
Из условия антипериодичности $\psi(\tau+\beta)=-\psi(\tau)$ 
следует
\[
	k=\frac{2\pi (n- 1/2)}{\beta},\qquad
	n \in \mathbb{Z}
,\] 
далее имеем
\begin{multline*}
	\det \left( \frac{\mathrm{d} }{\mathrm{d} \tau}+
	\omega\right) 
	= \prod_{n \in \mathbb{Z}}^{} 
	\left( \frac{2\pi \mathrm{i} \left( n- 1 /2 \right) }{\beta}+\omega \right) =\\=
	\prod_{n=1}^{\infty} \left( \left( 
	\frac{2\pi \left( n- 1 /2 \right) }{\beta}\right) ^2+\omega^2\right) =\\=
	\prod_{n'=1}^{\infty} \left( \frac{2\pi \left( n' -1 /2 \right) }{\beta} \right) ^2 \prod_{n=1}^{\infty} 
	\left( 1+ \left( \frac{\beta \omega}{2\pi\left(n- 1 /2  \right) } \right) ^2 \right)=\\=
	  \prod_{n=0}^{\infty} \left( \frac{2\pi\left( n+1 /2 \right) }{\beta} \right) ^{2}
\ch \left( \frac{\beta \omega}{2} \right)   
.\end{multline*} 
\[
	\zeta_1\left( s \right) 
	=\left( \frac{\beta}{2\pi} \right) ^{2s}
	\zeta\left( 2s,\,1 /2 \right) =
	\sum_{n=0}^{\infty} \left( \frac{2\pi\left( n+1 /2 \right) }{\beta} \right) ^{-2s}
.\] 
\begin{multline*}
	\zeta_1'(s)=2\left( \frac{\beta}{2\pi} \right) ^{2s}
	\left( \ln \left( \frac{\beta}{2\pi} \right) 
	\zeta\left( 2s,\,1 /2 \right) +
\zeta'(2s,\,1 /2)\right) =\\=
\sum_{n=0}^{\infty} \left( \frac{2\pi\left( n+1 /2 \right) }{\beta} \right) ^{-2s} \ln \left( \frac{2\pi\left( n+1 /2 \right) }{\beta} \right) ^{-2}
.\end{multline*} 
Вычисляя при $s=0$, получаем
\[
	\zeta_1(0)=2\ln \left( \frac{\beta}{2\pi} \right) 
	\zeta\left( 0,\,1 /2 \right) +2 \zeta'(0,\,1 /2)
	=\sum_{n=1}^{\infty} \ln \left( 
	\frac{2\pi (n+1 /2)}{\beta}\right) ^{-2}
.\] 
Экспоненциируя с двух сторон
\[
	\prod_{n=0}^{\infty} \left( \frac{2\pi(n+1 /2)}{\beta} \right) ^{-2}=\left( \frac{\beta}{2\pi} \right) ^{2\zeta(0,\,1 /2)}\mathrm{e} ^{2\zeta'(0,\,1 /2)} 
.\] 
Используя табличные данные $\zeta(0,\,1 /2)=0$,
$\zeta'(0,\,1 /2)=- \frac{1}{2}\ln 2$, заключаем, что
\[
\prod_{n=0}^{\infty} \left( \frac{2\pi(n+1 /2)}{\beta} \right) ^{-2}=\frac{1}{2}
.\] 
И
\[
	\left. \det \left( \frac{\mathrm{d} }{\mathrm{d} \tau}+\omega \right)  \right|_{\text{анти-пер.}}=
		2\ch \left( \frac{\beta\omega}{2} \right) 
.\] 
\end{sol}
\begin{hiProb}[Задача 4]
\end{hiProb}
\begin{sol}
Статсумма будет даваться выражением
\[
Z=\Tr \mathrm{e} ^{-\beta H}= \int
\mathcal{D}x \mathcal{D}\psi \mathcal{D}\psi^\dagger
\ \mathrm{e} ^{-S_E}=
\sum_{X}^{} \frac{\det \left( \mathrm{d} / \mathrm{d} \tau
+h''(X)\right) }{\det ^{1 /2}\left( -\mathrm{d} ^2 / \mathrm{d} \tau^2 +h''(X) \right) }
,\]
где, в отличие от индекса Виттена, граничные условия
для фермионов антипериодичны. Бозонный вклад
\[
	\det ^{1 /2} \left( - \frac{}{} \right) 
.\] 
\end{sol}
\end{document}
