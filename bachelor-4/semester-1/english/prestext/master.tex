\documentclass[a4paper]{article}
% Этот шаблон документа разработан в 2014 году
% Данилом Фёдоровых (danil@fedorovykh.ru) 
% для использования в курсе 
% <<Документы и презентации в \LaTeX>>, записанном НИУ ВШЭ
% для Coursera.org: http://coursera.org/course/latex .
% Исходная версия шаблона --- 
% https://www.writelatex.com/coursera/latex/5.3

% В этом документе преамбула

\usepackage{siunitx}
%%% Работа с русским языком
\usepackage{cmap}					% поиск в PDF
\usepackage{mathtext} 				% русские буквы в формулах
\usepackage[T2A]{fontenc}			% кодировка
\usepackage[utf8]{inputenc}			% кодировка исходного текста
\usepackage[english,russian]{babel}	% локализация и переносы
\usepackage{indentfirst}
\frenchspacing

\renewcommand{\epsilon}{\ensuremath{\varepsilon}}
\renewcommand{\phi}{\ensuremath{\varphi}}
\renewcommand{\kappa}{\ensuremath{\varkappa}}
\renewcommand{\le}{\ensuremath{\leqslant}}
\renewcommand{\leq}{\ensuremath{\leqslant}}
\renewcommand{\ge}{\ensuremath{\geqslant}}
\renewcommand{\geq}{\ensuremath{\geqslant}}
\renewcommand{\emptyset}{\varnothing}
\renewcommand{\Im}{\operatorname{Im}}
\renewcommand{\Re}{\operatorname{Re}}


%%% Дополнительная работа с математикой
\usepackage{amsmath,amsfonts,amssymb,amsthm,mathtools} % AMS
\usepackage{icomma} % "Умная" запятая: $0,2$ --- число, $0, 2$ --- перечисление

%% Номера формул
%\mathtoolsset{showonlyrefs=true} % Показывать номера только у тех формул, на которые есть \eqref{} в тексте.
%\usepackage{leqno} % Нумереация формул слева

%% Свои команды
\DeclareMathOperator{\sgn}{\mathop{sgn}}
\DeclareMathOperator{\sign}{\mathop{sign}}
\DeclareMathOperator*{\res}{\mathop{res}}
\DeclareMathOperator*{\tr}{\mathop{tr}}

%% Перенос знаков в формулах (по Львовскому)
\newcommand*{\hm}[1]{#1\nobreak\discretionary{}
{\hbox{$\mathsurround=0pt #1$}}{}}

%%% Работа с картинками
\usepackage{graphicx}  % Для вставки рисунков
\graphicspath{{figures/}}  % папки с картинками
\setlength\fboxsep{3pt} % Отступ рамки \fbox{} от рисунка
\setlength\fboxrule{1pt} % Толщина линий рамки \fbox{}
\usepackage{wrapfig} % Обтекание рисунков текстом

%%% Работа с таблицами
\usepackage{array,tabularx,tabulary,booktabs} % Дополнительная работа с таблицами
\usepackage{longtable}  % Длинные таблицы
\usepackage{multirow} % Слияние строк в таблице

%%% Теоремы
\theoremstyle{plain} % Это стиль по умолчанию, его можно не переопределять.
\newtheorem{theorem}{Теорема}
\newtheorem*{thm}{Теорема}
\newtheorem{prop}{Утверждение}
 
\theoremstyle{definition} % "Определение"
%\newtheorem{corollary}{Следствие}[theorem]
\newtheorem*{dfn}{Определение}
\newtheorem{problem}{Задача}
\newtheorem*{problem*}{Задача}

 
\theoremstyle{remark} % "Примечание"
\newtheorem*{sol}{Решение}
\newtheorem*{rem}{Замечание}

%%% Программирование
\usepackage{etoolbox} % логические операторы

%%% Страница
%\usepackage{extsizes} % Возможность сделать 14-й шрифт
%\usepackage{geometry} % Простой способ задавать поля
%	\geometry{top=25mm}
%	\geometry{bottom=35mm}
%	\geometry{left=35mm}
%	\geometry{right=20mm}
 
\usepackage{fancyhdr} % Колонтитулы
%	\pagestyle{fancy}
 %	\renewcommand{\headrulewidth}{0pt}  % Толщина линейки, отчеркивающей верхний колонтитул
	%\lfoot{Нижний левый}
	%\rfoot{Нижний правый}
	%\rhead{Верхний правый}
	%\chead{Верхний в центре}
	%\lhead{Верхний левый}
	%\cfoot{Нижний в центре} % По умолчанию здесь номер страницы

\usepackage{setspace} % Интерлиньяж
%\onehalfspacing % Интерлиньяж 1.5
%\doublespacing % Интерлиньяж 2
%\singlespacing % Интерлиньяж 1

\usepackage{lastpage} % Узнать, сколько всего страниц в документе.

\usepackage{soul} % Модификаторы начертания

\usepackage{hyperref}
%\usepackage[usenames,dvipsnames,svgnames,table,rgb]{xcolor}
\hypersetup{				% Гиперссылки
    unicode=true,           % русские буквы в раздела PDF
    pdftitle={Заголовок},   % Заголовок
    pdfauthor={Автор},      % Автор
    pdfsubject={Тема},      % Тема
    pdfcreator={Создатель}, % Создатель
    pdfproducer={Производитель}, % Производитель
    pdfkeywords={keyword1} {key2} {key3}, % Ключевые слова
    colorlinks=true,       	% false: ссылки в рамках; true: цветные ссылки
    linkcolor=red,          % внутренние ссылки
    citecolor=black,        % на библиографию
    filecolor=magenta,      % на файлы
    urlcolor=cyan           % на URL
}

\usepackage{csquotes} % Еще инструменты для ссылок

%\usepackage[style=apa,maxcitenames=2,backend=biber,sorting=nty]{biblatex}

\usepackage{multicol} % Несколько колонок

\usepackage{tikz} % Работа с графикой
\usepackage{pgfplots}
\usepackage{pgfplotstable}
%\usepackage{coloremoji}
\usepackage{floatrow}
\usepackage{subcaption}
\newcommand*{\N}{\mathbb{N}}
\newcommand*{\R}{\mathbb{R}}
\newcommand*{\K}{\mathbb{K}}
\newcommand*{\V}{\mathcal{V}}
\newcommand*{\A}{\mathcal{A}}
\newcommand*{\ii}{\mathbf{1}}
\newcommand*{\oo}{\mathbf{0}}
\newcommand*{\ba}{\mathbf{a}}
\newcommand*{\bb}{\mathbf{b}}
\newcommand*{\Q}{\mathbb{Q}}
\graphicspath{{figures/}}
%\usepackage{breqn}

\renewcommand\thesubfigure{\asbuk{subfigure}}
%\addbibresource{master.bib}

\usepackage{import}
\usepackage{pdfpages}
\usepackage{transparent}
\usepackage{xcolor}
\usepackage{xifthen}

%\newcommand{\incfig}[1]{%
%    \def\svgwidth{\columnwidth}
%    \import{./figures/}{#1.pdf_tex}
%}


\newcommand{\incfig}[2][1]{%
    \def\svgwidth{#1\columnwidth}
    \import{./figures/}{#2.pdf_tex}
}
\usepackage{titlesec}
%\titleformat{\section}{\normalfont\Large\bfseries}{}{0pt}{}
%----------------------STANDART:
%\titleformat{\chapter}[display]
%  {\normalfont\huge\bfseries}{\chaptertitlename\ \thechapter}{20pt}{\Huge}
%\titleformat{\section}{\normalfont\Large\bfseries}{\thesection}{1em}{}
%\titleformat{\subsection}
%  {\normalfont\large\bfseries}{\thesubsection}{1em}{}
%\titleformat{\subsubsection}
%  {\normalfont\normalsize\bfseries}{\thesubsubsection}{1em}{}
%\titleformat{\paragraph}[runin]
%  {\normalfont\normalsize\bfseries}{\theparagraph}{1em}{}
%\titleformat{\subparagraph}[runin]
%  {\normalfont\normalsize\bfseries}{\thesubparagraph}{1em}{}

\pdfsuppresswarningpagegroup=1
\pgfplotsset{compat=1.16}

\usepackage{xifthen}
\makeatother
%\def\@lecture{}%
%\newcommand{\lecture}[3]{
%    \ifthenelse{\isempty{#3}}{%
%        \def\@lecture{Неделя #1}%
%    }{%
%        \def\@lecture{Неделя #1: #3}%
%    }%
%    \section*{\@lecture}
%    \marginpar{\small\textsf{\mbox{#2}}}
%}
\makeatletter

%\newcommand{\lec}{\subsection{Лекция}}
%\newcommand{\sem}{\subsection{Семинар}}
%\newcommand{\hw}{\subsection{Домашняя работа}}
%\newcommand{\prob}[1]{\textbf{#1}}
%\renewcommand{\thesubsection}{}
%\renewcommand{\thesubsubsection}{}

%\setcounter{tocdepth}{1} % only parts,chapters,sections
%\titleformat{\subsection}{\normalfont\large\bfseries}{}{0em}{}
%\titleformat{\subsubsection}{\normalfont\normalsize\bfseries}{}{0em}{}

%\newcommand{\textover}[2]{\stackrel{\mathclap{\normalfont\mbox{#2}}}{#1}}

\author{Драчов Ярослав\\
Факультет общей и прикладной физики МФТИ}
\newcommand{\veq}{\mathrel{\rotatebox{90}{$=$}}}
%\newcommand{\teto}[1]{\stackrel{\mathclap{\normalfont\tiny\mbox{#1}}}{\to}}
%\renewcommand{\thesubsection}{\arabic{subsection}}

%%\setcounter{secnumdepth}{0}

\definecolor{tabblue}{RGB}{30, 119, 180}
\definecolor{taborange}{RGB}{255, 127, 15}
\definecolor{tabgreen}{RGB}{45, 160, 43}
\definecolor{tabred}{RGB}{214, 38, 40}
\definecolor{tabpurple}{RGB}{148, 103, 189}
\definecolor{tabbrown}{RGB}{140, 86, 76}
\definecolor{tabpink}{RGB}{227, 119, 193}
\definecolor{tabgray}{RGB}{127, 127, 127}
\definecolor{tabolive}{RGB}{188, 189, 33}
\definecolor{tabcyan}{RGB}{22, 190, 207}
\pgfplotscreateplotcyclelist{colorbrewer-tab}{
{tabblue},
{taborange},
{tabgreen},
{tabred},
{tabpurple},
{tabbrown},
{tabpink},
{tabgray},
{tabolive},
{tabcyan},
}
\usepackage{csvsimple}
\usepackage{extarrows}
%\renewcommand{\labelenumii}{\asbuk{enumii})}
%\renewcommand{\labelenumiv}{\Asbuk{enumiv}}
\newcommand{\prob}[1]{\subsubsection*{#1}}
\sisetup{output-decimal-marker = {,},separate-uncertainty = true,exponent-product = \cdot}

\usepackage{braket}
\usepackage{enumerate}
\usepackage{chngcntr}
%\counterwithin*{equation}{problem}
%\usepackage{bbold}

\newtheoremstyle{hiProb}% ⟨name ⟩ 
{3pt}% ⟨Space above ⟩1 
{3pt}% ⟨Space below ⟩1
{}% ⟨Body font ⟩
{}% ⟨Indent amount ⟩2
{\bfseries}% ⟨Theorem head font⟩
{.}% ⟨Punctuation after theorem head ⟩
{.5em}% ⟨Space after theorem head ⟩3
%{\thmname{#1} \thmnote{#3}}% ⟨Theorem head spec (can be left empty, meaning ‘normal’)⟩
{\thmnote{#3}}% ⟨Theorem head spec (can be left empty, meaning ‘normal’)⟩
\theoremstyle{hiProb} % "Определение"
%\newtheorem{hiProb}{Задача}
\newtheorem{hiProb}{}
\usepackage{mmacells}
\newcommand{\textover}[2]{\stackrel{\mathclap{\normalfont\scriptsize\mbox{#2}}}{#1}}
\usepackage{units}
\usepackage[math]{cellspace}%
\setlength\cellspacetoplimit{2pt}
\setlength\cellspacebottomlimit{2pt}

\renewcommand{\emph}{\textbf}
\title{Talking to My Daughter Can Be Harder Than Learning Quantum Mechanics}
\begin{document}
\begin{titlepage}
   \begin{center}
	   Moscow Institute of Physics and Technology\\
	   (State University)
       \vspace*{3cm}

       \textbf{Exam presentation at the
       department of foreign languages}

       \vspace{0.5cm}

       HUMAN DILEMMAS AND COMPLEXITY
        
       \vspace{0.5cm}
        
        based on the original article\\
	``Talking to my daughter can be harder than learning quantum mechanics''
            
       \vspace{0.5cm}

       by John Horgan\\
       \url{https://www.scientificamerican.com/article/talking-to-my-daughter-can-be-harder-than-learning-quantum-mechanics}

       \vfill
            
     
       \begin{flushright}
       Prepared by:\\
       Yaroslav Drachov\\
       4\textsuperscript{th} year student, 829 group
       \end{flushright} 
       
       \vspace{3cm}
        
       Moscow, 2021
   \end{center}
\end{titlepage}

\section*{Original article}
As a boy, I was a rock hound, and I learned how to identify minerals with the Mohs hardness test, named after the mineralogist who invented it. You take a known specimen, like quartz, and scratch an unknown specimen with it. If the quartz scratches the mystery specimen, you know it’s softer than quartz. It could be calcite or pyrite. If the quartz can’t scratch the specimen, it might be beryl or corundum, which are harder than quartz. Along with factors like color and crystalline structure, the hardness test can help you specify your specimen.

I loved the straightforward objectivity of the Mohs test. Recently, I’ve been brooding over a hardness --- call it cognitive hardness --- that is much harder to evaluate. Over the course of our lives, we face an enormous variety of cognitively hard tasks. For the past year, for example, I’ve been studying quantum mechanics, which is notoriously difficult to grasp. But is learning quantum mechanics harder, objectively, than chatting to my girlfriend about \#MeToo without irritating her? Or talking to my daughter about climate change without depressing her?

Subjective assessments of cognitive hardness aren’t much help, because they vary with each person’s experience and aptitude. You’re a whiz with differential equations, I’m better at riffing on Emily Dickinson’s poems. Is there a method, analogous to the Mohs test, for quantifying and hence ranking the cognitive hardness of various tasks? Such a method, perhaps, could yield insights that help us solve hard problems, or, conversely, accept their insolubility. At any rate, here are a few thoughts on cognitive hardness.
\subsection*{The traveling salesman’s problems}
Mathematicians and computer scientists rank problems by how long it would take a computer to find a solution. Problems are defined as NP-hard if there is no algorithmic shortcut to the best solution; you must laboriously check every possible solution to find the best one. (NP stands for ``nondeterministic polynomial time,'' which I’ve always thought of as meaning ``really, really.'')

[Post-publication note: Computer scientist Scott Aaronson objects to my description of NP-Hardness and defines it as follows: “NP is the class of problems for which ‘yes’ answers can be efficiently verified given a proof or witness. NP-hard is the class of problems where, if you had a magic box for solving them, it would let you solve every NP problem in polynomial time. Many problems are exponentially hard without being NP-hard. And if P=NP, there would be NP-hard problems (including traveling salesman) that weren’t ‘hard’ in the plain English sense. Other problems, like perfectly playing a generalization of chess to N*N boards, are known to be exponentially hard unconditionally.”]

One famous NP-hard problem involves a traveling salesman seeking the shortest route between many cities. The hardness of the problem balloons dramatically with the number of cities. If the salesman has to visit 15 cities, he has 87 billion possible routes to consider. Mathematicians have devised tricks for finding pretty short routes --- if not the shortest --- between many cities. But when the number of cities rises into the thousands, the world’s fastest computer would take virtually forever to find the shortest route.

Ironically, coming up with a time-saving itinerary is easy compared to other problems that the traveling salesman might face. For example: How long can he be on the road without endangering his marriage? If he is lonely, should he approach a woman in the hotel bar? If he feels bad about cheating on his wife, what should he tell himself to relieve his guilt?

What makes these problems especially hard is their moral dimension. Like most of us, the traveling salesman wants to believe he is a good person, but what does that even mean? In 2016 I attended a conference that explored whether artificial intelligence can solve ethical dilemmas. There were lots of droll variations on the trolley problem. For example, would you destroy a living thing, like a sparrow, to save a nonliving thing, like the Grand Canyon?

But philosophers have been arguing about morality for millennia without agreeing on what our moral rules should be. The famous play Death of a Salesman explores the moral dilemmas of a traveling salesman. Like most works of literature, Death of a Salesman does not solve moral problems; it rubs our faces in them.
\subsection*{Does hardness equal complexity?}
So-called complexity researchers equate hardness with complexity. Let’s say you are a scientist trying to model and hence explain some complicated phenomenon, like the propagation of gravitational waves from colliding blacks holes, or the spread of disinformation on social media. The hardness of your scientific problem, researchers suggest, is proportional to the complexity of the phenomenon you want to understand.

Moreover, dissimilar things might be complex, and hence hard to explain, for similar reasons. Ideally, modeling one hard phenomenon will yield insights that apply to very different ones. A better model of black holes might lead to a better model and deeper understanding of QAnon. Or so researchers hope.

Unfortunately, researchers cannot agree on a definition of complexity, which is crucial to their enterprise. Physicist Seth Lloyd has listed dozens of proposed definitions of complexity, based on information theory, thermodynamics, fractals and other measures. There are many definitions because none really suffices. I suspect that, just as unhappy families are unhappy in their own ways, different hard problems are hard for different reasons.

Some physicists insist that everything, including humanity, is ultimately explicable in terms of particles pushed and pulled by gravity, electromagnetism and other forces. Sabine Hossenfelder takes this position in a recent conversation with me. But physics has nothing to say about morality, meaning, emotions, choices and other significant features of human existence.
\subsection*{The hardness of math, english and fatherhood}
When I am struggling to understand the mathematical rules underpinning quantum mechanics, they often seem irritatingly arcane and arbitrary. Actually, the rules of calculus and linear algebra are quite sensible compared to the “rules” of ordinary language. To master English, you must first learn the letters of alphabet. Letters only acquire meaning when combined into words, and there are thousands of words, many of which have multiple meanings. Consider all the meanings of ``hard.''

Then you have all the rules for combining words into sentences, rules that are routinely bent and broken. The meaning of a sentence depends, again, according to rules that are hard to spell out precisely, on the context in which it is uttered and heard. Linguist Noam Chomsky has convinced most scientists that we have an innate talent for language, inherited from our ancestors; that’s why we learn language so quickly.

Sometimes, when I am engaged in conversation, my language instinct kicks in, and I chat with relative ease. I am displaying what philosopher Daniel Dennett calls ``competence without comprehension.'' Other times, I struggle to decipher the words of the person speaking to me, and I am overwhelmed by all the possible ways in which I can respond. This often happens when I am speaking to my daughter or son.

Becoming a father is, in the strict, biological sense, easy. Almost any idiot can do it. But what does it mean to be a good father? The answer varies across eras and cultures. My son and daughter are 28 and 26, and I’m still baffled by fatherhood. Almost every time I see my kids or talk to them over the phone, I second-guess myself afterward. Did I share too much? Not enough?

You can assess parents by looking at how their kids have done. But I know good parents (caring, well-intentioned) whose kids have died of drug overdoses, and I know bad parents (self-absorbed to the point of negligence) whose kids have thrived. These brutal facts have an upside: If your kids don’t turn out well, you can always blame bad luck. My larger point: Unlike the Schrödinger equation, the puzzles of parenting—and of all human relations—have no clear-cut solutions.
\subsection*{Race, gender and brain chips}
Cognitive scientists propose that we have an innate ability to intuit what others are thinking and feeling. This talent is called, confusingly, ``theory of mind.'' It is crucial for social success, that is, for getting what we want from others. It is also crucial for morality. We are more likely to feel compassion for others, and to treat them well, if we can empathize with them. But our theory-of-mind program can only take us so far.

Last year, I joined a Black Lives Matter march that passed through my hometown. Some white protestors carried signs that said, ``I understand that I will never understand. But I stand.'' The sign implies that white people like me cannot understand what it is like being Black in America. That task is too hard. If we say we understand, that means we don’t; we’re revealing our ignorance and arrogance. But we can still express support for Black Americans.

This situation applies to gender, too. I recently got into an argument with my girlfriend about one of the most famous passages in literature, Molly Bloom’s soliloquy, which concludes James Joyce’s novel Ulysses. I love this profane, sexy, poetic masterpiece within a masterpiece, in which Joyce imagines what it feels like to be a married woman and mother living in early 20th-century Dublin. My girlfriend hates the soliloquy, which she says is a male fantasy about what women think. Instead of arguing with my girlfriend, I should have just said, ``I understand that I will never understand. But I stand.''

Visionaries like Elon Musk hope that someday computer chips implanted in our brains will help us solve hard problems. That is why Musk founded Neuralink, which is building ``high-bandwidth brain-machine interfaces.'' The chips will link our brains to the internet and to powerful problem-solving programs, like Wolfram Alpha but much better.

I doubt that brain chips will help us with the problems that matter most. A brain chip might help the traveling salesman plan his itinerary, but it won’t tell him how to be a good husband and father, or how to avoid acting like a sexist or racist. It won’t tell him how to grab a little happiness without being a jerk. These problems are much harder than the hardest NP-hard problem.

While writing this column, I began to remember why, as a boy, I fantasized about becoming a mineralogist. I was already getting intimations of adulthood, and it didn’t appeal to me. Many adults seemed sad, or mean, or both. When I grow up, I thought, I will spend my days in a laboratory, alone, performing the Mohs test on crystalline specimens, testing their chemical reactivity, examining them through a microscope, admiring their perfect, symmetrical, inhuman beauty.
\section*{Script of speech}
	\noindent \textbf{Good afternoon, everyone! My name} is Yaroslav Drachov and \textbf{I’m a fourth year student at MIPT.}

\textbf{The reason we are here today} is to discuss the article from the magazine ``Scientific American'': ``Talking to My Daughter Can Be Harder Than Learning Quantum Mechanics''.\textbf{ I have chosen this topic because}
everyday we face many different problems and this article is just about the measure of the complexity of problems.
For many computational and Computer Science problems it
is very important to understand how much time
it will take to solve this problem on a given device and
people often can deal with it. However, describing the complexity 
of our daily problems can be much more sophisticated.

\emph{I'm going to develop three main points. First,}
the traveling salesman’s problems. \emph{Second, }hardness
and complexity. \emph{Third,} the hardness of Math, English and
fatherhood. And, \emph{finally}  we will discuss (surprise) race, gender and brain chips.

\emph{The presentation should last about seven minutes. If you have any questions, I’d be grateful if you could leave them until the end.}

\emph{Let me start.}

Over the course of our lives, we face an enormous variety of cognitively hard tasks. For the past year, for example, I’ve been studying quantum mechanics, which is notoriously difficult to grasp. But is learning quantum mechanics harder, objectively, than chatting to my girlfriend about \#MeToo without irritating her?
Or for a father to talk to his daughter about climate change without depressing her?

Subjective assessments of cognitive hardness aren’t much help, because they vary with each person’s experience and aptitude. You’re a whiz with differential equations, I’m better at riffing on Emily Dickinson’s poems. Is there a method for quantifying and hence ranking the cognitive hardness of various tasks? Such a method, perhaps, could yield insights that help us solve hard problems, or, conversely, accept their insolubility. At any rate, here are a few thoughts on cognitive hardness.

\emph{So, first,} the traveling salesman’s problems.
Mathematicians and computer scientists rank problems by how long it would take a computer to find a solution. Problems are defined as NP-hard if there is no algorithmic shortcut to the best solution; you must laboriously check every possible solution to find the best one. (NP stands for ``nondeterministic polynomial time.'')

One famous NP-hard problem involves a traveling salesman seeking the shortest route between cities. \emph{Despite} the simplicity of the problem statement, the hardness of the problem balloons dramatically with the number of cities. If the salesman has to visit 15 cities, he has 87 billion possible routes to consider. 
%Mathematicians have devised tricks for finding pretty short routes—if not the shortest—between many cities. But when the number of cities rises into the thousands, the world’s fastest computer would take virtually forever to find the shortest route.

Ironically, coming up with a time-saving itinerary is easy compared to other problems that the traveling salesman might face. For example: How long can he be on the road without endangering his marriage? If he is lonely, should he approach a woman in the hotel bar? If he feels bad about cheating on his wife, what should he tell himself to relieve his guilt?

What makes these problems especially hard is their moral dimension. Like most of us, the traveling salesman wants to believe he is a good person, but what does that even mean? \emph{Nevertheless}, philosophers have been arguing about morality for millennia without agreeing on what our moral rules should be.

\emph{Now, I’d like to move on to the next part of my presentation, which is about hardness and complexity.}
So-called complexity researchers equate hardness with complexity. Let’s say you are a scientist trying to model and hence explain some complicated phenomenon, like the propagation of gravitational waves from colliding blacks holes, or the spread of disinformation on social media. The hardness of your scientific problem, researchers suggest, is proportional to the complexity of the phenomenon you want to understand. \emph{Although} the definition of complexity is extremely important for researchers, there are dozens of different ones, based on information theory, thermodynamics, fractals and other measures because none really suffices.

\emph{Let's move on to} the hardness of Math, English and fatherhood. When I am struggling to understand the mathematical rules underpinning quantum mechanics, they often seem irritatingly arcane and arbitrary. Actually, the rules of calculus and linear algebra are quite sensible compared to the “rules” of ordinary language.
Scientists say that we have an innate talent for language, inherited from our ancestors. \emph{That’s why} we learn language so quickly.
\emph{Also,} becoming a father is, in the strict, biological sense, easy. Almost any idiot can do it. But what does it mean to be a good father? The answer varies across eras and cultures.

\emph{We have come to the final part of my talk about the race, gender and brain chips.}
Visionaries like Elon Musk hope that someday computer chips implanted in our brains will help us solve hard problems. That is why Musk founded Neuralink, which is building “high-bandwidth brain-machine interfaces.” The chips will link our brains to the internet and to powerful problem-solving programs, like Wolfram Alpha but much better.

I doubt that brain chips will help us with the problems that matter most. \emph{Although} a brain chip might help the traveling salesman plan his itinerary, it won’t tell him how to be a good husband and father, or how to avoid acting like a sexist or racist. It won’t tell him how to grab a little happiness without being a jerk. 

\emph{That brings me to the end of my talk. Let me go over the key points again. I have told you} about the traveling
salesman's problems, \emph{we have compared} the
hardness and complexity, and,\emph{ lastly} we have
discussed Math, English, fatherhood and brain chips.

\emph{To conclude I’d like to leave you with the following thought}.
The famous play ``Death of a Salesman'' explores the moral dilemmas of a traveling salesman. Like most works of literature, ``Death of a Salesman'' does not solve moral problems; it rubs our faces in them. These problems can be much harder than the hardest NP-hard problem.

\emph{Thank you for your attention. If you have any questions, I’ll be happy to answer them. }
\end{document}
