\documentclass[a4paper, 14pt]{extarticle}
\usepackage{fullpage}
% Этот шаблон документа разработан в 2014 году
% Данилом Фёдоровых (danil@fedorovykh.ru) 
% для использования в курсе 
% <<Документы и презентации в \LaTeX>>, записанном НИУ ВШЭ
% для Coursera.org: http://coursera.org/course/latex .
% Исходная версия шаблона --- 
% https://www.writelatex.com/coursera/latex/5.3

% В этом документе преамбула

\usepackage{siunitx}
%%% Работа с русским языком
%\usepackage{cmap}					% поиск в PDF
%\usepackage{mathtext} 				% русские буквы в формулах
%\usepackage[T2A]{fontenc}			% кодировка
%\usepackage[utf8]{inputenc}			% кодировка исходного текста
%\usepackage[english,russian]{babel}	% локализация и переносы
%\usepackage{indentfirst}
%\frenchspacing
%
%\renewcommand{\epsilon}{\ensuremath{\varepsilon}}
%\newcommand{\phibackup}{\ensuremath{\phi}}
%\renewcommand{\phi}{\ensuremath{\varphi}}
%\renewcommand{\varphi}{\ensuremath{\phibackup}}
%\renewcommand{\kappa}{\ensuremath{\varkappa}}
%\renewcommand{\le}{\ensuremath{\leqslant}}
%\renewcommand{\leq}{\ensuremath{\leqslant}}
%\renewcommand{\ge}{\ensuremath{\geqslant}}
%\renewcommand{\geq}{\ensuremath{\geqslant}}
%\renewcommand{\emptyset}{\varnothing}
%\renewcommand{\Im}{\operatorname{Im}}
%\renewcommand{\Re}{\operatorname{Re}}


%%% Дополнительная работа с математикой
\usepackage{amsmath,amsfonts,amssymb,amsthm,mathtools} % AMS
%\usepackage{icomma} % "Умная" запятая: $0,2$ --- число, $0, 2$ --- перечисление

%% Номера формул
%\mathtoolsset{showonlyrefs=true} % Показывать номера только у тех формул, на которые есть \eqref{} в тексте.
%\usepackage{leqno} % Нумереация формул слева

%% Свои команды
\DeclareMathOperator{\sgn}{\mathop{sgn}}
\DeclareMathOperator{\sign}{\mathop{sign}}
\DeclareMathOperator*{\res}{\mathop{res}}
\DeclareMathOperator*{\tr}{\mathop{tr}}
\DeclareMathOperator*{\rot}{\mathop{rot}}
\DeclareMathOperator*{\divop}{\mathop{div}}
\DeclareMathOperator*{\grad}{\mathop{grad}}

%% Перенос знаков в формулах (по Львовскому)
\newcommand*{\hm}[1]{#1\nobreak\discretionary{}
{\hbox{$\mathsurround=0pt #1$}}{}}

%%% Работа с картинками
\usepackage{graphicx}  % Для вставки рисунков
\graphicspath{{figures/}}  % папки с картинками
\setlength\fboxsep{3pt} % Отступ рамки \fbox{} от рисунка
\setlength\fboxrule{1pt} % Толщина линий рамки \fbox{}
\usepackage{wrapfig} % Обтекание рисунков текстом

%%% Работа с таблицами
\usepackage{array,tabularx,tabulary,booktabs} % Дополнительная работа с таблицами
\usepackage{longtable}  % Длинные таблицы
\usepackage{multirow} % Слияние строк в таблице

%%% Теоремы
\theoremstyle{plain} % Это стиль по умолчанию, его можно не переопределять.
\newtheorem{thm}{Теорема}
\newtheorem*{thm*}{Теорема}
\newtheorem{prop}{Предложение}
\newtheorem*{prop*}{Предложение}
 
\theoremstyle{definition} % "Определение"
%\newtheorem{corollary}{Следствие}[theorem]
\newtheorem{dfn}{Определение}
\newtheorem*{dfn*}{Определение}
\newtheorem{prob}{Задача}
\newtheorem*{prob*}{Задача}

 
\theoremstyle{remark} % "Примечание"
\newtheorem*{sol}{Решение}
\newtheorem*{rem}{Замечание}

%%% Программирование
\usepackage{etoolbox} % логические операторы

%%% Страница
%\usepackage{extsizes} % Возможность сделать 14-й шрифт
%\usepackage{geometry} % Простой способ задавать поля
%	\geometry{top=25mm}
%	\geometry{bottom=35mm}
%	\geometry{left=35mm}
%	\geometry{right=20mm}
 
\usepackage{fancyhdr} % Колонтитулы
%	\pagestyle{fancy}
 %	\renewcommand{\headrulewidth}{0pt}  % Толщина линейки, отчеркивающей верхний колонтитул
	%\lfoot{Нижний левый}
	%\rfoot{Нижний правый}
	%\rhead{Верхний правый}
	%\chead{Верхний в центре}
	%\lhead{Верхний левый}
	%\cfoot{Нижний в центре} % По умолчанию здесь номер страницы

\usepackage{setspace} % Интерлиньяж
%\onehalfspacing % Интерлиньяж 1.5
%\doublespacing % Интерлиньяж 2
%\singlespacing % Интерлиньяж 1

\usepackage{lastpage} % Узнать, сколько всего страниц в документе.

\usepackage{soul} % Модификаторы начертания

\usepackage{hyperref}
\usepackage[usenames,dvipsnames,svgnames,table,rgb]{xcolor}
\hypersetup{				% Гиперссылки
    unicode=true,           % русские буквы в раздела PDF
    pdftitle={Заголовок},   % Заголовок
    pdfauthor={Автор},      % Автор
    pdfsubject={Тема},      % Тема
    pdfcreator={Создатель}, % Создатель
    pdfproducer={Производитель}, % Производитель
    pdfkeywords={keyword1} {key2} {key3}, % Ключевые слова
%    colorlinks=true,       	% false: ссылки в рамках; true: цветные ссылки
    %linkcolor=red,          % внутренние ссылки
    %citecolor=black,        % на библиографию
    %filecolor=magenta,      % на файлы
    %urlcolor=cyan           % на URL
}

\usepackage{csquotes} % Еще инструменты для ссылок

%\usepackage[style=apa,maxcitenames=2,backend=biber,sorting=nty]{biblatex}

\usepackage{multicol} % Несколько колонок

\usepackage{tikz} % Работа с графикой
\usepackage{pgfplots}
\usepackage{pgfplotstable}
%\usepackage{coloremoji}
\usepackage{floatrow}
\usepackage{subcaption}
\graphicspath{{figures/}}

\renewcommand\thesubfigure{\asbuk{subfigure}}
%\addbibresource{master.bib}

\usepackage{import}
\usepackage{pdfpages}
\usepackage{transparent}
\usepackage{xcolor}
\usepackage{xifthen}

\newcommand{\incfig}[2][1]{%
    \def\svgwidth{#1\columnwidth}
    \import{./figures/}{#2.pdf_tex}
}
%\usepackage{titlesec}
%\titleformat{\section}{\normalfont\Large\bfseries}{}{0pt}{}
%----------------------STANDART:
%\titleformat{\chapter}[display]
%  {\normalfont\huge\bfseries}{\chaptertitlename\ \thechapter}{20pt}{\Huge}
%\titleformat{\section}{\normalfont\Large\bfseries}{\thesection}{1em}{}
%\titleformat{\subsection}
%  {\normalfont\large\bfseries}{\thesubsection}{1em}{}
%\titleformat{\subsubsection}
%  {\normalfont\normalsize\bfseries}{\thesubsubsection}{1em}{}
%\titleformat{\paragraph}[runin]
%  {\normalfont\normalsize\bfseries}{\theparagraph}{1em}{}
%\titleformat{\subparagraph}[runin]
%  {\normalfont\normalsize\bfseries}{\thesubparagraph}{1em}{}

\pdfsuppresswarningpagegroup=1
\pgfplotsset{compat=1.16}



%\setcounter{tocdepth}{1} % only parts,chapters,sections
%\titleformat{\subsection}{\normalfont\large\bfseries}{}{0em}{}
%\titleformat{\subsubsection}{\normalfont\normalsize\bfseries}{}{0em}{}

%\newcommand{\textover}[2]{\stackrel{\mathclap{\normalfont\mbox{#2}}}{#1}}

\author{Yaroslav Drachov\\
Moscow Institute of Physics and Technology}
%\author{Драчов Ярослав\\
%Факультет общей и прикладной физики МФТИ}
\newcommand{\veq}{\mathrel{\rotatebox{90}{$=$}}}
%\newcommand{\teto}[1]{\stackrel{\mathclap{\normalfont\tiny\mbox{#1}}}{\to}}
%\renewcommand{\thesubsection}{\arabic{subsection}}

%%\setcounter{secnumdepth}{0}

\definecolor{tabblue}{RGB}{30, 119, 180}
\definecolor{taborange}{RGB}{255, 127, 15}
\definecolor{tabgreen}{RGB}{45, 160, 43}
\definecolor{tabred}{RGB}{214, 38, 40}
\definecolor{tabpurple}{RGB}{148, 103, 189}
\definecolor{tabbrown}{RGB}{140, 86, 76}
\definecolor{tabpink}{RGB}{227, 119, 193}
\definecolor{tabgray}{RGB}{127, 127, 127}
\definecolor{tabolive}{RGB}{188, 189, 33}
\definecolor{tabcyan}{RGB}{22, 190, 207}
\pgfplotscreateplotcyclelist{colorbrewer-tab}{
{tabblue},
{taborange},
{tabgreen},
{tabred},
{tabpurple},
{tabbrown},
{tabpink},
{tabgray},
{tabolive},
{tabcyan},
}
\usepackage{csvsimple}
\usepackage{extarrows}
%\renewcommand{\labelenumii}{\asbuk{enumii})}
%\renewcommand{\labelenumiv}{\Asbuk{enumiv}}
%\newcommand{\prob}[1]{\subsubsection*{#1}}
\sisetup{output-decimal-marker = {,},separate-uncertainty = true,exponent-product = \cdot}

\usepackage{braket}
\usepackage{enumerate}
\usepackage{chngcntr}
%\counterwithin*{equation}{problem}
%\usepackage{bbold}

\newtheoremstyle{hiProb}% ⟨name ⟩ 
{3pt}% ⟨Space above ⟩1 
{3pt}% ⟨Space below ⟩1
{}% ⟨Body font ⟩
{}% ⟨Indent amount ⟩2
{\bfseries}% ⟨Theorem head font⟩
{.}% ⟨Punctuation after theorem head ⟩
{.5em}% ⟨Space after theorem head ⟩3
%{\thmname{#1} \thmnote{#3}}% ⟨Theorem head spec (can be left empty, meaning ‘normal’)⟩
{\thmnote{#3}}% ⟨Theorem head spec (can be left empty, meaning ‘normal’)⟩
\theoremstyle{hiProb} % "Определение"
%\newtheorem{hiProb}{Задача}
\newtheorem{hiProb}{}
%\usepackage{mmacells}
\newcommand{\textover}[2]{\stackrel{\mathclap{\normalfont\scriptsize\mbox{#2}}}{#1}}
\usepackage{units}
\usepackage[math]{cellspace}%
\setlength\cellspacetoplimit{2pt}
\setlength\cellspacebottomlimit{2pt}

\DeclareMathAlphabet{\mathbbold}{U}{bbold}{m}{n}

\newcommand{\normord}[1]{:\mathrel{#1}:}

\renewcommand{\emph}{\textbf}
\title{Talking to My Daughter Can Be Harder Than Learning Quantum Mechanics}
\begin{document}
\begin{titlepage}
   \begin{center}
	   Moscow Institute of Physics and Technology\\
	   (State University)
       \vspace*{3cm}

       \textbf{Exam presentation at the
       department of foreign languages}

       \vspace{0.5cm}

       WHICH FACE MASK TO CHOOSE TODAY
        
       \vspace{0.5cm}
        
        based on the original article\\
	``Why we need to upgrade our face masks --- and where to get them''
            
       \vspace{0.5cm}

       by Tanya Lewis\\
       \url{https://www.scientificamerican.com/article/why-we-need-to-upgrade-our-face-masks-and-where-to-get-them/}

       \vfill
            
     
       \begin{flushright}
       Prepared by:\\
       Yaroslav Drachov\\
       4\textsuperscript{th} year student, 829 group
       \end{flushright} 
       
       \vspace{3cm}
        
       Moscow, 2021
   \end{center}
\end{titlepage}

\section*{Original article}
A wealth of evidence has shown that wearing a face mask helps prevent people from spreading the virus that causes COVID, SARS-CoV-2, to others and from becoming sick themselves. But there has been less guidance from public health officials on what kind of masks provide the best protection.

Early on in the pandemic, the U.S. Centers for Disease Control and Prevention and the World Health Organization told the public not to wear N95 respirators, a type of mask that is made from high-tech synthetic fibers and provides a high level of protection against virus-laden airborne particles called aerosols. That was because there was then a shortage of such masks—and health care workers desperately needed them. At the same time, both agencies said there was little risk of aerosol transmission of SARS-CoV-2. They recommended cloth masks or other homemade face coverings that can stop some relatively large virus-carrying droplets even as it became clear that SARS-CoV-2 commonly spreads through aerosols—and as the supply of better-quality masks increased.

There is now a cornucopia of high-filtration respirator-style masks on the market, including N95s, Chinese-made KN95s and South Korean–made KF94s. They have been widely available and relatively affordable for months and provide better protection than cloth or surgical masks. Yet it was not until September 10 that the CDC finally updated its guidance to say the general public could wear N95s and other medical-grade masks now that they are in sufficient supply.

Still, however, the “CDC continues to recommend that N95 respirators should be prioritized for protection against COVID-19 in healthcare settings,” wrote CDC spokesperson Jade Fulce in an e-mail to Scientific American last week. “Essential workers and workers who routinely wore respirators before the pandemic should continue wearing N95 respirators,” she continued. “As N95s become more available they can be worn in non-healthcare settings, however, cloth masks are an acceptable and recommended option for masking.”

The agency announced in May that supplies of approved respirator masks had “increased significantly.” When asked why it only updated it guidance on N95 use by the public in September, Fulce replied that the “CDC regularly reviews and updates its guidance as more information becomes available.”
Scientific American spoke with several experts on aerosol transmission—some of whom have tested various masks available on the market—and they agree that health authorities should strongly recommend people wear well-fitted, high-filtration masks.

“A year ago we could say that we were concerned about shortages for health care workers, so we were telling people to make your cloth mask, and any mask is better than no mask,” says Linsey Marr, an environmental engineer and aerosol science expert at Virginia Tech. But given what scientists know now—especially with the virus’s highly transmissible Delta variant spreading and people spending more time indoors in schools, for example—“I think the CDC should be recommending high-performance masks for everyone when they’re in these risky indoor situations,” she says.

\subsection*{What makes a good mask?}
When it comes to mask effectiveness, the most important parameters are filtration, fit and comfort. Filtration generally refers to the percentage of particles the mask material blocks. For example, an N95 filters at least 95 percent of airborne particles. But that does little good if gaps around the mask let air in freely. A well-fitted mask should sit snugly against the face and over the chin, with no gaps around the nose or mouth. Comfort is also an extremely important metric: a mask does no good if people simply find it intolerable to wear.

A good mask is “the most important defense we have” against COVID, says aerosol expert Kimberly Prather, an atmospheric chemist at the University of California, San Diego.

There are a number of national standards for respirator quality. The U.S. gold standard, N95s, are certified by the CDC’s National Institute for Occupational Safety and Health (NIOSH). And the Occupational Safety and Health Administration (OSHA) sets standards for how they have to fit people in work settings (such as in hospitals). But there is no official standard for N95 use by the general public. The European equivalent of the N95 is the FFP2 respirator, which filters at least 94 percent of particles. China has the KN95, and South Korea has the KF94. All provide excellent filtration, so it really comes down to which fits an individual best and is most comfortable.

\subsection*{Which masks are best?}
In the absence of more specific guidance from health authorities such as the CDC as to which brands of respirators and other masks provide the best protection, some skilled amateurs have  stepped in to fill the gap. Aaron Collins, aka “Mask Nerd,” is a mechanical engineer at Seagate Technology with a background in aerosol science. In his free time, he makes YouTube videos in which he tests and reviews high-filtration masks made by various manufacturers. Collins says he does not earn any money from mask manufacturers or his videos themselves—he considers them a service and wants them to be objective.

Collins has a mask-testing setup in his bathroom, where he assesses masks’ filtration efficiency by generating aerosols of sodium chloride (salt). He then uses a condensation particle counter—a device that measures the concentration of particles inside and outside a mask he is wearing—to determine the total inward leakage through and around the mask. (For comparison, NIOSH’s N95 standard requires manufacturers to measure leakage through the respirator material itself. And OSHA measures how a respirator fits on someone’s face, which often involves wearing an N95 in an enclosed space with saccharin or another distinctly flavored test aerosol sprayed in: if the wearer reports tasting the substance, the mask fails the fit test.)

Collins also tests “pressure drop,” which is basically how easy it is to breathe while wearing a mask. If doing so is too difficult, a wearer might not only find the mask less comfortable but also suck in air around its sides, negating its filtration. Some cloth masks—including those outfitted with coffee filters—have this problem. “There’s a reason N95s aren’t made from cloth,” Collins says.

The Mask Nerd’s top picks can be found in this video. In general, he recommends KN95s made by Chinese company Powecom and others, a variety of KF94s such as the Bluna FaceFit and N95s made by reputable brands such as 3M, Moldex or Honeywell. All of these masks had close to 99 percent filtration efficiencies and fairly low pressure drops in Collins’s setup. (For comparison, he found that a surgical mask alone had between about 50 and 75 percent filtration efficiency, depending on the fit, and a good cloth mask had about 70 percent.) But when choosing the best mask, comfort should be a deciding factor, he says. Not everyone needs to wear an N95.

“To me, the minimum I want to see people wear is a KN95 or KF94 with the Delta variant,” Collins says. “I don’t think surgical masks are good enough anymore, and we should’ve gotten rid of cloth masks last summer—they’re not even in the spectrum” of good filtration. (To be clear, some studies have found that surgical and cloth masks can provide at least some protection against COVID. A recent large, randomized study in Bangladesh found that surgical masks significantly lowered the risk of infection; cloth masks did not have a measurable benefit, although other studies suggest they provide some protection.)

\subsection*{The best masks for kids}
With children starting school in-person, many parents are understandably worried about their kids, especially those who are too young to be eligible for vaccination—and particularly in states where politicians have tried to ban mask mandates in schools. These parents might find Collins’s recommendations for high-filtration kids’ masks particularly helpful. There is no N95 standard for children, but plenty of manufacturers make KF94 or KN95 masks for them. Such masks are designed for small faces and are easy to put on. Collins sees no reason why kids could not tolerate them. “I have my own son,” Collins says. “He’s five years old. He wore them all summer.”

\subsection*{Where to find legitimate masks}
An issue with commercially available high-filtration masks is that they may not come from reputable suppliers. The CDC’s Web site warns that about 60 percent of KN95 respirators available in the U.S. are counterfeit. To find ones that are legitimate, Prather recommends the Web site Project N95. Masks can also be ordered directly from suppliers such as Bona Fide Masks, which sells KN95s made by Powecom. “That’s the one people swear by,” Prather says. They cost around \$1 each. DemeTECH sells N95s for around \$4 apiece, as well as other types of masks.

\subsection*{Reusing masks}
One reason people may be reluctant to use KN95s and similar masks is because they are usually considered disposable. But several experts say they can in fact be worn multiple times. “You can probably reuse it until it becomes visibly damaged or soiled,” Marr says. Collins’s amateur testing suggests mask can be used for up 40 hours with no decrease in their filtration efficacy (he recommends using them within six months of opening a package). The virus likely does not survive long on these masks, but it is not a bad idea to have a few in rotation, reusing one every three days or so, Collins says.

\subsection*{Double masking}
One popular way to increase effectiveness is to wear a cloth mask on top of a surgical mask. This strategy, which the CDC has recommended, combines the filtration efficiency of the surgical mask material with the fit of a cloth mask. But how well does it actually work?
 
According to Collins, pretty well. He measured a filtration efficiency of upward of 90 percent for a cloth mask (with nose wire) over a surgical mask. But the pressure drop was almost twice as high as that of an N95. One reason the CDC and others have recommended against the use of N95s by the general public, apart from their previous scarcity, is that they can be difficult to breathe through—so Collins finds it “baffling” that the CDC would recommend double masking. “So does double masking work? Yes, but … I think there are better solutions,” he said in one of his videos.

Another way to get a better fit is to use masks with straps that go around the back of the head or to use a mask brace if one only has access to a surgical mask.

Not all experts agree that high-filtration masks are necessary for everyone. “What I usually say is ‘The best mask is the one you wear properly,’” says Judith Flores, a pediatrician and a fellow of the American Academy of Pediatrics and of the New York Academy of Medicine. Flores believes surgical masks are the most convenient and cleanest option if they are discarded after each use. Cloth masks are okay, too, she adds, as long as they have three layers. “Unless you are a health care worker or home care worker tending to a person who is COVID-positive,” Flores says, “you don’t need an N95.”

\subsection*{Facial hair}
What about the bewhiskered among us? How does facial hair influence the effectiveness of various masks? While there are not a great deal of data on this, some research suggests that the longer a person’s beard or mustache is, the less effective a mask will be because it makes an inferior seal with the face. The CDC has released a somewhat amusing graphic demonstrating styles of facial hair that are appropriate to wear with a respirator.

At this point in the pandemic, with supplies of high-quality masks readily available in many areas, perhaps it is time to ditch loose-fitting cloth or surgical masks for something that provides better protection. “The most important layer of protection,” Prather says, “is to never let the virus get out in the air in the first place.”

\section*{Script of speech}
	\noindent \textbf{Good afternoon, everyone! My name} is Yaroslav Drachov and \textbf{I’m a fourth year student at MIPT.}

\textbf{The reason we are here today} is to discuss the article from the magazine ``Scientific American'': ``Why We Need to Upgrade Our Face Masks—and Where to Get Them''.\textbf{ I have chosen this topic because}
almost two years have passed since the beginning of the pandemic, we are all tired enough to follow the daily updates of the recommendations of the world's medical organizations, but nevertheless, this can help save many lives. The article provides a brief overview of the current recommendations regarding face masks.

\emph{I'm going to develop three main points. First,}
important parameters of face masks. \emph{Second, } best masks on the market. And, \emph{finally,}  we will discuss facial hair, reusing masks and double masking.

\emph{The presentation should last about seven minutes. If you have any questions, I’d be grateful if you could leave them until the end.}

\emph{Let me start.}

A wealth of evidence has shown that wearing a face mask helps prevent people from spreading the virus that causes COVID to others and from becoming sick themselves. \emph{However,} there has been less guidance from public health officials on what kind of masks provide the best protection.

Early on in the pandemic, the U.S. Centers for Disease Control and Prevention and the World Health Organization told the public not to wear N95 respirators, a type of mask that is made from high-tech synthetic fibers and provides a high level of protection against virus particles. That was because there was then a lack of such masks --- and health care workers needed them.

\emph{However,} there is now a cornucopia of high-filtration respirator-style masks on the market, including N95s, Chinese-made KN95s and South Korean–made KF94s. And CDC finally updated its guidance on September 10 to say the general public could wear N95s and other medical-grade masks now that they are in sufficient supply.

\emph{In the first part of my talk I would like to consider} the important parameters of face masks.
When it comes to mask effectiveness, the most important parameters are filtration, fit and comfort. Filtration generally refers to the percentage of particles the mask material blocks. For example, an N95 filters at least 95 percent of airborne particles. \emph{However,} that does little good if gaps around the mask let air in freely. A well-fitted mask should sit snugly against the face and over the chin, with no gaps around the nose or mouth. \emph{Also,} comfort is an extremely important metric: a mask does no good if people simply find it intolerable to wear.

\emph{Now, I’d like to move on to the next part of my presentation, which is about the best masks on the market.}
In the absence of more specific guidance from health authorities such as the CDC as to which brands of respirators and other masks provide the best protection, some skilled amateurs have  stepped in to fill the gap. Aaron Collins, aka “Mask Nerd,” is a mechanical engineer at Seagate Technology with a background in aerosol science. In his free time, he makes YouTube videos in which he tests and reviews high-filtration masks made by various manufacturers. Collins says he does not earn any money from mask manufacturers or his videos themselves --- he considers them a service and wants them to be objective.

Collins has a mask-testing setup in his bathroom, where he evaluates masks filtration efficiency by generating aerosols of sodium chloride. \emph{Then} he uses a condensation particle counter --- a device that measures the concentration of particles inside and outside a mask he is wearing --- to determine the total inward leakage through and around the mask. \emph{Also,} Collins tests “pressure drop,” which is basically how easy it is to breathe while wearing a mask. In general, he recommends KN95s made by Chinese company Powecom and others.

\emph{Despite} there is no N95 standard for children, plenty of manufacturers make KF94 or KN95 masks for them. Such masks are designed for small faces and are easy to put on. Collins sees no reason why kids could not tolerate them.


\emph{We have come to the final part of my talk about} facial hair, reusing masks and double masking.
How does facial hair influence the effectiveness of various masks? While there are not a great deal of data on this, some research suggests that the longer a person’s beard or mustache is, the less effective a mask will be because it makes an worse seal with the face.


\emph{Next}, let's discuss reusing masks. One reason people may be reluctant to use KN95s and similar masks is because they are usually considered disposable. \emph{However,} several experts say they can in fact be worn multiple times.
The virus likely does not survive long on these masks, but it is not a bad idea to have a few in rotation, reusing one every three days or so, Collins says.

And, \emph{finally,} double masking. One popular way to increase effectiveness is to wear a cloth mask on top of a surgical mask. This strategy, which the CDC has recommended, combines the filtration efficiency of the surgical mask material with the fit of a cloth mask. But how well does it actually work? According to Collins, pretty well. \emph{Although} the pressure drop was almost twice as high as that of an N95 and can be difficult to breathe through this construction. 



\emph{That brings me to the end of my talk. Let me go over the key points again. I have told you} about the properties of a good mask, \emph{we have known} which masks are best, and,\emph{ lastly} we have
discussed reusing masks, double masking and facial hair.

\emph{To conclude I’d like to leave you with the following thought}.

At this point in the pandemic, when there are enough high quality masks for everyone, perhaps it is time to ditch loose-fitting cloth or surgical masks for something that provides better protection. As aerosol expert Kimberly Prather says, “the most important layer of protection, is to never let the virus get out in the air in the first place.”

\emph{Thank you for your attention. If you have any questions, I’ll be happy to answer them. }
\end{document}
