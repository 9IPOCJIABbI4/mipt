\documentclass[a4paper]{article}
% Этот шаблон документа разработан в 2014 году
% Данилом Фёдоровых (danil@fedorovykh.ru) 
% для использования в курсе 
% <<Документы и презентации в \LaTeX>>, записанном НИУ ВШЭ
% для Coursera.org: http://coursera.org/course/latex .
% Исходная версия шаблона --- 
% https://www.writelatex.com/coursera/latex/5.3

% В этом документе преамбула

\usepackage{siunitx}
%%% Работа с русским языком
%\usepackage{cmap}					% поиск в PDF
%\usepackage{mathtext} 				% русские буквы в формулах
%\usepackage[T2A]{fontenc}			% кодировка
%\usepackage[utf8]{inputenc}			% кодировка исходного текста
%\usepackage[english,russian]{babel}	% локализация и переносы
%\usepackage{indentfirst}
%\frenchspacing
%
%\renewcommand{\epsilon}{\ensuremath{\varepsilon}}
%\newcommand{\phibackup}{\ensuremath{\phi}}
%\renewcommand{\phi}{\ensuremath{\varphi}}
%\renewcommand{\varphi}{\ensuremath{\phibackup}}
%\renewcommand{\kappa}{\ensuremath{\varkappa}}
%\renewcommand{\le}{\ensuremath{\leqslant}}
%\renewcommand{\leq}{\ensuremath{\leqslant}}
%\renewcommand{\ge}{\ensuremath{\geqslant}}
%\renewcommand{\geq}{\ensuremath{\geqslant}}
%\renewcommand{\emptyset}{\varnothing}
%\renewcommand{\Im}{\operatorname{Im}}
%\renewcommand{\Re}{\operatorname{Re}}


%%% Дополнительная работа с математикой
\usepackage{amsmath,amsfonts,amssymb,amsthm,mathtools} % AMS
%\usepackage{icomma} % "Умная" запятая: $0,2$ --- число, $0, 2$ --- перечисление

%% Номера формул
%\mathtoolsset{showonlyrefs=true} % Показывать номера только у тех формул, на которые есть \eqref{} в тексте.
%\usepackage{leqno} % Нумереация формул слева

%% Свои команды
\DeclareMathOperator{\sgn}{\mathop{sgn}}
\DeclareMathOperator{\sign}{\mathop{sign}}
\DeclareMathOperator*{\res}{\mathop{res}}
\DeclareMathOperator*{\tr}{\mathop{tr}}
\DeclareMathOperator*{\rot}{\mathop{rot}}
\DeclareMathOperator*{\divop}{\mathop{div}}
\DeclareMathOperator*{\grad}{\mathop{grad}}

%% Перенос знаков в формулах (по Львовскому)
\newcommand*{\hm}[1]{#1\nobreak\discretionary{}
{\hbox{$\mathsurround=0pt #1$}}{}}

%%% Работа с картинками
\usepackage{graphicx}  % Для вставки рисунков
\graphicspath{{figures/}}  % папки с картинками
\setlength\fboxsep{3pt} % Отступ рамки \fbox{} от рисунка
\setlength\fboxrule{1pt} % Толщина линий рамки \fbox{}
\usepackage{wrapfig} % Обтекание рисунков текстом

%%% Работа с таблицами
\usepackage{array,tabularx,tabulary,booktabs} % Дополнительная работа с таблицами
\usepackage{longtable}  % Длинные таблицы
\usepackage{multirow} % Слияние строк в таблице

%%% Теоремы
\theoremstyle{plain} % Это стиль по умолчанию, его можно не переопределять.
\newtheorem{thm}{Теорема}
\newtheorem*{thm*}{Теорема}
\newtheorem{prop}{Предложение}
\newtheorem*{prop*}{Предложение}
 
\theoremstyle{definition} % "Определение"
%\newtheorem{corollary}{Следствие}[theorem]
\newtheorem{dfn}{Определение}
\newtheorem*{dfn*}{Определение}
\newtheorem{prob}{Задача}
\newtheorem*{prob*}{Задача}

 
\theoremstyle{remark} % "Примечание"
\newtheorem*{sol}{Решение}
\newtheorem*{rem}{Замечание}

%%% Программирование
\usepackage{etoolbox} % логические операторы

%%% Страница
%\usepackage{extsizes} % Возможность сделать 14-й шрифт
%\usepackage{geometry} % Простой способ задавать поля
%	\geometry{top=25mm}
%	\geometry{bottom=35mm}
%	\geometry{left=35mm}
%	\geometry{right=20mm}
 
\usepackage{fancyhdr} % Колонтитулы
%	\pagestyle{fancy}
 %	\renewcommand{\headrulewidth}{0pt}  % Толщина линейки, отчеркивающей верхний колонтитул
	%\lfoot{Нижний левый}
	%\rfoot{Нижний правый}
	%\rhead{Верхний правый}
	%\chead{Верхний в центре}
	%\lhead{Верхний левый}
	%\cfoot{Нижний в центре} % По умолчанию здесь номер страницы

\usepackage{setspace} % Интерлиньяж
%\onehalfspacing % Интерлиньяж 1.5
%\doublespacing % Интерлиньяж 2
%\singlespacing % Интерлиньяж 1

\usepackage{lastpage} % Узнать, сколько всего страниц в документе.

\usepackage{soul} % Модификаторы начертания

\usepackage{hyperref}
\usepackage[usenames,dvipsnames,svgnames,table,rgb]{xcolor}
\hypersetup{				% Гиперссылки
    unicode=true,           % русские буквы в раздела PDF
    pdftitle={Заголовок},   % Заголовок
    pdfauthor={Автор},      % Автор
    pdfsubject={Тема},      % Тема
    pdfcreator={Создатель}, % Создатель
    pdfproducer={Производитель}, % Производитель
    pdfkeywords={keyword1} {key2} {key3}, % Ключевые слова
%    colorlinks=true,       	% false: ссылки в рамках; true: цветные ссылки
    %linkcolor=red,          % внутренние ссылки
    %citecolor=black,        % на библиографию
    %filecolor=magenta,      % на файлы
    %urlcolor=cyan           % на URL
}

\usepackage{csquotes} % Еще инструменты для ссылок

%\usepackage[style=apa,maxcitenames=2,backend=biber,sorting=nty]{biblatex}

\usepackage{multicol} % Несколько колонок

\usepackage{tikz} % Работа с графикой
\usepackage{pgfplots}
\usepackage{pgfplotstable}
%\usepackage{coloremoji}
\usepackage{floatrow}
\usepackage{subcaption}
\graphicspath{{figures/}}

\renewcommand\thesubfigure{\asbuk{subfigure}}
%\addbibresource{master.bib}

\usepackage{import}
\usepackage{pdfpages}
\usepackage{transparent}
\usepackage{xcolor}
\usepackage{xifthen}

\newcommand{\incfig}[2][1]{%
    \def\svgwidth{#1\columnwidth}
    \import{./figures/}{#2.pdf_tex}
}
%\usepackage{titlesec}
%\titleformat{\section}{\normalfont\Large\bfseries}{}{0pt}{}
%----------------------STANDART:
%\titleformat{\chapter}[display]
%  {\normalfont\huge\bfseries}{\chaptertitlename\ \thechapter}{20pt}{\Huge}
%\titleformat{\section}{\normalfont\Large\bfseries}{\thesection}{1em}{}
%\titleformat{\subsection}
%  {\normalfont\large\bfseries}{\thesubsection}{1em}{}
%\titleformat{\subsubsection}
%  {\normalfont\normalsize\bfseries}{\thesubsubsection}{1em}{}
%\titleformat{\paragraph}[runin]
%  {\normalfont\normalsize\bfseries}{\theparagraph}{1em}{}
%\titleformat{\subparagraph}[runin]
%  {\normalfont\normalsize\bfseries}{\thesubparagraph}{1em}{}

\pdfsuppresswarningpagegroup=1
\pgfplotsset{compat=1.16}



%\setcounter{tocdepth}{1} % only parts,chapters,sections
%\titleformat{\subsection}{\normalfont\large\bfseries}{}{0em}{}
%\titleformat{\subsubsection}{\normalfont\normalsize\bfseries}{}{0em}{}

%\newcommand{\textover}[2]{\stackrel{\mathclap{\normalfont\mbox{#2}}}{#1}}

\author{Yaroslav Drachov\\
Moscow Institute of Physics and Technology}
%\author{Драчов Ярослав\\
%Факультет общей и прикладной физики МФТИ}
\newcommand{\veq}{\mathrel{\rotatebox{90}{$=$}}}
%\newcommand{\teto}[1]{\stackrel{\mathclap{\normalfont\tiny\mbox{#1}}}{\to}}
%\renewcommand{\thesubsection}{\arabic{subsection}}

%%\setcounter{secnumdepth}{0}

\definecolor{tabblue}{RGB}{30, 119, 180}
\definecolor{taborange}{RGB}{255, 127, 15}
\definecolor{tabgreen}{RGB}{45, 160, 43}
\definecolor{tabred}{RGB}{214, 38, 40}
\definecolor{tabpurple}{RGB}{148, 103, 189}
\definecolor{tabbrown}{RGB}{140, 86, 76}
\definecolor{tabpink}{RGB}{227, 119, 193}
\definecolor{tabgray}{RGB}{127, 127, 127}
\definecolor{tabolive}{RGB}{188, 189, 33}
\definecolor{tabcyan}{RGB}{22, 190, 207}
\pgfplotscreateplotcyclelist{colorbrewer-tab}{
{tabblue},
{taborange},
{tabgreen},
{tabred},
{tabpurple},
{tabbrown},
{tabpink},
{tabgray},
{tabolive},
{tabcyan},
}
\usepackage{csvsimple}
\usepackage{extarrows}
%\renewcommand{\labelenumii}{\asbuk{enumii})}
%\renewcommand{\labelenumiv}{\Asbuk{enumiv}}
%\newcommand{\prob}[1]{\subsubsection*{#1}}
\sisetup{output-decimal-marker = {,},separate-uncertainty = true,exponent-product = \cdot}

\usepackage{braket}
\usepackage{enumerate}
\usepackage{chngcntr}
%\counterwithin*{equation}{problem}
%\usepackage{bbold}

\newtheoremstyle{hiProb}% ⟨name ⟩ 
{3pt}% ⟨Space above ⟩1 
{3pt}% ⟨Space below ⟩1
{}% ⟨Body font ⟩
{}% ⟨Indent amount ⟩2
{\bfseries}% ⟨Theorem head font⟩
{.}% ⟨Punctuation after theorem head ⟩
{.5em}% ⟨Space after theorem head ⟩3
%{\thmname{#1} \thmnote{#3}}% ⟨Theorem head spec (can be left empty, meaning ‘normal’)⟩
{\thmnote{#3}}% ⟨Theorem head spec (can be left empty, meaning ‘normal’)⟩
\theoremstyle{hiProb} % "Определение"
%\newtheorem{hiProb}{Задача}
\newtheorem{hiProb}{}
%\usepackage{mmacells}
\newcommand{\textover}[2]{\stackrel{\mathclap{\normalfont\scriptsize\mbox{#2}}}{#1}}
\usepackage{units}
\usepackage[math]{cellspace}%
\setlength\cellspacetoplimit{2pt}
\setlength\cellspacebottomlimit{2pt}

\DeclareMathAlphabet{\mathbbold}{U}{bbold}{m}{n}

\newcommand{\normord}[1]{:\mathrel{#1}:}

\title{Massively Reducing Food Waste Could Feed the World\footnote{\url{https://www.scientificamerican.com/article/massively-reducing-food-waste-could-feed-the-world/}}}
\author{Chad Frischmann, Mamta Mehra}
\date{October 1, 2021}
\begin{document}
	\maketitle
\section*{Original text}
Imagine going to the market, leaving with three full bags of groceries and coming home. Before you step through your door, you stop and throw one of the bags into a trash bin, which later is hauled away to a landfill. What a waste. Collectively, that is exactly what we are doing today. Globally, 30 to 40 percent of food intended for human consumption is not eaten. Given that more than 800 million people go hungry every day, the scale of food loss fills many of us with a deep sense of anguish\footnote{a feeling of great physical or emotional pain (мучение, страдание).\emph{The rejection filled him with anguish.}}.

If population growth and economic development continue at their current pace, the world will have to produce 53 million more metric tons of food annually by 2050. That increase would require converting another 442 million hectares of forests and grassland --- far greater than the size of India --- into farmland over the next 30 years. The escalation would also release the equivalent of an additional 80 billion tons of carbon dioxide over the next 30 years --- about 15 times the emissions of the entire U.S. economy in 2019. Food waste already accounts for roughly 8 percent of the world's greenhouse gases.

There is another path, however. Our group at Project Drawdown\footnote{a reduction in the value of an investment (просадка).\emph{ The drawdown seemed to be concentrated in the smaller hedge funds}} , an international research and communications organization, completed an exhaustive study of existing technologies and practices that can significantly reduce greenhouse gas levels in the atmosphere while ushering in a more regenerative society and economy. Reducing food waste is one of the top-five means of achieving these goals among 76 we analyzed. Basic adjustments in how food is produced and consumed could help feed the entire world a healthy, nutrient-rich diet through 2050 and beyond without clearing, planting or grazing more land than is used today. Providing more food by eliminating waste, along with better ways of producing that food, would avoid deforestation and also save an enormous amount of energy, water, fertilizer\footnote{a natural or chemical substance added to soil in order to help plants grow (удобрение).\emph{ No synthetic fertilizers are used the growing of our products. }}, labor and other resources.

Opportunities to reduce waste exist at every step along the supply chain from farm to table. We harvest\footnote{the activity of collecting a crop (жать, собирать).\emph{ The corn/potato/grape harvest}} crops, raise livestock\footnote{animals such as cows, sheep, and pigs that are kept on farms (домашний скот).\emph{ Markets for the trading of livestock}}, and process these commodities\footnote{something that can be bought and sold, especially a basic food product or fuel (товар).\emph{ Commodities such as copper and coffee}} into products such as rice, vegetable oil, potato chips, perfectly cut carrots, cheese and New York strip steaks. Most of these products are packaged in cardboard boxes, plastic bags and bottles, tin cans\footnote{
	a tinplate or aluminum container for preserving food, especially an empty one (консервная банка). \emph{We found a tin can and filled it with water}} and glass jars\footnote{A wide-mouthed cylindrical container made of glass or pottery and typically having a lid, used especially for storing food (банка).\emph{ A large storage jar}} made from extracted materials in industrial factories, and then they are shipped on gas-guzzling\footnote{Eat or drink (something) greedily (жадно). \emph{This car guzzles gas}} trucks, trains and planes all over the world.

	After arriving at stores and restaurants, food is held in energy-hungry refrigerators and freezers that use hydrofluorocarbons --- powerful greenhouse gases --- until purchased by consumers, whose eyes are often bigger than their appetites, particularly in richer communities. In high-income countries, restaurants and households turn on their energy-consuming stoves and ovens, and in developing nations, billions of people burn biomass in noxious\footnote{Harmful, poisonous, or very unpleasant (ядовитый). \emph{They were overcome by the noxious fumes}} cookstoves that spew polluting, unhealthy smoke and black carbon.

After all these waste-producing activities, too much of the food that makes it to a consumer's table is thrown in the garbage, which then is typically transported by fossil-fueled trucks to landfills where it decomposes and emits methane, another potent greenhouse gas. Tossing that leftover lasagna accounts for far more emissions than a rotting tomato that never leaves the farm gate. We can do better.

\subsection*{Smaller foodprint}
At Project Drawdown, we poured global data from the Food and Agriculture Organization and many other sources into a detailed model of the entire food production and consumption system. The model took into account rising population projections, as well as greater consumption and more meat eating per person, particularly in developing countries, based on actual trends over the past several decades. According to our calculations, healthier diets and more regenerative agricultural production lead to a lower “foodprint” --- less waste, fewer emissions and a cleaner environment.

If half of the world's population consumes a healthy 2,300 kilocalories a day, built around a plant-rich diet, and puts into practice already proven actions that cut waste across the supply chain, food losses could decline from the current 40 percent to 20 percent, an incredible savings. If we were even more ambitious in following the same practices, food waste could be cut to 10 percent.

These hefty savings would result partly from shifts in basic habits. In the developed world, embracing an average daily 2,300-kilocalorie diet instead of consumption that often reaches more than 3,000 kilocalories lessens food waste in the first place. In the developing world, caloric and protein intake generally need to rise to reach nutritious levels, which may increase some waste across the system. But overall, if everyone on the planet adopted healthy consumption practices and a plant-rich (not necessarily vegetarian) diet, 166 million metric tons of food waste could be avoided over the next 30 years. Feedback would be sent across the supply chain to increase crop production and decrease animal production.

Reducing waste by adjusting how food is produced and consumed can greatly help the environment as well. Different types of foods such as grains, vegetables, fish, meat and dairy have very different environmental footprints. On average, growing and harvesting one kilogram of tomatoes creates about 0.35 kilogram of carbon dioxide emissions. Producing the same amount of beef creates an average of 36 kilograms of emissions. With the entire supply chain taken into account, greenhouse gas emissions from plant-based commodities are 10 to 50 times lower than from most animal-based products.

Additionally, industrial agriculture has spread monocropping, excessive tillage, and widespread use of synthetic fertilizers and pesticides. These practices degrade soil and emit a vast amount of greenhouse gases. Staples are still destroyed in the field by pests and disease and can rot in storage. Livestock consumption of grasses and feed adds further emissions.

Agroecological pest-management practices, such as planting different crops together, and smarter crop rotation can suppress pests and weeds, reducing these losses. Improved livestock-management practices, such as silvopasture, which integrates trees into foraging land, can improve the quality and quantity of animal-based products: more food from fewer hooves in the field and thus fewer resources used and fewer losses. And because regenerative farming practices --- which can increase yield from 5 to 35 percent, restore soils and pull more carbon from the air --- use compost and manure instead of artificial fertilizers, any food that fails to leave the farm gate can be recycled as natural fertilizer or can be converted by anaerobic digestors into biogas for energy on the farm. More farms need to convert to such practices. Restaurants across the U.S. are helping them through one interesting organization called Zero Foodprint, started by chef Anthony Myint, which takes a few cents added to patrons' bills to fund regenerative farms in the making.
\subsection*{Saving the third bag}
In low-income countries, most food is lost before ever getting to market. Improving education and professional training for farmers and producers there, along with innovative technologies, can minimize waste. India's state of Jharkhand, for example, has installed solar-powered refrigeration units that allow farmers who produce vegetables, fruits and other perishables to store their products without sacrificing quality --- a project led by the United Nations Development Program and the Global Environment Facility. In Africa, the Consortium of International Agricultural Research Centers has expanded training that will help local farmers grow more food under conditions being created by climate change, using crops that better tolerate drought and no-till farming to protect withering soil.

In high- and medium-income countries, most waste occurs at the end of the supply chain --- markets and households. There consumers have a tremendous amount of power to prevent waste. A good first step is to reflect on what and how much we are buying. This begins with conscious decisions to purchase what we intend to eat and to eat what we purchase. Rather than overstocking on perishables and other products, buying appropriate quantities of food reduces waste. If too much is cooked for the dinner table, properly storing leftovers reduces spoilage, or they can be shared with neighbors, building stronger community ties.

Broader cultural shifts are also required. The “inglorious fruits and vegetables” campaign launched by the French supermarket chain Intermarché in 2014 aimed to avoid waste by changing cultural attitudes toward “imperfect” foods. Markets tend to procure only fruits and vegetables that meet an idealized cultural perception of shape and color. Imperfect produce that does not match these false traits accounts for up to 40 percent of edible fruits and vegetables being discarded before they leave the farm gate. Instead Intermarché sells these fruits and vegetables in special aisles and runs a national marketing campaign glorifying the inglorious. Other retailers are going even further: All the shelves at Danish supermarket WeFood are stocked with products that would have gone to a landfill. Pittsburgh-based 412 Food Rescue distributes nutritious food that was destined for landfills because of imperfections, limited freshness (such as day-old bread) and unclear labeling to communities in need --- for free.

Wholesalers, retailers and restaurants can play a significant role in shrinking the waste piles. They can demand that suppliers use more food from local regenerative farms. Ensuring that food items are sold with clear, standardized “sell by/use by” labels helps store managers know when to mark down items, and it helps consumers know when and when not to dispose of food. Restaurant owners can offer different portion sizes and fewer menu items and can encourage patrons to take leftovers home.

Governments and companies that offer food services to employees can jump in, too. U.S. federal government cafeterias serve more than two million people; imagine if the kitchen managers chose to offer plant-rich fare made from perfectly imperfect produce procured from regenerative suppliers. Google is already doing more of that in its cafeterias today.

No matter how conscientious we all are, some food will inevitably be lost across the supply chain. Anaerobic digesters and composting are better ways of disposal than dumping food in landfills because they create soil or generate electricity. Eight states across the U.S. now have laws requiring that organic waste be diverted from landfills to avoid potent methane emissions. The latest Project Drawdown analysis shows that implementing these solutions globally can reduce greenhouse gas emissions by around 14 billion metric tons over the next 30 years.

The real magic happens when a variety of solutions are adopted in parallel and sustained over time. The decisions people make as farmers, executives, grocers, chefs and consumers can prevent enough food loss to feed the world through 2050 without converting any more land. That means together we can eliminate hunger and support a healthier global population. And there would still be enough cropland available to grow plants for organic materials such as bioplastics, insulation and biofuels.

Revamping the food chain and adjusting eating habits will not happen overnight. Nor should we expect to immediately become perfect, regeneratively minded, plant-rich connoisseurs who are fastidious about our purchases and what we waste. Our most fundamental task is to be conscientious about the choices we make --- to try to be “solutionists” as much as we can. Together we can save that third bag of groceries.

\section*{Questions}
\begin{enumerate}
\item How much food for human consumption is left uneaten?
\item How can reducing food waste help the planet?
\item How do different stages of the supply chain contribute to greenhouse gas emissions?
\item What can help cut food waste in half?
\item What creates more carbon dioxide emissions, one kilogram of tomatoes or the same amount of beef?
\item What should we do with food that went bad before leaving the farm?
\item Where is most of the food lost in low- and high-income countries?
\item How many imperfect but edible fruits and vegetables are thrown away before leaving the farm gate?
\item How can we reduce the loss of imperfect but edible foods?
\item How can we feed the world until 2050 without additional cultivation of the land?
\end{enumerate}
\section*{Summary}
%The article I’m going to speak about is taken from scientificamerican.com. The headline of the article is "Massively Reducing Food Waste Could Feed the World".
%The topic of the article is food waste these days and ways to reduce it.
%The author starts by telling the reader that globally, 30 to 40 percent of food intended for human consumption is not eaten.
%However, reducing food waste can significantly reduce greenhouse gas levels in the atmosphere. Greenhouse gas emissions occur at all stages of the supply chain, from unsustainable packaging of goods, storing them in energy-intensive refrigerators and freezers, to the delivery of food in greedy vehicles.
%
%After that the author points out that if half of the world's population consumes a healthy 2,300 kilocalories a day, built around a plant-rich diet, and puts into practice already proven actions that cut waste across the supply chain, food losses could decline from the current 40 percent to 20 percent. Author notes that 0.35 kg emissions from the production of one kilogram of tomatoes and 36 kg of emissions from the production of the same amount of beef. Also, any food that fails to leave the farm gate can be recycled as natural fertilizer or can be converted by anaerobic digestors into biogas for energy on the farm.
%
%Then the author states that in low-income countries, most food is lost before ever getting to market. But in high- and medium-income countries, most waste occurs at the end of the supply chain—markets and households. Also, imperfect produce that does not match these false traits accounts for up to 40 percent of edible fruits and vegetables being discarded before they leave the farm gate.
%
%In conclusion the author says that we can do better and when a variety of solutions will be adopted in parallel and sustained over time the real magic can happen.

The article I’m going to speak about is taken from scientificamerican.com. The headline of the article is "Massively Reducing Food Waste Could Feed the World". The topic of the article is food waste these days and ways to reduce it.

The author begins by telling the reader that globally 30 to 40 percent of food remains uneaten. However, reducing food waste can significantly reduce the level of greenhouse gases in the atmosphere. Greenhouse gas emissions occur at all stages of the supply chain, from unsustainable packaging of goods, storing them in energy-intensive refrigerators and freezers, to the delivery of food in greedy vehicles.

The author then notes that we could cut food waste in half if only half of the world's population consumed a healthy 2,300 kilocalories per day, built around a plant-rich diet. The diet should be plant-rich at least because producing 1 kg of beef produces 100 times more emissions than producing the same amount of tomatoes. We can also reduce waste by processing food that fails to leave the farm gate, for example, into fertilizer.

The author also provides interesting statistics: in low-income countries, most food is lost before ever getting to market. But in high-income countries, most waste occurs at markets and households. Also, up to 40 percent of edible fruit and vegetables never leave the farm gate due to their imperfect appearance. These losses can also be reduced.

In conclusion the author says that we can do better and when a variety of solutions will be adopted in parallel and sustained over time the real magic can happen.
\end{document}
