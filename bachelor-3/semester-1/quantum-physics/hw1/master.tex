\documentclass[a4paper]{article}
% Этот шаблон документа разработан в 2014 году
% Данилом Фёдоровых (danil@fedorovykh.ru) 
% для использования в курсе 
% <<Документы и презентации в \LaTeX>>, записанном НИУ ВШЭ
% для Coursera.org: http://coursera.org/course/latex .
% Исходная версия шаблона --- 
% https://www.writelatex.com/coursera/latex/5.3

% В этом документе преамбула

\usepackage{siunitx}
%%% Работа с русским языком
%\usepackage{cmap}					% поиск в PDF
%\usepackage{mathtext} 				% русские буквы в формулах
%\usepackage[T2A]{fontenc}			% кодировка
%\usepackage[utf8]{inputenc}			% кодировка исходного текста
%\usepackage[english,russian]{babel}	% локализация и переносы
%\usepackage{indentfirst}
%\frenchspacing
%
%\renewcommand{\epsilon}{\ensuremath{\varepsilon}}
%\newcommand{\phibackup}{\ensuremath{\phi}}
%\renewcommand{\phi}{\ensuremath{\varphi}}
%\renewcommand{\varphi}{\ensuremath{\phibackup}}
%\renewcommand{\kappa}{\ensuremath{\varkappa}}
%\renewcommand{\le}{\ensuremath{\leqslant}}
%\renewcommand{\leq}{\ensuremath{\leqslant}}
%\renewcommand{\ge}{\ensuremath{\geqslant}}
%\renewcommand{\geq}{\ensuremath{\geqslant}}
%\renewcommand{\emptyset}{\varnothing}
%\renewcommand{\Im}{\operatorname{Im}}
%\renewcommand{\Re}{\operatorname{Re}}


%%% Дополнительная работа с математикой
\usepackage{amsmath,amsfonts,amssymb,amsthm,mathtools} % AMS
%\usepackage{icomma} % "Умная" запятая: $0,2$ --- число, $0, 2$ --- перечисление

%% Номера формул
%\mathtoolsset{showonlyrefs=true} % Показывать номера только у тех формул, на которые есть \eqref{} в тексте.
%\usepackage{leqno} % Нумереация формул слева

%% Свои команды
\DeclareMathOperator{\sgn}{\mathop{sgn}}
\DeclareMathOperator{\sign}{\mathop{sign}}
\DeclareMathOperator*{\res}{\mathop{res}}
\DeclareMathOperator*{\tr}{\mathop{tr}}
\DeclareMathOperator*{\rot}{\mathop{rot}}
\DeclareMathOperator*{\divop}{\mathop{div}}
\DeclareMathOperator*{\grad}{\mathop{grad}}

%% Перенос знаков в формулах (по Львовскому)
\newcommand*{\hm}[1]{#1\nobreak\discretionary{}
{\hbox{$\mathsurround=0pt #1$}}{}}

%%% Работа с картинками
\usepackage{graphicx}  % Для вставки рисунков
\graphicspath{{figures/}}  % папки с картинками
\setlength\fboxsep{3pt} % Отступ рамки \fbox{} от рисунка
\setlength\fboxrule{1pt} % Толщина линий рамки \fbox{}
\usepackage{wrapfig} % Обтекание рисунков текстом

%%% Работа с таблицами
\usepackage{array,tabularx,tabulary,booktabs} % Дополнительная работа с таблицами
\usepackage{longtable}  % Длинные таблицы
\usepackage{multirow} % Слияние строк в таблице

%%% Теоремы
\theoremstyle{plain} % Это стиль по умолчанию, его можно не переопределять.
\newtheorem{thm}{Теорема}
\newtheorem*{thm*}{Теорема}
\newtheorem{prop}{Предложение}
\newtheorem*{prop*}{Предложение}
 
\theoremstyle{definition} % "Определение"
%\newtheorem{corollary}{Следствие}[theorem]
\newtheorem{dfn}{Определение}
\newtheorem*{dfn*}{Определение}
\newtheorem{prob}{Задача}
\newtheorem*{prob*}{Задача}

 
\theoremstyle{remark} % "Примечание"
\newtheorem*{sol}{Решение}
\newtheorem*{rem}{Замечание}

%%% Программирование
\usepackage{etoolbox} % логические операторы

%%% Страница
%\usepackage{extsizes} % Возможность сделать 14-й шрифт
%\usepackage{geometry} % Простой способ задавать поля
%	\geometry{top=25mm}
%	\geometry{bottom=35mm}
%	\geometry{left=35mm}
%	\geometry{right=20mm}
 
\usepackage{fancyhdr} % Колонтитулы
%	\pagestyle{fancy}
 %	\renewcommand{\headrulewidth}{0pt}  % Толщина линейки, отчеркивающей верхний колонтитул
	%\lfoot{Нижний левый}
	%\rfoot{Нижний правый}
	%\rhead{Верхний правый}
	%\chead{Верхний в центре}
	%\lhead{Верхний левый}
	%\cfoot{Нижний в центре} % По умолчанию здесь номер страницы

\usepackage{setspace} % Интерлиньяж
%\onehalfspacing % Интерлиньяж 1.5
%\doublespacing % Интерлиньяж 2
%\singlespacing % Интерлиньяж 1

\usepackage{lastpage} % Узнать, сколько всего страниц в документе.

\usepackage{soul} % Модификаторы начертания

\usepackage{hyperref}
\usepackage[usenames,dvipsnames,svgnames,table,rgb]{xcolor}
\hypersetup{				% Гиперссылки
    unicode=true,           % русские буквы в раздела PDF
    pdftitle={Заголовок},   % Заголовок
    pdfauthor={Автор},      % Автор
    pdfsubject={Тема},      % Тема
    pdfcreator={Создатель}, % Создатель
    pdfproducer={Производитель}, % Производитель
    pdfkeywords={keyword1} {key2} {key3}, % Ключевые слова
%    colorlinks=true,       	% false: ссылки в рамках; true: цветные ссылки
    %linkcolor=red,          % внутренние ссылки
    %citecolor=black,        % на библиографию
    %filecolor=magenta,      % на файлы
    %urlcolor=cyan           % на URL
}

\usepackage{csquotes} % Еще инструменты для ссылок

%\usepackage[style=apa,maxcitenames=2,backend=biber,sorting=nty]{biblatex}

\usepackage{multicol} % Несколько колонок

\usepackage{tikz} % Работа с графикой
\usepackage{pgfplots}
\usepackage{pgfplotstable}
%\usepackage{coloremoji}
\usepackage{floatrow}
\usepackage{subcaption}
\graphicspath{{figures/}}

\renewcommand\thesubfigure{\asbuk{subfigure}}
%\addbibresource{master.bib}

\usepackage{import}
\usepackage{pdfpages}
\usepackage{transparent}
\usepackage{xcolor}
\usepackage{xifthen}

\newcommand{\incfig}[2][1]{%
    \def\svgwidth{#1\columnwidth}
    \import{./figures/}{#2.pdf_tex}
}
%\usepackage{titlesec}
%\titleformat{\section}{\normalfont\Large\bfseries}{}{0pt}{}
%----------------------STANDART:
%\titleformat{\chapter}[display]
%  {\normalfont\huge\bfseries}{\chaptertitlename\ \thechapter}{20pt}{\Huge}
%\titleformat{\section}{\normalfont\Large\bfseries}{\thesection}{1em}{}
%\titleformat{\subsection}
%  {\normalfont\large\bfseries}{\thesubsection}{1em}{}
%\titleformat{\subsubsection}
%  {\normalfont\normalsize\bfseries}{\thesubsubsection}{1em}{}
%\titleformat{\paragraph}[runin]
%  {\normalfont\normalsize\bfseries}{\theparagraph}{1em}{}
%\titleformat{\subparagraph}[runin]
%  {\normalfont\normalsize\bfseries}{\thesubparagraph}{1em}{}

\pdfsuppresswarningpagegroup=1
\pgfplotsset{compat=1.16}



%\setcounter{tocdepth}{1} % only parts,chapters,sections
%\titleformat{\subsection}{\normalfont\large\bfseries}{}{0em}{}
%\titleformat{\subsubsection}{\normalfont\normalsize\bfseries}{}{0em}{}

%\newcommand{\textover}[2]{\stackrel{\mathclap{\normalfont\mbox{#2}}}{#1}}

\author{Yaroslav Drachov\\
Moscow Institute of Physics and Technology}
%\author{Драчов Ярослав\\
%Факультет общей и прикладной физики МФТИ}
\newcommand{\veq}{\mathrel{\rotatebox{90}{$=$}}}
%\newcommand{\teto}[1]{\stackrel{\mathclap{\normalfont\tiny\mbox{#1}}}{\to}}
%\renewcommand{\thesubsection}{\arabic{subsection}}

%%\setcounter{secnumdepth}{0}

\definecolor{tabblue}{RGB}{30, 119, 180}
\definecolor{taborange}{RGB}{255, 127, 15}
\definecolor{tabgreen}{RGB}{45, 160, 43}
\definecolor{tabred}{RGB}{214, 38, 40}
\definecolor{tabpurple}{RGB}{148, 103, 189}
\definecolor{tabbrown}{RGB}{140, 86, 76}
\definecolor{tabpink}{RGB}{227, 119, 193}
\definecolor{tabgray}{RGB}{127, 127, 127}
\definecolor{tabolive}{RGB}{188, 189, 33}
\definecolor{tabcyan}{RGB}{22, 190, 207}
\pgfplotscreateplotcyclelist{colorbrewer-tab}{
{tabblue},
{taborange},
{tabgreen},
{tabred},
{tabpurple},
{tabbrown},
{tabpink},
{tabgray},
{tabolive},
{tabcyan},
}
\usepackage{csvsimple}
\usepackage{extarrows}
%\renewcommand{\labelenumii}{\asbuk{enumii})}
%\renewcommand{\labelenumiv}{\Asbuk{enumiv}}
%\newcommand{\prob}[1]{\subsubsection*{#1}}
\sisetup{output-decimal-marker = {,},separate-uncertainty = true,exponent-product = \cdot}

\usepackage{braket}
\usepackage{enumerate}
\usepackage{chngcntr}
%\counterwithin*{equation}{problem}
%\usepackage{bbold}

\newtheoremstyle{hiProb}% ⟨name ⟩ 
{3pt}% ⟨Space above ⟩1 
{3pt}% ⟨Space below ⟩1
{}% ⟨Body font ⟩
{}% ⟨Indent amount ⟩2
{\bfseries}% ⟨Theorem head font⟩
{.}% ⟨Punctuation after theorem head ⟩
{.5em}% ⟨Space after theorem head ⟩3
%{\thmname{#1} \thmnote{#3}}% ⟨Theorem head spec (can be left empty, meaning ‘normal’)⟩
{\thmnote{#3}}% ⟨Theorem head spec (can be left empty, meaning ‘normal’)⟩
\theoremstyle{hiProb} % "Определение"
%\newtheorem{hiProb}{Задача}
\newtheorem{hiProb}{}
%\usepackage{mmacells}
\newcommand{\textover}[2]{\stackrel{\mathclap{\normalfont\scriptsize\mbox{#2}}}{#1}}
\usepackage{units}
\usepackage[math]{cellspace}%
\setlength\cellspacetoplimit{2pt}
\setlength\cellspacebottomlimit{2pt}

\DeclareMathAlphabet{\mathbbold}{U}{bbold}{m}{n}

\newcommand{\normord}[1]{:\mathrel{#1}:}

\title{Домашняя работа по квантовой физике}
\begin{document}
	\maketitle
\prob{Задача 0-6-1}
\begin{sol}
\[
	E_l= \frac{\hbar^2}{2I}l(l+1)
.\] 
\[
E_1 -E_0= \frac{\hbar^2}{I}
.\] 
\[
kT= \frac{\hbar^2}{I}= \frac{\hbar^2}{\mu a^2}= \frac{2 \hbar^2}{m_O
a^2}
.\] 
\[
	T= \frac{2 \hbar^2}{m_O k a^2}= \frac{2 \left( 
	1,0546 \cdot 10^{-27}\right) ^2}{16 \cdot 1,66 \cdot
10^{-24}\cdot 1,38 \cdot 10^{-16}\cdot\left( 1,2 \cdot 10^{-8} \right) ^2}
= 4,2 \text{ К}
.\] 
\end{sol}
\prob{Задача 0-6-2}
\begin{sol}
\[
	E_1= - \frac{m e^4}{2 \hbar^2} \cdot \frac{1}{1}= - \operatorname{Ry}=
	-13,6 \text{ эВ}
.\] 
\[
	E_2= - \frac{m e^4}{2 \hbar^2} \cdot \frac{1}{4}= - \operatorname{Ry}=
	-3,4 \text{ эВ} \to \Delta E_{12}= 10,2 \text{ эВ}
.\] 
\[
	E_3= - \frac{m e^4}{2 \hbar^2} \cdot \frac{1}{9}= - \operatorname{Ry}=
	-1,51 \text{ эВ} \to \Delta E_{13}= 12,09 \text{ эВ}
.\] 
\[
	E_4= - \frac{m e^4}{2 \hbar^2} \cdot \frac{1}{16}= - \operatorname{Ry}=
	-0,85 \text{ эВ} \to \Delta E_{14}= 12,75 \text{ эВ}
.\] 
\[
	\mathcal{E}=\mathcal{E}_0-\Delta E_{13}=
	12,5-12,09=0,41 \text{ эВ}
.\] 
\end{sol}
\prob{Задача 4.29}
\begin{sol}
\[
\mu= \frac{m_p}{2}
.\] 
\[
E_n = - \frac{\mu e^4}{2 \hbar^2 n^2}= - \operatorname{Ry} \frac{m_p}{
m_e} \frac{1}{2 n^2} \approx -12,5 \frac{1}{n^2} \text{ кэВ}
.\] 
Следовательно, вклад кулоноского взаимодействия в энергию перехода
$2p \to 1s$ в атоме протониума составляет
\[
	\Delta \mathcal{E}_\text{кул}= 12,5 \left( \frac{1}{1^2}-
	\frac{1}{2^2}\right) \approx 9,4 \text{ кэВ}
.\] 
Расхоождение с экспериментальным значением обусловлено вкладом
сильного взаимодействия. Таким образом,
\[
	\Delta \mathcal{E}_\text{сил} = \Delta \mathcal{E}_\text{эксп}
	- \Delta \mathcal{E}_\text{кул}= 10,1-9,4 \approx 0,7 \text{ кэВ}
.\] 
В силу короткодействия ядерных сил их влияние на положение 
$2p$-состояния незначительно по сравнению с $1s$ состоянием, поскольку
 в кулоновском потенциале вероятность частице в $2p$-состоянии
попасть в окрестность начала координат близка к нулю.
\end{sol}
\prob{Задача 4.38}
\begin{sol}
	Согласно закону Мозли энергия кванта, излученного при переходе электрона с уровня $n_2$ на уровень $n_1$ (заряд ядра $Z$, а $\sigma$ --- поправка на экранирование заряда ядра электронами $K$- оболочки)
	\[
		\mathcal{E}=\hbar \omega  = \operatorname{Ry}
		(Z-\sigma)^2 \left( \frac{1}{n_1^2}-\frac{1}{n_2^2} \right) 
	.\] 
Для линии $K_\alpha$ $n_1=1$, $n_2=2$. Поскольку энергия такого
кванта в спектре излучения серебра известна и равна $\mathcal{E}=
21,6 \text{ кэВ}$, то из приведённой формулы найдём поправку $\sigma$ 
на экранирование заряда ядра электронами на $K$-оболочке (в обоих
случаях $Z<50$ )
\[
	\sigma=Z_{\text{Ag}}- \sqrt{\frac{4 \mathcal{E}}{3 \operatorname{Ry}}} \approx 1
.\] 
Энергия, необходимая для освобождения электрона из $K$-оболочки атома
$_{30}\text{Zn}$ (переход с $n=1$  в $n= \infty$), равна
 \[
	 (\hbar \omega)_\text{Zn}= \operatorname{Ry}
	 (Z_{Zn}-1)^2 = 13,6 \cdot 29^2=11,4 \text{ кэВ}
 ,\] 
а значит, кинетическая энергия $T_e$ вылетевшего оттуда электрона
равна
\[
	T_e= \mathcal{E}- (\hbar \omega)_{\operatorname{Zn}}=
	21,6-11,4=10,2 \text{ кэВ}
.\] 
\end{sol}
\prob{Задача 4.42}
\begin{sol}
%Для кулоновского потенциала
%\[
%|E|= \frac{\mu_e Z^2 e^4}{2 \hbar^2},\qquad
%r_1 = \frac{\hbar^2}{Z \mu_e e^2}
%.\] 
%\[
%\Psi = \frac{1}{\sqrt{\pi r_1^3} }e ^{- \frac{r}{r_1}}
%.\] 
%Если считать ядро равномерно заряженным шаром, то
%\begin{multline*}
%	E(r)\cdot 4 \pi r^2= 4 \pi Q \cdot \frac{r^3}{r^3}
%	\implies E = \frac{Q r }{R^3} \implies
%	U(r)= U(R)+ \left. \frac{e Q r^2}{2 R^3} \right|_{r}^R=\\=
%		\frac{eQ}{R}+\frac{eQ}{2R}-\frac{eQr^2}{2R^3}=
%		\frac{3Qe}{2R}-\frac{Qr^2 e}{2R^3}
%.\end{multline*} 
%\[
%\left<E \right> = \int\limits_{V}^{} \Psi^* \hat{H} \Psi dV, \qquad
%\hat{H} = -\frac{\hbar^2}{2m} \Delta r + \hat{U} (r)
%.\]
%\begin{multline*}
%\Delta E = \left<E \right>- |E|=
%\int\limits_{V_\text{Я}}^{} \Psi \Delta \hat{U}(r) \Psi dV=\\=
%\int\limits_{0}^{R} 4 \pi x^2 \cdot \frac{1}{\pi r_1^3}e^{
%-\frac{2x}{r_1}}\cdot \left( \frac{3 Qe}{2R}- \frac{eQx^2}{2R^3}-
%\frac{Qe}{x}\right) dx=\\= \int\limits_{0}^{R} \frac{4}{r_1^3}
%e ^{-\frac{2x}{r_1}}\left( \frac{3Qx^2e}{2R}-\frac{Qx^4e}{2R^3}-
%Qxe\right) dx\approx\\ \approx \frac{4e}{r_1^3}\left( \frac{QR^2}{2}-
%\frac{QR^2}{10}-\frac{QR^2}{2}\right) =
%-\frac{2QR^2e}{5r_1}
%.\end{multline*} 
%Следовательно
%\[
%	\frac{|\Delta E|}{|E|}= \frac{2Q R^2 e}{5 r_1^3 \cdot \left( \frac{Ze^2}{2r_1} \right) }= \frac{4}{5} \left( \frac{R^2}{r_1^2} \right) 
%.\] 
Считаем, чт решение задачи об уровнях энергии в кулоновском поле
точечного заряда известно: $\hat{H} \phi = \mathcal{E} \psi$, где
$\hat{H} = \frac{\hat{p}^2}{2m}-\frac{Ze^2}{r}$, $\mathcal{E}=
-\frac{Z^2 m e^4}{2\hbar^2} \frac{1}{n^2}$. Рассмотрим $n=1$,
тогда $\psi_1= \frac{1}{\sqrt{\pi r_1^3} } e^{-r /r_1}$. Истинный
потенциал при $r \ge R_\text{я}$ совпадает с потенциалом точечного
ядра и отличается от него при $r< R_\text{я}$. Если считать ядро
равномерно заряженной по перхности сферой, то $U(r< R_\text{я}=
-\frac{Ze^2}{R_\text{я}}$; если равномерно заряженным шаром, то
$U(r<R_\text{я})= - \frac{3}{2} \frac{Ze^2}{R_{\text{я}}}\left( 
1- \frac{r^2}{3R_\text{я}^2}\right) $.

Представим истинный гамильтониан в виде
\[
\hat{H}= \frac{\hat{p}^2}{2m}- \frac{Ze^2}{r}+ \delta U
,\] 
где для шара
\[
\delta U = \begin{cases}
	0,& r \ge R_\text{я},\\
	\frac{Z e^2}{r}- \frac{3}{2} \frac{Z e^2}{R_\text{я}}
	\left( 1- \frac{r^2}{3 R_\text{я}^2} \right) ,& r<R_\text{я}
\end{cases}
\] 
и для сферы
\[
\delta U = \begin{cases}
	0, & r \ge R_\text{я},\\
	\frac{Ze^2}{r}- \frac{Z e^2}{R_\text{я}}, & r<R_\text{я}.
\end{cases}
\]
Так как всё отличие происходит при $r <R_\text{я}\ll r_1$, то 
рассматриваем $\delta U$ как поправку. Среднее значение $\delta U$ в
основном состоянии и есть сдвиг уровня:
\[
\left< \psi_1  \left| \hat{H} \right|  \psi_1 \right> =  
\left< \psi_1 \left| \frac{\hat{p}^2}{2m}- \frac{Z e^2}{r} \right| \psi_1 \right>+ \left<\psi_1 \left| \delta U \right| \psi_1 \right>= 
- \frac{m Z^2 e^4}{2 \hbar^2} +\Delta \mathcal{E}
 ,\]
\[
	\Delta \mathcal{E} = \int dV \psi_1^* \delta U \psi_1 =
	\frac{1}{\pi r_1^3} 4 \pi \int\limits_{0}^{R_\text{я}} 
	e^{-2r /r_1} 
	\left\{
	\begin{gathered}
	\frac{Ze^2}{r}- \frac{Ze^2}{R_{\text{я}}}\\
	\frac{Ze^2}{r} - \frac{3}{2} \frac{Ze^2}{R_\text{я}} \left( 
	1 - \frac{r^2}{3 R_\text{я}^2}\right) 
	\end{gathered}
\right\}r^2 dr
 ,\] 
т.\:к. $\frac{R_\text{я}}{r_1}\le \frac{3,5 \cdot 10^{-13}}{25,6
\cdot 10 ^{-13}} \sim 0,1,$ то $e ^{- 2r /r_1} \approx 1$. Тогда
\[
	\Delta \mathcal{E} \approx 
	\frac{4}{r_1^3} \int\limits_{0}^{R_{\text{я}}} 
	\left\{
	\begin{gathered}
		Ze^2 \left( r - \frac{r^2}{R_{\text{я}}} \right) \\
		Ze^2 \left[ r - \frac{3}{2} \frac{r^2}{R_\text{я}}
		\left( 1- \frac{r^2}{3 R_\text{я}^2} \right) \right] 
	\end{gathered}
\right\} dr=
\left\{
\begin{aligned}
	&\frac{2}{3} \frac{Ze^2}{r_1} \left( \frac{R_\text{я}}{r_1} \right) ^2 \text{ --- сфера,}\\
	&\frac{2}{5} \frac{Ze^2}{r_1} \left( \frac{R_\text{я}}{r_1} \right) ^2 \text{ --- шар.}
\end{aligned}
\right.
\] 
Видно, что пооправка положительна, следовательно,
уровни сдвигаются вверх. Относительная поправка
\[
	\delta = \left| \frac{\Delta \mathcal{E}}{\mathcal{E}_1} \right| = \frac{\Delta \mathcal{E}}{\frac{Z e^2}{2 r_1}}=
	\left\{
	\begin{aligned}
		&\frac{4}{3} \left( \frac{R_\text{я}}{r_1} \right) ^2
		\text{ --- сфера,}\\
		&\frac{4}{5} \left( \frac{R_\text{я}}{r_1} \right) ^2
		\text{ --- шар,}
	\end{aligned}
	\right.
\] 
где $r_1=\frac{\hbar^2}{Z m_e e^2}$, откуда
\[
	 \frac{\Delta \mathcal{E}}{\mathcal{E}}=
	\left\{
	\begin{aligned}
		&5,94 \cdot 10^{-7}
		\text{ --- сфера,}\\
		&3,56 \cdot 10^{-7}
		\text{ --- шар.}
	\end{aligned}
	\right.
\] 

\end{sol}
\prob{Задача 4.45}
\begin{sol}
\[
r_\mu= \frac{\hbar^2}{m_\mu Z e^2} \frac{m_e}{m_e}=r_\text{Б}
\frac{m_e}{Z m_\mu}= 1,27 \cdot 10^{-11} \text{ см} \ll r_{\text{Б}}
.\] 
\[
	\frac{1}{\lambda_{32}}= R_\infty \left( \frac{1}{2^2}-\frac{1}{3^2} \right) 
	= \frac{5}{36} R_\infty = \frac{5}{36}= 1,52 \cdot 10^4 \text{ см}^{-1}
.\] 
\[
\lambda_{32}= 656 \text{ нм}
.\] 
\[
	\mathcal{E}_{32}= \frac{hc}{\lambda_{32}}=1,89 \text{ эВ}
.\] 
\end{sol}
\prob{Задача 5.16}
\begin{sol}
\[
I = \mu d^2 \approx m d^2
.\] 
\[
	\Delta E_l = \frac{\hbar^2}{I}(l+1)=\frac{\hbar^2}{I}
.\] 
\[
	\Delta E = \frac{\hbar^2}{m d^2}= \frac{hc}{\lambda}= hc \left( \frac{1}{\lambda} \right) 
.\] 
\[
	d^2 = \frac{\hbar^2}{m h c \left( \frac{1}{\lambda} \right) }=\frac{\hbar^2}{2 \pi m c \Delta}= 1,97 \cdot 10^{-16} \text{ см}^2
.\] 
\[
d= 1,4 \cdot 10^{-8 } \text{ см}
.\] 
\end{sol}
\prob{Задача 5.25}
\begin{sol}
\[
\overline{E}= \overline{E}_\text{кин}+ \overline{E}_\text{пот}
.\] 
\[
\overline{E}_\text{к}= \overline{E}_\text{п}=
\frac{\mu \omega^2 \overline{x^2}}{2}
.\] 
\[
\overline{E}= \frac{\hbar \omega}{2}
.\] 
\[
\overline{x^2}= \overline{A_0^2 \cos^2 \omega t} =
\frac{\overline{A_0^2}}{2} \to  \frac{\hbar \omega}{2}=
2 \cdot \frac{\mu \omega^2}{2} \frac{A_0^2}{2}
.\]
\[
A_0= \sqrt{\frac{\hbar}{\mu \omega}} 
.\] 
\[
\mu= \frac{m_O m_C}{m_O+ m _C}= \frac{16\cdot 12}{16+12}\cdot
1,66 \cdot 10^{-24} \text{ г}= 11,45\cdot 10^{-24} \text{ г}
.\] 
\[
\omega= \frac{2 \pi c}{\lambda} \to  A_0= \sqrt{\frac{\hbar }{\mu
\omega}} = \sqrt{ \frac{\hbar \cdot \lambda}{2 \pi c \mu}} \approx
4,7 \cdot 10^{-10} \text{ см}
.\]
\[
kT \ge \hbar \omega
.\] 
\[
T\ge  \frac{2 \pi \hbar c}{k \cdot \lambda}\approx 3100 \text{ К}
.\] 
\end{sol}
\prob{Задача 5.51}
\begin{sol}
	$\operatorname{H}\prescript{35}{}{\text{Cl}}$ :
$\mu_{35}= \frac{1 \cdot 35 }{1+35}= 0,9722 m_p$.

$\operatorname{H} \prescript{37}{}{\text{Cl}}$: $\mu_{37}=
\frac{1\cdot 37}{1+37}= 0,9737 m_p$
\[
\Delta \mu = 0,0015 m_p
.\] 
\[
	\lambda = \frac{hc}{\Delta \mathcal{E}_\text{кол}}=
	\frac{hc}{\hbar \sqrt{\frac{k}{\mu}} }=
	2 \pi c \sqrt{\frac{\mu}{k}} \propto \sqrt{\mu}  
.\] 
\[
\frac{\Delta\lambda}{\lambda}= \frac{1}{2}\frac{\Delta\mu}{\mu}=
7,7\cdot 10^{-4}
.\] 
\end{sol}
\prob{Задача 5.55}
\begin{sol}
При $n > n_\text{max}$ не $\exists$ реальных $n$.
 \[
	 \frac{d E_n}{dn}= \hbar \omega \left[ 1 - 2\alpha \left( 
	 n+\frac{1}{2}\right)  \right] =0
.\] 
\[
n^{\operatorname{max}}=\frac{1-\alpha}{2\alpha}=80,9
.\] 
\[
	N_{\operatorname{max}}= 81
.\] 
\end{sol}
\prob{Задача 0-7-1}
\begin{sol}
См. табл. \ref{tab:1}.
\begin{table}[htpb]
	\centering
	\caption{}
	\label{tab:1}
	\begin{tabular}{c|c|c}
		$n$ & $l$ & $j$ \\ \hline
		& 0 & $\pm 1 /2$\\
		3 & 1 &  $3 /2;\, 1 /2$ \\
		  & 2&  $5 /2;\, 3/2$
	\end{tabular}
\end{table}
\[l= 0;\ldots;n-1=0;1;2.\]
\[
j=l+s, \text{ т.\:е. } j= l+ \frac{1}{2}, \text{ либо } j=l-\frac{1}{2}
.\] 
\[
j= \pm \frac{1}{2};\, \frac{3}{2};\, \frac{5}{2}
.\] 
\end{sol}
\prob{Задача 0-7-2}
\begin{sol}
$2p$-сост: $n=2$; $l=1$
 \[
j= \frac{1}{2};\, \frac{3}{2}
.\] 
\[
	J= m_j \hbar; \quad m_j = \pm j;\,\pm(j-1);\ldots;0=
	\pm \frac{3}{2};\, \pm \frac{1}{2};\, 0
.\] 
\[
J=0;\,\pm \frac{1}{2}\hbar;\, \pm \frac{3}{2}\hbar
.\] 
\end{sol}
\prob{Задача 6.10}
\begin{sol}
\[
I\omega=2lN\to \omega= \frac{2lN}{I}
.\] 
\[
\omega= \frac{2N\cdot \hbar }{0,5 m r^2}; \qquad
\nu= \frac{m}{A}= \frac{N}{N_\text{А}} \to N = \frac{m N_\text{А}}{A}
.\] 
\[
\omega= \frac{4N \hbar }{m r^2}= \frac{4 \hbar }{m r^2} \cdot 
\frac{m N_{\operatorname{A}}}{A} \cdot \frac{\pi \rho L}{\pi \rho L}=
\frac{2 h N_{\operatorname{A}}\rho L}{A \rho \pi r^2 L}=
\frac{2h N_{\operatorname{A}}\rho L}{m A}= 1,1 \cdot 10 ^{-3} \text{ с}^{-1}
.\] 
\end{sol}
\prob{Задача 6.15}
\[
f_z = \mu \frac{\partial B_z}{\partial z} = m_n \cdot a_\perp
.\] 
\[
a_\perp = \frac{v_\perp}{\tau};\qquad \tau = \frac{L}{v_{\parallel}}
.\] 
\[
f_z = m_n\cdot \frac{v_{\parallel}v_\perp}{L}= \mu \frac{\partial B_z}{\partial z} 
.\] 
\[
v_\perp= \frac{\mu \frac{\partial B_z}{\partial z} }{m_n} \frac{L}{v_{\parallel}}
.\] 
Дифф. уширение: $\alpha_\text{диф}\sim \frac{\lambda}{d}=\frac{h}{d
\sqrt{2 m E} }$ 
\[
\alpha_\text{маг}= \frac{v_\perp}{v_\parallel}= \frac{\mu \frac{\partial B_z}{\partial z} }{m} \frac{L}{v_\parallel^2}= \alpha_\text{диф}=
\frac{h}{d \sqrt{2 m E} }
.\] 
\[
\frac{\partial B_z}{\partial z} = \frac{2 E h}{L \mu d \sqrt{2mE} }=
150 \frac{\text{Гс}}{\text{см}}
.\] 
\prob{Задача 6.20}
\begin{sol}
\[
	\mathcal{E}_1= \hbar \omega_1= \frac{hc}{\lambda_1}
.\] 
\[
	\mathcal{E}_2= \frac{hc}{\lambda_2}
.\] 
\[
	\Delta\mathcal{E}= hc \left( \frac{1}{\lambda_2}-\frac{1}{\lambda_1} \right) =2 \cdot 10^{-3} \text{ эВ}
.\] 
\[
	\Delta\mathcal{E}=2 \mu B
.\] 
\[
	B=\frac{\Delta\mathcal{E}}{2\mathfrak{m}_\text{Б}}=1,8 \cdot 10^5 \text{ Гс}
.\] 
\end{sol}
\prob{Задача 6.48}
\begin{sol}
\[
\mathbf{H}=-\beta \mathbf{M};\qquad \beta= \frac{4\pi}{3}
.\] 
\[
	\mathbf{B}= \mathbf{H}+4 \pi \mathbf{M}=\left( 
	-\frac{4\pi}{3}+4\pi\right) \mathbf{M}=\frac{8\pi}{3} \frac{
g_s\mathfrak{m}_\text{Б}\cdot \mathbf{S}_e}{\frac{4}{3}\pi r_\text{Б}^3}
.\] 
Энергия взаимодействия:
\begin{multline*}
	U_{ep}= - \left( \boldsymbol{\mu}_p,\,\mathbf{B} \right)=
	-g_p \mathfrak{m}_\text{яд} \mathbf{S}_p \mathbf{B}=
	-g_p \mathfrak{m}_\text{яд} \mathbf{S}_p \cdot 2 g_s
	\frac{\mathfrak{m}_\text{Б}}{r_\text{Б}^3}\mathbf{S}_e=\\=
	- 2g_s g_p \frac{\mathfrak{m}_\text{яд}\mathfrak{m}_\text{Б}
	}{r_{\text{Б}}^3}(\mathbf{S}_p,\,\mathbf{S}_e)
.\end{multline*} 
\[
	\mathbf{S}=\mathbf{S}_e+\mathbf{S}_e,\quad S=1,\,0
.\] 
\[
	\Delta E=\left| U_{ep}(S=1)-U_{ep}(S=0) \right| 
.\] 
\[
	\overline{S^2}=\overline{\left( \mathbf{S}_e+\mathbf{S}_p \right) ^2}=\overline{S_e^2}+\overline{S_p^2}+2 \overline{\mathbf{S}_e
	\mathbf{S}_p}
.\] 
\[
	2\mathbf{S}_e\mathbf{S}_p=\left( \overline{S^2}-
	\overline{S^2_e}-\overline{S_p^2}\right) 
.\] 
\[
	\overline{S^2}= S(S+1) \xlongequal[]{S=1}1\cdot 2=2
.\] 
\[
	\overline{S_e^2}=S(S+1)=\frac{1}{2}\cdot \frac{3}{2}=\frac{3}{4}
.\]

При $S=1$: $2 \overline{\mathbf{S}_e \mathbf{S}_p}=\left(1\cdot 2-\frac{3}{4}-\frac{3}{4}\right)=\frac{1}{2}$.

При $S=0$: $2 \overline{\mathbf{S}_e \mathbf{S}_p}=\left( 0-
\frac{3}{4}-\frac{3}{4}\right) =-\frac{3}{2}$
\[
	\Delta E= g_s g_p \frac{\mathfrak{m}_\text{яд}\mathfrak{m}_\text{Б}}{r_{\text{Б}}^3}\left( \frac{1}{2}+\frac{3}{2} \right)=
	2 g_s g_p \frac{\mathfrak{m}_\text{Б}^2}{r_{\text{Б}}^3}
	\frac{m_e}{m_p}
.\] 
\[
\lambda= \frac{hc}{\Delta E}=\frac{hcr_\text{Б}^3 m_p}{2 g_s g_p
\mathfrak{m}_\text{Б}^2 m_e}= 28,2 \text{ см}
.\] 
\end{sol}
\prob{Задача 6.78}
\begin{sol}
	Энерг. спин-орб. вз-я: $\mathcal{E}_{SL}=A \overline{(\mathbf{L},\,\mathbf{S}}$.
\begin{multline*}
	\mathbf{J}=\mathbf{S}+\mathbf{L}\implies
	\overline{J^2}= \overline{S^2}+\overline{L^2}+
	2 \overline{(\mathbf{S},\,\mathbf{S})}\implies
	\overline{(\mathbf{L},\,\mathbf{S})}=\\=\frac{1}{2}
	\left(\overline{J^2}-\overline{S^2}-\overline{L^2}\right)=
	\frac{1}{2} \left( J(J+1)-S(S+1)-L(L+1) \right) 
.\end{multline*} 
\[
	^1D_2:\quad \mathcal{E}_{SL}(^1D_2)=\frac{A}{2}(2\cdot 3-
	0-2\cdot 3)=0
.\] 
\[
	^3 P_2:\quad \mathcal{E}_{SL} (^3P_2)=
	\frac{A}{2}(2 \cdot 3-1 \cdot 2- 1\cdot 2)=A
.\] 
\[
	^3 P_1:\quad \mathcal{E}_{SL} (^3P_2)=
	\frac{A}{2}(1 \cdot 3-1 \cdot 2- 1\cdot 2)=-A
.\] 
\[
	^3 P_0:\quad \mathcal{E}_{SL} (^3P_2)=
	\frac{A}{2}(0 \cdot 3-1 \cdot 2- 1\cdot 2)=-2A
.\] 
\[
\left\{
\begin{aligned}
	\frac{hc}{\lambda_1}= \mathcal{E}(^1D_2)-\mathcal{E}(^3P_2)=
	\mathcal{E}(^1D_2)-\left( \mathcal{E}(^3P)+\mathcal{E}_{SL}(
	^3P_2)\right) \\
	\frac{hc}{\lambda_2}= \mathcal{E}(^1D_2)-\mathcal{E}(^3P_1)=
	\mathcal{E}(^1D_2)-\left( \mathcal{E}(^3P)+\mathcal{E}_{SL}(
	^3P_1)\right) \\
\end{aligned}
\right.
.\] 
\[
	\frac{hc}{\lambda_1}-\frac{hc}{\lambda_2}=\mathcal{E}_{SL}
	(^3P_1)- \mathcal{E}_{SL}(^3P_2)=-2A
.\] 
\[
	\mathcal{E}(^3P_0)-\mathcal{E}(^3P_1)=
	\mathcal{E}(^3P)+\mathcal{E}_{SL}(^3P_0)-
	\mathcal{E}(^3P)-\mathcal{E}_{SL}(^3P_1)=
	\mathcal{E}_{SL}(^3P_0)-\mathcal{E}_{SL}(^3P_1)=-A
.\] 
\[
	-A= \frac{hc}{\lambda}= \left( \frac{hc}{\lambda_1}-
	\frac{hc}{\lambda_2}\right) \cdot 2 \implies
	\lambda=\frac{2\lambda_1 \lambda_2}{\lambda_2-\lambda_1}=
	16634 \text{ \AA}
.\] 
\end{sol}

\prob{Задача 6.78}
\begin{sol}
$S=1$ --- ортогелий, $S=0$ --- парагелий.
\[
E_{\text{пара}}= - W_{\text{пара}}= E + E_\text{кул}+ V_\text{пара}
.\] 
\[
E_\text{орто}= - W_{\text{орто}}= E+E_\text{кул}+ V_{\text{орто}}
.\] 
\[
	\overline{\mathbf{S}_1+\mathbf{S}_2}= \overline{S_1^2}+
	\overline{S_2^2}+ 2 \overline{\mathbf{S}_1 \mathbf{S}_2}
.\] 
\[
	2 \overline{\mathbf{S}_1 \mathbf{S}_2}=
	\overline{(S_1+S_2)^2}- \overline{S_1^2}-\overline{S_2^2}=
	S(S+1)- \frac{3}{4}-\frac{3}{4}
.\] 
\begin{multline*}
	V= -\frac{A}{2}\left[ 1+4 \overline{\mathbf{S}_1 \mathbf{S}_2} \right]=
	-\frac{A}{2} \left[ 1+2 \left( S(S+1)-\frac{6}{4} \right)  \right] =\\=-A\left[ \frac{1}{2}+S(S+1)-\frac{3}{2} \right] =-A
	\left[ S(S+1)-1 \right] 
.\end{multline*} 
\[
V_{\text{пара}}=+A;\qquad V_\text{орто}=-A
.\] 
\[
E_{\text{пара}}= - W_{\text{пара}}= E + E_\text{кул}+ A
.\] 
\[
E_\text{орто}= - W_{\text{орто}}= E+E_\text{кул} -A
.\]
В поле $Z=2$ при $1s^12s^1$
 \[
	 E=-13,6\cdot Z^2-13,6 \frac{Z^2}{n^2}=-13,6\left( 4+
	 \frac{4}{4}\right) =-68 \text{ эВ}
.\]
\[
	A=\left( E_\text{пара}- E_\text{орто} \right) \frac{1}{2}=
	0,4 \text{ эВ}
.\] 
\[
E_\text{кул}= \frac{E_\text{пара}+E_\text{орто}}{2}-E=-9,2 \text{ эВ}
.\] 
\end{sol}
\prob{Задача Т2}
\begin{sol}
Термом называют макс. возможные $M_L$ при конкретных $M_S$
или  $M_S$ при одном $M_L$. См. рис.~\ref{fig:2}.
\begin{figure}[ht]
    \centering
    \incfig{2}
    \caption{}
    \label{fig:2}
\end{figure}
\[
	\begin{pmatrix} \underline{^3P} \\ ^3 S \end{pmatrix} \qquad
	\begin{pmatrix} \underline{^1 D} \\ ^1 P \\ ^1 S \end{pmatrix} \qquad
	\begin{pmatrix} \underline{^1 S} \end{pmatrix} 
.\] 
\end{sol}
\end{document}
