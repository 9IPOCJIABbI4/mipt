\lecture{1}{Вт 08 сен 2020 12:51}{Гипотеза Планка}
Ядерная физика: Ушханов

Атомная физика: все в задавальнике

\[
k_x= \frac{\pi}{L_x}l;\qquad k_y= \frac{\pi}{L_y}n; \qquad k_z= \frac{\pi}{L_z}p,
\]
\[
\Delta \Omega = \frac{\pi^3}{V};
\] 
\[
	d N_{\vec{k}}=2 \cdot \frac{1}{8} \frac{4\pi k^2 dk}{
	\Delta \Omega}= \frac{V k^2 dk}{\pi^2}
.\]
\[
	d N_\omega = \frac{V \omega^2 d \omega}{\pi^2 c^3}
.\] 
\begin{figure}[ht]
    \centering
    \incfig{1}
    \caption{}
    \label{fig:1}
\end{figure}

\[
E_{clas}= \frac{VT}{\pi^2 c^3}\int\limits_{0}^{\infty} \omega^2 d\omega \to  \infty 
.\] 
\begin{multline*}
	\left< \epsilon_{quan}(\omega) \right> =
	\frac{\sum_{n=0}^{\infty}n \hbar \omega A \exp\left( 
	-\frac{n \hbar \omega}{T}\right) }{\sum_{n=0}^{\infty} A
\exp \left( - \frac{n \hbar \omega}{T} \right) }=
\frac{\sum_{n=0}^{\infty} n \hbar \omega \exp\left( - \frac{
n \hbar \omega}{T} \right) }{\sum_{n=0}^{\infty} \exp \left( -
\frac{n \hbar \omega}{T}\right) }=\\=
\frac{\sum_{n=0}^{\infty} n \hbar \omega \exp \left(-\beta n \hbar \omega\right)}{\sum_{n=0}^{\infty} \exp\left(-\beta n \hbar \omega\right)}\equiv \frac{
S_1}{S_0}, \text{ где } \beta=\frac{1}{T}
.\end{multline*} 
\[
\epsilon=\hbar \omega= h \nu
.\] 
\[
h= 2 \pi \hbar
.\] 
\[
	\hbar=1,05 \cdot 10^{-27} \text{ эрг}\cdot \text{c}=
	1,05 \cdot 10^{-34} \text{ Дж}\cdot \text{с}
.\] 
\[
w_E = A e^{-\frac{E}{T}}
.\]
\[
	S_0= \frac{1}{1-\exp(-\beta \hbar \omega)}
.\] 
\[
S_1=-\frac{\partial S_0}{\partial \beta} =\frac{\hbar \omega
\exp(-\beta \hbar \omega)}{[1-\exp(-\beta \hbar \omega)]^2}
.\] 
\[
	\left<e_{quan}(\omega) \right> = \frac{\hbar \omega}{\exp
	\left( \frac{\hbar\omega}{T} \right) -1}
.\]
\[
	\left<n(\omega) \right> = \frac{1}{\exp\left( \frac{\hbar
	\omega}{T} \right) -1}
.\] 
\[
	E_{quan}= \frac{V}{\pi^2 c^3}\int\limits_{0}^{\infty} 
	\frac{\hbar \omega^3 d \omega}{\exp\left( \frac{\hbar \omega}{T} \right) -1}= \frac{V T^4}{\pi^2 c^3 \hbar^3} \int\limits_{0}^{\infty} \frac{x^3 dx}{\exp(x) -1}= \frac{\pi^2 V T^4}{15c^3 \hbar^3} 
.\]
\[
u = \frac{E_{quan}}{V}=\frac{\pi^2 T^4}{15 c^3 \hbar^3}
.\] 
$\alpha$--- коэффициент поглощения, если  $\alpha(\omega)=\text{const}$, то тело --- серое.
\begin{figure}[ht]
    \centering
    \incfig{2}
    \caption{}
    \label{fig:2}
\end{figure}
\[
	I(\theta)=I(0) \cos \theta
.\] 
\[
	I(\theta)=\frac{d\Phi}{dS\cdot d\theta}
.\] 
\[
R= \sigma T^4 \qquad \sigma = \frac{\pi^2 k_B^4}{60 c^2 \hbar^3}=
5,67\cdot 10^{-8} \frac{\text{Вт}}{\text{м}^2 \text{К}^4}
.\] 
\begin{figure}[ht]
    \centering
    \incfig{3}
    \caption{}
    \label{fig:3}
\end{figure}
\[
\hbar \omega_{max}=2,8 T=2,8 k_B T
.\] 
Рассмотрим тело, которое и излучает и поглощает, тогда 
\[
	\Phi(\omega)+(1-A(\omega))R(\omega)=R(\omega)
.\] 
\[
	\frac{\Phi(\omega)}{A(\omega)}=R(\omega)
.\]
$\Phi(\omega)$ --- испускательная способность в узком интервале частот
$(\omega,\, \omega+d\omega)$. 
