\documentclass[a4paper]{article}
% Этот шаблон документа разработан в 2014 году
% Данилом Фёдоровых (danil@fedorovykh.ru) 
% для использования в курсе 
% <<Документы и презентации в \LaTeX>>, записанном НИУ ВШЭ
% для Coursera.org: http://coursera.org/course/latex .
% Исходная версия шаблона --- 
% https://www.writelatex.com/coursera/latex/5.3

% В этом документе преамбула

\usepackage{siunitx}
%%% Работа с русским языком
\usepackage{cmap}					% поиск в PDF
\usepackage{mathtext} 				% русские буквы в формулах
\usepackage[T2A]{fontenc}			% кодировка
\usepackage[utf8]{inputenc}			% кодировка исходного текста
\usepackage[english,russian]{babel}	% локализация и переносы
\usepackage{indentfirst}
\frenchspacing

\renewcommand{\epsilon}{\ensuremath{\varepsilon}}
\renewcommand{\phi}{\ensuremath{\varphi}}
\renewcommand{\kappa}{\ensuremath{\varkappa}}
\renewcommand{\le}{\ensuremath{\leqslant}}
\renewcommand{\leq}{\ensuremath{\leqslant}}
\renewcommand{\ge}{\ensuremath{\geqslant}}
\renewcommand{\geq}{\ensuremath{\geqslant}}
\renewcommand{\emptyset}{\varnothing}
\renewcommand{\Im}{\operatorname{Im}}
\renewcommand{\Re}{\operatorname{Re}}


%%% Дополнительная работа с математикой
\usepackage{amsmath,amsfonts,amssymb,amsthm,mathtools} % AMS
\usepackage{icomma} % "Умная" запятая: $0,2$ --- число, $0, 2$ --- перечисление

%% Номера формул
%\mathtoolsset{showonlyrefs=true} % Показывать номера только у тех формул, на которые есть \eqref{} в тексте.
%\usepackage{leqno} % Нумереация формул слева

%% Свои команды
\DeclareMathOperator{\sgn}{\mathop{sgn}}
\DeclareMathOperator{\sign}{\mathop{sign}}
\DeclareMathOperator*{\res}{\mathop{res}}
\DeclareMathOperator*{\tr}{\mathop{tr}}

%% Перенос знаков в формулах (по Львовскому)
\newcommand*{\hm}[1]{#1\nobreak\discretionary{}
{\hbox{$\mathsurround=0pt #1$}}{}}

%%% Работа с картинками
\usepackage{graphicx}  % Для вставки рисунков
\graphicspath{{figures/}}  % папки с картинками
\setlength\fboxsep{3pt} % Отступ рамки \fbox{} от рисунка
\setlength\fboxrule{1pt} % Толщина линий рамки \fbox{}
\usepackage{wrapfig} % Обтекание рисунков текстом

%%% Работа с таблицами
\usepackage{array,tabularx,tabulary,booktabs} % Дополнительная работа с таблицами
\usepackage{longtable}  % Длинные таблицы
\usepackage{multirow} % Слияние строк в таблице

%%% Теоремы
\theoremstyle{plain} % Это стиль по умолчанию, его можно не переопределять.
\newtheorem{theorem}{Теорема}
\newtheorem*{thm}{Теорема}
\newtheorem{prop}{Утверждение}
 
\theoremstyle{definition} % "Определение"
%\newtheorem{corollary}{Следствие}[theorem]
\newtheorem*{dfn}{Определение}
\newtheorem{problem}{Задача}
\newtheorem*{problem*}{Задача}

 
\theoremstyle{remark} % "Примечание"
\newtheorem*{sol}{Решение}
\newtheorem*{rem}{Замечание}

%%% Программирование
\usepackage{etoolbox} % логические операторы

%%% Страница
%\usepackage{extsizes} % Возможность сделать 14-й шрифт
%\usepackage{geometry} % Простой способ задавать поля
%	\geometry{top=25mm}
%	\geometry{bottom=35mm}
%	\geometry{left=35mm}
%	\geometry{right=20mm}
 
\usepackage{fancyhdr} % Колонтитулы
%	\pagestyle{fancy}
 %	\renewcommand{\headrulewidth}{0pt}  % Толщина линейки, отчеркивающей верхний колонтитул
	%\lfoot{Нижний левый}
	%\rfoot{Нижний правый}
	%\rhead{Верхний правый}
	%\chead{Верхний в центре}
	%\lhead{Верхний левый}
	%\cfoot{Нижний в центре} % По умолчанию здесь номер страницы

\usepackage{setspace} % Интерлиньяж
%\onehalfspacing % Интерлиньяж 1.5
%\doublespacing % Интерлиньяж 2
%\singlespacing % Интерлиньяж 1

\usepackage{lastpage} % Узнать, сколько всего страниц в документе.

\usepackage{soul} % Модификаторы начертания

\usepackage{hyperref}
%\usepackage[usenames,dvipsnames,svgnames,table,rgb]{xcolor}
\hypersetup{				% Гиперссылки
    unicode=true,           % русские буквы в раздела PDF
    pdftitle={Заголовок},   % Заголовок
    pdfauthor={Автор},      % Автор
    pdfsubject={Тема},      % Тема
    pdfcreator={Создатель}, % Создатель
    pdfproducer={Производитель}, % Производитель
    pdfkeywords={keyword1} {key2} {key3}, % Ключевые слова
    colorlinks=true,       	% false: ссылки в рамках; true: цветные ссылки
    linkcolor=red,          % внутренние ссылки
    citecolor=black,        % на библиографию
    filecolor=magenta,      % на файлы
    urlcolor=cyan           % на URL
}

\usepackage{csquotes} % Еще инструменты для ссылок

%\usepackage[style=apa,maxcitenames=2,backend=biber,sorting=nty]{biblatex}

\usepackage{multicol} % Несколько колонок

\usepackage{tikz} % Работа с графикой
\usepackage{pgfplots}
\usepackage{pgfplotstable}
%\usepackage{coloremoji}
\usepackage{floatrow}
\usepackage{subcaption}
\newcommand*{\N}{\mathbb{N}}
\newcommand*{\R}{\mathbb{R}}
\newcommand*{\K}{\mathbb{K}}
\newcommand*{\V}{\mathcal{V}}
\newcommand*{\A}{\mathcal{A}}
\newcommand*{\ii}{\mathbf{1}}
\newcommand*{\oo}{\mathbf{0}}
\newcommand*{\ba}{\mathbf{a}}
\newcommand*{\bb}{\mathbf{b}}
\newcommand*{\Q}{\mathbb{Q}}
\graphicspath{{figures/}}
%\usepackage{breqn}

\renewcommand\thesubfigure{\asbuk{subfigure}}
%\addbibresource{master.bib}

\usepackage{import}
\usepackage{pdfpages}
\usepackage{transparent}
\usepackage{xcolor}
\usepackage{xifthen}

%\newcommand{\incfig}[1]{%
%    \def\svgwidth{\columnwidth}
%    \import{./figures/}{#1.pdf_tex}
%}


\newcommand{\incfig}[2][1]{%
    \def\svgwidth{#1\columnwidth}
    \import{./figures/}{#2.pdf_tex}
}
\usepackage{titlesec}
%\titleformat{\section}{\normalfont\Large\bfseries}{}{0pt}{}
%----------------------STANDART:
%\titleformat{\chapter}[display]
%  {\normalfont\huge\bfseries}{\chaptertitlename\ \thechapter}{20pt}{\Huge}
%\titleformat{\section}{\normalfont\Large\bfseries}{\thesection}{1em}{}
%\titleformat{\subsection}
%  {\normalfont\large\bfseries}{\thesubsection}{1em}{}
%\titleformat{\subsubsection}
%  {\normalfont\normalsize\bfseries}{\thesubsubsection}{1em}{}
%\titleformat{\paragraph}[runin]
%  {\normalfont\normalsize\bfseries}{\theparagraph}{1em}{}
%\titleformat{\subparagraph}[runin]
%  {\normalfont\normalsize\bfseries}{\thesubparagraph}{1em}{}

\pdfsuppresswarningpagegroup=1
\pgfplotsset{compat=1.16}

\usepackage{xifthen}
\makeatother
%\def\@lecture{}%
%\newcommand{\lecture}[3]{
%    \ifthenelse{\isempty{#3}}{%
%        \def\@lecture{Неделя #1}%
%    }{%
%        \def\@lecture{Неделя #1: #3}%
%    }%
%    \section*{\@lecture}
%    \marginpar{\small\textsf{\mbox{#2}}}
%}
\makeatletter

%\newcommand{\lec}{\subsection{Лекция}}
%\newcommand{\sem}{\subsection{Семинар}}
%\newcommand{\hw}{\subsection{Домашняя работа}}
%\newcommand{\prob}[1]{\textbf{#1}}
%\renewcommand{\thesubsection}{}
%\renewcommand{\thesubsubsection}{}

%\setcounter{tocdepth}{1} % only parts,chapters,sections
%\titleformat{\subsection}{\normalfont\large\bfseries}{}{0em}{}
%\titleformat{\subsubsection}{\normalfont\normalsize\bfseries}{}{0em}{}

%\newcommand{\textover}[2]{\stackrel{\mathclap{\normalfont\mbox{#2}}}{#1}}

\author{Драчов Ярослав\\
Факультет общей и прикладной физики МФТИ}
\newcommand{\veq}{\mathrel{\rotatebox{90}{$=$}}}
%\newcommand{\teto}[1]{\stackrel{\mathclap{\normalfont\tiny\mbox{#1}}}{\to}}
%\renewcommand{\thesubsection}{\arabic{subsection}}

%%\setcounter{secnumdepth}{0}

\definecolor{tabblue}{RGB}{30, 119, 180}
\definecolor{taborange}{RGB}{255, 127, 15}
\definecolor{tabgreen}{RGB}{45, 160, 43}
\definecolor{tabred}{RGB}{214, 38, 40}
\definecolor{tabpurple}{RGB}{148, 103, 189}
\definecolor{tabbrown}{RGB}{140, 86, 76}
\definecolor{tabpink}{RGB}{227, 119, 193}
\definecolor{tabgray}{RGB}{127, 127, 127}
\definecolor{tabolive}{RGB}{188, 189, 33}
\definecolor{tabcyan}{RGB}{22, 190, 207}
\pgfplotscreateplotcyclelist{colorbrewer-tab}{
{tabblue},
{taborange},
{tabgreen},
{tabred},
{tabpurple},
{tabbrown},
{tabpink},
{tabgray},
{tabolive},
{tabcyan},
}
\usepackage{csvsimple}
\usepackage{extarrows}
%\renewcommand{\labelenumii}{\asbuk{enumii})}
%\renewcommand{\labelenumiv}{\Asbuk{enumiv}}
\newcommand{\prob}[1]{\subsubsection*{#1}}
\sisetup{output-decimal-marker = {,},separate-uncertainty = true,exponent-product = \cdot}

\usepackage{braket}
\usepackage{enumerate}
\usepackage{chngcntr}
%\counterwithin*{equation}{problem}
%\usepackage{bbold}

\newtheoremstyle{hiProb}% ⟨name ⟩ 
{3pt}% ⟨Space above ⟩1 
{3pt}% ⟨Space below ⟩1
{}% ⟨Body font ⟩
{}% ⟨Indent amount ⟩2
{\bfseries}% ⟨Theorem head font⟩
{.}% ⟨Punctuation after theorem head ⟩
{.5em}% ⟨Space after theorem head ⟩3
%{\thmname{#1} \thmnote{#3}}% ⟨Theorem head spec (can be left empty, meaning ‘normal’)⟩
{\thmnote{#3}}% ⟨Theorem head spec (can be left empty, meaning ‘normal’)⟩
\theoremstyle{hiProb} % "Определение"
%\newtheorem{hiProb}{Задача}
\newtheorem{hiProb}{}
\usepackage{mmacells}
\newcommand{\textover}[2]{\stackrel{\mathclap{\normalfont\scriptsize\mbox{#2}}}{#1}}
\usepackage{units}
\usepackage[math]{cellspace}%
\setlength\cellspacetoplimit{2pt}
\setlength\cellspacebottomlimit{2pt}

\usepackage{mmacells}
\title{Второе задание по вычислительной математике}
\begin{document}
	\maketitle
	\section*{Тема IV. Нелинейные уравнения}
\begin{hiProb}[11.3]
Построить метод Ньютона для вычисления числа $1 /a$ так, чтобы
расчётные формулы не содержали операций деления. Определить
область сходимости метода при $a>0$.
\end{hiProb}
\begin{sol}
%Пусть
%\[
%	F(x)= \frac{1}{ax}-1
% ,\] 
%\[
%	F(x)=af(x),\quad F'(x)=\frac{f(x)}{a}
%.\] 
%\[
%	\frac{dF}{dx}=a f'(x)= \frac{f(x)}{a}
%.\] 
%\[
%\frac{df}{f}=\frac{dx}{a^2}
%.\] 
%\[
%\ln f=\frac{x}{a^2}+C
%.\] 
%\[
%	f =\tilde{C}e^{\frac{x}{a^2}}
%.\] 
%\[
%	F(x)=a \tilde{C}e^{\frac{x}{a^2}}
%.\] 
%\[
%\frac{F}{F'}=e^{-n}
%.\] 
%тогда корнем данного уравнения будет являться число $x =1 /a$.
%
%Строим итерационный процесс
%\[
%x^{n+1}
%.\] 
	Пусть
\[
	F(x)= \frac{1}{x^2}-a^2=0
.\] 
Строим итерационный процесс
\[
	x^{n+1}= \frac{x_n \left(3-a^2 (x^n)^2\right)}{2}
.\] 
%Функция $F(x)$ определена и дважды непрерывно дифференцируема
%на любом интервале $|x-x^0|<r\; (r>0)$ который не включает
%точку $x=0$. $F(x^0)\neq 0$, если $x^0 \neq a$,
%$F'(x)= - 2 /x^3\neq 0 \;\forall x$.
%\[
%	M=\sup \left| \frac{F''(x)}{F'(x^0)} \right|=
%	\sup \left| \frac{(3 x^0)^3}{x^4} \right|
%	=\sup \left| \frac{(3x^0)^3}{\inf\limits_{|x-x^0|<r}|x|} \right| >0
%.\] 
%\[
%	2M \cdot \left| \frac{F(x^0)}{F'(x^0)} \right| =
%	2M\cdot \left| \frac{1}{4}x^0\left(a^2 (x^0)^2-2\right) \right| 
%.\] 
Т.\:к. $F'(1 /a)=-2a^3\neq 0$ то $\exists U_\delta$:
\[
	\forall x \in U_\delta: \left| \frac{FF''(x)}{(F'(x))^2} \right|= \left| \frac{3}{2}(1-a^2 x^2) \right|  <1
.\]
Откуда находим область сходимости
\[
	\left(-\frac{1}{a} \sqrt{\frac{5}{3}} ,\,-\frac{1}{\sqrt{3} a} \right) \cup \left( \frac{1}{\sqrt{3} a},\, \frac{1}{a}\sqrt{\frac{5}{3}}  \right) 
.\] 
\end{sol}
\begin{hiProb}[11.14]
Для нахождения положительного корня нелинейного уравнения
\[
	x \ln (x+2) +x^2-1=0,
\] 
предложено несколько вариантов МПИ:
\begin{enumerate}
	\item $x_{n+1}=(1-x_n^2) / \ln(x_n+2)$,
	\item $x_{n+1}=\exp (1 /x_n-x_n)-2$,
	\item  $x_{n+1}=\sqrt{1-x_n \ln(x_n+2)} $,
	\item $x_{n+1}= 1 /x_n -\ln (x_n+1)$
\end{enumerate}
Исследовать эти методы
и сделать выводы о целесообразности использования каждого
из них.
\end{hiProb}
\begin{sol}
График данного нелинейного уравнения, а также графики
всех $f'(x)$ задаваемых соотношениями $x_{n+1}=f(x_n)$
представлены на рис.~\ref{fig:1}.\begin{figure}[]
	\centering
	\begin{tikzpicture}
		\begin{axis}[
			xmin= 0.5, xmax= 0.7,
			ymin= -25, ymax= 1,
			axis lines = middle,
legend pos=south east,
		]
			\addplot[tabbrown,domain=0.5:0.7, samples=100]{x*
				ln(x+2)+x^2-1};
			\addplot[tabblue,domain=0.5:0.7, samples=100]{
			(-1+x^2-2*x*(2+x)*ln(2+x))/((2+x)*(ln(2+x))^2};
			\addplot[tabred,domain=0.5:0.7, samples=100]{exp(1/x-x)*(-1-1/(x^2))};
			\addplot[taborange,domain=0.5:0.7, samples=100]{
		(-(x/(2 + x)) - ln(2 + x))/(2.*sqrt(1 - x*ln(2 + x)))	};
			\addplot[tabgreen,domain=0.5:0.7, samples=100]{
	-x^(-2) - 1/(1 + x)	};
			\addplot[dashed,domain=0.5:0.7, samples=100]{
	-1};
\legend{$F(x)$,$f'_1(x)$,$f_2'(x)$,$f_3'(x)$, $f_4'(x)$,$-1$},
		\end{axis}
	\end{tikzpicture}
	\caption{}
	\label{fig:1}
\end{figure}
В некоторой окрестности единственного
положительного корня лишь одна $f'_3(x)$ может принимать значения
меньше единицы. Следовательно лишь третий МПИ будет сходиться
в окрестности положительного корня данного нелинейного уравнения.
\end{sol} 
\begin{hiProb}[11.1а]
Определить область изменения параметров $a$, $b$ и $c$, при
которых метод простой итерации $x^{(n+1)}=\phi\left( x^{(n)} \right) $ сходится для любого начального приближения $x_0 \in \mathbb{R}$, если 
\[
	\phi(x)=a \sin x +b \cos x+c
.\] 
\end{hiProb}
\begin{sol}
Найдём
\[
	\phi'(x)=a \cos x-b \sin x
 ,\] 
\[
	\phi''(x)=-a \sin x- b \cos x
.\] 
Для нахождения экстремумов $\phi'(x)$ решим уравнение
\[
	\phi''(x)=0 \implies x_\text{extr} = -\arctg \frac{b}{a}
.\]
Тогда
\[
	\phi'(x_{\text{extr}})=\sqrt{a^2+b^2} 
.\] 
И условие сходимости будет выглядеть как
\[
\sqrt{a^2+b^2} <1
.\] 
\end{sol}
\begin{hiProb}[12.4б]
Отделить корни уравнения
\[
3x+4x^3-12x^2-5=0
 ,\] 
а затем уточнить один из них с помощью подходящего итерационного
процесса.
\end{hiProb}
\begin{sol}
Построим график функции
\[
	f(x)=3x+4x^3-12x^2-5
\] 
(см. рис.~\ref{fig:2}).
\begin{figure}[]
	\centering
	\begin{tikzpicture}
		\begin{axis}[
			xmin= -2, xmax= 4,
			ymin= -50, ymax = 30,
			xlabel={$x$},
			ylabel={$f(x)$},
			axis lines = middle,
		]
			\addplot[domain=-2:4, samples=100]{3*x+4*x^3-12*x^2-5};
		\end{axis}
	\end{tikzpicture}
	\caption{}
	\label{fig:2}
\end{figure}
Из рисунка очевидно, что исследуемое уравнение имеет лишь один
корень на действительной прямой.

График функции
\[
	 \frac{f(x) f''(x)}{(f'(x))^2}=
\frac{8 (x-1) \left(4 x^3-12 x^2+3 x-5\right)}{3 \left(4 x^2-8 x+1\right)^2}
\]
в окрестности $(2,8;\,3,0)$ ($f(2,8)<0,\; f(3,0)>0$) изображён
на рис.~\ref{fig:3}.
\begin{figure}[]
	\centering
	\begin{tikzpicture}
		\begin{axis}[
			xmin= 2.8, xmax= 3,
			axis lines = middle,
			xlabel={$x$},
			ylabel={$\dfrac{f(x) f''(x)}{(f'(x))^2}$},
		]
			\addplot[domain=2.8:3, samples=100]{(8*(-1 + x)*(-5 + 3*x - 12*x^2 + 4*x^3))/(3.*(1 - 8*x + 4*x^2)^2)};
		\end{axis}
	\end{tikzpicture}
	\caption{}
	\label{fig:3}
\end{figure}
Заметим, что
\[
	\left| \frac{f(x)f''(x)}{(f'(x))^2} \right| <1
\]
на $(2,8;\,3,0)$ поэтому метод Ньютона в данном случае будет
сходиться. Применим его:
\[
x_{n+1}=\frac{8 x_n^3-12 x_n^2+5}{12 x_n^2-24 x_n+3} 
.\] 
На 12-й итерации метода получим значение, не отличающеюся от
11-й с точностью до машинной погрешности
\[
x_{12}=2,8901453858415
.\] 
\end{sol}
\begin{hiProb}[12.5а]
Вычислить с точностью $\epsilon=10^{-3}$ координаты точек пересечения кривых
\[
	\sin(x+1)-y=1.2,\quad 2x+\cos y=2
.\] 
\end{hiProb}
\begin{sol}
Построение к данной задаче приведено на рис.~\ref{fig:4}.
\begin{figure}[]
	\centering
	\begin{tikzpicture}
		\begin{axis}[
			xmin= -3, xmax= 3,
			ymin= -3, ymax = 4,
			%axis lines = middle,
			grid=both,
			legend pos= north west
		]
		\addplot[tabgreen,domain=-3:3, samples=100]{sin(deg(x+1))-1.2};
		\addplot[tabblue,domain=0.5:1.5, samples=100]{-rad(acos(2*(1-x)))};
		\addplot[taborange,domain=0.5:1.5, samples=100]{rad(acos(2*(1-x)))};
		\legend{$\sin(1+x)-1.2$,$-\arccos (2(x-1))$,$\arccos (2(x-1))$}
		\end{axis}
	\end{tikzpicture}
	\caption{}
	\label{fig:4}
\end{figure}
Видно, что у данной системы есть единственное решение на пересечении $\sin (1+x)-1.2$ и $-\arccos(2(x-1))$.
В нашем случае
\[
	\mathbf{u}= \begin{pmatrix} x\\y \end{pmatrix} ,\quad
	\mathbf{F}(\mathbf{u})=
	\begin{pmatrix} y+1.2-\sin(x+1)\\2x+\cos y -2 \end{pmatrix} 
.\] 
\[
	\mathbf{J}=	\frac{d\mathbf{F}}{d\mathbf{u}}= \begin{pmatrix} 
	\dfrac{\partial F_1}{\partial x} &
\dfrac{\partial F_1}{\partial y}\\
\dfrac{\partial F_2}{\partial x} &
\dfrac{\partial F_2}{\partial y} \end{pmatrix} 
=\begin{pmatrix} 
-\cos(1+x)& 1\\ 2& -\sin y\end{pmatrix} 
.\] 
Как начальное приближение выберем
\[
	\mathbf{u}^0= \begin{pmatrix} 0.5 \\-0.2 \end{pmatrix} 
.\] 
Следующее приближение к корню согласно методу Ньютона тогда
будет построено как
\[
	\mathbf{u}^{k+1}\approx \mathbf{u}^k +\mathbf{J}^{-1}\cdot
	\mathbf{F}\left( \mathbf{u}^k \right) 
.\]
Данную систему можно представить в эквивалентном виде
\[
	\mathbf{u}=\mathbf{f}(\mathbf{u})
,\]
где
\[
	\mathbf{f}(\mathbf{u})=\begin{pmatrix} 
	1- 0.5 \cos y\\ \sin (x+1)-1.2\end{pmatrix} 
.\] 
Значит для неё можно построить МПИ
\[
	\mathbf{u}^{n+1}=\mathbf{f}\left( \mathbf{u}^n \right) 
.\] 
Ниже приведена программная реализация данного алгоритма
на Mathematica.
\begin{mmaCell}[moredefined={RealAbs,testcell, f, u, i},morepattern={x_, y_, y, x, k_, k}]{Input}
  f[x_,y_]:=\{1-0.5 Cos[y], Sin[x+1]-1.2\}
  u[0] = \{0.5, -0.2\};
  u[k_]:=u[k]=f[x,y]/.\{x\(\pmb{\to}\)u[k-1][[1]],y\(\pmb{\to}\)u[k-1][[2]]\}
  i=1;
  While[
  RealAbs[u[i][[1]]-u[i-1][[1]]]>\mmaSup{10}{-3}u[i-1][[1]]&&
  RealAbs[u[i][[2]]-u[i-1][[2]]]>\mmaSup{10}{-3}u[i-1][[2]],
  i++
  ]
  i
  u[i]
\end{mmaCell}
\begin{mmaCell}[]{Output}
  16
\end{mmaCell}
\begin{mmaCell}{Output}
  \{0.5101297456452708`,-0.20183868954136786`\}
\end{mmaCell}
Откуда с точностью до $\epsilon=10^{-3}$ находим координаты
точки пересечения:
\[
	\mathbf{u}=\begin{pmatrix} 
	0.510\\ -0.202\end{pmatrix} 
.\] 
\end{sol}
\begin{hiProb}[6.1]
	Привести пример функции $f(x)$, заданной на множестве
	$U=R$ и обладающей следующим свойством
\renewcommand{\labelenumi}{\asbuk{enumi})}
	\begin{enumerate}
		\item глобальный минимум $f(x)$ достигается на
			счётном множестве точек;
		\item  $f(x)$ имеет бесконечное число точек
			локального минимума, но глобальный
			минимум $f(x)$ на $U$ не достигается;
		\item $f(x)$ ограничена снизу на $U$, но не
			достигает минимума.
	\end{enumerate}
\end{hiProb}
\begin{sol}
\renewcommand{\labelenumi}{\asbuk{enumi})}
\begin{enumerate}
\item $\sin x$ ;
\item $x \sin x$ ;
\item $\arctg x$.
\end{enumerate}
\end{sol}
\section*{Тема V. Задачи нахождения экстремума функции}
\begin{hiProb}[7.1а]
Используя методы дихотомии и сведения вариационной задачи к
задаче алгебраического уравнения, найти точку локального минимума
функции
\[
	f(x)= 2x^2 -\ln x
.\] 
\end{hiProb}
\begin{sol}
График данного уравнения приведён на рис.~\ref{fig:5}.
\begin{figure}[htpb]
	\centering
	\begin{tikzpicture}
		\begin{axis}[
			xmin= 0, xmax= 2.25,
			ymin= 0, ymax = 9,
			axis lines = middle,
			ylabel={$f(x)$},
			xlabel={$x$}
		]
		\addplot[domain=0:2.25, samples=100]{2*x^2-ln(x)};
		\end{axis}
	\end{tikzpicture}
	\caption{}
	\label{fig:5}
\end{figure}
Из построения видно, что минимум локализован на отрезке $[0,\,1]$.
Далее представлен код в Mathematica для метода дихотомии.
\begin{mmaCell}[moredefined={u,x,min,f, a, b, i},morepattern={x_, n_, n}]{Input}
  f[x_]:=2\mmaSup{\mmaPat{x}}{2}-Log[\mmaPat{x}]
  a[0]=0.;
  b[0]=1.;
  \mmaDef{\(\pmb{\Delta}\)}[n_]:=\mmaFrac{b[n]-a[n]}{100}
  \mmaSub{u}{1}[n_]:=\mmaFrac{b[n-1]+a[n-1]-\mmaDef{\(\pmb{\Delta}\)}[n-1]}{2}
  \mmaSub{u}{2}[n_]:=\mmaFrac{b[n-1]+a[n-1]+\mmaDef{\(\pmb{\Delta}\)}[n-1]}{2}
  a[n_]:=a[n]=Piecewise[\{\{a[n-1],f[\mmaSub{u}{1}[n]]\(\pmb{\leq}\) f[\mmaSub{u}{2}[n]]\},
  \{\mmaSub{u}{1}[n],f[\mmaSub{u}{1}[n]]\(\pmb{\geq}\)f[\mmaSub{u}{2}[n]]\}\}]
  b[n_]:=b[n]=Piecewise[\{\{\mmaSub{u}{2}[n],f[\mmaSub{u}{1}[n]]\(\pmb{\leq}\) f[\mmaSub{u}{2}[n]]\},
  \{b[n-1],f[\mmaSub{u}{1}[n]]\(\pmb{\geq}\)f[\mmaSub{u}{2}[n]]\}\}]
  \mmaSub{x}{min}[n_]:=\mmaFrac{a[n]+b[n]}{2}
  i=1;While[\mmaSub{x}{min}[i]=!=\mmaSub{x}{min}[i-1],i++] 
  i
  \mmaSub{x}{min}[i]
\end{mmaCell}
\begin{mmaCell}[]{Output}
  61
\end{mmaCell}
\begin{mmaCell}{Output}
  0.49999988529080724`
\end{mmaCell}
Теперь найдём точку локального минимума методом сведения
вариационной задачи к задаче алгебраического уравнения:
\[
	f'(x)=4x-\frac{1}{x}=0 \implies x_{\text{min}}= \frac{1}{2}
.\] 
\end{sol}
\begin{hiProb}[7.5а]
Методом покоординатного спуска найти точки локального минимума
функции
\[
	f(x,\,y)=\left( x^2+y-11 \right) +(x+y^2-7)^2
.\] 
\end{hiProb}
\begin{sol}
	График данной функции представлен на рис.~\ref{fig:6}.
	\begin{figure}[htpb]
\begin{tikzpicture}
\begin{axis}
\addplot3[
    surf,
]
{
	( x^2+y-11 ) +(x+y^2-7)^2
};
\end{axis}
\end{tikzpicture}
\caption{}
\label{fig:6}
\end{figure}
	Если рассмотреть функцию $f(x,\,y)$ как функцию одной
	переменной $x$ при фиксированном $y$, то минимум
	для неё можно найти следующим образом
	\[
		f_y'(x_{\text{min}})=0 \implies x_\text{min}=
		\frac{1}{2}\left( 7-y^2 \right) 
	.\]
	Аналогично для $y$ 
	\[
		f_x'(y_\text{min})=0\implies4 x y_{\min }+4 y_{\min }^3-28 y_{\min }+1=0
	.\]
	Аналитические выражения можно получить для всех трёх
	вообще говоря комплексных корней данного уравнения.
	Из них как минимум один будет действительным, в
	процессе метода будем выбирать именно его (один
	из пары, если таковых два).
%	Откуда
%	\[y_{\text{min}}=
%\frac{1}{12} \sqrt[3]{\sqrt{(48 x-336)^3+46656}-216}-\frac{4 (x-7)}{\sqrt[3]{\sqrt{(48 x-336)^3+46656}-216}}	
%	.\] 
	За начальное приближение возьмём пару
	$(x,\,y)=(0,\,0)$. Код в Mathematica для метода
	покоординатного спуска приведён далее
\begin{mmaCell}[moredefined={x, y, i},morepattern={n_, n}]{Input}
  x[0]=0.;
  y[0]=0.;
  x[n_]:=x[n]=\mmaFrac{1}{2} (7-\mmaSup{y[n-1]}{2})
  y[n_]:=y[n]=y/.First[
  Solve[1+4 y(-7+x+\mmaSup{y}{2})==0,y,\(\mmaDef{\mathbb{R}}\)]/.x\(\pmb{\to}\)x[n]]
  i=1;While[x[i]=!=x[i-1]&&y[i]=!=y[i-1],i++] 
  i
  x[i]
  y[i]
\end{mmaCell}
\begin{mmaCell}[]{Output}
  51
\end{mmaCell}
\begin{mmaCell}{Output}
  -0.09325687922159798`
\end{mmaCell}
\begin{mmaCell}{Output}
  -2.680767382381992`
\end{mmaCell}
Откуда координаты минимума
\[
x_\text{min} = -0.09325687922159798
,\] 
\[
y_\text{min} =
  -2.680767382381992
.\] 
\end{sol}
\section*{Тема VI. Интерполяция}
\begin{hiProb}[7.3]
Доказать, что если узлы интерполяции расположены симметрично
относительно некоторой точки $c$, а значения интерполируемой
функции в симметричных узлах равные, то интерполяционный
многочлен в форме Лагранжа --- чётная функция аргумента $x-c$
\end{hiProb}
\begin{proof}
Интерполяционный многочлен в форме Лагранжа имеет следующий вид
\[
	P_n(x)=P_n(x,\,f,\,x_0,\ldots,\,x_n)=f(x_0)l_0(x)+
	f(x_1)l_1(x)+\ldots+f(x_{n})l_n(x)
,\] 
где
\[
	l_k= \frac{(x-x_0)\ldots(x-x_{k-1})(x-x_{k+1})\ldots
	(x-x_{n})}{(x_k-x_0)\ldots(x_k-x_{k-1})(x_k-x_{k+1})\ldots
(x_k-x_n)}
.\] 
Пусть $y=x-c$. Тогда
\[
	l_k= \frac{(y+c-x_0)\ldots(y+c-x_{k-1})(y+c-x_{k+1})\ldots
	(y+c-x_{n})}{(x_k-x_0)\ldots(x_k-x_{k-1})(x_k-x_{k+1})\ldots
(x_k-x_n)}
.\]
Пронумеруем узлы следующим образом: узел находящийся в точке $c$ (если таковой имеется) будет иметь индекс 0, все остальные узлы
будут нумероваться по порядку удаления от $c$, т.\:е. 1, 2, 3 \ldots, $N$ справа и $-1$, $-2$, $-3$ \ldots, $N$ слева.
Также введём обозначение
$\Delta_n=x_n-c$.
Тогда
\[
	l_k= \frac{y\ldots (y^2 -\Delta_{k-1}^2)(y+\Delta_k)
	(y^2-\Delta_{k+1}^2)\ldots (y^2-\Delta_{N}^2)}{
(\Delta_k-\Delta_{-N}) \ldots(\Delta_k-\Delta_{k-1})(\Delta_k-\Delta_{k+1})\ldots(\Delta_k-\Delta_N)}
.\] 
Из симметрии задачи следует, что для любых $k$ и $n$ выполняются
соотношения
\[
\Delta_k=-\Delta_{-k},\quad\Delta_k-\Delta_n=\Delta_{-n}-\Delta_{-k}
.\] 
Следовательно
\[
	l_k(-y)=l_{-k}(y)
.\] 
Для случая отсутствия точки $x_0$ аналогично доказывается
такое же тождество.
Окончательно
\begin{multline*}
	P_n(-y)=f(x_0)l_0(-y)+
	f(x_1)l_1(-y)+\ldots+f(x_{n})l_n(-y)=\\=
	f(x_0)l_0(y)+ f(x_1)l_1 (y)+\ldots +
	f(x_{n})l_n(y)=P_n(y)
.\end{multline*}
%\[
%	l_k(-y)= \frac{(2c-x-x_0)\ldots(2c-x-x_{k-1})(2c-x-x_{k+1})\ldots
%	(2c-x-x_{n})}{(x_k-x_0)\ldots(x_k-x_{k-1})(x_k-x_{k+1})\ldots
%(x_k-x_n)}
%.\]
%\[
%c+
%.\] 
\end{proof}
\begin{hiProb}[7.11]
	Показать, что на равномерной сетке норма многочлена
	\[
		\omega_{n+1}(x)=(x-x_0)(x-x_1)\ldots (x-x_n)
	,\]
	\[
		\| \omega_{n+1}(x)\|\approx n! h^{n+1},\quad
		h=(x_n-x_0)/n
	.\] 
\end{hiProb}
\begin{sol}
%Пусть $x_k=x_0+k h$, $y=x-x_0$. Тогда
%\[
%	\omega_{n+1}(x)=(x-x_0)\left(x-x_0-h\right)\ldots\left(x-x_0 -n h\right)
%.\] 
%\[
%	\omega_{n+1}(x)=y\left(y-h\right)\ldots\left(y-n h\right)
%.\]
Очевидно, что для $x \in [x_{i-1},\,x_i]$
\[
|x-x_{i-1}| |x-x_i|\le \frac{h^2}{4}
.\] 
Для $j<i-1$ очевидна оценка $|x-x_j|\le (i-j)h$, а
для $j>i$ --- $|x-x_i|\le (j+i-i)h$. Учитывая все
эти оценки, получим для $x \in [x_{i-1},\,x_i]$ 
\[
	|\omega_{n+1}(x)|= \prod_{j=0}^{n} |x-x_j|\le 
	\frac{h^2}{4}i! h^{i-1} (n+1-i)!h^{n-i}\le 
	\frac{i!(n+1-i)!}{4}h^{n+1}
.\] 

Пусть $n\ge 2$ и $i \in \{1,\,2,\ldots,\,n-1\} $ обозначает номер внутреннего интервала. Для произвольной точки
$x \in (x_{i-1},\,x_i)$ построим симметричную относительно
середины отрезка $[x_0,\,x_i]$ точку $y \in (x_0,\,x_1)$,
т.\:е. возьмём $y=x_0+x_i-x$. Тогда $|x-x_j|=|y-x_{i-j}|$ 
для всех $j = \overline{0,\,i}$, так что $|\omega_{i+1}(y)|=|\omega_{i+1}(x)|$. Но $|y-x_j|>|x-x_j|$ для
$j=i + \overline{1,\,n}$. Поэтому $|\omega_{n+1}(y)|>
\omega_{n+1}(x)$. В силу произвольности $x \in (x_{i-1},\,
 x_i)$ это означает, что максимальное по модулю
 значение  $\omega_{n+1}$ на первом отрезке строго больше
максимального значения $|\omega_{n+1}|$ на любом внутреннем. Наконец, осталось заметить, что в силу симметричного
относительно середины отрезка $[x_0,\,x_n]=[a,\,b]$
расположения узлов, этом максимум достигается также
на отрезке $[x_{n-1},\,x_n]$. Тогда, применяя предыдущую
оценку для $i=1$, получаем
\[
	\| \omega_{n+1}(x)\|\le \frac{n!}{4}h^{n+1}\approx n! h^{n+1}
.\] 
\end{sol}
\begin{hiProb}[8.7]
С какой точностью можно извлечь кубический корень из $1200$,
интерполируя функцию $f(x)= \sqrt[3]{x} $ между узлами
$x_0=10^3$, $x_1=11^3=1331$, $x_2=12^3=1728$?
\end{hiProb}
\begin{sol}
Выпишем вспомогательные многочлены
\[
	l_0=\frac{(x-x_1)(x-x_2)}{(x_0-x_1)(x_0-x_2)}
,\] 
\[
	l_1=\frac{(x-x_0)(x-x_2)}{(x_1-x_0)(x_1-x_2)}
,\]
\[
	l_2=\frac{(x-x_0)(x-x_1)}{(x_2-x_0)(x_2-x_1)}
.\]
Интерполяционный многочлен в форме Лагранжа тогда будет иметь вид
\[
	P_2(x)=f(x_0)l_0+f(x_1)l_1+f(x_2)l_2
.\] 
Погрешность алгебраической интерполяции можно оценить как
\[
	\| R_2(x)\|\le  \frac{\left\| \left( \sqrt[3]{x}  \right)''' \right\|}{3!}\| (x-x_0)(x-x_1)(x-x_2)\|
.\] 
\[
	\| R_2(1200)\|\approx0,005
.\] 
%\[
%	(x-x_0)(x-x_1)(x-x_2)'=0\implies x_{1,\,2}^\text{extr}=
%\frac{1}{3} \left(4059\pm\sqrt{398577}\right)
%.\] 
%\[
%\| (x-x_0)(x-x_1)(x-x_2)\|=\| (x^\text{extr}_1-x_0)(x^\text{extr}_1-x_1)(x^\text{extr}_1-x_2)\|\approx 2.16 \cdot 10^7
%.\] 
%\[
%	\left( \sqrt[3]{x}  \right) '''=\frac{10}{27 x^{8/3}},\quad
%	\left\|\left( \sqrt[3]{x}  \right) '''\right\|=
%	\frac{10}{27x_0^{8 /3}}=3.70\cdot 10^{-9}
%.\] 
%\[
%	\| R_2 (x)\|\le 1.33 \cdot 10^{-2}
%.\] 
\end{sol}
\begin{hiProb}[8.13]
Задана табличная функция.
\begin{table}[htpb]
	\centering
	\label{tab:1}
	\begin{tabular}{|c|c|c|c|c|}
	\hline
	$x$&0  &$\pi /6$  &$\pi /4$  &$\pi /3$  \\ \hline
	$\sin (x)$ & 0& 0.5 & 0.71&0.87 \\ \hline
	\end{tabular}
\end{table}
С какой точностью можно восстановить значение в точке $x = 7 \pi /24$,
если известно, что функция в узлах задана с абсолютной погрешностью, не превосходящей $10^{-2}$?
\end{hiProb}
\begin{sol}
\[
	l_0= \frac{(x-x_1)(x-x_2)(x-x_3)}{(x_0-x_1)(x_0-x_2)
	(x_0-x_3)}
.\] 
\[
	l_1=\frac{(x-x_0)(x-x_2)(x-x_3)}{(x_1-x_0)(x_1-x_2)(
	x_1-x_3)}
.\] 
\[
	l_2= \frac{(x-x_0)(x-x_1)(x-x_3)}{(x_2-x_0)(x_2-x_1)
	(x_2-x_3)}
.\] 
\[
	l_3= \frac{(x-x_0)(x-x_1)(x-x_2)}{(x_3-x_0)(x_3-x_1)(x_3-x_2)}
.\] 
\[
	P_3(x)=f(x_0)l_0+f(x_1)l_1+f(x_2)l_2+f(x_3)l_3
.\] 
\[
	L_3(x)= \sum_{k=0}^{3} |l_k(x)|
.\] 
Возмущение интерполяционного многочлена при возмущении задания
функции в узлах на $10^{-2}$ можно оценить как
\[
	\left\| P_3\left(\frac{7\pi}{24},\,\delta f\right)\right\|\le \left\| \delta f\left(\frac{7\pi}{24}\right)\right\|L_3\left(\frac{7\pi}{24}\right)=
	10^{-2}\cdot 1.44
.\] 
Погрешность интерполяции
\[
	\left\| R_3\left( \frac{7\pi}{24} \right) \right\|
	\le \frac{\left\lVert  \left. \sin  x ^{(4)} \right|_{x= \frac{7\pi}{24}}\right\rVert}{4!}\left\| \omega_4\left( \frac{7\pi}{24} \right) \right\|\approx 2 \cdot 10^{-4}
.\] 
\end{sol}
\begin{hiProb}[8.18в]
Выбрать вид аппроксимации Паде и найти коэффициенты аппроксимации
для функции
\[
	f(x)=x^2 \exp \left( -x^2 \right) 
.\] 
\end{hiProb}
\begin{sol}
\[
	f(x) = x^2-x^4-\ldots
.\] 
Вычислим аппроксимацию Паде $[5 /1]$ этой функции
 \[
	 \frac{a_0+a_1 x+a_2 x^2+a_3 x^3}{1+b_1 x}=x^2-x^4 +O(x^6)
,\] 
или, умножая на знаменатель,
\[
	a_0+a_1 x+a_2 x^2+a_3 x^3+a_4 x^4+a_5 x^5=(1+b_1 x)\left(x^2-x^4+O(x^6)\right)
.\] 
Откуда
\[
	[5 /1]= \frac{x^2+x^3-x^4-x^5}{1+x}
.\] 
\end{sol}
\begin{hiProb}[9.1в]
Методом обратной интерполяции найти корень нелинейного
уравнения
\[
4x- \cos x=0
,\] 
используя приведённую таблицу. Оценить точность
полученного решения.
\begin{table}[htpb]
	\centering
	\label{tab:2}
	\begin{tabular}{|c|c|c|c|c|}
		\hline
	$x$ & $x_1=0$  & $x_2=0.1$ & $x_3=0.3$ & $x_4=0.5$  \\
	\hline $f(x)$ & $-1$ & $-0.595$ &$0.245$ & $1.122$ \\ \hline
	\end{tabular}
\end{table}
\end{hiProb}
\begin{sol}
\[
	f'(x)=4+\sin x \neq 0\; \forall x\implies \exists f^{-1}(x)=
	x(f)
.\] 
\[
	l_0= \frac{(f-f_1)(f-f_2)(f-f_3)}{(f_0-f_1)(f_0-f_2)
	(f_0-f_3)}
.\] 
\[
	l_1=\frac{(f-f_0)(f-f_2)(f-f_3)}{(f_1-f_0)(f_1-f_2)(
	f_1-f_3)}
.\] 
\[
	l_2= \frac{(f-f_0)(f-f_1)(f-f_3)}{(f_2-f_0)(f_2-f_1)
	(f_2-f_3)}
.\] 
\[
	l_3= \frac{(f-f_0)(f-f_1)(f-f_2)}{(f_3-f_0)(f_3-f_1)(f_3-f_2)}
.\] 
\[
	P_3(f)=x(f_0)l_0+x(f_1)l_1+x(f_2)l_2+x(f_3)l_3
.\] 
\[
	P_3(0)\approx 0.243
.\] 
\[
	L_3(x)= \sum_{k=0}^{3} |l_k(x)|
.\] 
Входные данные даны с точностью $10^{-3}$. Погрешность, обусловленная возмущением такого порядка можно оценить следующим образом
\[
	\| P_3(0,\,\delta f)\|\le  \|\delta f \| 
	L_3(0)=10^{-3}\cdot 1.40
.\] 
Погрешность интерполяции
\[
	\| R_3\left( 0 \right) \|\le \frac{\| - \cos 0\|}{4!}
	\| \omega_4 (0)\|\approx 6.81 \cdot 10^{-3}
.\] 
\end{sol}
\begin{hiProb}[9.2c]
	Для функции, заданной таблично, найти значение $f'(3.)$ с максимально возможной точностью
\begin{enumerate}
\item с помощью интерполяции
\item методом неопределённых коэффициентов
\end{enumerate}
\begin{table}[htpb]
	\centering
	\label{tab:3}
	\begin{tabular}{|c|c|c|c|c|c|}
		\hline
	$x$ & $x_0=0$ & $x_1=1$ & $x_2=2$ & $x_3=3$ & $x_4=4$\\
	\hline $f(x)$& $4$& $2.5$ &$1$ & $-1$  &$-2$ \\
	\hline
	\end{tabular}
\end{table}
\end{hiProb}
\begin{sol}
\begin{enumerate}
\item 
\begin{gather*}
l_0=\frac{\left(x-x_1\right) \left(x-x_2\right) \left(x-x_3\right) \left(x-x_4\right)}{\left(x_0-x_1\right) \left(x_0-x_2\right) \left(x_0-x_3\right)
   \left(x_0-x_4\right)},\\l_1=
   \frac{\left(x-x_0\right) \left(x-x_2\right) \left(x-x_3\right) \left(x-x_4\right)}{\left(x_1-x_0\right) \left(x_1-x_2\right)
   \left(x_1-x_3\right) \left(x_1-x_4\right)},\\l_2=
   \frac{\left(x-x_0\right) \left(x-x_1\right) \left(x-x_3\right) \left(x-x_4\right)}{\left(x_2-x_0\right)
   \left(x_2-x_1\right) \left(x_2-x_3\right) \left(x_2-x_4\right)},\\l_3=
   \frac{\left(x-x_0\right) \left(x-x_1\right) \left(x-x_2\right)
   \left(x-x_4\right)}{\left(x_3-x_0\right) \left(x_3-x_1\right) \left(x_3-x_2\right) \left(x_3-x_4\right)},\\l_4=
   \frac{\left(x-x_0\right) \left(x-x_1\right)
   \left(x-x_2\right) \left(x-x_3\right)}{\left(x_4-x_0\right) \left(x_4-x_1\right) \left(x_4-x_2\right) \left(x_4-x_3\right)}.
\end{gather*}
\[
	P_4(x)=f(x_0)l_0+f(x_1)x_1+f(x_2)x_2+f(x_3)x_3+
	f(x_4)x_4
.\] 
\[
	P'_4(3)\approx -1.9
.\] 
\item \[
		f'(x)= a_0 f(x_0)+a_1 f(x_1)+a_2 f(x_2)+
		a_3 f(x_3)+ a_4 f(x_4)
.\] 
\[
	f(x_0)=f(x)+f'(x)(x_0-x)+\frac{1}{2} f''(x)(x_0-x)^2+
	\frac{1}{6}f'''(x)(x_0-x)^3+\ldots
.\] 
\[
	f(x_1)=f(x)+f'(x)(x_1-x)+\frac{1}{2} f''(x)(x_1-x)^2+
	\frac{1}{6}f'''(x)(x_1-x)^3+\ldots
.\]
\[
	f(x_2)=f(x)+f'(x)(x_2-x)+\frac{1}{2} f''(x)(x_2-x)^2+
	\frac{1}{6}f'''(x)(x_2-x)^3+\ldots
.\]
\[
	f(x_3)=f(x)+f'(x)(x_3-x)+\frac{1}{2} f''(x)(x_3-x)^2+
	\frac{1}{6}f'''(x)(x_3-x)^3+\ldots
.\]
\[
	f(x_4)=f(x)+f'(x)(x_4-x)+\frac{1}{2} f''(x)(x_4-x)^2+
	\frac{1}{6}f'''(x)(x_4-x)^3+\ldots
.\]
\[
\left\{
\begin{aligned}
a_0+a_1+a_2+a_3+a_4=0\\
a_0(x_0-x)+a_1(x_1-x)+a_2(x_2-x)+a_3(x_3-x)+a_4(x_4-x)=1\\
a_0(x_0-x)^2+a_1(x_1-x)^2+a_2(x_2-x)^2+a_3(x_3-x)^2+a_4(x_4-x)^2=0\\
a_0(x_0-x)^3+a_1(x_1-x)^3+a_2(x_2-x)^3+a_3(x_3-x)^3+a_4(x_4-x)^3=0\\
a_0(x_0-x)^4+a_1(x_1-x)^4+a_2(x_2-x)^4+a_3(x_3-x)^4+a_4(x_4-x)^4=0\\
\end{aligned}
\right.
.\] 
Решая данную систему, находим
\[
a_0= -\frac{1}{12},\quad a_1= \frac{1}{2},\quad a_2= -\frac{3}{2},\quad a_3= \frac{5}{6},\quad a_4= \frac{1}{4}
.\] 
Откуда
\[
	f'(x)=4 a_0+2.5 a_1+a_2-a_3-2 a_4= -\frac{23}{12}\approx -1.9
.\] 
\end{enumerate}
\end{sol}
\section*{Тема VII. Численное интегрирование}
\begin{hiProb}[6.1]
Доказать, что вычисление интеграла от строго выпуклой функции
$f''>0$ по формулам прямоугольников со средней точкой даёт
заниженное значение интеграла.
\end{hiProb}
\begin{proof}
%:При вычислении интеграла по формуле прямоугольников
%со средней точкой
%мы приближаем функцию на всём промежутке $[x_k,\,x_{k+1})$ 
%её значением в средней точке $\xi_k=0.5(x_{k+1}+x_k)$,
%а соответственно интеграл на данном промежутке принимаем
%приближённо равным
%\[
%	I_k=(x_{k+1}-x_k)f(\xi_k)
%.\] 
%В то время, как точное значение интеграла на данном промежутке
%\[
%	I_k =\int\limits_{x_k}^{x_{k+1}} f(x)dx 
%.\] 
По определению строго выпуклой функции на
промежутке $[a,\,b]$
\[
	\forall u,\,v \in [a,\,b],\, u\le v,\, \forall t \in [0,\,1]:
	f(tu+(1-t)v)\le  t f(u)+(1-t)f(v)
.\] 
Тогда для $t=\frac{1}{2}$ 
\[
	f\left( \frac{x_1+x_2}{2} \right) \le \frac{f(x_1)+f(x_2)}{2}
.\] 
\begin{multline*}
	\frac{1}{b-a} \int\limits_{a}^{b} f(x) dx=
	\frac{1}{b-a} \left( \int\limits_{a}^{(a+b) /2} f(x)dx +\int\limits_{(a+b) /2}^{b} f(x) dx   \right) =\\=
	\frac{1}{2} \int\limits_{0}^{1} \left[ 
	f\left( \frac{a+b-t(b-a)}{2} \right) +
f\left( \frac{a+b+t(b-a)}{2} \right) \right] dt \ge f
\left( \frac{a+b}{2} \right) 
.\end{multline*} 
Откуда тривиально следует доказываемое утверждение.
\end{proof}
\begin{hiProb}[8.3]
	Пусть $\triangle ABC$ --- треугольник в плоскости $(x,\,y)$, точки $M,\,N,\,K$ --- середины его сторон. Показать, что
	квадратурная формула
\[
	\iint\limits_{\triangle ABC}^{}  f(x,\,y) dxdy=\frac{1}{3}
	S_{\triangle ABC} \left( f(M)+f(N)+f(K) \right) 
\]
точна для всех многочленов второй степени
\[
	f(x,\,y)=a_{11}x^2 +a_{12}xy +a_{22}y^2 +b_1x+b_2y+c
.\] 
\end{hiProb}
\begin{sol}
%Для дальнейших вычислений нужно заменой координат привести треугольник к более простому виду. Назовём координаты $i$-й вершины
%треугольника $(x_i,\,y_{i})$.
%Нам нужно построить отображение вида
%\[
%	(x_i,\,y_i)\mapsto (\tilde{x}_i,\,\tilde{y}_i)\quad
%	\tilde{x}_i=c_{11} x_i+c_{12} y_i +c_{13},\quad
%	\tilde{y}_i=c_{21}x_i+c_{22}y_i+c_{23}
%,\] 
%переводящее
%\[
%	(x_1,\,y_1)\mapsto (1,\,0),\quad
%	(x_2,\,y_2)\mapsto (0,\,1),\quad
%	(x_3,\,y_3)\mapsto (0,\,0)
%.\] 
%Составляя и решая систему уравнений относительно $c_{ij}$ 
%получаем
%\[
%c_{11}= -\frac{y_3-y_2}{y_1-2 y_2+y_3},\quad c_{12}= \frac{1}{y_1-2 y_2+y_3},\quad
%c_{13}= -\frac{3 y_2-2 y_3}{y_1-2 y_2+y_3},
% \] 
%\[
%c_{21}= -\frac{y_1-y_3}{y_1-2
%   y_2+y_3},\quad
%c_{22}= -\frac{2}{y_1-2 y_2+y_3},\quad
%c_{23}= -\frac{y_3-3 y_1}{y_1-2 y_2+y_3}
%.\] 
Для упрощения интеграла в левой части данного равенства
сначала отобразим первую вершину в начало координат,
вторую вершину в точку $(1,\,0)$ и третью вершину в
точку $(0,\,1)$. Т.\:е.
\[
	x(s,\,t)=x_1+(x_2-x_1)s+(x_3-x_1)t
 ,\]
\[
	y(s,\,t)=y_1+(y_2-y_1)s+(y_3-y_1)t
.\] 
Откуда
\[
	\frac{\partial (x,\,y)}{\partial (s,\,t)} =
	(x_2-x_1)(y_3-y_1)-(x_3-x_1)(y_2-y_1)=
	2S_{\triangle ABC}
.\] 
Обозначим $\phi(s,\,t)=f(x(s,\,t),\,y(s,\,t))$ и
\[
	\phi(s,\,t)=\xi_{11}s^2 +\xi_{12} st+
	\xi_{22}t^2+\eta_1 s+\eta_2 t +\zeta
.\]
Тогда
\[
	f(M)=\phi\left(0,\, \frac{1}{2}\right)=
	\zeta +\frac{\eta _2}{2}+\frac{\xi _{22}}{4},\quad
,\] 
\[
	f(N)=\phi\left( \frac{1}{2},\,0 \right)=
\zeta +\frac{\eta _1}{2}+\frac{\xi _{11}}{4}
.\] 
\[
	f(K)=\phi\left(\frac{1}{2},\frac{1}{2}\right)=
	\zeta +\frac{\eta _1}{2}+\frac{\eta _2}{2}+\frac{\xi _{11}}{4}+\frac{\xi _{12}}{4}+\frac{\xi _{22}}{4}
.\] 
Следовательно
\begin{multline*}
	\iint\limits_{\triangle ABC}^{}  f(x,\,y)dxdy=
	2S_{\triangle ABC} \int\limits_{0}^{1} 
	\int\limits_{0}^{1-s} \phi(s,\,t)dt ds 
	=\\=2S_{\triangle ABC}\left( 
\frac{\zeta }{2}+\frac{\eta _1}{6}+\frac{\eta _2}{6}+\frac{\xi _{11}}{12}+\frac{\xi _{12}}{24}+\frac{\xi _{22}}{12}	\right) =\\=
\frac{1}{3}S_{\triangle ABC}(f(M)+f(N)+f(K))
.\end{multline*} 
\end{sol}
\begin{hiProb}[8.12з]
Построить квадратуру Гаусса-Кристоффеля с двумя узлами для
вычисления интеграла:
\[
	\int\limits_{-1}^{1} \sqrt{1-x^2} f(x) dx 
.\] 
\end{hiProb}
\begin{sol}
Весовая функция
\[
	p(x)=\sqrt{1-x^2} 
.\] 
Найдём $c_1,\; c_2,\; x_1,\; x_2$:
\[
\left\{
\begin{aligned}
	\int\limits_{-1}^{1} \sqrt{1-x^2}\,dx&= c_1+c_2,\\
	\int\limits_{-1}^{1} x \sqrt{1-x^2}\,dx&=c_1 x_1+c_2 x_2,\\
	\int\limits_{-1}^{1} x^2\sqrt{1-x^2}\,dx &=c_1 x_1^2+c_2 x_2^2,\\
	\int\limits_{-1}^{1} x^3 \sqrt{1-x^2}\,dx &=c_1 x_1^3+c_2 x_2^3. 
\end{aligned}
\right.
\Leftrightarrow 
\left\{
\begin{aligned}
c_1+c_2 &= \frac{\pi}{2}, \\
c_1 x_1+c_2 x_2 &= 0, \\
c_1 x_1^2+c_2 x_2^2 &= \frac{\pi}{8}, \\
c_1 x_1^3+c_2 x_2^3 &= 0, \\
\end{aligned}
\right.
\Leftrightarrow
\left\{
\begin{aligned}
	c_{1,\,2}&=\frac{\pi}{4},\\
	x_{1,\,2}&=\pm\frac{1}{2}.\\
\end{aligned}
\right.
\] 
Следовательно квадратурная формула Гаусса-Кристоффеля будет
иметь следующий вид
\[
	\int\limits_{-1}^{1} f(x) \sqrt{1-x^2} \, dx=
	\frac{\pi}{4}\left( f\left(\frac{1}{2}\right)+f
	\left( -\frac{1}{2} \right) \right) 
.\] 
\end{sol}
\begin{hiProb}[8.25б]
Предложить алгоритм вычисления интеграла
\[
	\int\limits_{0}^{3} \frac{\sin \left( \sqrt{x}  \right) }{\sqrt{3x^2-x^3} }dx 
\] с заданной точностью
$\epsilon $, используя метод регуляризации подынтегральной
функции.
\end{hiProb}
\begin{sol}
%	На заданном промежутке $3x^2-x^3\ge 0 $, следовательно
%\begin{multline*}
%	I= \int\limits_{0}^{3} \frac{\sin\left( \sqrt{x}  \right) }{\sqrt{3x^2-x^3} }dx= \int\limits_{0}^{3} \frac{\sin \left( \sqrt{x}  \right) }{x \sqrt{3-x} }  dx\xlongequal[]{3-x=t^2}
%	\int\limits_{\sqrt{3} }^{0} \frac{\sin\left( 
%	\sqrt{3-t^2} \right) }{3-t^2}dt =\\=
%	\int\limits_{\sqrt{3} }^{ 0  } \frac{\sin\left( 
%	\sqrt{\left(\sqrt{3} -t\right)\left(\sqrt{3} +t\right)} \right) }{\left(\sqrt{3} -t\right)\left(\sqrt{3}+t \right)}dt
%	\xlongequal[]{y=\sqrt{3} -t}
%	\int\limits_{0 }^{\sqrt{3} } \frac{
%	}{}  
%.\end{multline*}
У данного интеграла две особенности: $x=0$ и $x=3$.
В рассматриваемом случае  представим его в виде
суммы трёх интегралов:
\[
I=I_1+I_2+I_3,
\] 
\[
I_1= \int\limits_{0}^{\delta_1} \frac{
\sin\left( \sqrt{x}  \right) }{\sqrt{3x^2-x^3} }dx,\quad
I_2= \int\limits_{\delta_1}^{3-\delta_2} \frac{\sin\left( \sqrt{x}  \right) }{\sqrt{3x^2-x^3} }dx ,\quad
I_3= \int\limits_{3-\delta_2}^{3} \frac{\sin\left( \sqrt{x}  \right) }{\sqrt{3x^2-x^3} }dx 
.\] 

Второй интеграл особенности не содержит и вычисляется по
любой квадратурной формуле. Вопрос о выборе величин $\delta_1$ 
и $\delta_2$
обсуждается ниже.

Первый интеграл с требуемой точностью вычисляем аналитически,
используя представление подынтегральной функции в окрестности
особой точки $(x=0)$ в виде отрезка ряда по степеням $x$
\begin{multline*}
I_1= \int\limits_{0}^{\delta_1} dx
\frac{\dfrac{x^{1/2}}{1!}-\dfrac{x^{3/2}}{3!}+\ldots+(-1)^m \dfrac{x^{(2m+1)/2}}{
(2m+1)!}}{\sqrt{3} x}\times\\ \times\left( 1+ \frac{-1 /2}{1!}\left( -\frac{x}{3} \right) + \frac{(-1 /2)(-3 /2)}{2!}\left( - \frac{x}{3} \right) ^2 +\ldots+\frac{C_{-1 /2}^n}{n!}\left( - \frac{x}{3} \right) ^n\right) =
\\=\int\limits_{0}^{\delta_1} dx \left(\frac{1}{\sqrt{3x} }+
%\frac{x^{3 /2}}{45\sqrt{3} }+
O\left( x^{3 /2} \right) \right)=
\frac{2 \sqrt{\delta_1} }{\sqrt{3} }+
%\frac{2\delta^{5 /2}}{225\sqrt{3} }+
O\left( \delta_1^{5 /2} \right) 
.\end{multline*} 
%\[
%I_1= \int\limits_{0}^{\delta}
%\frac{\dfrac{x^{1/2}}{1!}-\dfrac{x^{3/2}}{3!}+\ldots+(-1)^m \dfrac{x^{(2m+1)/2}}{
%(2m+1)!}}{\sqrt{3x^2-x^3}} dx
%.\]
%Пусть
%\[
%I_3=\int\limits_{0}^{\delta}
%\frac{x^{1/2}}{\sqrt{3x^2-x^3}} dx
%.\] 
Для выбора параметра $\delta_1$ имеем следующий критерий
\[
\frac{2\sqrt{\delta_1} }{\sqrt{3} }\le \frac{\epsilon}{3}
.\]

Третий интеграл также с требуемой точностью  вычислим аналогично
аналитически
\begin{multline*}
	I_3=\int\limits_{3-\delta_2}^{3} \frac{\sin\left( \sqrt{x}  \right) }{x\sqrt{3-x} }dx \xlongequal[]{t=3-x} 
	\int\limits_{0}^{\delta_2} \frac{\sin\left( 
	\sqrt{3-t} \right) }{(3-t)\sqrt{t} }dt =
	\int\limits_{0}^{\delta_2} dt\left(\frac{\sin \sqrt{3}   }{3\sqrt{t} }+O\left(t^{1 /2}\right) \right)=\\=
	\frac{2}{3} \sqrt{\delta_2} \sin \sqrt{3} 
.\end{multline*}
Критерий для $\delta_2$:
\[
\frac{2}{3}\sqrt{\delta_2} \sin\sqrt{3} \le \frac{\epsilon}{3}
.\] 
Оставшуюся $\epsilon /3$ отведём в качестве допустимого уровня
погрешности  при вычислении $I_2$.
\end{sol}
\begin{hiProb}[9.5б]
Для функции, заданной таблично, вычислить значение определённого
интеграла методом трапеций, сделать уточнение результата
экстраполяцией Ричардсона. Сравнить уточнённый результат с
вычислениями по методу Симпсона.
\begin{table}[htpb]
	\centering
	\label{tab:95}
	\begin{tabular}{|c|c|c|c|c|c|}
		\hline
	 $x$& $0$ & $0.125$ & $0.25$ & $0.375$ &$0.5$   \\ \hline
	 $f(x)$  &0.000000& 0.021470 &0.293050  & 0.494105 &0.541341  \\\hline
	 $x$&$0.625$  &$0.75$  & $0.875$ & $1$&\\ \hline
$f(x)$ & 
0.516855&0.468617  & 0.416531 &0.367879&  \\ \hline
	\end{tabular}
\end{table}
\end{hiProb}
\begin{sol}
Метод трапеций:
\begin{mmaCell}[morepattern={i_, i, k_},morefunctionlocal={k},moredefined={Ih}]{Input}
  \mmaSub{x}{i_}:=0.125i
  \mmaSub{f}{i_}:=\{0.,0.021470,0.293050,0.494105,0.541341,
  0.516855,0.468617,0.416531,0.367879\}[[i+1]]
  \mmaSub{h}{k_}:=\mmaSub{x}{\mmaPat{k}+1}-\mmaSub{x}{\mmaPat{k}}
  Ih=\mmaUnderOver{\(\pmb{\sum}\)}{k=0}{7}\mmaFrac{\mmaSub{h}{k}}{2}(\mmaSub{f}{k+1}+\mmaSub{f}{k})
\end{mmaCell}
\begin{mmaCell}[]{Output}
  0.3669885625`
\end{mmaCell}
Уточнение результата экстраполяцией Ричардсона:
\begin{mmaCell}[moredefined={I2h, IR, Ih},morefunctionlocal={k}]{Input}
  I2h=\mmaUnderOver{\(\pmb{\sum}\)}{k=0}{3}\mmaFrac{\mmaSub{h}{2k}+\mmaSub{h}{2k+1}}{2}(\mmaSub{f}{2k+2}+\mmaSub{f}{2k});
  IR= Ih+\mmaFrac{Ih-I2h}{\mmaSup{2}{2}-1}
\end{mmaCell}
\begin{mmaCell}[]{Output}
  0.3654057916666667`
\end{mmaCell}
Вычисления по методу Симпсона:
\begin{mmaCell}[moredefined={IS},morefunctionlocal={k}]{Input}
  IS=\mmaUnderOver{\(\pmb{\sum}\)}{k=0}{3}\mmaFrac{\mmaSub{h}{2k}+\mmaSub{h}{2k+1}}{6}(\mmaSub{f}{2k+2}+4\mmaSub{f}{2k+1}+\mmaSub{f}{2k})
\end{mmaCell}
\begin{mmaCell}{Output}
  0.36540579166666665`
\end{mmaCell}
Уточнённый экстраполяцией Ричардсона результат совпадает со
значением, полученным методом Симпсона, чего и следовало ожидать.
\end{sol}
\begin{hiProb}[9.10г]
Вычислить несобственный интеграл
\[
	\int\limits_{0}^{1} \frac{\ln \left( x^2+1 \right) }{\sqrt{x} }dx 
\] 
с точностью $10^{-4}$. Укажите и сравните различные приёмы
для решения данной задачи.
\end{hiProb}
\begin{sol}
Сделаем замену $x=t^2$:
\begin{multline*}
	I=2\int\limits_{0}^{1}  \ln\left( t^4+1 \right) dt=
	\int\limits_{-1}^{1} \ln(t^4+1)dt=
	\left. 2 \ln (t^4+1)t
 \right|_0^1-2 \int\limits_{0}^{1} \frac{4t^3dt}{t^4+1}t=\\=
 2 \ln (2)-8+8 \int\limits_{0}^{1} \frac{dt}{t^4+1}=\ldots
 \\ =
2\ln 2 - 8+\sqrt{2} \left( \ln \frac{2+\sqrt{2} }{2-\sqrt{2} }+
2 \arctg \left(1+\sqrt{2} \right)-2 \arctg\left(1-\sqrt{2}\right) \right)\approx 0.322078 .\end{multline*} 
Для достижения требуемой точности методом
Гаусса проведём оценку
остаточного члена интегрирования:
\[
	r_2^{\Gamma}= \frac{1}{135}\max_{\xi \in [-1,\,1]} \left|f^{(4)}(\xi)\right|\approx 0.18,\quad
	r_3^{\Gamma}= \frac{1}{15750}\max_{\xi \in [-1,\,1]} \left|f^{(6)}(\xi)\right|\approx 0.12
\] 
\[
	r_4^{\Gamma}= \frac{2}{3472875}\max_{\xi \in [-1,\,1]} \left|f^{(8)}(\xi)\right|\approx 0.053,\quad
	r_5^{\Gamma}= \frac{13}{1237732650}\max_{\xi \in [-1,\,1]} \left|f^{(10)}(\xi)\right|\approx 0.24\ldots
\] 
\[
	r_{10}^\Gamma= \frac{2^{21}(10!)^4}{21(20!)^3}\max_{\xi \in [-1,\,1]} \left|f^{(20)}(\xi)\right|\approx 6.2 \cdot 10^{-5}
.\] 
Возможно, искать нули и веса для полиномов Лежандра 10-й
степени не лучшая идея, так что перейдём к другим методам\ldots

На первый взгляд особой точки в данном интеграле нет, т.\:к. в окрестности нуля $\ln\left( x^2+1 \right) \sim x^2$ .
Однако $f''(0)\to \infty$, поэтому воспользуемся методом замены
переменных, т.\:е. будем искать интеграл в виде
\[
	I=2 \int\limits_{0}^{1} \ln(t^4+1)dt 
.\] Как было показано выше, методами замены переменных
и интегрирования по частям интеграл вычислился точно.

Будем вычислять данный интеграл по формуле трапеций.
Сперва найдём шаг сетки для заданной точности
\[
	10^{-4}=\epsilon \le \frac{1}{12} \max_{0\le x\le 1} |f''(x)|h^2\implies h<0.007
.\] 
Шаг $h=0.005$ удовлетворит нашу потребность в точности.
Далее приведён код в Mathematica для формулы трапеций. 
\begin{mmaCell}[morepattern={i_, i, k_},morefunctionlocal={k},moredefined={Ih}]{Input}
  \mmaSub{x}{i_}:=0.005i
  \mmaSub{f}{i_}:=2Log[\mmaSup{\mmaSub{x}{i}}{4}+1]
  \mmaSub{h}{k_}:=\mmaSub{x}{\mmaPat{k}+1}-\mmaSub{x}{\mmaPat{k}}
  Ih=\mmaUnderOver{\(\pmb{\sum}\)}{k=0}{199}\mmaFrac{\mmaSub{h}{k}}{2}(\mmaSub{f}{k+1}+\mmaSub{f}{k})
\end{mmaCell}
\begin{mmaCell}[]{Output}
  0.3220865931864013`
\end{mmaCell}
Результат с нужной точностью согласуется с аналитическим, чего и следовало ожидать.
\end{sol}
\begin{hiProb}[9.13б]
	Функция $f(x)$ задана своими сеточными значениями.
	Найти $\int\limits_{a}^{b} f(x) \sin 80x \, dx $ 
	построением сплайна для аппроксимации $f(x)$.
	\begin{table}[htpb]
		\centering
		\label{tab:913}
		\begin{tabular}{|c|c|c|c|c|c|}\hline
		 $i$& $0$ & $1$ &$2$  &$3$  &$4$ \\ \hline 
		 $x_{i}$& $0.1$  & $0.5$ & $0.9$ &$1.3$  &$1.7$ \\ \hline 
		 $f_{i}$ & $-2.3026$ & $-0.69315$ & $-0.10536$ & $0.26236$ & $0.53063$ \\ \hline 
		\end{tabular}
	\end{table}
\end{hiProb}
\begin{sol}
Код для построения свободного кубического сплайна и его последующего интегрирования в Mathematica:
\begin{mmaCell}[morepattern={i_, x_, x},morefunctionlocal={i, j},moredefined={f, sol}]{Input}
  \mmaSub{p}{i_}[x_]:=\mmaSub{a}{\mmaPat{i}} \mmaSup{x}{3}+\mmaSub{b}{\mmaPat{i}} \mmaSup{x}{2}+\mmaSub{c}{\mmaPat{i}} x+\mmaSub{d}{\mmaPat{i}}
  \mmaSub{\mmaUnd{x}}{i_}:=\{0.1,0.5,0.9,1.3,1.7\}[[\mmaPat{i}+1]]
  \mmaSub{f}{i_}:=\{-2.3026,-0.69315,-0.10536,0.26236,0.53063\}[[\mmaPat{i}+1]]
  sol=Solve[\{\mmaSub{p}{1}[\mmaSub{\mmaUnd{x}}{0}]==\mmaSub{f}{0},\mmaSub{p}{4}[\mmaSub{\mmaUnd{x}}{4}]==\mmaSub{f}{4}, \mmaSub{p}{1}''[\mmaSub{\mmaUnd{x}}{0}]==0,\mmaSub{p}{4}''[\mmaSub{\mmaUnd{x}}{4}]==0,
  Table[\mmaSub{p}{i}[\mmaSub{\mmaUnd{x}}{i}]==\mmaSub{p}{i+1}[\mmaSub{\mmaUnd{x}}{i}]==\mmaSub{f}{i},\{i,3\}],
  Table[Derivative[j][\mmaSub{p}{i}][\mmaSub{\mmaUnd{x}}{i}]==
  Derivative[j][\mmaSub{p}{i+1}][\mmaSub{\mmaUnd{x}}{i}],\{i,3\},\{j,2\}]\}//Flatten,
  Table[\{\mmaSub{a}{i},\mmaSub{b}{i},\mmaSub{c}{i},\mmaSub{d}{i}\},\{i,4\}]//Flatten]
  f[x_]:=Piecewise[\{\{\mmaSub{p}{1}[x],\mmaSub{x}{0}<x<\mmaSub{x}{1}\},\{\mmaSub{p}{2}[x],\mmaSub{x}{1}<x<\mmaSub{x}{2}\},
  \{\mmaSub{p}{3}[x],\mmaSub{x}{2}<x<\mmaSub{x}{3}\},\{\mmaSub{p}{4}[x],\mmaSub{x}{3}<x<\mmaSub{x}{4}\}\}]/.sol
  \mmaSubSupM{\int}{\mmaSub{\mmaFnc{x}}{0}}{\mmaSub{\mmaFnc{x}}{4}}f[\mmaFnc{x}]Sin[80\mmaFnc{x}]d\mmaFnc{x}
\end{mmaCell}
\begin{mmaCell}[]{Output}
  \{\{\mmaSub{a}{1}\(\to\)-4.0580552455357095`,\mmaSub{b}{1}\(\to\)1.217416573660713`,
  \mmaSub{c}{1}\(\to\)4.551172181919641`,\mmaSub{d}{1}\(\to\)-2.7658333286830357`,
  \mmaSub{a}{2}\(\to\)4.3268387276785525`,\mmaSub{b}{2}\(\to\)-11.35992438616068`,
  \mmaSub{c}{2}\(\to\)10.839842661830337`,\mmaSub{d}{2}\(\to\)-3.8139450753348183`,
  \mmaSub{a}{3}\(\to\)-0.7244559151785411`,\mmaSub{b}{3}\(\to\)2.27857114955348`,
  \mmaSub{c}{3}\(\to\)-1.4348033203124082`,\mmaSub{d}{3}\(\to\)-0.13155128069199595`,
  \mmaSub{a}{4}\(\to\)0.45567243303569954`,\mmaSub{b}{4}\(\to\)-2.3239294084820488`,
  \mmaSub{c}{4}\(\to\)4.5484474051338415`,\mmaSub{d}{4}\(\to\)-2.724293261718706`\}\}
\end{mmaCell}
\begin{mmaCell}[]{Output}
  \{0.007454746884931084`\}
\end{mmaCell}
Т.\:е.
\[
	\int\limits_{a}^{b} f(x) \sin 80 x \, dx \approx
	0.007455
.\] 
\end{sol}
\section*{Тема VIII. Решение ОДУ}
\begin{hiProb}[7.6]
Нелинейное автономное дифференциальное уравнение решается
с помощью явного метода Рунге-Кутты второго порядка аппроксимации
с числом стадий, равным двум. Указать коэффициенты метода, имеющего
 минимальную погрешность аппроксимации для данного класса задач.
\end{hiProb}
\begin{sol}
В указанном методе разностная схема имеет вид
\[
	\frac{y_{n+1}-y_n}{\tau}=b_1 f(y_n)+b_2 f\left( y_n+
	\tau a_{21} f(y_n)\right) 
.\] 
В таком случае невязка в точке $x$ вычисляется как
\[
	\phi(x)= \frac{y(x+\tau)-y(x)}{\tau} - b_1 f\left( 
	y(x)\right) -b_2 f\left( y(x)+\tau a_{21} f\left( y(x) \right)  \right) 
 ,\] 
 где $y(\cdot)$ есть точное решение дифференциального уравнения.

Найдём условия на коэффициенты $b_1,\,b_2,\,a_{21}$ необходимые
для условий $\phi(0)$, $\phi'(0)=0$. Это и будет означать, что
 разностная схема имеет второй порядок аппроксимации.

Имеем, что
\[
	\phi(0)= y''(x)-2 a_{12} b_2 f(y(x))f'(y(x))
.\] 
Так как 
\[
	y''(x)= f'(y(x))y'(x)=f'(y(x))f(y(x))
 ,\] 
 то верно, что $\phi'(0)=0$ при $1= 2 a_{12} b_2$.

Проверим, что в общем случае вторую производную $\phi$ 
невозможно занулить. Имеем, что  
\begin{multline*}
	\phi''(0)= y'''(x)- 3a_{21}^2 b_2 f(y(x))^2f''(y(x))=
\\=	f(y(x))^2 f''(y(x))(1-3a_{21}^2b_2)+f'(y(x))^2f(y(x))
.\end{multline*} 
Видим, что засчёт выбора коэффициентов невозможно для всякой
функции $f$ занулить величину $\phi''(0)$.

Введём параметр $\lambda$. Тогда можно выразить все коэффициенты
через данный параметр:
\[
	b_1=\lambda,\quad b_2=1-\lambda,\quad a_{12}=\frac{1}{2(1-\lambda)}
.\] 
Видим, что параметр $\lambda$ может пробегать множество $\mathbb{R}\setminus {1}$.

Таким образом получаем однопараметрическое семейство
коэффициентов, которые дают требуемый порядок аппроксимации:
\[
	b_1=\lambda,\quad b_2=1-\lambda,\quad a_{12}=\frac{1}{2(1-\lambda)}
,\]
где $\lambda$ пробегает множество $\mathbb{R} \setminus {1}$.
\end{sol}
\begin{hiProb}[9.7-2]
Приближенное решение задачи Коши
\[
	\frac{dx}{dt}=ax;\quad x(0)=X_0;\quad 0\le t\le T
\]
вычисляется по разностной схеме
\[
\frac{x_{n+1}-\frac{4}{3}x_n+ \frac{1}{3}x_{n-1}}{2\tau}=\frac{1}{3}
ax_{n+1};\quad x_0=X_0
.\] 
Способ задания дополнительного краевого условия $x_1$ предложить
самостоятельно.

Найти порядок аппроксимации разностной схемы. Исследовать влияние способа
 задания $x_1$ на порядок аппроксимации. Исследовать на
устойчивость разностную схему. Найти точное решение разностной
задачи. Исследовать его сходимость к точному решению
дифференциальной задачи.
\end{hiProb}
\begin{sol}
%Разностное уравнение можно переписать в следующем виде
%\[
%	\frac{3x_{n+1}-4x_n+x_{n-1}}{2\tau}=f(x_{n+1})\equiv
%	f_{n+1}
%.\] 
%	\[
%		\frac{1}{2}\left(\frac{x_{n+1}- x_n}{\tau}-\frac{1}{3} \frac{x_{n}
%		-x_{n-1}}{\tau}\right)= \frac{1}{3}a x_{n+1}
%	.\] 
Найдём порядок аппроксимации разностной схемы.
\begin{multline*}
r_\tau= \frac{3u_{m+1}- 4u_m +u_{m-1}}{2\tau}-
f_{m+1}=\\=
\frac{3 u_{m+1}-4\left(u_{m+1}-u'_{m+1}\tau+u''_{m+1}\frac{\tau^2}{2}+
	u'''_{m+1}\frac{\tau^3}{6}+O\left(\tau^4\right)\right)
}{2\tau}
+\\+
\frac{u_{m+1}- 2u'_{m+1}\tau+2u''_{m+1}\tau^2-
\frac{4}{3}u'''_{m+1}\tau^3+O\left(\tau^4\right)}{2\tau}-f_{m+1}=
-u'''_{m+1}\tau^2
,\end{multline*}  
--- второй порядок.

Если мы зададим $x_1$ с точностью до второго порядка или выше,
то и у всей схемы будет второй порядок, если же точность будет
первого порядка, то и вся схема будет такой же точности. Поэтому
для нахождения $x_1$ решим на первом шаге уравнение
(метод трапеций)
\[
	\frac{x_1-x_0}{\tau}=\frac{a}{2}\left(x_0+x_1  \right) 
	\implies x_1=-\frac{x_0 (a \tau +2)}{a \tau -2}
.\] 
Предположим,
что для собственных значений оператора перехода $\lambda$
последовательные значения $x$ связаны соотношением
\[
x_{n+1}=\lambda x_n,\quad x_n=\lambda x_{n-1}
.\] 
Перепишем разностную схему следующим образом
\[
	x_{n+1}\left( 1-\frac{2}{3}a\tau \right) -\frac{4}{3}
	x_n+\frac{1}{3}x_{n-1}=0
.\] 
Составим характеристическое уравнение
\[
	\lambda\left( 1-\frac{2}{3}a\tau \right) -\frac{4}{3}+
	\frac{1}{3\lambda}=0
.\] 
Его корнями являются числа
\[
\lambda_{1,\,2}= \frac{2\pm\sqrt{1+2a\tau} }{3-2a\tau}
.\] 
Таким образом получаем общее решение разностного уравнения
\[
	x_n= C_1 \left( \frac{2 -\sqrt{1+2a\tau} }{3-2a\tau} \right) ^n+C_2 \left( \frac{2+\sqrt{1+2a\tau} }{3-2a\tau} \right) ^n
,\]
где $C_1$ и $C_2$ --- произвольные константы.

Исследуем на устойчивость данную разностную схему.  Имеют
место следующие разложения
\[
	\frac{2-\sqrt{1+2a\tau} }{3-2a\tau}= \frac{1}{3}+ O(\tau)
	,\quad
	\frac{2+\sqrt{1+2a \tau} }{3-2a\tau}=1+a\tau +O\left( \tau^2 \right) 
.\] 
Видим, что при всех достаточно малых положительных значениях
$\tau$ верно, что 
\[
	|\lambda_1|\le 1+\left(|a|+1\right) \tau,\quad
	|\lambda_2|\le 1+\left( |a|+1 \right) \tau
.\] 
Т.\:е. данная разностная схема нестого устойчива.

Если $a\le 0$, то при всех достаточно малых $\tau$ верно, что
$|\lambda_1|\le 1$ и $|\lambda_2|\le 1$, поэтому в таком
случае имеет место строгая устойчивость. Если $a>0$, то $\lambda_2>1$ при всех достаточно малых положительных значения $\tau$,
поэтому в таком случае строгая устойчивость отсутствует. Таким
образом имеет место нестрогая устойчивость разностной схемы
во всех случаях, а строгая устойчивость имеет место только
при $a\le 0$.
\end{sol}
\begin{hiProb}[9.9а]
	Для решения задачи Коши на отрезке $[0,\,T]$ введена
	равномерная сетка с шагом $\tau$. Рассматривается
	дифференциальное уравнение, которому в соответствие
	ставится численный метод
	\[
		y'-y=0,\quad 0\le x\le 1,\quad y(0)=1,
	\] 
	\[
		(y_{n+1}-y_n) /h - (3y_{n+1}-y_n) /2=0,\quad
		y_0=1
	.\] 
	Выписать общее решении разностного уравнения и
	исследовать сходимость численного решения к решению
	дифференциальной задачи на основе определения сходимости.

	Пусть по данной схеме проводятся вычисления на компьютере,
	где для хранения мантиссы отводится 10 бит. Оценить
	максимальную ошибку округления для данной задачи.
\end{hiProb}
\begin{sol}
Запишем характеристическое уравнение
\[
y_{n+1}=\lambda y_n,\quad \frac{\lambda-1}{h}= \frac{3\lambda-1}{2}\implies \lambda=\frac{h-2}{3 h-2}
.\] 
Разностное уравнение с учётом начального условия имеет общее решение
\[
	y_n=\left( \frac{h-2}{3h-2} \right) ^n
.\] 
Исследуем сходимость решения разностного уравнения к решению
дифференциального уравнения. Общее решение дифференциального
уравнения имеет вид $y(x)=e^x$. Обозначим через $N_h$ число
узлов сетки, в которых вычисляется искомое решение, при
данном значении $h$.

Сходимость решения разностного уравнения к решению дифференциального
уравнения означает, что
\[
\lim_{h \to +0} \max_{k \in  \overline{0,\,N_h}}|y_k-
y(kh)|=0
.\]
Имеем следующие разложения по формуле Тейлора:
\[
	e^{h}=1+\tau +\frac{h^2}{2}+O\left( h^3 \right) ,\quad \frac{h-2}{3h-2}=1+h+\frac{3h^2}{2}+
	O\left( h^3 \right) 
.\] 
Видим, что при всех достаточно малых положительных значениях
$h$ верно, что $e^{h}< \frac{h-2}{3h-2}$.
Отсюда следует, что
\[
	\max_{k \in  \overline{0,\, N_h}}|y_k-y(kh)|=\left| \frac{h-2}{3h-2}-e^h \right| 
.\] 
Можно считать, что при малых $h$ верно, что $N_h h=1$, поэтому
$e^{N_h h}=e$ и $\left(\frac{h-2}{3h-2}\right)^{N_h}=
\left( \frac{h-2}{3h-2} \right) ^{2 /h}$.

Имеем следующее разложение
\[
	\left( \frac{h-2}{3h-2} \right) ^{1 /h}=e+O(h)
.\] 
Таким образом получаем, что
\[
	\lim_{h \to +0}  \max_{k \in  \overline{0,\, N_h}}|y_k-y(kt))=0
.\] 
Получаем, что решение разностного уравнения сходится к решению
дифференциального уравнения.

Пусть $y_n$ есть точное решение разностного уравнения, пусть
$\tilde{y}_n$ есть соответствующие записанные на компьютере
значения.

Имеет место следующая оценка для ошибки округления
\[
	R= \max_{k \in  \overline{0,\,N_h}}|y_k -\tilde{y}_k|
\simeq e^L |y_0-\tilde{y}_0|+2 \epsilon \frac{e^L}{L}
,\] 
где $L$ --- это константа Липшица правой части исходного
дифференциального уравнения, $\epsilon$ есть максимальная
ошибка округления на компьютере.

В нашем случае правая часть дифференциального уравнения имеет вид $y$, поэтому $L=1$. Т.\:к. в нашем случае под мантиссу отведено
10 бит, то  $\epsilon= 2^{-10}$. Т.\:к. $y_0=y(0)=1$, то
$|y_0- \tilde{y}_0|\le  \epsilon |y_0|=\epsilon$. Получаем, что
\[
	R \simeq \epsilon e^L \left( 1+\frac{2}{L} \right) 
	= \frac{1}{2^10}e(1+2) \simeq 10^{-2}
.\] 
Таким образом получаем оценку для максимальной погрешности
округления: $R \approx 10^{-2}$.
\end{sol}
\begin{hiProb}[10.3]
Приближённо решить задачу Коши:
\[
	\frac{d^2 y}{dt^2}= y \sin x,\quad y(0)=0;\;
	y'(0)=1;\, 0\le x\le 1
.\] 
\renewcommand{\labelenumi}{\asbuk{enumi})}
\begin{enumerate}
\item Описать алгоритм, основанный на переходе к системе двух
	уравнений первого порядка с последующим решением
	этой системы.
\item Описать алгоритм, основанный на замене уравнения $y''=
	y \sin x$ разностным уравнением второго порядка.
\item Решить задачу любым из этих методов
\end{enumerate}
\end{hiProb}
\begin{sol}
\renewcommand{\labelenumi}{\asbuk{enumi})}
\begin{enumerate}
\item Обозначим $y'=z$, тогда получаем систему из двух
	уравнений
	 \[
	\left\{
	\begin{aligned}
		y'&=z \\
		z'&=y \sin x
	\end{aligned}
	\right.
	.\] 
	Явная схема Эйлера для данной системы ($x_n=\tau n$)
	\[
	\left\{
	\begin{aligned}
		\frac{y_{n+1}-y_n}{\tau}&=z_n\\
	\frac{z_{n+1}-z_n}{\tau}&=y_n \sin x_n
	\end{aligned}
	\right.
	.\] 
\item Вторая производная вычисляется по приближённой формуле
	\[
		y''(x)= \frac{y(x+\tau)-2f(x)+f(x-\tau)}{\tau^2}
	.\]
Значит явная схема Эйлера для данного уравнения будет иметь вид
\[
\frac{y_{n+1}-2 y_n+y_{n-1}}{\tau^2}=y_n \sin x_n
.\] 
\item Алгоритм в Mathematica для первого метода:
\begin{mmaCell}[morepattern={n_, n},moredefined={sol,ListLinePlot},morefunctionlocal={i}]{Input}
  \mmaSub{y}{0}=0;
  \mmaSub{z}{0}=1;
  \mmaDef{\(\pmb{\tau}\)}=0.01;
  \mmaSub{x}{n_}:=\mmaSub{x}{n}=\mmaDef{\(\pmb{\tau}\)} n
  \mmaSub{y}{n_}:=\mmaSub{y}{n}=\mmaSub{z}{n-1}\mmaDef{\(\pmb{\tau}\)}+\mmaSub{y}{n-1}
  \mmaSub{z}{n_}:=\mmaSub{z}{n}=\mmaSub{y}{n-1}Sin[\mmaSub{x}{n-1}]\mmaDef{\(\pmb{\tau}\)}+\mmaSub{z}{n-1}
  sol=Table[\{\mmaSub{x}{i},\mmaSub{y}{i}\},\{i,0,100\}];
  ListLinePlot[sol]
  
\end{mmaCell}
 \begin{mmaCell}[moregraphics={moreig={scale=.5}}]{Output}
  \mmaGraphics{1}
 \end{mmaCell}
\end{enumerate}
\end{sol}
\end{document}
