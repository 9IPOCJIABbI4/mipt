\documentclass[a4paper]{article}
% Этот шаблон документа разработан в 2014 году
% Данилом Фёдоровых (danil@fedorovykh.ru) 
% для использования в курсе 
% <<Документы и презентации в \LaTeX>>, записанном НИУ ВШЭ
% для Coursera.org: http://coursera.org/course/latex .
% Исходная версия шаблона --- 
% https://www.writelatex.com/coursera/latex/5.3

% В этом документе преамбула

\usepackage{siunitx}
%%% Работа с русским языком
\usepackage{cmap}					% поиск в PDF
\usepackage{mathtext} 				% русские буквы в формулах
\usepackage[T2A]{fontenc}			% кодировка
\usepackage[utf8]{inputenc}			% кодировка исходного текста
\usepackage[english,russian]{babel}	% локализация и переносы
\usepackage{indentfirst}
\frenchspacing

\renewcommand{\epsilon}{\ensuremath{\varepsilon}}
\renewcommand{\phi}{\ensuremath{\varphi}}
\renewcommand{\kappa}{\ensuremath{\varkappa}}
\renewcommand{\le}{\ensuremath{\leqslant}}
\renewcommand{\leq}{\ensuremath{\leqslant}}
\renewcommand{\ge}{\ensuremath{\geqslant}}
\renewcommand{\geq}{\ensuremath{\geqslant}}
\renewcommand{\emptyset}{\varnothing}
\renewcommand{\Im}{\operatorname{Im}}
\renewcommand{\Re}{\operatorname{Re}}


%%% Дополнительная работа с математикой
\usepackage{amsmath,amsfonts,amssymb,amsthm,mathtools} % AMS
\usepackage{icomma} % "Умная" запятая: $0,2$ --- число, $0, 2$ --- перечисление

%% Номера формул
%\mathtoolsset{showonlyrefs=true} % Показывать номера только у тех формул, на которые есть \eqref{} в тексте.
%\usepackage{leqno} % Нумереация формул слева

%% Свои команды
\DeclareMathOperator{\sgn}{\mathop{sgn}}
\DeclareMathOperator{\sign}{\mathop{sign}}
\DeclareMathOperator*{\res}{\mathop{res}}
\DeclareMathOperator*{\tr}{\mathop{tr}}

%% Перенос знаков в формулах (по Львовскому)
\newcommand*{\hm}[1]{#1\nobreak\discretionary{}
{\hbox{$\mathsurround=0pt #1$}}{}}

%%% Работа с картинками
\usepackage{graphicx}  % Для вставки рисунков
\graphicspath{{figures/}}  % папки с картинками
\setlength\fboxsep{3pt} % Отступ рамки \fbox{} от рисунка
\setlength\fboxrule{1pt} % Толщина линий рамки \fbox{}
\usepackage{wrapfig} % Обтекание рисунков текстом

%%% Работа с таблицами
\usepackage{array,tabularx,tabulary,booktabs} % Дополнительная работа с таблицами
\usepackage{longtable}  % Длинные таблицы
\usepackage{multirow} % Слияние строк в таблице

%%% Теоремы
\theoremstyle{plain} % Это стиль по умолчанию, его можно не переопределять.
\newtheorem{theorem}{Теорема}
\newtheorem*{thm}{Теорема}
\newtheorem{prop}{Утверждение}
 
\theoremstyle{definition} % "Определение"
%\newtheorem{corollary}{Следствие}[theorem]
\newtheorem*{dfn}{Определение}
\newtheorem{problem}{Задача}
\newtheorem*{problem*}{Задача}

 
\theoremstyle{remark} % "Примечание"
\newtheorem*{sol}{Решение}
\newtheorem*{rem}{Замечание}

%%% Программирование
\usepackage{etoolbox} % логические операторы

%%% Страница
%\usepackage{extsizes} % Возможность сделать 14-й шрифт
%\usepackage{geometry} % Простой способ задавать поля
%	\geometry{top=25mm}
%	\geometry{bottom=35mm}
%	\geometry{left=35mm}
%	\geometry{right=20mm}
 
\usepackage{fancyhdr} % Колонтитулы
%	\pagestyle{fancy}
 %	\renewcommand{\headrulewidth}{0pt}  % Толщина линейки, отчеркивающей верхний колонтитул
	%\lfoot{Нижний левый}
	%\rfoot{Нижний правый}
	%\rhead{Верхний правый}
	%\chead{Верхний в центре}
	%\lhead{Верхний левый}
	%\cfoot{Нижний в центре} % По умолчанию здесь номер страницы

\usepackage{setspace} % Интерлиньяж
%\onehalfspacing % Интерлиньяж 1.5
%\doublespacing % Интерлиньяж 2
%\singlespacing % Интерлиньяж 1

\usepackage{lastpage} % Узнать, сколько всего страниц в документе.

\usepackage{soul} % Модификаторы начертания

\usepackage{hyperref}
%\usepackage[usenames,dvipsnames,svgnames,table,rgb]{xcolor}
\hypersetup{				% Гиперссылки
    unicode=true,           % русские буквы в раздела PDF
    pdftitle={Заголовок},   % Заголовок
    pdfauthor={Автор},      % Автор
    pdfsubject={Тема},      % Тема
    pdfcreator={Создатель}, % Создатель
    pdfproducer={Производитель}, % Производитель
    pdfkeywords={keyword1} {key2} {key3}, % Ключевые слова
    colorlinks=true,       	% false: ссылки в рамках; true: цветные ссылки
    linkcolor=red,          % внутренние ссылки
    citecolor=black,        % на библиографию
    filecolor=magenta,      % на файлы
    urlcolor=cyan           % на URL
}

\usepackage{csquotes} % Еще инструменты для ссылок

%\usepackage[style=apa,maxcitenames=2,backend=biber,sorting=nty]{biblatex}

\usepackage{multicol} % Несколько колонок

\usepackage{tikz} % Работа с графикой
\usepackage{pgfplots}
\usepackage{pgfplotstable}
%\usepackage{coloremoji}
\usepackage{floatrow}
\usepackage{subcaption}
\newcommand*{\N}{\mathbb{N}}
\newcommand*{\R}{\mathbb{R}}
\newcommand*{\K}{\mathbb{K}}
\newcommand*{\V}{\mathcal{V}}
\newcommand*{\A}{\mathcal{A}}
\newcommand*{\ii}{\mathbf{1}}
\newcommand*{\oo}{\mathbf{0}}
\newcommand*{\ba}{\mathbf{a}}
\newcommand*{\bb}{\mathbf{b}}
\newcommand*{\Q}{\mathbb{Q}}
\graphicspath{{figures/}}
%\usepackage{breqn}

\renewcommand\thesubfigure{\asbuk{subfigure}}
%\addbibresource{master.bib}

\usepackage{import}
\usepackage{pdfpages}
\usepackage{transparent}
\usepackage{xcolor}
\usepackage{xifthen}

%\newcommand{\incfig}[1]{%
%    \def\svgwidth{\columnwidth}
%    \import{./figures/}{#1.pdf_tex}
%}


\newcommand{\incfig}[2][1]{%
    \def\svgwidth{#1\columnwidth}
    \import{./figures/}{#2.pdf_tex}
}
\usepackage{titlesec}
%\titleformat{\section}{\normalfont\Large\bfseries}{}{0pt}{}
%----------------------STANDART:
%\titleformat{\chapter}[display]
%  {\normalfont\huge\bfseries}{\chaptertitlename\ \thechapter}{20pt}{\Huge}
%\titleformat{\section}{\normalfont\Large\bfseries}{\thesection}{1em}{}
%\titleformat{\subsection}
%  {\normalfont\large\bfseries}{\thesubsection}{1em}{}
%\titleformat{\subsubsection}
%  {\normalfont\normalsize\bfseries}{\thesubsubsection}{1em}{}
%\titleformat{\paragraph}[runin]
%  {\normalfont\normalsize\bfseries}{\theparagraph}{1em}{}
%\titleformat{\subparagraph}[runin]
%  {\normalfont\normalsize\bfseries}{\thesubparagraph}{1em}{}

\pdfsuppresswarningpagegroup=1
\pgfplotsset{compat=1.16}

\usepackage{xifthen}
\makeatother
%\def\@lecture{}%
%\newcommand{\lecture}[3]{
%    \ifthenelse{\isempty{#3}}{%
%        \def\@lecture{Неделя #1}%
%    }{%
%        \def\@lecture{Неделя #1: #3}%
%    }%
%    \section*{\@lecture}
%    \marginpar{\small\textsf{\mbox{#2}}}
%}
\makeatletter

%\newcommand{\lec}{\subsection{Лекция}}
%\newcommand{\sem}{\subsection{Семинар}}
%\newcommand{\hw}{\subsection{Домашняя работа}}
%\newcommand{\prob}[1]{\textbf{#1}}
%\renewcommand{\thesubsection}{}
%\renewcommand{\thesubsubsection}{}

%\setcounter{tocdepth}{1} % only parts,chapters,sections
%\titleformat{\subsection}{\normalfont\large\bfseries}{}{0em}{}
%\titleformat{\subsubsection}{\normalfont\normalsize\bfseries}{}{0em}{}

%\newcommand{\textover}[2]{\stackrel{\mathclap{\normalfont\mbox{#2}}}{#1}}

\author{Драчов Ярослав\\
Факультет общей и прикладной физики МФТИ}
\newcommand{\veq}{\mathrel{\rotatebox{90}{$=$}}}
%\newcommand{\teto}[1]{\stackrel{\mathclap{\normalfont\tiny\mbox{#1}}}{\to}}
%\renewcommand{\thesubsection}{\arabic{subsection}}

%%\setcounter{secnumdepth}{0}

\definecolor{tabblue}{RGB}{30, 119, 180}
\definecolor{taborange}{RGB}{255, 127, 15}
\definecolor{tabgreen}{RGB}{45, 160, 43}
\definecolor{tabred}{RGB}{214, 38, 40}
\definecolor{tabpurple}{RGB}{148, 103, 189}
\definecolor{tabbrown}{RGB}{140, 86, 76}
\definecolor{tabpink}{RGB}{227, 119, 193}
\definecolor{tabgray}{RGB}{127, 127, 127}
\definecolor{tabolive}{RGB}{188, 189, 33}
\definecolor{tabcyan}{RGB}{22, 190, 207}
\pgfplotscreateplotcyclelist{colorbrewer-tab}{
{tabblue},
{taborange},
{tabgreen},
{tabred},
{tabpurple},
{tabbrown},
{tabpink},
{tabgray},
{tabolive},
{tabcyan},
}
\usepackage{csvsimple}
\usepackage{extarrows}
%\renewcommand{\labelenumii}{\asbuk{enumii})}
%\renewcommand{\labelenumiv}{\Asbuk{enumiv}}
\newcommand{\prob}[1]{\subsubsection*{#1}}
\sisetup{output-decimal-marker = {,},separate-uncertainty = true,exponent-product = \cdot}

\usepackage{braket}
\usepackage{enumerate}
\usepackage{chngcntr}
%\counterwithin*{equation}{problem}
%\usepackage{bbold}

\newtheoremstyle{hiProb}% ⟨name ⟩ 
{3pt}% ⟨Space above ⟩1 
{3pt}% ⟨Space below ⟩1
{}% ⟨Body font ⟩
{}% ⟨Indent amount ⟩2
{\bfseries}% ⟨Theorem head font⟩
{.}% ⟨Punctuation after theorem head ⟩
{.5em}% ⟨Space after theorem head ⟩3
%{\thmname{#1} \thmnote{#3}}% ⟨Theorem head spec (can be left empty, meaning ‘normal’)⟩
{\thmnote{#3}}% ⟨Theorem head spec (can be left empty, meaning ‘normal’)⟩
\theoremstyle{hiProb} % "Определение"
%\newtheorem{hiProb}{Задача}
\newtheorem{hiProb}{}
\usepackage{mmacells}
\newcommand{\textover}[2]{\stackrel{\mathclap{\normalfont\scriptsize\mbox{#2}}}{#1}}
\usepackage{units}
\usepackage[math]{cellspace}%
\setlength\cellspacetoplimit{2pt}
\setlength\cellspacebottomlimit{2pt}

\title{Работа над ошибками\\
Вариант 2}
\begin{document}
	\maketitle
\prob{Задача 3}
Далее будем считать, что
\[
	\mathbf{f}=\begin{pmatrix} 2 \\ -2 \\ 2 \end{pmatrix} 
.\] 
\begin{enumerate}
	\item Вычислим евклидовы нормы матриц $\mathbf{A}$ и $\mathbf{A}^{-1}$:
\[
	\| \mathbf{A}\|_3=5,\qquad \| \mathbf{A}^{-1}\|=1
.\] 
Тогда число обусловленности $\mu_3$ матрицы $\mathbf{A}$ будет равно:
\[
	\mu_3(\mathbf{A})=\| \mathbf{A}\|\| \mathbf{A}^{-1}\|=5
.\] 
\item Приведём разложение матрицы $\mathbf{A}$ на диагональную  $\mathbf{D}$,
нижнюю и верхнюю треугольные матрицы с нулевыми элементами на главной
диагонали ($\mathbf{L}$ и $\mathbf{U}$ --- соответственно):
\[
	\mathbf{D}= \operatorname{diag}(1,\,1,\,4),\qquad
	\mathbf{L}=\begin{pmatrix}0 & 0 & 0 \\ -2 & 0 & 0 \\ -1 & 1 & 0
	\end{pmatrix}, \qquad \mathbf{U} =\mathbf{L}^T
.\]
Пользуясь следующими обозначениями для общего вида итерационного
метода
\[
	\mathbf{u}^{(k+1)}=\mathbf{R}\mathbf{u}^{(k)}+\mathbf{F}
\] 
запишем вычислительные формулы для
метода Якоби:
\[
	\mathbf{R}=-\mathbf{D}^{-1}(\mathbf{L}+\mathbf{U})
	=\begin{pmatrix} 0 & 2 &1\\2&0&-1\\\frac{1}{4}&
		-\frac{1}{4}&0\end{pmatrix} ,\qquad \mathbf{F}=\mathbf{D}^{-1}\mathbf{f}= \begin{pmatrix} 2 \\ -2 \\ \frac{1}{2}
	\end{pmatrix} 
 ,\]
и метода Зейделя:
\[
	\mathbf{R}=-(\mathbf{L}+\mathbf{D})^{-1}\mathbf{U}
	=\begin{pmatrix} 0 & 2&1\\0&4&1\\0&-\frac{1}{2}&0
		\end{pmatrix} ,\qquad \mathbf{F}=(\mathbf{L}+\mathbf{D})^{-1}\mathbf{f}= \begin{pmatrix} 2 \\ 2 \\ \frac{1}{2} \end{pmatrix} 
.\] 
Далее воспользуемся критерием сходимости метода Якоби
(для сходимости необходимо и достаточно, чтобы все корни уравнения
\[
	\begin{vmatrix}\lambda a_{11} & a_{12}& a_{13}\\
	a_{21} & \lambda a_{22} & a_{23} \\
a_{31} & a_{32} & \lambda a_{33}\end{vmatrix} =0
\]
по модулю не превосходили единицы):
\[
\begin{vmatrix} 
 \lambda  & -2 & -1 \\
 -2 & \lambda  & 1 \\
 -1 & 1 & 4 \lambda  \\
\end{vmatrix} =4-18\lambda +4\lambda^3=
2 (\lambda -2) \left(2 \lambda ^2+4 \lambda -1\right)=0
.\] 
Заметим, что данное уравнение имеет как минимум один корень $\lambda_1=2$, по модулю превосходящий единицу. Следовательно итерационный
метод Якоби по приведённому критерию не сходится.

Для сходимости метода Зейделя необходимо и достаточно, чтобы
все корни уравнения
 \[
	 \begin{vmatrix} \lambda a_{11} & a_{12} & a_{13} \\
	 \lambda a_{21} & \lambda a_{22} & a_{23} \\
 \lambda a_{31} & \lambda a_{32} & \lambda a_{33}\end{vmatrix}=0 
\]
по модулю не превосходили единицы.
\[
\begin{vmatrix} 
 \lambda  & -2 & -1 \\
 -2 \lambda  & \lambda  & 1 \\
 -\lambda  & \lambda  & 4 \lambda  
\end{vmatrix} =
4 \lambda ^3-16 \lambda ^2+2 \lambda=
2 \lambda  \left(2 \lambda ^2-8 \lambda +1\right)=0
.\] 
Решением данного уравнения, кроме прочих, будет корень \[\lambda_1=
\frac{1}{2}\left(4 +\sqrt{14}\right)>1. \]
Следовательно метод Зейделя по критерию в данном случае также не сходится.
\item Считая начальным приближением вектор
\[
	\mathbf{u}^{(0)}=\mathbf{x}=\begin{pmatrix} 0 \\ 2\\0 \end{pmatrix} 
\] 
и пользуясь полученными ранее значениями $\mathbf{R}$ и $\mathbf{F}$ 
для методов Якоби и Зейделя для итерационного метода вида
$\mathbf{u}^{(k+1)}= \mathbf{R} \mathbf{u}^{(k)}+\mathbf{F}$,
выполним по три итерации данных методов:
\begin{itemize}
	\item \emph{Метод Якоби}:
\[
	\mathbf{u}^{(1)}=\mathbf{R}\mathbf{u}^{(0)}+\mathbf{F}=
	\begin{pmatrix} 6 \\ -2 \\ 0 \end{pmatrix} 
,\] 
\[
	\mathbf{u}^{(2)}=\mathbf{R}\mathbf{u}^{(1)}+\mathbf{F}=
	\begin{pmatrix} -2 \\ 10 \\ \frac{5}{2} \end{pmatrix} 
,\] 

\[
	\mathbf{u}^{(3)}=\mathbf{R}\mathbf{u}^{(2)}+\mathbf{F}=
	\begin{pmatrix} \frac{49}{2} \\ -\frac{17}{2} \\ -\frac{5}{2} \end{pmatrix} 
.\] 
\item \emph{Метод Зейделя}:

\[
	\mathbf{u}^{(1)}=\mathbf{R}\mathbf{u}^{(0)}+\mathbf{F}=
	\begin{pmatrix} 6 \\ 10 \\ -\frac{1}{2} \end{pmatrix} 
,\] 

\[
	\mathbf{u}^{(2)}=\mathbf{R}\mathbf{u}^{(1)}+\mathbf{F}=
	\begin{pmatrix} \frac{43}{2} \\ \frac{83}{2} \\ -\frac{9}{2} \end{pmatrix} 
,\] 
\[
	\mathbf{u}^{(3)}=\mathbf{R}\mathbf{u}^{(2)}+\mathbf{F}=
	\begin{pmatrix} \frac{161}{2} \\ \frac{327}{2} \\ -\frac{81}{4} \end{pmatrix} 
.\] 
\end{itemize}
\end{enumerate}
\end{document}
