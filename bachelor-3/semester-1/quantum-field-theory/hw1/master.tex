\documentclass[a4paper]{article}
% Этот шаблон документа разработан в 2014 году
% Данилом Фёдоровых (danil@fedorovykh.ru) 
% для использования в курсе 
% <<Документы и презентации в \LaTeX>>, записанном НИУ ВШЭ
% для Coursera.org: http://coursera.org/course/latex .
% Исходная версия шаблона --- 
% https://www.writelatex.com/coursera/latex/5.3

% В этом документе преамбула

\usepackage{siunitx}
%%% Работа с русским языком
\usepackage{cmap}					% поиск в PDF
\usepackage{mathtext} 				% русские буквы в формулах
\usepackage[T2A]{fontenc}			% кодировка
\usepackage[utf8]{inputenc}			% кодировка исходного текста
\usepackage[english,russian]{babel}	% локализация и переносы
\usepackage{indentfirst}
\frenchspacing

\renewcommand{\epsilon}{\ensuremath{\varepsilon}}
\renewcommand{\phi}{\ensuremath{\varphi}}
\renewcommand{\kappa}{\ensuremath{\varkappa}}
\renewcommand{\le}{\ensuremath{\leqslant}}
\renewcommand{\leq}{\ensuremath{\leqslant}}
\renewcommand{\ge}{\ensuremath{\geqslant}}
\renewcommand{\geq}{\ensuremath{\geqslant}}
\renewcommand{\emptyset}{\varnothing}
\renewcommand{\Im}{\operatorname{Im}}
\renewcommand{\Re}{\operatorname{Re}}


%%% Дополнительная работа с математикой
\usepackage{amsmath,amsfonts,amssymb,amsthm,mathtools} % AMS
\usepackage{icomma} % "Умная" запятая: $0,2$ --- число, $0, 2$ --- перечисление

%% Номера формул
%\mathtoolsset{showonlyrefs=true} % Показывать номера только у тех формул, на которые есть \eqref{} в тексте.
%\usepackage{leqno} % Нумереация формул слева

%% Свои команды
\DeclareMathOperator{\sgn}{\mathop{sgn}}
\DeclareMathOperator{\sign}{\mathop{sign}}
\DeclareMathOperator*{\res}{\mathop{res}}
\DeclareMathOperator*{\tr}{\mathop{tr}}

%% Перенос знаков в формулах (по Львовскому)
\newcommand*{\hm}[1]{#1\nobreak\discretionary{}
{\hbox{$\mathsurround=0pt #1$}}{}}

%%% Работа с картинками
\usepackage{graphicx}  % Для вставки рисунков
\graphicspath{{figures/}}  % папки с картинками
\setlength\fboxsep{3pt} % Отступ рамки \fbox{} от рисунка
\setlength\fboxrule{1pt} % Толщина линий рамки \fbox{}
\usepackage{wrapfig} % Обтекание рисунков текстом

%%% Работа с таблицами
\usepackage{array,tabularx,tabulary,booktabs} % Дополнительная работа с таблицами
\usepackage{longtable}  % Длинные таблицы
\usepackage{multirow} % Слияние строк в таблице

%%% Теоремы
\theoremstyle{plain} % Это стиль по умолчанию, его можно не переопределять.
\newtheorem{theorem}{Теорема}
\newtheorem*{thm}{Теорема}
\newtheorem{prop}{Утверждение}
 
\theoremstyle{definition} % "Определение"
%\newtheorem{corollary}{Следствие}[theorem]
\newtheorem*{dfn}{Определение}
\newtheorem{problem}{Задача}
\newtheorem*{problem*}{Задача}

 
\theoremstyle{remark} % "Примечание"
\newtheorem*{sol}{Решение}
\newtheorem*{rem}{Замечание}

%%% Программирование
\usepackage{etoolbox} % логические операторы

%%% Страница
%\usepackage{extsizes} % Возможность сделать 14-й шрифт
%\usepackage{geometry} % Простой способ задавать поля
%	\geometry{top=25mm}
%	\geometry{bottom=35mm}
%	\geometry{left=35mm}
%	\geometry{right=20mm}
 
\usepackage{fancyhdr} % Колонтитулы
%	\pagestyle{fancy}
 %	\renewcommand{\headrulewidth}{0pt}  % Толщина линейки, отчеркивающей верхний колонтитул
	%\lfoot{Нижний левый}
	%\rfoot{Нижний правый}
	%\rhead{Верхний правый}
	%\chead{Верхний в центре}
	%\lhead{Верхний левый}
	%\cfoot{Нижний в центре} % По умолчанию здесь номер страницы

\usepackage{setspace} % Интерлиньяж
%\onehalfspacing % Интерлиньяж 1.5
%\doublespacing % Интерлиньяж 2
%\singlespacing % Интерлиньяж 1

\usepackage{lastpage} % Узнать, сколько всего страниц в документе.

\usepackage{soul} % Модификаторы начертания

\usepackage{hyperref}
%\usepackage[usenames,dvipsnames,svgnames,table,rgb]{xcolor}
\hypersetup{				% Гиперссылки
    unicode=true,           % русские буквы в раздела PDF
    pdftitle={Заголовок},   % Заголовок
    pdfauthor={Автор},      % Автор
    pdfsubject={Тема},      % Тема
    pdfcreator={Создатель}, % Создатель
    pdfproducer={Производитель}, % Производитель
    pdfkeywords={keyword1} {key2} {key3}, % Ключевые слова
    colorlinks=true,       	% false: ссылки в рамках; true: цветные ссылки
    linkcolor=red,          % внутренние ссылки
    citecolor=black,        % на библиографию
    filecolor=magenta,      % на файлы
    urlcolor=cyan           % на URL
}

\usepackage{csquotes} % Еще инструменты для ссылок

%\usepackage[style=apa,maxcitenames=2,backend=biber,sorting=nty]{biblatex}

\usepackage{multicol} % Несколько колонок

\usepackage{tikz} % Работа с графикой
\usepackage{pgfplots}
\usepackage{pgfplotstable}
%\usepackage{coloremoji}
\usepackage{floatrow}
\usepackage{subcaption}
\newcommand*{\N}{\mathbb{N}}
\newcommand*{\R}{\mathbb{R}}
\newcommand*{\K}{\mathbb{K}}
\newcommand*{\V}{\mathcal{V}}
\newcommand*{\A}{\mathcal{A}}
\newcommand*{\ii}{\mathbf{1}}
\newcommand*{\oo}{\mathbf{0}}
\newcommand*{\ba}{\mathbf{a}}
\newcommand*{\bb}{\mathbf{b}}
\newcommand*{\Q}{\mathbb{Q}}
\graphicspath{{figures/}}
%\usepackage{breqn}

\renewcommand\thesubfigure{\asbuk{subfigure}}
%\addbibresource{master.bib}

\usepackage{import}
\usepackage{pdfpages}
\usepackage{transparent}
\usepackage{xcolor}
\usepackage{xifthen}

%\newcommand{\incfig}[1]{%
%    \def\svgwidth{\columnwidth}
%    \import{./figures/}{#1.pdf_tex}
%}


\newcommand{\incfig}[2][1]{%
    \def\svgwidth{#1\columnwidth}
    \import{./figures/}{#2.pdf_tex}
}
\usepackage{titlesec}
%\titleformat{\section}{\normalfont\Large\bfseries}{}{0pt}{}
%----------------------STANDART:
%\titleformat{\chapter}[display]
%  {\normalfont\huge\bfseries}{\chaptertitlename\ \thechapter}{20pt}{\Huge}
%\titleformat{\section}{\normalfont\Large\bfseries}{\thesection}{1em}{}
%\titleformat{\subsection}
%  {\normalfont\large\bfseries}{\thesubsection}{1em}{}
%\titleformat{\subsubsection}
%  {\normalfont\normalsize\bfseries}{\thesubsubsection}{1em}{}
%\titleformat{\paragraph}[runin]
%  {\normalfont\normalsize\bfseries}{\theparagraph}{1em}{}
%\titleformat{\subparagraph}[runin]
%  {\normalfont\normalsize\bfseries}{\thesubparagraph}{1em}{}

\pdfsuppresswarningpagegroup=1
\pgfplotsset{compat=1.16}

\usepackage{xifthen}
\makeatother
%\def\@lecture{}%
%\newcommand{\lecture}[3]{
%    \ifthenelse{\isempty{#3}}{%
%        \def\@lecture{Неделя #1}%
%    }{%
%        \def\@lecture{Неделя #1: #3}%
%    }%
%    \section*{\@lecture}
%    \marginpar{\small\textsf{\mbox{#2}}}
%}
\makeatletter

%\newcommand{\lec}{\subsection{Лекция}}
%\newcommand{\sem}{\subsection{Семинар}}
%\newcommand{\hw}{\subsection{Домашняя работа}}
%\newcommand{\prob}[1]{\textbf{#1}}
%\renewcommand{\thesubsection}{}
%\renewcommand{\thesubsubsection}{}

%\setcounter{tocdepth}{1} % only parts,chapters,sections
%\titleformat{\subsection}{\normalfont\large\bfseries}{}{0em}{}
%\titleformat{\subsubsection}{\normalfont\normalsize\bfseries}{}{0em}{}

%\newcommand{\textover}[2]{\stackrel{\mathclap{\normalfont\mbox{#2}}}{#1}}

\author{Драчов Ярослав\\
Факультет общей и прикладной физики МФТИ}
\newcommand{\veq}{\mathrel{\rotatebox{90}{$=$}}}
%\newcommand{\teto}[1]{\stackrel{\mathclap{\normalfont\tiny\mbox{#1}}}{\to}}
%\renewcommand{\thesubsection}{\arabic{subsection}}

%%\setcounter{secnumdepth}{0}

\definecolor{tabblue}{RGB}{30, 119, 180}
\definecolor{taborange}{RGB}{255, 127, 15}
\definecolor{tabgreen}{RGB}{45, 160, 43}
\definecolor{tabred}{RGB}{214, 38, 40}
\definecolor{tabpurple}{RGB}{148, 103, 189}
\definecolor{tabbrown}{RGB}{140, 86, 76}
\definecolor{tabpink}{RGB}{227, 119, 193}
\definecolor{tabgray}{RGB}{127, 127, 127}
\definecolor{tabolive}{RGB}{188, 189, 33}
\definecolor{tabcyan}{RGB}{22, 190, 207}
\pgfplotscreateplotcyclelist{colorbrewer-tab}{
{tabblue},
{taborange},
{tabgreen},
{tabred},
{tabpurple},
{tabbrown},
{tabpink},
{tabgray},
{tabolive},
{tabcyan},
}
\usepackage{csvsimple}
\usepackage{extarrows}
%\renewcommand{\labelenumii}{\asbuk{enumii})}
%\renewcommand{\labelenumiv}{\Asbuk{enumiv}}
\newcommand{\prob}[1]{\subsubsection*{#1}}
\sisetup{output-decimal-marker = {,},separate-uncertainty = true,exponent-product = \cdot}

\usepackage{braket}
\usepackage{enumerate}
\usepackage{chngcntr}
%\counterwithin*{equation}{problem}
%\usepackage{bbold}

\newtheoremstyle{hiProb}% ⟨name ⟩ 
{3pt}% ⟨Space above ⟩1 
{3pt}% ⟨Space below ⟩1
{}% ⟨Body font ⟩
{}% ⟨Indent amount ⟩2
{\bfseries}% ⟨Theorem head font⟩
{.}% ⟨Punctuation after theorem head ⟩
{.5em}% ⟨Space after theorem head ⟩3
%{\thmname{#1} \thmnote{#3}}% ⟨Theorem head spec (can be left empty, meaning ‘normal’)⟩
{\thmnote{#3}}% ⟨Theorem head spec (can be left empty, meaning ‘normal’)⟩
\theoremstyle{hiProb} % "Определение"
%\newtheorem{hiProb}{Задача}
\newtheorem{hiProb}{}
\usepackage{mmacells}
\newcommand{\textover}[2]{\stackrel{\mathclap{\normalfont\scriptsize\mbox{#2}}}{#1}}
\usepackage{units}
\usepackage[math]{cellspace}%
\setlength\cellspacetoplimit{2pt}
\setlength\cellspacebottomlimit{2pt}

\title{Домашняя работа по теории поля}
%\titleformat{\subsubsection}{\normalfont\normalsize\bfseries}{}{0em}{}
%\newcommand{\prob}[1]{\subsubsection{#1}}
\begin{document}
	\maketitle
\prob{Задача 2.5}
\begin{sol}
%	\[
%		\mathcal{L}=a( \partial_\mu \phi)^2 +b \partial_\mu
%		\partial_\mu \phi+ c \phi\partial_\mu \partial_\mu \phi
%		+\frac{m^2}{2} \phi^2 +d\phi
%	.\] 
%	\begin{multline*}
%		S=\int d^4x \left[a( \partial_\mu \phi)^2 +b \partial_\mu
%		\partial_\mu \phi+ c \phi\partial_\mu \partial_\mu \phi
%	+\frac{m^2}{2} \phi^2 +d\phi\right]=\\
%		=\int d^4x \left[a( \partial_\mu \phi)^2 +b \partial_\mu
%		\partial_\mu \phi+ c \phi\partial_\mu \partial_\mu \phi
%	+\frac{m^2}{2} \phi^2 +d\phi\right]
%	.\end{multline*} 
%	\begin{multline*}
%		\delta S= S[\phi+\delta \phi,\,\partial_\mu \phi
%		+\delta \partial_ \mu\phi,\, \partial_\mu\partial_\mu \phi+
%		\delta\partial_\mu \partial_\mu \phi]-
%		S[\phi,\,\partial_\mu \phi
%		,\, \partial_\mu\partial_\mu \phi]=\\=
%		\int d^4x \left[2a\partial_\mu \phi \delta\partial_\mu \phi
%			+b\delta \partial_\mu
%			\partial_\mu \phi+ c\left(\partial_\mu \partial_\mu \phi\delta\phi
%			+\phi\delta\partial_\mu \partial_\mu \phi\right)
%		+m^2 \phi\delta\phi +d\delta\phi\right]
%	.\end{multline*}
%	Далее, $\delta\partial_\mu \phi= \partial_\mu\delta \phi$ и 
%	$\delta\partial_\mu\partial_\mu \phi=\partial_\mu\partial_\mu
%	\delta\phi$.
%	В первом слагаемом интегрируем по частям,
%	\begin{multline*}
%	\delta S=\\=\int d^4x \left[-2a\partial_\mu \partial_\mu\phi\delta\phi
%		+b\delta \partial_\mu
%		\partial_\mu \phi+ c\left(\partial_\mu \partial_\mu \phi\delta\phi
%		+\phi\delta\partial_\mu \partial_\mu \phi\right)
%	+m^2 \phi\delta\phi +d\delta\phi\right]+\\+\int d \Sigma^\mu[
%	\partial_\mu \phi \delta \phi]
%	,\end{multline*}
%где второй интеграл берётся по границе объёма. Как обычно, полагаем $\delta\phi=0$ 
%на границе четырёхмерного объёма, тогда второй интеграл исчезает,
%	\begin{multline*}
%		\delta S=\int d^4x \left[\left(-2a\partial_\mu\partial_\mu \phi
%		+ c\partial_\mu \partial_\mu \phi
%	+m^2 \phi +d\right)\delta\phi+
%	(c\phi+b)\delta \partial_\mu
%\partial_\mu \phi\right]=\\=
%\int d^4x \left[\left(-2a\partial_\mu\partial_\mu \phi
%		+ c\partial_\mu \partial_\mu \phi
%	+m^2 \phi +d\right)\delta\phi-
%	c\partial_\mu\phi
%\partial_\mu \delta\phi\right]+\\+\int d\Sigma^\mu\left[ (c \phi+b)
%\delta\partial_\mu
%\phi\right]\xlongequal{\text{аналог.}}
%\int d^4x \left[2(c-a)\partial_\mu\partial_\mu \phi
%	+m^2 \phi +d\right]\delta\phi
%\end{multline*}
%Требуя $\delta S=0$ для всех  $\delta \phi$, получаем уравнение
%\[
%	2(c-a)\partial_\mu\partial_\mu\left(\phi+\frac{d}{m^2}\right)+m^2 \left(\phi+\frac{d}{m^2}
%	\right)=0
%.\] 
%Пусть $\tilde{\phi}(m)$ --- решение уравнения Клейна-Гордона-Фока
%\[
%\partial_\mu\partial_\mu \phi+m^2 \phi=0
%,\]
%Тогда решение полученного нами уравнения будет представимо в виде
%\[
%	\phi(a,\,b,\,c,\,d,\,m)=
%	\begin{cases}
%		\tilde{\phi}\left( \sqrt{ \frac{m^2}{2(c-a)}} \right) -\frac{d}{m^2} & a\neq c,\\
%		-\frac{d}{m^2}& a=c.
%	\end{cases}\] 
%Аналог функции Лагранжа $L$ в теории скалярногоо поля:
%\[
%	L=\int d^3 x \mathcal{L}= \int d^3 x\left[a( \partial_\mu \phi)^2 +b \partial_\mu
%		\partial_\mu \phi+ c \phi\partial_\mu \partial_\mu \phi
%	+\frac{m^2}{2} \phi^2 +d\phi\right]=
%	\int d^3 x \left[ a \dot{\phi}^2- a \partial_i \phi
%	\partial_i \phi +b \ddot{\phi}+b \Delta \phi+ c\right] 
%.\] 
\begin{multline*}
	S= \int d^4 x\left[a( \partial_\mu \phi)^2 +b \partial_\mu
		\partial_\mu \phi+ c \phi\partial_\mu \partial_\mu \phi
	+\frac{m^2}{2} \phi^2 +d\phi\right]=\\=
	 \int d^4 x\left[a( \partial_\mu \phi)^2 +(b+ c \phi)\partial_\mu \partial_\mu \phi
	+\frac{m^2}{2} \phi^2 +d\phi\right]\xlongequal{\text{по частям}}\\
	= \int d^4 x\left[a( \partial_\mu \phi)^2 -c \partial_\mu \phi\partial_\mu \phi
	+\frac{m^2}{2} \phi^2 +d\phi\right]\xlongequal{+C}
	\\= \int d^4 x\left[(a-c)( \partial_\mu \phi)^2
+\frac{m^2}{2} \left(\phi+\frac{d}{m^2}\right)^2\right]
.\end{multline*}
Здесь видно что $b$ пропадает после интегрирования по частям, а именно,
после взятия производной $\partial_\mu(b+c\phi)$. После интегрирования по частям
также появляется член 
\[
	\int d\Sigma^\mu (b+c\phi)\partial_\mu \phi
,\]
где данный интеграл берётся по границе объёма. Однако, мы ограничиваемся
 рассмотрением случая $\delta \partial_\mu \phi=0$ на границе четырёхмерного
 объёма, и поэтому данное слагаемое обращается в нуль.

Значит
\[
	\mathcal{L}=(a-c)( \partial_\mu \phi)^2
+\frac{m^2}{2} \left(\phi+\frac{d}{m^2}\right)^2
.\] 
Аналог функции Лагранжа $L$  в теории скалярного поля
\begin{multline*}
	L= \int d^3 x \mathcal{L}= \int d^3 x \left[(a-c)( \partial_\mu \phi)^2
+\frac{m^2}{2} \left(\phi+\frac{d}{m^2}\right)^2
  \right] =\\=
  \int d^3 x \left[ 
  (a-c) \dot{\phi}^2-(a-c)\partial_i \phi \partial_i \phi
+\frac{m^2}{2} \left(\phi+\frac{d}{m^2}\right)^2
  \right] 
.\end{multline*} 
Используя
\[
	\frac{\delta \mathcal{L}}{\delta \dot{\phi}(\mathbf{x})}=
	2(a-c)\dot{\phi}(\mathbf{x}),
\]
получим для энергии
\[
	E= \int d^3 x  \frac{\delta \mathcal{L}}{\delta \dot{\phi}
	(\mathbf{x})}\dot{\phi}(\mathbf{x})-L
\]
выражение  вида
\[
E=\int d^3 x \left[ 
  (a-c) \dot{\phi}^2+(a-c)\partial_i \phi \partial_i \phi
-\frac{m^2}{2} \left(\phi+\frac{d}{m^2}\right)^2
\right] 
.\] 
Кажется, что на физические решения в данном случае стоит наложить условие
$a<c$ с тем, чтобы в ситуации большого и быстро осциллирующего поля  $\phi$
энергия не равнялась нулю.

Случай $a=c$ отвечает равенству
\[
	\frac{\delta \mathcal{L}}{\delta \dot{\phi}(\mathbf{x})}=0
.\]
По аналогии с классической механикой это можно интерпретировать как
равенство нулю пространственной плотности обобщённого импульса.
Нетрудно видеть, что в данном случае мы получаем стационарные решения
вида
\[
\phi=-\frac{d}{m^2}
.\] 
\end{sol}
\prob{Задача 2.6}
\begin{sol}
Из выражения для энергии получаем
\[
	[E]=\frac{L^3 [\phi]^2}{T^2}
,\]
\[
	[\hbar][\omega]=L[\phi]^2
.\] 
Т.к. $[\hbar]=1$, $[\omega]=1 /T$ и  $L=T=1 /M$, то
 \[
	 [\phi]=M
.\] 
Следовательно
\[
	[m]=\left( \frac{[E]}{L^3 [\phi]^2} \right) ^\frac{1}{2}=M
.\] 
\end{sol}
\prob{Задача 2.7}
\begin{sol}
	\[
		E=\int d^3 x \frac{\delta \mathcal{L}}{\delta \dot{\phi}(\mathbf{x})}
		\dot{\phi}(\mathbf{x})-L
	.\] 
%	\[
%	S=-\frac{1}{4} \int d^4 x F_{\mu \nu} F_{\mu\nu}
%	.\] 
%	\[
%	F_{\mu\nu}=\partial_\mu A_\nu-\partial_\nu A_\mu
%	.\]
%	\[
%	\partial_\mu F_{\mu \nu}=0
%	.\]
%	\[
%		E= \frac{1}{2} \int \left(\mathbf{E}^2 + \mathbf{H}^2\right) d^3x
%	.\] 
%	\begin{multline*}
%		L= \int d^3x \mathcal{L}=-\frac{1}{4} \int d^3x F_{\mu\nu}
%		F_{\mu\nu}=-\frac{1}{4} \int d^3 x (\partial_\mu A_\nu-
%		\partial_\nu A_\mu)
%		(\partial_\mu A_\nu-\partial_\nu A_\mu)=\\
%		=-\frac{1}{4} \int d^3 x\left[ (\partial_\mu A_\nu)^2+
%		(\partial_\nu A_\mu)^2-\partial_\nu A_\mu \partial_\mu
%	A_\nu-\partial_\mu A_\nu \partial_\nu A_\mu\right] 
%	.\end{multline*} 
%\[
%	E= \int d^3x \frac{\delta \mathcal{L}}{\delta \dot{A}(\mathbf{x},\,t)}\dot{A}(\mathbf{x},\,t)-L
%.\] 
%\begin{multline*}
%	\mathcal{L}=-\frac{1}{4}F_{\mu \nu}F_{\mu\nu}=
%	-\frac{1}{4}\left( \partial_\mu A_\nu -\partial_\nu A_\mu \right) \left( \partial_\mu A_\nu -\partial_\nu A_\mu \right) =\\=
%	-\frac{1}{4}\left( \partial_\mu A_\nu \partial_\mu A_\nu+
%	\partial_\nu A_\mu \partial_\nu A_\mu-
%\partial_\nu A_\mu \partial_\mu A_\nu -\partial_\mu A_\nu
%\partial_\nu A_\mu\right) = \\
%=\frac{1}{2}\left(
%\partial_\mu A_\nu \partial_\nu A_\mu-\partial_\mu A_\nu \partial_\mu A_\nu\right) 
%.\end{multline*} 
% 
%\begin{multline*}
%	E=\int d^3x \left( \left( \partial_\mu A_0-\partial_0 A_\mu \right) \partial_0 A_\mu - \frac{1}{2}(\partial_\mu A_\nu \partial_\nu A_\mu-\partial_\mu A_\nu
%	\partial_\mu A_\nu )\right)=\\=
%	\frac{1}{2}\int d^3 x\left(\partial_\mu A_0 \partial_0 A_\mu-\partial_\mu A_j \partial_j A_\mu -\partial_0A_\mu\partial_0
%	A_\mu +\partial_i A_\nu \partial_i A_\nu\right) =\\=
%	\frac{1}{2}\int d^3 x \left( 
%	\partial_i A_0 \partial_0 A_i -\partial_0 A_i
%\partial_i A_0-\partial_0 A_i \partial_i A_0+
%\partial_i A_0 \partial_i  A_0\right) =\\=
%\frac{1}{2} \int d^3 x\left(\partial_i A_0 -\partial_0 A_i \right) \left( \partial \right) 
%.\end{multline*} 
	Обозначим пространственную плотность энергии $\mathcal{H}$ следующим образом
\[
	E= \int d^3 x \mathcal{H}
.\] 
Тогда
\[
	\mathcal{H}= \frac{\delta \mathcal{L}}{\delta \dot{A}(\mathbf{x},\,t)}\dot{A}(\mathbf{x},\,t)-\mathcal{L}
.\] 
\[
	\frac{\delta \mathcal{L}}{\delta \dot{A}(\mathbf{x},\,t)}=
	\frac{\delta \mathcal{L}}{\delta \left( \partial_0 A_\mu \right) }= \frac{1}{2}\left( 2 \partial_\mu A_0-2 \partial_0 A_\mu \right) =F_{\mu 0}
.\]
\begin{multline*}
	\mathcal{H}=F_{\mu 0}\partial_{0}A_\mu-\mathcal{L}=
	F_{\mu 0} \partial_0 A_\mu +\frac{1}{4} F_{\mu \nu} F_{
	\mu \nu}=\\=
	F_{\mu 0}\left( F_{0\mu}+\partial_\mu A_0 \right) +
	\frac{1}{4}F_{\mu\nu}F_{\mu\nu}
	-F_{0\mu}\left( F_{0\mu}+\partial_\mu A_0 \right) =
	\frac{1}{4} F_{\mu\nu}F_{\mu\nu}=\\=
	-F_{0\mu}F_{0\mu}-F_{0\mu}\partial_\mu A_0+\frac{1}{4}
	F_{\mu\nu}F_{\mu\nu}=
	\mathbf{E}^2- F_{0i} \partial_i A_0+ \frac{1}{2}
	\left( \mathbf{H}^2-\mathbf{E}^2 \right) =\\=
	\frac{1}{2}\left(\mathbf{E}^2+\mathbf{H}^2\right)
	+E_i \partial_i\phi
.\end{multline*} 
Используя уравнения поля получаем
\[
\int d^3x E_i \partial_i \phi=- \int d^3 x\phi \partial_i E_i
=- \int d^3x\phi \nabla \cdot \mathbf{E}=0
.\] 
\[
	E= \frac{1}{2} \int \left( \mathbf{E}^2+\mathbf{H}^2 \right) d^3x
.\]
Запишем составляющие 4-вектора плотности тока для покоящейся в
$\mathbf{x}_0$ точечной
частицы с зарядом $e$
\[
	j_0(\mathbf{x})=e\delta^{(3)}(\mathbf{x}-\mathbf{x}_0),\quad
	\mathbf{j}=\mathbf{0}
.\] 
Добавка в лагранжеву плотность, обусловленная наличием точечного заряда
\[
	\Delta \mathcal{L}=j_\mu A_\mu
.\] 
Т.\:к. данная добавка не содержит $\dot{A}(\mathbf{x},\,t)$, то
\[
	\Delta \mathcal{H}=-\Delta \mathcal{L}
.\] 
\[
	\Delta E=\int d^3 x\Delta \mathcal{H}=-eA_0 (\mathbf{x}_0)
	=-e\phi(\mathbf{x}_0)
.\] 
\end{sol}
\prob{Ещё задача}
\begin{multline*}
S=\int d^3 x \left[ a F_{ij}F^{ij}+b \epsilon_{ijk}A^i \partial^j A^k \right] =\\=
\int d^3 x\left[2a\left(\partial_i A_j \partial^i A^j -\partial_i A_j \partial^j
A^i\right)
+b \epsilon_{ijk}A^i \partial^j A^k\right]=\int d^3x \mathcal{L}
.\end{multline*} 
Воспользуемся уравнениями Эйлера-Лагранжа
\[
	\partial_l \frac{\partial \mathcal{L}}{\partial (\partial_l A_m)}=\frac{\partial \mathcal{L}}{\partial A_m}
.\] 
Найдём сперва
\[
	\frac{\partial \mathcal{L}}{\partial A_m} =
	b\epsilon_{ijk} \eta^{in}\delta^m_n \partial^j A^k=
	b \epsilon_{jk}^m\partial^j A^k
.\] 
Далее
\begin{multline*}
	\frac{\partial \mathcal{L}}{\partial (\partial_l A_m)} =\\=
	\frac{\partial}{\partial(\partial_l A_m)}\left[a
	\eta^{ij}\eta^{kn}(\partial_i A_k-\partial_k A_i)(
\partial_j A_n -\partial_n A_j)+b \epsilon_{ijk} A^i
\eta^{jn}\eta^{kp}\partial_n A_p \right] =\\=
a\eta^{ij}\eta^{kn}[(\delta^l_i \delta^m_k-\delta^l_k\delta^m_i)
(\partial_j A_n -\partial_n A_j)+(\partial_i A_k-\partial_k A_i)
(\delta^l_j \delta^m_n-\delta^l_n\delta_j^m)]+\\+
b\epsilon_{ijk}A^i \eta^{jn}\eta^{kp}\delta^l_n\delta^m_p=\\=
a\left[(\eta^{lj}\eta^{mn}-\eta^{mj}\eta^{ln})(\partial_j A_n-
\partial_n A_j)+(\partial_i A_k -\partial_k A_i)(\eta^{il}\eta^{km}-
\eta^{im}\eta^{kl})\right]+\\+b \epsilon_{ijk}A^i\eta^{jl}\eta^{km}=\\
a \left[ \partial^l A^m-\partial^m A^l-\partial^mA^l+\partial^l
A^m+\partial^l A^m-\partial^mA^l-\partial^mA^l+\partial^lA^m\right] +b \epsilon^{ml}_i A^i=\\=
4aF^{lm}+b\epsilon_i^{ml}A^i
.\end{multline*} 
Откуда уравнения движения
\[
4a \partial_l F^{lm}+b\epsilon_i^{ml}\partial_l A^i=b \epsilon_{li}^m \partial^l A^i
.\] 
\[
2a \partial_l F^{lm}=b \epsilon^m_{li}\partial^lA^i
.\] 
\[
2a \partial^l F_{lm}=b \epsilon_{mli}\partial^l A^i
.\] 
\[
2a \partial^l F_{li}=b \epsilon_{ijk}\partial^j A^k
.\]
Будем искать решение в виде свободной монохроматической
плоской волны
\[
	A_i=\tilde{A}_i e^{i p_i x^i}
.\] 
\[
\partial^j A_i=
	ip^j A_i
.\] 
Зафиксируем калибровку Лоренца
\[
\partial_i A^i=0
.\] 
\[
\partial^l F_{li}=\partial^l \partial_l A_i -\partial^l \partial_i
A_l=\partial^l \partial_l A_i=\square A_i
.\] 
%\[
%F_{ij}=\partial_i A_j -\partial_j A_i=
%\eta_{ik}\partial^kA_j-\eta_{jk}\partial^k A_i=
%i\left(\eta_{ik}p^kA_j-\eta_{jk}p^kA_i\right)
%.\] 
\[
	-2a p^lp_lA_i 
	=ib \epsilon_{ijk}\eta^{kl}p^jA_l
.\] 
\[
	2a p^lp_l\tilde{A}_i 
	+ib \epsilon_{ij}^lp^j\tilde{A}_l=0
.\]
\[
	\tilde{A}_i=-\frac{ib}{2ap^lp_l}\epsilon^l_{ij}p^j
	\tilde{A}_l
.\] 
\[
	2ap^lp_l \tilde{A}_i+b \epsilon^l_{ij}p^j
	\frac{b}{2ap^l p_l}\epsilon^n_{lm}p^m\tilde{A}_n=0
.\]
\[
	4(ap^lp_l)^2\tilde{A}_i +b^2 \epsilon^l_{ij}p^j
	\epsilon^n_{lm}p^m\tilde{A}_n=0
.\]
\[
	4(ap^lp_l)^2\tilde{A}_i +b^2 p^j
	p^m\tilde{A}_n\left( \delta^n_i \delta_{jm}-\delta^n_j \delta_{im}\right) =0
.\]
\[
	4(ap^lp_l)^2\tilde{A}_i +b^2\left( p_m
	p^m\tilde{A}_i-p^n p_i \tilde{A}_n\right) =0
.\]
Вследствие выбранной калибровки последнее слагаемое зануляется, тогда
\[
4a^2 p^4+b^2 p^2=0
.\] 
Откуда нетривиальное решение 
\[
p= \pm\frac{b}{2a}
.\] 
Переходим в СО, где $p_2=p_3=0$. В ней $p_i =p\delta^1_i $.
Тогда
 \[
\partial_i A^i=0 \implies A^1=0
.\] 
\[
	2ap^2\tilde{A}_i+ib \epsilon^l_{ij}p\delta^j_1\tilde{A}_l=0
.\] 
\[
	\sign(p)	\tilde{A}_i+i\epsilon^l_{i1}\tilde{A}_l=0
.\] 
Решая данную систему, получаем
\[
	\tilde{A}_i=\alpha \begin{pmatrix} 0\\1 \\ -i\sign(p) \end{pmatrix} 
.\] 
%\[
%	2a p^lp_l\tilde{A}_i 
%	+ib \epsilon_{ijk}p^j\tilde{A}^k=0
%.\]
%\[
%	\tilde{A}_i=- \frac{ib}{2a p^l p_l}\epsilon_{ijk}p^j
%	\tilde{A}^k
%.\] 
%\[
%	\epsilon_{ijk} p^j \tilde{A}^k\left( ib+ \right) 
%.\] 
%\[
%	2a \mathbf{p}^2 \tilde{\mathbf{A}}=-ib \mathbf{p}\times
%	\tilde{\mathbf{A}}
%.\] 
%\[
%	p^l A_i (ib \epsilon^i_{jl}+2 a p_l)=0
%.\]
%\[
%-\frac{ib}{2a p^l p_l}\epsilon_{ijk}p^j=\eta_{ik}
%.\] 
%\[
%- \frac{ib}{2ap^l p_l}
%.\] 
\prob{Задача 4.4}
\begin{sol}
\begin{multline*}
\left[ D_\mu,\,D_\nu \right] =D_\mu D_\nu- D_\nu D_\mu=\\=
(\partial_\mu -i e A_\mu)(\partial_\nu- i e A_\nu)-
(\partial_\nu -i e A_\nu)(\partial_\mu-ie A_\mu)=\\=
\partial_\mu \partial_\nu-i e\partial_\mu A_\nu-
i e A_\mu \partial_\nu- \partial_\nu \partial_\mu
-eA_\mu A_\nu+
ie\partial_\nu A_\mu+ie A_\nu \partial_\mu+
e A_\nu A_\mu=\\=
-ie(\partial_\mu A_\nu-\partial_\nu A_\mu)=
-ie F_{\mu\nu}
.\end{multline*} 
\end{sol}
\prob{Задача 4.8}
\begin{sol}
\begin{multline*}
	\epsilon_{\mu\nu\lambda\rho}\tr(F_{\mu\nu}F_{\lambda\rho})=\epsilon_{\mu\nu\lambda\rho}F_{\mu\nu}^aF_{\lambda\rho}^a=
	\epsilon_{\mu\nu\lambda\rho}\left( \partial_\mu A_\nu^a
	-\partial_\nu A_\mu^a\right) (\partial_\lambda A^a_\rho-
	\partial_\rho A_\lambda^a)=\\=
	\epsilon_{\mu\nu\lambda\rho}\left( 
	\partial_\mu A^a_\nu\partial_\lambda A^a_\rho-
\partial_\mu A_\nu^a \partial_\rho A_\lambda^a-
\partial_\nu A^a_\mu \partial_\lambda A^a_\rho+
\partial_\nu A_\mu^a\partial_\rho A^a_\lambda\right) =\\=
	\partial_\mu A^a_\nu\partial_\lambda A^a_\rho
	\epsilon_{\mu\nu\lambda\rho} -
\partial_\mu A_\nu^a \partial_\rho A_\lambda^a\epsilon_{\mu\nu\lambda\rho} -
\partial_\nu A^a_\mu \partial_\lambda A^a_\rho\epsilon_{\mu\nu\lambda\rho} +
\partial_\nu A_\mu^a\partial_\rho A^a_\lambda\epsilon_{\mu\nu\lambda\rho} =\\=
	\partial_\mu A^a_\nu\partial_\lambda A^a_\rho
	\epsilon_{\mu\nu\lambda\rho} -
\partial_\mu A_\nu^a \partial_\rho A_\lambda^a\epsilon_{\mu\nu\lambda\rho} -
\partial_\mu A^a_\nu \partial_\lambda A^a_\rho\epsilon_{\nu\mu\lambda\rho} +
\partial_\mu A_\nu^a\partial_\rho A^a_\lambda\epsilon_{\nu\mu\lambda\rho} =\\=
	\partial_\mu A^a_\nu\partial_\lambda A^a_\rho
	\epsilon_{\mu\nu\lambda\rho} -
\partial_\mu A_\nu^a \partial_\rho A_\lambda^a\epsilon_{\mu\nu\lambda\rho} +
\partial_\mu A^a_\nu \partial_\lambda A^a_\rho\epsilon_{\mu\nu\lambda\rho} -
\partial_\mu A_\nu^a\partial_\rho A^a_\lambda\epsilon_{\mu\nu\lambda\rho} =\\=
\partial_\mu \left(A^a_\nu\partial_\lambda A^a_\rho
	\epsilon_{\mu\nu\lambda\rho} -
A_\nu^a \partial_\rho A_\lambda^a\epsilon_{\mu\nu\lambda\rho} +
A^a_\nu \partial_\lambda A^a_\rho\epsilon_{\mu\nu\lambda\rho} -
A_\nu^a\partial_\rho A^a_\lambda\epsilon_{\mu\nu\lambda\rho}\right)
.\end{multline*} 
Откуда
\begin{multline*}
K_\mu=
A^a_\nu\partial_\lambda A^a_\rho
	\epsilon_{\mu\nu\lambda\rho} -
A_\nu^a \partial_\rho A_\lambda^a\epsilon_{\mu\nu\lambda\rho} +
A^a_\nu \partial_\lambda A^a_\rho\epsilon_{\mu\nu\lambda\rho} -
A_\nu^a\partial_\rho A^a_\lambda\epsilon_{\mu\nu\lambda\rho}=\\=
2\epsilon_{\mu\nu\lambda\rho}\left(A^a_{\nu}\partial_\lambda
A_\rho^a-A^a_\nu\partial_\rho A^a_\lambda\right) =
2\epsilon_{\mu\nu\lambda\rho}A^a_{\nu}\left(\partial_\lambda
A_\rho^a-\partial_\rho A^a_\lambda\right) =\\=
2\epsilon_{\mu\nu\lambda\rho}A^a_{\nu}F_{\lambda\rho}^a
.\end{multline*} 
Покажем, что добавление в лагранжиан полной дивергенции от любого
вектора, зависящего от полей, не изменяет уравнения поля.
Для $K_\mu(A_\mu,\,\partial_\mu A_\nu)$ и $G=\left\{\mathbb{R}^3\times 
[t_1,\,t_2] \right\} $запишем его вклад в действие.
%Проварьируем добавку
%в лагранжиан $\mathcal{L}=\partial_\mu K_\mu(A_\mu,\,\partial_\mu A_\rho)$
%\[
%	\frac{\partial \mathcal{L}}{\partial A_\nu}=\frac{\partial }{\partial A_\nu} \partial_\mu K_\mu,\quad
%	\partial_\lambda \frac{\partial \mathcal{L}}{\partial (
%	\partial_\lambda A_\nu)} =\delta_{\mu\lambda}\delta_{
%\rho\nu}
%.\] 
\[
	\int\limits_{G}^{}   d^4 x\partial_\mu K_\mu= \int\limits_{\partial G}^{} d^3xK_\mu n_\mu \xlongequal[A_\mu|_{\partial G}=0]{\partial_\mu A_\nu|_{\partial G}=0}0
.\] 
Что и требовалось доказать.
\end{sol}
\prob{И ещё задача}
Поле $A_\mu$ принимает значения в алгебре Ли группы $G$, т.\:е.
\[
	A_\mu (x)=g t^a A^a_\mu (x)
 ,\]
где $g$ --- некоторое число, $t^a$ --- генераторы группы $G$.
Требуется доказать, что
\[
A_\mu'= \omega A_\mu \omega^{-1}+\omega \partial_\mu \omega^{-1}
 ,\quad\omega\in G,\]
также лежит в алгебре Ли группы $G$, т.\:е.
\[
	A_\mu'(x)=g t^a A_\mu'^a(x)
.\] 
\prob{Задача 5.5}
\begin{sol}
В задаче рассматривается теория трёх комплексных скалярных
полей $f_i (x),\ i=1,\,2,\,3$, с лагранжианом
\[
	\mathcal{L}= \partial_\mu f_i^*\partial_\mu f_i+
	\mu^2 f_i^* f_i -\lambda (f_i ^* f_i)^2
.\] 
Пусть
\[
	f_i=\frac{1}{\sqrt{2} }(f_{i1}+i f_{i 2})
.\] 
Также введём константу $c$ для удобства в дальнейшем
\[
	\mathcal{L}= \partial_\mu f_i^*\partial_\mu f_i+
	\mu^2 f_i^* f_i -\lambda (f_i ^* f_i)^2-c
.\]
\renewcommand{\labelenumi}{\asbuk{enumi})}
\begin{enumerate}
\item Группой глобальной симметрии этого лагранжиана будет являться
	$U(3)$, т.\:е. он инвариантен относительно преобразований
\[
	f_i(x)\to f'_i(x)=\omega_{ij}f_j (x),
\] 
где $\omega$ --- произвольная матрица из $U(3)$. Свойство
инвариантности лагранжиана относительно данных преобразований
очевидно из тождества
\[
f_i'^*f_i'=f_k^*\omega_{ik}^* \omega_{ij} f_j=
f_k^*\left( \omega^\dagger \omega \right) _{kj} f_j=
f_k^* f_k
.\] 
\item Рассмотрим энергию поля
	\[
		E= \int d^3 x \left(\partial_0 f^*_i \partial_0
		f_i +\partial_j f^*_i \partial_j f_i+ V(f_i^*,\,
		f_i)\right)
	,\] 
где 
 \[
	 V(f_i^*,\,f_i)=-\mu^2 f_i^* f_i +\lambda(f_i^*f_i)^2+c
.\] 
Основное состояние однородно в пространстве-времени,
$f_i= \text{const}$, и представляет собой минимум потенциала.
Потенциал $V(f_i^*,\,f_i)$
имеет минимум при
\[
f_i^* f_i=f_0^2, \quad \text{где}\quad f_0= \frac{\mu}{\sqrt{\lambda} }
.\] 
%Потенциал зависит только от одной переменной
%\[
%	|f|= \sqrt{f_i^* f_i} 
%\] 
%и имеет непрерывный набор минимумов
%\[
%f_i= e^{i\alpha} \frac{f_0}{\sqrt{2} }
% ,\]
%где $f_0$ определяется из  условия
%\[
%	\left( \frac{\partial V}{\partial |f|}  \right) 
%	\left( \frac{f_0}{\sqrt{2} } \right) =0
%\]
%и равно
%\[
%f_0= \frac{\mu}{\sqrt{\lambda}}
%.\] 
Рассмотрим основное состояние
\[
f_i =\delta_{i 1} \frac{f_0}{\sqrt{2} }
 ,\]
и возмущения около него, описываемые полями
\[
	f_{i1}(x)=\delta_{i1} f_0 +\chi_i (x),\quad
	f_{i2}(x)=\theta_i (x)
.\] 
Вакуумный вектор $\mathbf{f}^{(0)}=(f_0,\,0,\,0)$ не полностью
нарушает симметрию: имеется нетривиальная подгруппа группы
$U(3)$, относительно которой вакуумный вектор инвариантен:
\[
	\omega \mathbf{f}^{(0)}=\mathbf{f}^{(0)}
.\] 
Эта подгруппа представляет собой группу $U(2)$,
\[
	f_1\to \omega_{1j}f_j,\quad
	f_2\to \omega_{2j}f_j,\quad
	f_3\to f_3
.\] 
\item Ограничимся малыми возмущениями, для  чего выделим  слагаемые
в лагранжиане, квадратичные относительно возмущений $\chi_i$ и
$\theta_i$. Имеем $\partial_\mu f_{i1}=\partial_\mu \chi_i$,
$\partial_\mu f_{i 2}=\partial_\mu \theta_i$ и
\begin{multline*}
V= -\frac{\mu^2}{2}\left[ 
(f_0 +\chi_1)^2+ \chi_2^2+\chi_3^2 +\theta_i\theta_i\right] +\\+
\frac{\lambda}{4}\left[ 
(f_0 +\chi_1)^2+ \chi_2^2+\chi_3^2 +\theta_i\theta_i\right] ^2+
\frac{\mu^4}{4\lambda}
,\end{multline*} 
где константа $c$ подобрана так, что энергия основного состояния
равна нулю.

В квадратичном порядке по полям $\chi_i$ и $\theta_i$ получим
 \[
V=\mu^2 \chi_1^2
.\] 

Итак, квадратичный лагранжиан равен
\[
	\mathcal{L}_{\chi_i,\,\theta_i}^{(2)}=\frac{1}{2}
	\partial_\mu \chi_i \partial_\mu \chi_i +
	\frac{1}{2} \partial_\mu \theta_i \partial_\mu \theta_i-
	\mu^2 \chi_1^2
.\] 
Поле $\chi_1$ имеет массу $m_{\chi_1}=\sqrt{2} \mu$, остальные
же поля остаются безмассовыми, т.\:е. являются намбу-голдстоуновскими модами.
\item Из всего вышесказанного получается, что безмассовые моды
	преобразуются по тривиальному представлению $U(2)$,
	а намбу-голдстоуновские по фундаментальному.
\end{enumerate}
\end{sol}
\prob{Задача 7.1}
\begin{sol}
Движущиеся кинки описываются семейством решений
\[
	\phi_{\text{k}}(x-x_0;\,t;\, u)= \frac{\mu}{\sqrt{\lambda} }\th \left( \frac{\mu}{\sqrt{2} }\frac{(x-x_0)-ut}{\sqrt{1-u^2} } \right) 
,\] 
где $u$ --- это скорость солитона, $x_0$ --- это положение центра
кинка в момент $t=0$.

Классическую энергию кинка можно определить как
\[
	E_{\text{k}}= \int\limits_{-\infty}^{\infty}   \epsilon(x)dx,\quad \epsilon(x)=\frac{1}{2} (\partial_\nu \phi_{\text{k}})^2+V(\phi_{\text{k}}) 
,\] 
где 
\[
	V(\phi)=\frac{\lambda}{4}(\phi^2-v^2)^2,\quad
	v=\frac{\mu}{\sqrt{\lambda} }
.\] 
Непосредственными расчётами получаем
\[
	\epsilon(x)=\frac{1}{1-u^2} v^4\frac{\lambda}{2}\ch^{-4}\left( 
	\frac{\mu}{\sqrt{2} } \frac{(x-x_0)-ut}{\sqrt{1-u^2} }\right)
,\] 
и
\[
	E_{\text{k}}=\frac{2mv^2}{3\sqrt{1-u^2} }
,\] 
где $m=\sqrt{2} \mu$ --- масса элементарного возбуждения.

Лагранжиан кинка
\[
	L_{\text{k}}=\int\limits_{-\infty}^{\infty} \mathcal{L}\,dx,\quad
	\mathcal{L}=\frac{1}{2}(\partial_{\nu}\phi_{\text{k}})^2
	-V(\phi_{\text{k}})
.\] 
Непосредственными вычислениями получаем
\[
	\mathcal{L}(x)=\frac{3u^2}{4(1-u^2)}\lambda v^4\ch^{-4}\left( 
	\frac{\mu}{\sqrt{2} } \frac{(x-x_0)-ut}{\sqrt{1-u^2} }\right)
.\]
Плотность импульса кинка
\[
	\rho(x)= T^{01}=- \frac{\partial \mathcal{L}}{\partial
	\dot{\phi}}\phi' 
.\] 
Непосредственными вычислениями получаем
\[
	\rho(x)=\frac{u}{1-u^2} v^4\frac{\lambda}{2}\ch^{-4}\left( 
	\frac{\mu}{\sqrt{2} } \frac{(x-x_0)-ut}{\sqrt{1-u^2} }\right)
,\]
и
\[
p_\text{k}=\frac{2umv^2}{3\sqrt{1-u^2} }
.\] 
Заметим, что для найденных $E_{\text{k}}$, $p_\text{k}$ и
$M_\text{k}=\frac{2}{3}mv^2$ выполняется релятивистское
соотношение
\[
M_\text{k}^2=E_\text{k}^2-p_\text{k}^2
.\] 
\end{sol}
\prob{Задача 12.7}
\begin{sol}
На сферически-симметричных конфигурациях
\[
E_\text{stat}=4\pi \int\limits_{0}^{\infty} r^2 dr \left[ 
\frac{1}{2}(\phi')^2+V_0(\phi)-\epsilon V_1 (\phi)\right]  
.\] 
Внешняя область пузыря не даёт вклада в $E_\text{stat}$, а внутренняя
область даёт вклад
\[
E_{\text{stat}}^\text{in}=-\frac{4}{3}\pi R^3 \epsilon
.\] 
Вклад стенки пропорционален $R^2$ и в главном порядке не зависит
от  $\epsilon$; для его вычисления положим $r=R$ в мере $E_\text{stat}
$ и пренебрежём $\epsilon V_1$. Получим
\[
	E_\text{stat}^\text{wall}=4\pi R^2 \mu
,\] 
где
\[
\mu= \int\limits_{-\infty}^{\infty} \left[ 
\frac{1}{2}(\phi_k'(r))^2+V_0 (\phi_k(r))\right]  
.\] 
Итак, в главном порядке по $\epsilon$ и $R$ статическая энергия
пузыря
размера $R$ равна
\[
	E_{\text{stat}}(R)=4\pi R^2 \mu-\frac{4}{3}\pi R^3 \epsilon 
.\] 
Экстремум этого выражения достигается при
\[
R=R_\text{sph}= \frac{2\mu}{\epsilon}
.\] 
Для статической энергии сфалерона получим окончательно
\[
E_{\text{stat} }=\frac{16\pi \mu^3}{3\epsilon^2}
.\] 
Т.\:к. $\mu>0$ и $\epsilon>0$, то статическая энергия пузыря
будет  иметь вид, качественно изображённый на рис.~\ref{fig:1}.
\begin{figure}[ht]
    \centering
    \incfig{1}
    \caption{}
    \label{fig:1}
\end{figure}
Откуда можно заключить, что состояние пузыря с радиусом $R_\text{sph}$ --- неустойчиво.

Запишем возмущение энергии из-за возмущения поля сфалерона
\[
	\delta E= E(\phi+\delta \phi)-E(\phi)
.\] 
\[
\delta E= 4\pi \int\limits_{0}^{\infty} r^2 dr
\left[ \phi' \delta\phi'+\frac{\partial V}{\partial \phi}  \delta \phi\right] 
.\] 
\begin{multline*}
\phi' \delta \phi'+ \frac{\partial V}{\partial \phi} \delta \phi=
\phi'\delta \phi'+ \left(\phi''+ \frac{2}{r}\phi'\right)\delta\phi
=\phi'\delta \phi'-\phi' \delta \phi'+\frac{2}{r}\phi'\delta\phi=
\\=\frac{2}{r}\phi'\delta \phi
.\end{multline*} 
Для того, чтобы возмущение $\delta E$ оказалось отрицательным
можем взять, например, $\delta \phi=-\epsilon \phi'$, тогда
\[
	\delta E= -4 \pi \int\limits_{0}^{\infty} 2r\epsilon(\phi')^2 dr <0
.\] 
Уравнения движения поля
\[
\phi''+ \frac{2}{r} \phi'= \frac{\partial V}{\partial \phi} 
.\] 
Уравнения движения возмущённого поля
\[
	\phi''+\delta\phi''+ \frac{2}{r} (\phi'+\delta \phi')=
	\frac{\partial V}{\partial (\phi+\delta \phi)} 
.\] 
Произведем замену $\delta \phi=g(r) e^{i \omega t}$.
\[
	\phi''+g'' e^{i\omega t}+ \frac{2}{r}(\phi'+
	g' e^{i\omega t})= \frac{\partial V(\phi+g e^{i\omega t})}{\partial (\phi+
	g e^{i\omega t})} 
.\] 
Сделаем замену $\phi+g ^{i\omega t}=\xi$
\[
	\phi''+g'' e^{i\omega t}+ \frac{2}{r}(\phi'+
	g' e^{i\omega t})= \frac{\partial V(\xi)}{\partial \xi}
.\]

Уравнение на кинк
\[
	\phi''- \frac{\partial V(\phi)}{\partial \phi} =0
.\] 
\[
	V(\phi)= \frac{\lambda}{4}(\phi^2-v^2)^2
.\] 
Статическое уравнение поля в $d$-мерном пространстве-времени
имеет вид
\[
-\partial_i \partial_i \phi + \frac{\partial V}{\partial \phi} =0
,\] 
где индекс $i$ --- пространственный и пробегает значения 
$i=1,\,2,\,3$. Его решения имеют сферическую симметрию
$\phi=\phi\left( \sqrt{x_ix_i}  \right) $ и уравнение для
$\phi$ имеет вид
\[
\phi''+ \frac{2}{r}\phi'= \frac{\partial V}{\partial \phi} 
.\] 
Пусть $\phi_\text{k}(r)$ --- статическое решение уравнений поля.
Рассмотрим малые возмущения $f(\tau,\mathbf{x})$ около статического
решения, так что исходное поле имеет вид
\[
	\phi(\tau,\,\mathbf{x})=\phi_\text{k}(r)+f(\tau,\,\mathbf{x})
.\] 
Поле $\phi(\tau,\,\mathbf{x})$ должно удовлетворять уравнениям
\[
	-\partial_\mu \partial^\mu \phi+ \frac{\partial V}{\partial \phi}  =0
\]
или
\[
	-\partial_\mu\partial^\mu(\phi_\text{k}+f)+
	\frac{\partial V}{\partial \phi} (\phi_\text{k})+
	\frac{\partial^2 V}{\partial \phi^2} (\phi_\text{k})
	f+\ldots=0
.\] 
Поле $\phi_\text{k}$ уравнениям выше,
поэтому в линейном порядке по $f$ получим
\[
	-\partial_\mu \partial^\mu f+ \frac{\partial ^2 V}{\partial \phi^2} (\phi_\text{k})f=0
.\] 
Т.\:к. $\frac{\partial ^2 V}{\partial \phi^2} (\phi_\text{k})$
зависит только от $r$, переменные разделяются и решение можно
искать в виде
\[
	f(\tau,\,r)= e^{i\omega t}g_\omega (r)
 ,\]
где $g_\omega$ удовлетворяет уравнению
\[
\omega^2 g_\omega+ g_\omega''+ \frac{\partial ^2V}{\partial \phi^2} 
(\phi_\text{k})g_\omega=0
 ,\] 
или
\[
	-g''_\omega +U(r)g_\omega=\omega^2 g_\omega
 ,\]
где
\[
	U(r)= \frac{\partial ^2 V}{\partial \phi^2} (\phi_\text{k})
.\] 
Нулевая мода имеет вид
\[
	g_0(r)= \frac{\partial \phi_\text{k}(r)}{\partial r} 
.\] 
Действительно, $\phi_\text{k}$ удовлетворяет уравнению
\[
-\phi
.\] 
\end{sol}
\end{document}
