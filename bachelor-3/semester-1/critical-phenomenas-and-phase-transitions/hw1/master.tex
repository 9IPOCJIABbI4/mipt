\documentclass[a4paper]{article}
% Этот шаблон документа разработан в 2014 году
% Данилом Фёдоровых (danil@fedorovykh.ru) 
% для использования в курсе 
% <<Документы и презентации в \LaTeX>>, записанном НИУ ВШЭ
% для Coursera.org: http://coursera.org/course/latex .
% Исходная версия шаблона --- 
% https://www.writelatex.com/coursera/latex/5.3

% В этом документе преамбула

\usepackage{siunitx}
%%% Работа с русским языком
%\usepackage{cmap}					% поиск в PDF
%\usepackage{mathtext} 				% русские буквы в формулах
%\usepackage[T2A]{fontenc}			% кодировка
%\usepackage[utf8]{inputenc}			% кодировка исходного текста
%\usepackage[english,russian]{babel}	% локализация и переносы
%\usepackage{indentfirst}
%\frenchspacing
%
%\renewcommand{\epsilon}{\ensuremath{\varepsilon}}
%\newcommand{\phibackup}{\ensuremath{\phi}}
%\renewcommand{\phi}{\ensuremath{\varphi}}
%\renewcommand{\varphi}{\ensuremath{\phibackup}}
%\renewcommand{\kappa}{\ensuremath{\varkappa}}
%\renewcommand{\le}{\ensuremath{\leqslant}}
%\renewcommand{\leq}{\ensuremath{\leqslant}}
%\renewcommand{\ge}{\ensuremath{\geqslant}}
%\renewcommand{\geq}{\ensuremath{\geqslant}}
%\renewcommand{\emptyset}{\varnothing}
%\renewcommand{\Im}{\operatorname{Im}}
%\renewcommand{\Re}{\operatorname{Re}}


%%% Дополнительная работа с математикой
\usepackage{amsmath,amsfonts,amssymb,amsthm,mathtools} % AMS
%\usepackage{icomma} % "Умная" запятая: $0,2$ --- число, $0, 2$ --- перечисление

%% Номера формул
%\mathtoolsset{showonlyrefs=true} % Показывать номера только у тех формул, на которые есть \eqref{} в тексте.
%\usepackage{leqno} % Нумереация формул слева

%% Свои команды
\DeclareMathOperator{\sgn}{\mathop{sgn}}
\DeclareMathOperator{\sign}{\mathop{sign}}
\DeclareMathOperator*{\res}{\mathop{res}}
\DeclareMathOperator*{\tr}{\mathop{tr}}
\DeclareMathOperator*{\rot}{\mathop{rot}}
\DeclareMathOperator*{\divop}{\mathop{div}}
\DeclareMathOperator*{\grad}{\mathop{grad}}

%% Перенос знаков в формулах (по Львовскому)
\newcommand*{\hm}[1]{#1\nobreak\discretionary{}
{\hbox{$\mathsurround=0pt #1$}}{}}

%%% Работа с картинками
\usepackage{graphicx}  % Для вставки рисунков
\graphicspath{{figures/}}  % папки с картинками
\setlength\fboxsep{3pt} % Отступ рамки \fbox{} от рисунка
\setlength\fboxrule{1pt} % Толщина линий рамки \fbox{}
\usepackage{wrapfig} % Обтекание рисунков текстом

%%% Работа с таблицами
\usepackage{array,tabularx,tabulary,booktabs} % Дополнительная работа с таблицами
\usepackage{longtable}  % Длинные таблицы
\usepackage{multirow} % Слияние строк в таблице

%%% Теоремы
\theoremstyle{plain} % Это стиль по умолчанию, его можно не переопределять.
\newtheorem{thm}{Теорема}
\newtheorem*{thm*}{Теорема}
\newtheorem{prop}{Предложение}
\newtheorem*{prop*}{Предложение}
 
\theoremstyle{definition} % "Определение"
%\newtheorem{corollary}{Следствие}[theorem]
\newtheorem{dfn}{Определение}
\newtheorem*{dfn*}{Определение}
\newtheorem{prob}{Задача}
\newtheorem*{prob*}{Задача}

 
\theoremstyle{remark} % "Примечание"
\newtheorem*{sol}{Решение}
\newtheorem*{rem}{Замечание}

%%% Программирование
\usepackage{etoolbox} % логические операторы

%%% Страница
%\usepackage{extsizes} % Возможность сделать 14-й шрифт
%\usepackage{geometry} % Простой способ задавать поля
%	\geometry{top=25mm}
%	\geometry{bottom=35mm}
%	\geometry{left=35mm}
%	\geometry{right=20mm}
 
\usepackage{fancyhdr} % Колонтитулы
%	\pagestyle{fancy}
 %	\renewcommand{\headrulewidth}{0pt}  % Толщина линейки, отчеркивающей верхний колонтитул
	%\lfoot{Нижний левый}
	%\rfoot{Нижний правый}
	%\rhead{Верхний правый}
	%\chead{Верхний в центре}
	%\lhead{Верхний левый}
	%\cfoot{Нижний в центре} % По умолчанию здесь номер страницы

\usepackage{setspace} % Интерлиньяж
%\onehalfspacing % Интерлиньяж 1.5
%\doublespacing % Интерлиньяж 2
%\singlespacing % Интерлиньяж 1

\usepackage{lastpage} % Узнать, сколько всего страниц в документе.

\usepackage{soul} % Модификаторы начертания

\usepackage{hyperref}
\usepackage[usenames,dvipsnames,svgnames,table,rgb]{xcolor}
\hypersetup{				% Гиперссылки
    unicode=true,           % русские буквы в раздела PDF
    pdftitle={Заголовок},   % Заголовок
    pdfauthor={Автор},      % Автор
    pdfsubject={Тема},      % Тема
    pdfcreator={Создатель}, % Создатель
    pdfproducer={Производитель}, % Производитель
    pdfkeywords={keyword1} {key2} {key3}, % Ключевые слова
%    colorlinks=true,       	% false: ссылки в рамках; true: цветные ссылки
    %linkcolor=red,          % внутренние ссылки
    %citecolor=black,        % на библиографию
    %filecolor=magenta,      % на файлы
    %urlcolor=cyan           % на URL
}

\usepackage{csquotes} % Еще инструменты для ссылок

%\usepackage[style=apa,maxcitenames=2,backend=biber,sorting=nty]{biblatex}

\usepackage{multicol} % Несколько колонок

\usepackage{tikz} % Работа с графикой
\usepackage{pgfplots}
\usepackage{pgfplotstable}
%\usepackage{coloremoji}
\usepackage{floatrow}
\usepackage{subcaption}
\graphicspath{{figures/}}

\renewcommand\thesubfigure{\asbuk{subfigure}}
%\addbibresource{master.bib}

\usepackage{import}
\usepackage{pdfpages}
\usepackage{transparent}
\usepackage{xcolor}
\usepackage{xifthen}

\newcommand{\incfig}[2][1]{%
    \def\svgwidth{#1\columnwidth}
    \import{./figures/}{#2.pdf_tex}
}
%\usepackage{titlesec}
%\titleformat{\section}{\normalfont\Large\bfseries}{}{0pt}{}
%----------------------STANDART:
%\titleformat{\chapter}[display]
%  {\normalfont\huge\bfseries}{\chaptertitlename\ \thechapter}{20pt}{\Huge}
%\titleformat{\section}{\normalfont\Large\bfseries}{\thesection}{1em}{}
%\titleformat{\subsection}
%  {\normalfont\large\bfseries}{\thesubsection}{1em}{}
%\titleformat{\subsubsection}
%  {\normalfont\normalsize\bfseries}{\thesubsubsection}{1em}{}
%\titleformat{\paragraph}[runin]
%  {\normalfont\normalsize\bfseries}{\theparagraph}{1em}{}
%\titleformat{\subparagraph}[runin]
%  {\normalfont\normalsize\bfseries}{\thesubparagraph}{1em}{}

\pdfsuppresswarningpagegroup=1
\pgfplotsset{compat=1.16}



%\setcounter{tocdepth}{1} % only parts,chapters,sections
%\titleformat{\subsection}{\normalfont\large\bfseries}{}{0em}{}
%\titleformat{\subsubsection}{\normalfont\normalsize\bfseries}{}{0em}{}

%\newcommand{\textover}[2]{\stackrel{\mathclap{\normalfont\mbox{#2}}}{#1}}

\author{Yaroslav Drachov\\
Moscow Institute of Physics and Technology}
%\author{Драчов Ярослав\\
%Факультет общей и прикладной физики МФТИ}
\newcommand{\veq}{\mathrel{\rotatebox{90}{$=$}}}
%\newcommand{\teto}[1]{\stackrel{\mathclap{\normalfont\tiny\mbox{#1}}}{\to}}
%\renewcommand{\thesubsection}{\arabic{subsection}}

%%\setcounter{secnumdepth}{0}

\definecolor{tabblue}{RGB}{30, 119, 180}
\definecolor{taborange}{RGB}{255, 127, 15}
\definecolor{tabgreen}{RGB}{45, 160, 43}
\definecolor{tabred}{RGB}{214, 38, 40}
\definecolor{tabpurple}{RGB}{148, 103, 189}
\definecolor{tabbrown}{RGB}{140, 86, 76}
\definecolor{tabpink}{RGB}{227, 119, 193}
\definecolor{tabgray}{RGB}{127, 127, 127}
\definecolor{tabolive}{RGB}{188, 189, 33}
\definecolor{tabcyan}{RGB}{22, 190, 207}
\pgfplotscreateplotcyclelist{colorbrewer-tab}{
{tabblue},
{taborange},
{tabgreen},
{tabred},
{tabpurple},
{tabbrown},
{tabpink},
{tabgray},
{tabolive},
{tabcyan},
}
\usepackage{csvsimple}
\usepackage{extarrows}
%\renewcommand{\labelenumii}{\asbuk{enumii})}
%\renewcommand{\labelenumiv}{\Asbuk{enumiv}}
%\newcommand{\prob}[1]{\subsubsection*{#1}}
\sisetup{output-decimal-marker = {,},separate-uncertainty = true,exponent-product = \cdot}

\usepackage{braket}
\usepackage{enumerate}
\usepackage{chngcntr}
%\counterwithin*{equation}{problem}
%\usepackage{bbold}

\newtheoremstyle{hiProb}% ⟨name ⟩ 
{3pt}% ⟨Space above ⟩1 
{3pt}% ⟨Space below ⟩1
{}% ⟨Body font ⟩
{}% ⟨Indent amount ⟩2
{\bfseries}% ⟨Theorem head font⟩
{.}% ⟨Punctuation after theorem head ⟩
{.5em}% ⟨Space after theorem head ⟩3
%{\thmname{#1} \thmnote{#3}}% ⟨Theorem head spec (can be left empty, meaning ‘normal’)⟩
{\thmnote{#3}}% ⟨Theorem head spec (can be left empty, meaning ‘normal’)⟩
\theoremstyle{hiProb} % "Определение"
%\newtheorem{hiProb}{Задача}
\newtheorem{hiProb}{}
%\usepackage{mmacells}
\newcommand{\textover}[2]{\stackrel{\mathclap{\normalfont\scriptsize\mbox{#2}}}{#1}}
\usepackage{units}
\usepackage[math]{cellspace}%
\setlength\cellspacetoplimit{2pt}
\setlength\cellspacebottomlimit{2pt}

\DeclareMathAlphabet{\mathbbold}{U}{bbold}{m}{n}

\newcommand{\normord}[1]{:\mathrel{#1}:}

\title{Домашняя работа по фазовым переходам}
\begin{document}
\maketitle
\section{Гауссовы интегралы}
\subsection{Размерность 1}
\begin{itemize}
\item \[
	\left<x^n \right>Z_1 (a)= \int\limits_{-\infty}^{\infty} dx\,
	x^n \exp\left(-ax^2\right)
	.\]
\[
	Z_1(a) \frac{b^n}{n!}\left<x^n \right> = \frac{b^n}{n!} \int\limits_{-\infty}^{\infty} dx\,
	x^n \exp\left(-ax^2\right)
	.\]
\begin{multline*}
	Z_1(a) \sum_{n=0}^{\infty} \frac{b^n}{n!}\left<x^n \right> =
	\sum_{n=0}^{\infty} \frac{b^n}{n!} \int\limits_{-\infty}^{\infty} 
	dx \, x^n \exp\left(-ax^2\right) \xlongequal{\text{инт. сх.}}\\=
	\int\limits_{-\infty}^{\infty} dx \, \exp \left( -a x^2 \right) 
	\sum_{n=0}^{\infty} \frac{(bx)^n}{n!}=
	\int\limits_{-\infty}^{\infty} \, dx \exp \left(-a x^2\right) 
	\exp (bx) =\\= Z_1(a,\,b)
.\end{multline*} 
\item \begin{multline*}
		Z_1 (a,\,b)=\\= \int\limits_{-\infty}^{\infty} 
		dx \exp (-a x^2 +bx)=
		\int\limits_{-\infty}^{\infty} dx \exp
		\left[-a\left(x-\frac{b}{2a}\right)^2+\frac{b^2}{4}
		\right] \xlongequal{y= x - b / 2a} \\=
		\int\limits_{-\infty}^{\infty} dy
		\exp \left( -ay^2+ \frac{b^2}{4a} \right) =
		e^{\frac{b^2}{4a}}\int\limits_{-\infty}^{\infty} 
		dy \exp(-a y^2)=
		e^{\frac{b^2}{4a}}\sqrt{\frac{\pi}{a}} 
.\end{multline*} 
\item 
	\[
		Z_1 (a,\,b)=\sqrt{\frac{\pi}{a}}
		\sum_{n=0}^{\infty} \frac{b^n}{n!}\left.\left(e^{\frac{b^2}{4a}}\right)^{(n)}\right|_{b=0}
	.\]
	\begin{equation*}
	\left<x^n \right> = \left.\left(e^{\frac{b^2}{4a}}\right)^{(n)}\right|_{b=0}
	\label{eq:1}
	\tag{*}
	.\end{equation*} 
\item Результаты непосредственного вычисления $\left<x^1 \right>,\,
	\left<x^2 \right>,\,\left<x^3 \right>,\,\left<x^4 \right>,\,
	\left<x^5 \right>$ по формуле (\ref{eq:1}) представлены в таблице \ref{tab:1}
	и совпадают с значениями, полученными с помощью
	Mathematica.
\begin{table}[htpb]
	\centering
	\caption{}
	\label{tab:1}
	\begin{tabular}{|c|c|c|c|c|c|c|}\hline
		$n$ & 1 & 2 & 3 & 4 & 5\\ \hline
		$\left<x^n \right>$ & 0 &  $1 / 2a$ & 0 & $3 / 4a^2$ & 0 \\ \hline
	\end{tabular}
\end{table}
\end{itemize}
\section{Размерность $N$}
\begin{itemize}
\item Матрица $S$, которая диагонализует матрицу $A_{ij}$
	может иметь, например, следующий вид
	\[
	S=
\begin{pmatrix}
% 1 & -1 & 1 \\
% 1 & -1 & -1 \\
% 1 & 2 & 0 \\
 -\frac{1}{\sqrt{3}} & -\frac{1}{\sqrt{6}} & -\frac{1}{\sqrt{2}} \\
 -\frac{1}{\sqrt{3}} & -\frac{1}{\sqrt{6}} & \frac{1}{\sqrt{2}} \\
 -\frac{1}{\sqrt{3}} & \sqrt{\frac{2}{3}} & 0 \\
\end{pmatrix}
.\]
Нетрудно заметить, что данная матрица --- ортогональна (такое
возможно благодаря положительной определённости $A$).

Диагональный вид матрицы $A$
 \[
	 \operatorname{diag}(\lambda_i)=
	 \begin{pmatrix} 12 & 0 & 0 \\ 0 & 6 & 0 \\ 0 & 0 & 2 \end{pmatrix} 
.\] 
%\[
%	S^{-1}DS=\left(S^{-1}\right)_{ij}D_{jk}S_{kl}
%.\] 
%\[
%	A_{ij}x_i x_j=\left(S^{-1}\right)_{ik}D_{kl}S_{lj}x_i x_j=
%	\left[x_i \left(S^{-1}\right)_{ik} \right] D_{kl} \left[ S_{lj}
%	x_j\right]=
%	\left[x_i \left(S^{-1}\right)_{ik} \right] D_{kk} \left[ S_{kj}
%	x_j\right]=
%.\] 
\begin{multline*}
	A_{ij}x_i x_{j}=\mathbf{x}^TA\mathbf{x}=
	\mathbf{x}^T S^{-1} \operatorname{diag}(\lambda_i)S \mathbf{x}
	\xlongequal[]{S \text{ --- орт.}}
	\mathbf{x}^T S^T\operatorname{diag}(\lambda_i)S\mathbf{x}=\\=
	(S\mathbf{x})^T\operatorname{diag}(\lambda_i)(S\mathbf{x})\xlongequal[]{\mathbf{y}=S\mathbf{x}}\sum_{i=1}^{3} \lambda_i y_i^2
.\end{multline*} 
Удачной заменой координат в данном случае кажется $y_i=S_{ij} x_j$. Тогда выражение для произвольного кореллятора принимает вид
\begin{multline*}
\left<x_{a_1}x_{a_2}x_{a_3}\ldots x_{a_n} \right> =\\=
\frac{1}{Z_N(A_{ij})}
\int |\det S| \prod_{i=1}^{N}  dy_i \prod_{i=1}^{n} y_{a_i}
\exp \left(- \sum_{i=1}^{N} \lambda_i y_i^2\right)\xlongequal[]{S \text{ --- орт.}}\\=
\frac{1}{Z_N(A_{ij})}
\int \prod_{i=1}^{N}  dy_i \prod_{i=1}^{n} y_{a_i}
\exp \left(- \sum_{i=1}^{N} \lambda_i y_i^2\right)
,\end{multline*}
а также
\begin{multline*}
	Z_N (A_{ij},\,B_i)= \int |\det S| \prod_{i=1}^{N} dy_i
	\exp \left(-\sum_{i=1}^{N} \lambda_i y_i^2+ B_iy_i  \right) \xlongequal[]{S \text{ --- орт.}}\\= \int \prod_{i=1}^{N} dy_i
	\exp \left(-\sum_{i=1}^{N} \lambda_i y_i^2+ B_iy_i  \right) 
.\end{multline*} 
\item Последовательно интегрируя по каждой переменной, пользуясь
	расчётами для одномерного случая, получаем
\begin{multline*}
	Z_N (A_{ij},\,B_i)=\int \prod_{i=1}^{N} dy_i
	\exp \left(-\sum_{i=1}^{N} \lambda_i y_i^2+ B_iy_i  \right) =\\=\sqrt{\frac{\pi^N}{\prod_{i=1}^{N} \lambda_i }} \exp\left( \sum_{i=1}^{N} \frac{B_i^2}{4\lambda_i} \right) =
	\sqrt{\frac{\pi^N}{\det A}} \exp \left( \frac{1}{4}
	B_i\left(A^{-1}\right)_{ij}B_j\right) 
.\end{multline*} 
\item \[
\left<x_{a_1}x_{a_2}x_{a_3}\ldots x_{a_n} \right>Z_N(A_{ij}) =
\int \prod_{i=1}^{N}  dx_i \prod_{i=1}^{n} x_{a_i}
\exp \left(- A_{ij} x_i x_j\right)
.\]
\begin{multline*}
Z_N(A_{ij}) \frac{\prod_{i=1}^{n} B_{a_i} }{n!}\left<x_{a_1}x_{a_2}x_{a_3}\ldots x_{a_n} \right> =\\=
\frac{\prod_{i=1}^{n} B_{a_i} }{n!}\int \prod_{i=1}^{N}  dx_i \prod_{i=1}^{n} x_{a_i}
\exp \left(- A_{ij} x_i x_j\right)
.\end{multline*}
\begin{multline*}
Z_N(A_{ij})\sum_{n=0}^{\infty} \frac{\prod_{i=1}^{n}B_{a_i}  }{n!} \left<x_{a_1}x_{a_2}x_{a_3}\ldots x_{a_n} \right> =\\=
\sum_{n=0}^{\infty} \frac{\prod_{i=1}^{n} B_{a_i} }{n!}
\int \prod_{i=1}^{N} dx_i \prod_{i=1}^{n} x_{a_i}\exp (-A_{ij}
x_i x_j)\xlongequal[]{ \text{инт. сх}}\\=
\int \prod_{i=1}^{N}  dx_i\sum_{n=0}^{\infty} \frac{\prod_{i=1}^{n} 
B_{a_i}x_{a_i}}{n!}
\exp \left(- A_{ij} x_i x_j\right)= \\=
\int \prod_{i=1}^{N}  dx_i 
\exp(B_i x_i)\exp \left(- A_{ij} x_i x_j\right)= \\=
\int \prod_{i=1}^{N}  dx_i 
\exp \left(- A_{ij} x_i x_j+B_i x_i\right)
.\end{multline*}
Следовательно
\[
	Z_N (A_{ij},\,B_i)= Z_N (A_{ij}) \sum_{n=0}^{\infty} \frac{\prod_{i=1}^{n} B_{a_i} }{n!} \left<x_{a_1}x_{a_2}x_{a_3}\ldots
	x_{a_n}\right>
.\] 
\item 
	\begin{multline*}
Z_N(A_{ij},\,B_i)=
\sqrt{\frac{\pi^N}{\det A}} \exp\left( \frac{1}{4}B_i \left( A^{-1} \right) _{ij}B_i \right) =\\=Z_N(A_{ij})\exp \left( \frac{1}{4}
B_i \left( A^{-1} \right) _{ij}B_j\right) 
=\\= Z_N(A_{ij}) \sum_{n=0}^{\infty}  \frac{\prod_{i=1}^{n} B_{a_i} }{n!}\left. \frac{\partial ^n}{\partial B_{a_1}\partial B_{a_2}\ldots\partial B_{a_n}} \exp\left(B_i \left( A^{-1} \right) _{ij}B_j\right)
\right|_{B_i=\mathbf{0}}	.\end{multline*} 
Значит
\[
\left<x_{a_1}x_{a_2}x_{a_3}\ldots x_{a_n} \right>=
\left. \frac{\partial^n}{\partial B_{a_1}\partial B_{a_2}\ldots\partial B_{a_n}} \exp\left(B_i \left( A^{-1} \right) _{ij}B_j\right)
\right|_{B_i=\mathbf{0}}
.\] 
\item Для непосредственного вычисления корелляторов $\left<x_{a_1} \right>$, $\left<x_{a_1} x_{a_2}\right>$, $\left<x_{a_1} x_{a_2}x_{a_3} \right>$, $\left<x_{a_1} x_{a_2}x_{a_3} x_{a_4}\right>$, $\left<x_{a_1} x_{a_2}x_{a_3} x_{a_4} x_{a_5}\right>$
найдём
\begin{multline*}
	\frac{\partial }{\partial B_k} \exp\left( \frac{1}{4} B_i
	\left(A^{-1}\right)_{ij} B_j\right)=\\=
	\frac{1}{4}\exp \left( \frac{1}{4} B_i \left( A^{-1} \right)_{ij} B_j  \right) \left( 2B_k \left( A^{-1} \right) _{kk}
	+B_{i \neq k} \left( A^{-1} \right) _{ik}+
B_{j\neq k}\left( A^{-1} \right) _{kj}\right)=\\=
\frac{1}{4} \exp\left( \frac{1}{4}B_i \left( A \right)_{ij}B_j  \right) 
\left( B_i \left( A^{-1} \right) _{ik}+ B_j \left( A^{-1} \right) _{kj} \right) \xlongequal[]{A \text{ --- сим.}} \\=
\frac{B_i\left( A^{-1} \right) _{ik}}{2} \exp \left( \frac{1}{4}B_i \left( A^{-1}\right)_{ij}B_j   \right) 
,\end{multline*} 
\[
\frac{\partial }{\partial B_k} \prod_{i=1}^{n} B_{a_i}=
\sum_{j=1}^{n} \prod_{i=1}^{n} \delta_{a_j k}B_{a_i}
.\]
\[
	 \frac{\partial}{\partial B_{a_1}}\exp\left( B_i\left(A^{-1}\right)_{ij} B_j \right)  =
		\frac{B_i\left( A^{-1} \right) _{ia_1}}{2} \exp \left( \frac{1}{4}B_i \left( A^{-1}\right)_{ij}B_j\right)
.\] 
\begin{multline*}
	\frac{\partial^2}{\partial B_{a_1}\partial B_{a_2}}\exp\left( B_i\left(A^{-1}\right)_{ij} B_j \right)  =
		\frac{\left( A^{-1} \right) _{a_2a_1}}{2} \exp \left( \frac{1}{4}B_i \left( A^{-1}\right)_{ij}B_j\right)+\\+
\frac{B_{i_1}\left( A^{-1} \right) _{i_1a_1}B_{i_2}\left( A^{-1} \right) _{i_2a_2}}{4} \exp \left( \frac{1}{4}B_i \left( A^{-1}\right)_{ij}B_j\right)
.\end{multline*}
\begin{multline*}
	\frac{\partial^3}{\partial B_{a_1}\partial B_{a_2}\partial B_{a_3}}\exp\left( B_i\left(A^{-1}\right)_{ij} B_j \right)  =\\=
	\frac{\left( A^{-1} \right) _{a_2a_1}B_i (A^{-1})_{ia_3}}{4} \exp \left( \frac{1}{4}B_i \left( A^{-1}\right)_{ij}B_j\right)+\\+
\frac{\left( A^{-1} \right) _{a_3a_1}B_{i}\left( A^{-1} \right) _{ia_2}}{4} \exp \left( \frac{1}{4}B_i \left( A^{-1}\right)_{ij}B_j\right)+\\+
\frac{B_{i}\left( A^{-1} \right) _{ia_1}\left( A^{-1} \right) _{a_3a_2}}{4} \exp \left( \frac{1}{4}B_i \left( A^{-1}\right)_{ij}B_j\right)+\\+
\frac{B_{i_1}\left( A^{-1} \right) _{i_1a_1}B_{i_2}\left( A^{-1} \right) _{ia_2}B_{i_3}\left( A^{-1} \right) _{i_3a_2}}{6} \exp \left( \frac{1}{4}B_i \left( A^{-1}\right)_{ij}B_j\right)
.\end{multline*}
\begin{multline*}
	\frac{\partial^4}{\partial B_{a_1}\partial B_{a_2}\partial B_{a_3}\partial B_{a_4}}\exp\left( B_i\left(A^{-1}\right)_{ij} B_j \right)  =\\=
	\frac{\left( A^{-1} \right) _{a_2a_1}(A^{-1})_{a_4a_3}}{4} \exp \left( \frac{1}{4}B_i \left( A^{-1}\right)_{ij}B_j\right)+\\+
\frac{\left( A^{-1} \right) _{a_3a_1}\left( A^{-1} \right) _{a_4a_2}}{4} \exp \left( \frac{1}{4}B_i \left( A^{-1}\right)_{ij}B_j\right)+\\+
\frac{\left( A^{-1} \right) _{a_4a_1}\left( A^{-1} \right) _{a_3a_2}}{4} \exp \left( \frac{1}{4}B_i \left( A^{-1}\right)_{ij}B_j\right)+\\+
\frac{B_{i_1}\left( A^{-1} \right) _{i_1a_1}B_{i_2}\left( A^{-1} \right) _{ia_2}B_{i_3}\left( A^{-1} \right) _{i_3a_2}B_{i_4}\left( A^{-1} \right) _{i_4a_4}}{8} \exp \left( \frac{1}{4}B_i \left( A^{-1}\right)_{ij}B_j\right)
.\end{multline*}
\begin{multline*}
	\frac{\partial^5}{\partial B_{a_1}\partial B_{a_2}\partial B_{a_3}\partial B_{a_4}\partial B_{a_5}}\exp\left( B_i\left(A^{-1}\right)_{ij} B_j \right)  =\\=
	\frac{\left( A^{-1} \right) _{a_2a_1}(A^{-1})_{a_4a_3}
	B_i (A^{-1})_{ia_5}}{6} \exp \left( \frac{1}{4}B_i \left( A^{-1}\right)_{ij}B_j\right)+\\+
\frac{\left( A^{-1} \right) _{a_3a_1}\left( A^{-1} \right) _{a_4a_2}B_i (A^{-1})_{ia_5}}{6} \exp \left( \frac{1}{4}B_i \left( A^{-1}\right)_{ij}B_j\right)+\\+
\frac{\left( A^{-1} \right) _{a_4a_1}\left( A^{-1} \right) _{a_3a_2}B_i (A^{-1})_{ia_5}}{6} \exp \left( \frac{1}{4}B_i \left( A^{-1}\right)_{ij}B_j\right)+\\+
\frac{\left( A^{-1} \right) _{a_5a_1}B_{i_2}\left( A^{-1} \right) _{ia_2}B_{i_3}\left( A^{-1} \right) _{i_3a_2}B_{i_4}\left( A^{-1} \right) _{i_4a_4}}{8} \exp \left( \frac{1}{4}B_i \left( A^{-1}\right)_{ij}B_j\right)+\\+
\frac{B_{i_1}\left( A^{-1} \right) _{i_1a_1}\left( A^{-1} \right) _{a_5a_2}B_{i_3}\left( A^{-1} \right) _{i_3a_2}B_{i_4}\left( A^{-1} \right) _{i_4a_4}}{8} \exp \left( \frac{1}{4}B_i \left( A^{-1}\right)_{ij}B_j\right)+\\+
\frac{B_{i_1}\left( A^{-1} \right) _{i_1a_1}B_{i_2}\left( A^{-1} \right) _{ia_2}\left( A^{-1} \right) _{a_5a_2}B_{i_4}\left( A^{-1} \right) _{i_4a_4}}{8} \exp \left( \frac{1}{4}B_i \left( A^{-1}\right)_{ij}B_j\right)+\\+
\frac{B_{i_1}\left( A^{-1} \right) _{i_1a_1}B_{i_2}\left( A^{-1} \right) _{ia_2}B_{i_3}\left( A^{-1} \right) _{i_3a_2}\left( A^{-1} \right) _{a_5a_4}}{8} \exp \left( \frac{1}{4}B_i \left( A^{-1}\right)_{ij}B_j\right)
.\end{multline*}
%\[
%\left<x_{a_1}x_{a_2}x_{a_3} \right> =
%\frac{(A^{-1})_{a_1 a_}}{2}
%.\] 
\end{itemize}
\end{document}
