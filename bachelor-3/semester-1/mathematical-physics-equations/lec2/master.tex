\documentclass[a4paper]{article}
% Этот шаблон документа разработан в 2014 году
% Данилом Фёдоровых (danil@fedorovykh.ru) 
% для использования в курсе 
% <<Документы и презентации в \LaTeX>>, записанном НИУ ВШЭ
% для Coursera.org: http://coursera.org/course/latex .
% Исходная версия шаблона --- 
% https://www.writelatex.com/coursera/latex/5.3

% В этом документе преамбула

\usepackage{siunitx}
%%% Работа с русским языком
%\usepackage{cmap}					% поиск в PDF
%\usepackage{mathtext} 				% русские буквы в формулах
%\usepackage[T2A]{fontenc}			% кодировка
%\usepackage[utf8]{inputenc}			% кодировка исходного текста
%\usepackage[english,russian]{babel}	% локализация и переносы
%\usepackage{indentfirst}
%\frenchspacing
%
%\renewcommand{\epsilon}{\ensuremath{\varepsilon}}
%\newcommand{\phibackup}{\ensuremath{\phi}}
%\renewcommand{\phi}{\ensuremath{\varphi}}
%\renewcommand{\varphi}{\ensuremath{\phibackup}}
%\renewcommand{\kappa}{\ensuremath{\varkappa}}
%\renewcommand{\le}{\ensuremath{\leqslant}}
%\renewcommand{\leq}{\ensuremath{\leqslant}}
%\renewcommand{\ge}{\ensuremath{\geqslant}}
%\renewcommand{\geq}{\ensuremath{\geqslant}}
%\renewcommand{\emptyset}{\varnothing}
%\renewcommand{\Im}{\operatorname{Im}}
%\renewcommand{\Re}{\operatorname{Re}}


%%% Дополнительная работа с математикой
\usepackage{amsmath,amsfonts,amssymb,amsthm,mathtools} % AMS
%\usepackage{icomma} % "Умная" запятая: $0,2$ --- число, $0, 2$ --- перечисление

%% Номера формул
%\mathtoolsset{showonlyrefs=true} % Показывать номера только у тех формул, на которые есть \eqref{} в тексте.
%\usepackage{leqno} % Нумереация формул слева

%% Свои команды
\DeclareMathOperator{\sgn}{\mathop{sgn}}
\DeclareMathOperator{\sign}{\mathop{sign}}
\DeclareMathOperator*{\res}{\mathop{res}}
\DeclareMathOperator*{\tr}{\mathop{tr}}
\DeclareMathOperator*{\rot}{\mathop{rot}}
\DeclareMathOperator*{\divop}{\mathop{div}}
\DeclareMathOperator*{\grad}{\mathop{grad}}

%% Перенос знаков в формулах (по Львовскому)
\newcommand*{\hm}[1]{#1\nobreak\discretionary{}
{\hbox{$\mathsurround=0pt #1$}}{}}

%%% Работа с картинками
\usepackage{graphicx}  % Для вставки рисунков
\graphicspath{{figures/}}  % папки с картинками
\setlength\fboxsep{3pt} % Отступ рамки \fbox{} от рисунка
\setlength\fboxrule{1pt} % Толщина линий рамки \fbox{}
\usepackage{wrapfig} % Обтекание рисунков текстом

%%% Работа с таблицами
\usepackage{array,tabularx,tabulary,booktabs} % Дополнительная работа с таблицами
\usepackage{longtable}  % Длинные таблицы
\usepackage{multirow} % Слияние строк в таблице

%%% Теоремы
\theoremstyle{plain} % Это стиль по умолчанию, его можно не переопределять.
\newtheorem{thm}{Теорема}
\newtheorem*{thm*}{Теорема}
\newtheorem{prop}{Предложение}
\newtheorem*{prop*}{Предложение}
 
\theoremstyle{definition} % "Определение"
%\newtheorem{corollary}{Следствие}[theorem]
\newtheorem{dfn}{Определение}
\newtheorem*{dfn*}{Определение}
\newtheorem{prob}{Задача}
\newtheorem*{prob*}{Задача}

 
\theoremstyle{remark} % "Примечание"
\newtheorem*{sol}{Решение}
\newtheorem*{rem}{Замечание}

%%% Программирование
\usepackage{etoolbox} % логические операторы

%%% Страница
%\usepackage{extsizes} % Возможность сделать 14-й шрифт
%\usepackage{geometry} % Простой способ задавать поля
%	\geometry{top=25mm}
%	\geometry{bottom=35mm}
%	\geometry{left=35mm}
%	\geometry{right=20mm}
 
\usepackage{fancyhdr} % Колонтитулы
%	\pagestyle{fancy}
 %	\renewcommand{\headrulewidth}{0pt}  % Толщина линейки, отчеркивающей верхний колонтитул
	%\lfoot{Нижний левый}
	%\rfoot{Нижний правый}
	%\rhead{Верхний правый}
	%\chead{Верхний в центре}
	%\lhead{Верхний левый}
	%\cfoot{Нижний в центре} % По умолчанию здесь номер страницы

\usepackage{setspace} % Интерлиньяж
%\onehalfspacing % Интерлиньяж 1.5
%\doublespacing % Интерлиньяж 2
%\singlespacing % Интерлиньяж 1

\usepackage{lastpage} % Узнать, сколько всего страниц в документе.

\usepackage{soul} % Модификаторы начертания

\usepackage{hyperref}
\usepackage[usenames,dvipsnames,svgnames,table,rgb]{xcolor}
\hypersetup{				% Гиперссылки
    unicode=true,           % русские буквы в раздела PDF
    pdftitle={Заголовок},   % Заголовок
    pdfauthor={Автор},      % Автор
    pdfsubject={Тема},      % Тема
    pdfcreator={Создатель}, % Создатель
    pdfproducer={Производитель}, % Производитель
    pdfkeywords={keyword1} {key2} {key3}, % Ключевые слова
%    colorlinks=true,       	% false: ссылки в рамках; true: цветные ссылки
    %linkcolor=red,          % внутренние ссылки
    %citecolor=black,        % на библиографию
    %filecolor=magenta,      % на файлы
    %urlcolor=cyan           % на URL
}

\usepackage{csquotes} % Еще инструменты для ссылок

%\usepackage[style=apa,maxcitenames=2,backend=biber,sorting=nty]{biblatex}

\usepackage{multicol} % Несколько колонок

\usepackage{tikz} % Работа с графикой
\usepackage{pgfplots}
\usepackage{pgfplotstable}
%\usepackage{coloremoji}
\usepackage{floatrow}
\usepackage{subcaption}
\graphicspath{{figures/}}

\renewcommand\thesubfigure{\asbuk{subfigure}}
%\addbibresource{master.bib}

\usepackage{import}
\usepackage{pdfpages}
\usepackage{transparent}
\usepackage{xcolor}
\usepackage{xifthen}

\newcommand{\incfig}[2][1]{%
    \def\svgwidth{#1\columnwidth}
    \import{./figures/}{#2.pdf_tex}
}
%\usepackage{titlesec}
%\titleformat{\section}{\normalfont\Large\bfseries}{}{0pt}{}
%----------------------STANDART:
%\titleformat{\chapter}[display]
%  {\normalfont\huge\bfseries}{\chaptertitlename\ \thechapter}{20pt}{\Huge}
%\titleformat{\section}{\normalfont\Large\bfseries}{\thesection}{1em}{}
%\titleformat{\subsection}
%  {\normalfont\large\bfseries}{\thesubsection}{1em}{}
%\titleformat{\subsubsection}
%  {\normalfont\normalsize\bfseries}{\thesubsubsection}{1em}{}
%\titleformat{\paragraph}[runin]
%  {\normalfont\normalsize\bfseries}{\theparagraph}{1em}{}
%\titleformat{\subparagraph}[runin]
%  {\normalfont\normalsize\bfseries}{\thesubparagraph}{1em}{}

\pdfsuppresswarningpagegroup=1
\pgfplotsset{compat=1.16}



%\setcounter{tocdepth}{1} % only parts,chapters,sections
%\titleformat{\subsection}{\normalfont\large\bfseries}{}{0em}{}
%\titleformat{\subsubsection}{\normalfont\normalsize\bfseries}{}{0em}{}

%\newcommand{\textover}[2]{\stackrel{\mathclap{\normalfont\mbox{#2}}}{#1}}

\author{Yaroslav Drachov\\
Moscow Institute of Physics and Technology}
%\author{Драчов Ярослав\\
%Факультет общей и прикладной физики МФТИ}
\newcommand{\veq}{\mathrel{\rotatebox{90}{$=$}}}
%\newcommand{\teto}[1]{\stackrel{\mathclap{\normalfont\tiny\mbox{#1}}}{\to}}
%\renewcommand{\thesubsection}{\arabic{subsection}}

%%\setcounter{secnumdepth}{0}

\definecolor{tabblue}{RGB}{30, 119, 180}
\definecolor{taborange}{RGB}{255, 127, 15}
\definecolor{tabgreen}{RGB}{45, 160, 43}
\definecolor{tabred}{RGB}{214, 38, 40}
\definecolor{tabpurple}{RGB}{148, 103, 189}
\definecolor{tabbrown}{RGB}{140, 86, 76}
\definecolor{tabpink}{RGB}{227, 119, 193}
\definecolor{tabgray}{RGB}{127, 127, 127}
\definecolor{tabolive}{RGB}{188, 189, 33}
\definecolor{tabcyan}{RGB}{22, 190, 207}
\pgfplotscreateplotcyclelist{colorbrewer-tab}{
{tabblue},
{taborange},
{tabgreen},
{tabred},
{tabpurple},
{tabbrown},
{tabpink},
{tabgray},
{tabolive},
{tabcyan},
}
\usepackage{csvsimple}
\usepackage{extarrows}
%\renewcommand{\labelenumii}{\asbuk{enumii})}
%\renewcommand{\labelenumiv}{\Asbuk{enumiv}}
%\newcommand{\prob}[1]{\subsubsection*{#1}}
\sisetup{output-decimal-marker = {,},separate-uncertainty = true,exponent-product = \cdot}

\usepackage{braket}
\usepackage{enumerate}
\usepackage{chngcntr}
%\counterwithin*{equation}{problem}
%\usepackage{bbold}

\newtheoremstyle{hiProb}% ⟨name ⟩ 
{3pt}% ⟨Space above ⟩1 
{3pt}% ⟨Space below ⟩1
{}% ⟨Body font ⟩
{}% ⟨Indent amount ⟩2
{\bfseries}% ⟨Theorem head font⟩
{.}% ⟨Punctuation after theorem head ⟩
{.5em}% ⟨Space after theorem head ⟩3
%{\thmname{#1} \thmnote{#3}}% ⟨Theorem head spec (can be left empty, meaning ‘normal’)⟩
{\thmnote{#3}}% ⟨Theorem head spec (can be left empty, meaning ‘normal’)⟩
\theoremstyle{hiProb} % "Определение"
%\newtheorem{hiProb}{Задача}
\newtheorem{hiProb}{}
%\usepackage{mmacells}
\newcommand{\textover}[2]{\stackrel{\mathclap{\normalfont\scriptsize\mbox{#2}}}{#1}}
\usepackage{units}
\usepackage[math]{cellspace}%
\setlength\cellspacetoplimit{2pt}
\setlength\cellspacebottomlimit{2pt}

\DeclareMathAlphabet{\mathbbold}{U}{bbold}{m}{n}

\newcommand{\normord}[1]{:\mathrel{#1}:}

\title{Введение в функциональное исчисление операторов в гильбертовом пр-ве}
\begin{document}
	\maketitle
Рассмотрим сначала конечно-мерное евклидово пространство $\mathcal{E}^n$ размерности $n \in \mathbb{N}$ 
(оно автоматически  полное относительно евклидовой нормы)
и рассмотрим линейный оператор $A : \mathcal{E}_n\to \mathcal{E}_n$.
Пусть $e=\{e_1,\ldots,e_n\} $ --- ОНБ в $\mathcal{E}_n$. Тогда в базисе $e$ оператор $A$ задаётся $n \times n$ 
матрицей $A \in  \mathbb{C}^{n \times n}$, так что $\forall x \in  \mathcal{E}_n$
с координатным  столбцом $\xi \in  \mathbb{C}^n$ в базисе $e$ координаты $A(x)$ 
образуют столбец $A \xi \in  \mathbb{C}^n$.
Алгебраически мы можем определить оператор
\[
	A^* : \mathcal{E}_n \to  \mathcal{E}_n
 ,\]
 сопряжённый к оператору $A$, так, чтобы
 \begin{align*}
	 (A(&x),\,y)= (x,\, A^* (y)) \quad \forall x,\, y \in \mathcal{E}_n\\
	    & \veq\\
	 (A &\xi)^T \overline{\eta}= \eta^T A^T \overline{\eta} = \xi \overline{\overline{
	 A^T}\eta},
 \end{align*} 
 где $x= e \xi$ и $y = e \eta$, $\xi,\,\eta \in \mathbb{C}^n$.
 То есть матрица $A^*$ оператора  $A^*$ в базисе $e$ равна
 \[
	 \overline{A^T} \in  \mathbb{C}^{n \times n}
 .\] 
\begin{dfn}
Оператор $A$ называется \emph{самосопряжённым}, если
\[
A=A^* \Leftrightarrow A= \overline{A^T}
.\] 
\end{dfn} 
Как хорошо известно из линейной алгебры, все собственные
числа самосопряжённого оператора $A: \mathcal{E}_n \to  \mathcal{E}_n$ вещественны, а его собственные векторы, отвечающие различным
собственным числам --- ортогональны в $\mathcal{E}_n$.
Обозначим $\Lambda \subset \mathbb{R}$ --- все собственные
числа ССО $A$, это конечное множество,
\[
	\forall \lambda \in \Lambda \implies \underbrace{\operatorname{Ker}
(A- \lambda I)}_{\substack{\text{собственное
подпр-во,}\\ \text{ соответствующее}\\
\lambda \in \Lambda}} \neq 0
\] 
(здесь $I : \mathcal{E}_n \to \mathcal{E}_n$  --- тождественный
оператор). Обозначим $A_\lambda\equiv A- \lambda I$.
Таким образом $\forall \lambda,\,\mu \in \Lambda : \lambda \neq
 \mu$, следовательно
\[
	\operatorname{Ker} A_\lambda \perp \operatorname{Ker}
	A_\mu
.\] 
Более того, хорошо известно, что
\[
\sum_{\lambda \in \Lambda}^{} \dim  \operatorname{Ker}
A_\lambda = n,
\]
т.\:е. имеет место равенство
\[
	\underset{\lambda \in \Lambda}{\oplus} \operatorname{Ker}
	A_\lambda = \mathcal{E}_n \text{ !}
\] 
Этот факт ещё формулируется как существование у ССО
в конечно-мерном евкл. $\mathcal{E}_n$ ортогонального базиса
из его собственных векторов. Можно это переформулировать
на языке ортогональных  проекторов в $\mathcal{E}_n$:
$\forall \lambda \in  \Lambda$ обозначим $P(\lambda):
 \mathcal{E}_n \to  \operatorname{Ker}A_\lambda$ --- ортопроектор
 из $\mathcal{E}_n$ на $\operatorname{Ker}A_\lambda$,
 то есть $\forall x \in  \mathcal{E}_n \implies P(\lambda) x$ ---
 ортогональная проекция $x$ на собственное подпр-во
$\operatorname{Ker}A_\lambda$ оператора $A$
 \begin{align*}
	  \lambda,\, \mu \in  \Lambda : \lambda \neq \mu &\implies
	 \operatorname{Ker} A_{\lambda} \perp \operatorname{Ker}A_\mu \\
&\Downarrow \\
	 \forall x \in \mathcal{E}_n &\implies P(\lambda) x
\perp P(\mu) x,\end{align*} 
то есть образы  $P(\lambda)$ и $P(\mu)$ ортогональны.
При этом 
\[
	\underset{\lambda \in \Lambda}{\oplus} \operatorname{Ker}
	A_\lambda = \mathcal{E}_n \implies \forall
	x \in \mathcal{E}_n \implies x = \sum_{\lambda \in \Lambda}^{
	} P(\lambda) x
.\] 
То есть $I = \sum_{\lambda \in \Lambda}^{} P(\lambda)$ 
--- это разложение тождественного в $\mathcal{E}_n$ ( то есть 
единичного) оператора в конечную сумму ортогональных проекторов
на собственные подпространства ССО $A$. Так как
$\forall \lambda \in  \Lambda \; \forall x \in \mathcal{E}_n$ 
имеем $P(\lambda) x \in  \operatorname{Ker}A_\lambda$, то
\begin{align*}
	\underbrace{A_\lambda P(\lambda) x}_{\veq} &= 0\\
	AP(\lambda)x - \lambda P (&\lambda) x =0
,\end{align*} 
т.\:е.
\[
	AP(\lambda) x = \lambda P(\lambda) x
.\] 
Тогда $\forall x \in  \mathcal{E}_n$ имеем
\[
	x = \sum_{\lambda \in \Lambda}^{} P(\lambda) x
	\implies Ax= \sum_{\lambda \in \Lambda}^{} A P(\lambda) x=
	\sum_{\lambda \in  \Lambda}^{} \lambda P (\lambda) x,
\] 
т.\:е.
\[
	A = \sum_{\lambda \in  \Lambda}^{} \lambda P(\lambda):
	\mathcal{E}_n \to \mathcal{E}_n
\]
--- это разложение $A$ в сумму ортопроекторов на собственные подпространства ССО $A$ с коэфф. из соотв. собственных чисел $A$.
\begin{dfn}
Это разложение наз. \emph{спектральным разложением} ССО $A$.
\end{dfn}
Итак, если $A: \mathcal{E}_n \to \mathcal{E}_{n}$ --- ССО (самосопряжённый оператор),
\[
\Lambda = \left\{ \substack{\text{все собств.}\\
\text{числа} A} \right\}  \subset \mathbb{R}
,\] 
\[
	\forall \lambda \in \Lambda \hookrightarrow P(\lambda)
	: \mathcal{E}_n \to  \operatorname{Ker}A_\lambda
\text{ --- ортопроектор}
,\] 
то
\[
	A = \sum_{\lambda \in \Lambda}^{} \lambda P(\lambda)
.\] 
Заметим, что $\forall \lambda,\,\mu : \lambda \neq \mu$ имеем
$P(\lambda)P(\mu)=0$ и $P(\mu)P(\lambda)=0$, так как
$\operatorname{Im} P(\lambda)= \operatorname{Ker}A_\lambda$ и
$\operatorname{Im}P(\mu)= \operatorname{Ker}A_\mu$ --- \emph{ортогональные подпр-ва}!

При этом также по определению ортопроектора очевидно, что
$\forall \lambda \in \Lambda$ вып.
\[
	(P(\lambda))^2=P(\lambda)\implies
	(P(\lambda))^k =P(\lambda) \quad \forall k \in \mathbb{N}
.\] 
Следовательно, из разложения
\[
	A= \sum_{\lambda \in \Lambda}^{} \lambda P(\lambda):
	\mathcal{E}_n \to  \mathcal{E}_n
\]
мы $\forall k \in \mathbb{N}$ находим:
\[
	A^k= \sum_{\lambda \in \Lambda}^{} \lambda^k P(\lambda):
	\mathcal{E}_n \to \mathcal{E}_n
.\] 
Тогда для любого многочлена
\[
	T(z)= a_0 +a_1 z + \ldots+ a_N z^N,\qquad 
	a_0, \ldots, a_n \in \mathbb{C},\, N \in \mathbb{N}
\]
получаем равенство
\[
	T(A) = \sum_{\lambda \in \Lambda}^{} T(\lambda)
	P(\lambda): \mathcal{E}_n \to \mathcal{E}_n
.\] 
Теперь для любой целой функции $f : \mathbb{C}\to \mathbb{C}$ 
вида $f(z) = \sum_{k=0}^{\infty} a_k z^k, \; z \in \mathbb{C}$ с
радиусом сходимости $\infty$ рассмотрим оператор
\[
	f(A)= \sum_{\lambda \in \Lambda} f(\lambda)P(\lambda):
	\mathcal{E}_n\to \mathcal{E}_n
.\] 
Для частичной суммы
$f_N(z)=\sum_{k=0}^{N} a_k z^k$ мы имеем
 \[
	 f_N (A)=\sum_{\lambda \in \Lambda}^{} f_N (\lambda)
	 P(\lambda), 
\]
и тогда получим сходимость
$f_N(A)$ и $f(A)$ по операторной норме:
\[
	\| f(A) -f_N (A)\|= \left\lVert \sum_{\lambda \in \Lambda
	}^{} \left( f(\lambda) - f_N (\lambda) \right) P(\lambda)\right\rVert\le  \sum_{\lambda \in \Lambda}^{}|f(\lambda)-
f_N(\lambda)| \| P(\lambda)\|\xrightarrow[]{N\to \infty}0
,\] 
т.\:к. $|f(\lambda)-f_N(\lambda)|\xrightarrow[N\to \infty]{}0 $(
$\forall \lambda \in \Lambda$ --- конечное мн-во),
$\|P(\lambda)\|=1$ (как норма  $\forall$ нетривиального
ортопроектора).
Т.\:о. $\| f(A)-f_N(A)\|\xrightarrow[]{N\to  \infty}0$.

Можно пойти дальше и вообще для любой функции $\phi: \mathbb{C}\to 
\mathbb{C}$ (не обязательно даже непрерывной), определить
\[
	\phi(A)= \sum_{\lambda \in \Lambda}^{} \phi(\lambda)
	P(\lambda) :
	\mathcal{E}_n \to  \mathcal{E}_n
.\] 
При этом $\phi(\Lambda)$ --- это все собств. числа $\phi(A)$.
Итак, для самосопряжённых операторов, действующих в конечномерном
евклидовом пространстве, мы в терминах
их спектрального разложения в сумму ортопроекторов мы определили
суперпозицию $\forall$ комплексной функции с $\forall$ такими
оператором.

Чтобы обобщить этот подход и на бесконечномерное гильбертово пространство,
требуется для любого оператора $A: D(A)\to H$ определить понятие
сопряжённого оператора \[\underbrace{A^*}_{?}: \underbrace{D(A^*)}_{?} \to  H,\]
и далее для самосопр. операторов поолучить их спектр. разложение.

Приступим к определению сопряжённого оператора для произвольного
линейного оператора
$A : D(A) \to H$.

Хотим прежде всего понять, где должен быть определён сопряжённый
оператор, \underline{$D(A^*)=?$}

\underline{Желание}: $\forall f \in  D(A) \; \forall g \in 
В(A^*)=?$ иметь равенство 
 \[
	 \left( Af,\,g \right) =\left(f,\,\underbrace{A^*g}_{?}\right) \in \mathbb{C}
 .\] 

 \underline{Вопрос}: для каких $g \in  H$ можно найти вектор
 $h \in H$ такой, что $\forall f \in  D(A)$ выполнено
 $\left( Af,\,g \right) =(f,\,h)$. Так как отображение
\[
	D(A) \ni f \mapsto (f,\,h) \in \mathbb{C}
\]
является непрерывны по $f \in  D(A)$ в силу КБШ:
\[
	|(f,\,h)|\le \| f\| \| h\| \qquad F \in D(A)
\]
--- липшицево с конст. $\| h\|$,
то естественно определяем подпространство
\[
	D(A^*) = \left\{  g \in H :
\substack{ \text{отображение } \\ 
D(A) \ni f \mapsto (Af,\,g) \in \mathbb{C}\\
\text{ непрерывно}\\ \text{по } f \in D(A)}\right\} 
=
\left\{ g \in H:
\substack{\exists C_g > 0 \quad \forall f \in D(A) \text{ вып.}\\
|(Af,\,g)|\le C_g \| f\|}\right\} 
\]
($C_g>0$ --- это константа Липшица линейного непр. ф-ла
$D(A) \ni f \mapsto (Af,\,g) \in \mathbb{C}$).
Итак, $g \in  D(A^*)\subset H \Leftrightarrow
 D(A) \ni f \mapsto  (Af,\,g) \in \mathbb{C}$ непрерывно по 
$f \in D(A)$. Очевидно, что $0 \in D(A^*)$, и 
$\forall g_{1,\,2} \in  D(A^*)$ и $\alpha_{1,\,2} \in \mathbb{C}$ 
будет $g=\alpha_1 g_1 + \alpha_2 g_2 \in D(A^*)$, т.\:к.
$\forall f \in D(A)$ 
\[
	|(Af,\,g)|\le |\alpha_1|
	\underbrace{|(Af,\,g_1)|}_{\le C_{g_1}\| f\|}+
	|\alpha_2| \underbrace{|(Af,\,g_2)}_{\le C_{g_2}\| f\|}\le 
	(|\alpha_1| C_{g_1}+ |\alpha_2| C_{g_2}) \| f\|
.\] 
Теперь линейный непрерывный функционал $\phi : D(A) \to \mathbb{C}$ 
вида 
\[
	\phi(f) =(Af,\,g), \quad f \in D(A)
\]
(Здесь $g \in D(A^*)$ ) можно по непрерывности продолжить на
замыкание $D(A)$ --- замкнутое подпространство $\overline{D(A)}
 \subset H$. Действительно, мы имеем $C_g>0$:
 \[
	 |\phi(f)|= |(Af,\,g)|\le 
	 C_g \| f\| \forall f \in D(A)
 .\] 
 Теперь берём $\forall f \in \overline{D(A)}$, имеем последовательность
 \[
	 \{f_n\} \subset D(A): f_n \xrightarrow[]{\| \cdot\|}f,
\]
и тогда $|\phi(f_n) -\phi(f_m)=|\phi(f_n -f_m)|\le 
 C_g \| f_n -f_m\|\xrightarrow[(n,\,m\to \infty)]{} 0$.
Следовательно $\{\phi(f_n)\} $ --- фундам. числовая посл-ть,
т.\:е. $\exists \lim_{n \to \infty}\phi(f_n) \textover{=}{def} 
\mathrm{hi o}
\psi (t) \in \mathbb{C}$
Заметим, что число $\psi(t) \in \mathbb{C}$ не зависит от
выбора посл-ти $D(A) \ni f_n \xrightarrow[]{\| \cdot\|}f$.
Если $\tilde{f}_n \in D(A),\,\tilde{f}_n \xrightarrow[]{\| \cdot\|}$,
то
\begin{multline*}
	\left|\phi(f_n)- \phi \left( \tilde{f}_n \right) \right|
	\le  C_g \left\lVert  f_n - \tilde{f}_n\right\rVert\le 
	\underbrace{\| f_n - f\|}_{\to 0}+
	\underbrace{\left\lVert \tilde{f}_n - f\right\rVert}_{\to 0}\to 0 \implies \\ \implies
	\lim_{n \to \infty} \phi\left(\tilde{f}_n\right)=
	\lim_{n \to \infty} \phi(f_n) = \psi(t) \qquad
	(n\to  \infty)
.\end{multline*} 
Итак, мы определили функционал $\psi: \overline{D(A)}\to \mathbb{C}$, и $\forall f \in  \overline{D(A)} \implies |\psi(f)|\le 
 C_g \| f\|$, так как для $f_n \in D(A)\quad f_n \xrightarrow[]{\| \cdot\|} f $ имеем
 \[
	 |\psi(f)|= \lim_{n \to \infty} \underbrace{|\phi(f_n)}_{\le 
	 C_g \| f_n\|}\le C_g \underbrace{\lim_{n \to \infty} \| f_n\|}_{= \| f\|}=C_g \| f\|
 .\] 
 Также $\psi : \overline{D(A)}\to  \mathbb{C}$ очевидно линеен
 на $\overline{D(A)}$, т.\:к. для $\tilde{f}$ и $\hat{f} \in 
 \overline{D(A)}$ строим $\tilde{f}_n \in D(A)$ и 
 $\hat{f}_n \in D(A)$ вида 
 $\tilde{f}_n \xrightarrow[]{\| \cdot \|}\tilde{f}$ и $
 \hat{f}_n \xrightarrow[]{\| \cdot \|}\hat{f}$,
 и тогда $\forall \tilde{\alpha}$ и $\hat{\alpha}\in \mathbb{C}$
 имеем
\begin{multline*}
	\psi\left( \tilde{\alpha} \tilde{f}+ \hat{\alpha}
	\hat{f}\right) = \lim_{n \to \infty} \phi\left( 
\tilde{\alpha}\tilde{f}_n+ \hat{\alpha}\hat{f}_n\right) =
\lim_{n \to \infty} \left( \tilde{\alpha}\underbrace{
\phi\left( \tilde{f}_n \right) }_{\to \psi\left( \tilde{f} \right) }+\hat{\alpha} \underbrace{\phi\left( \tilde{f}_n \right) }_{\to 
\psi\left( \tilde{f} \right) } \right) =\\=
\tilde{\alpha} \psi\left( \tilde{f} \right) +
\hat{\alpha} \psi\left( \hat{f} \right) 
.\end{multline*} 

Итак, $\psi: \overline{D(A)}\to \mathbb{C}$ линейный
непрерывный функционал, $\psi\mid _{D(A)}\equiv \phi$.
При этом $\overline{D(A)}$ как замкнутое подпространство в $H$ 
само является Гильбертом пространством.

\underline{Вопрос}: каков общий вид линейного непрерывного функционала в гильб. пр-ве?

Ответ даёт теорема \emph{Рисса-Фреше}.
\begin{thm}[Рисса-Фреше]
Пусть $H$ --- гильб.
пр-во, $\phi : H \to \mathbb{C}$ линейный непрерывный
функционал. Тогда $\exists! h_\phi \in H: \phi(f) =(f,\,h_\phi)
 \; \forall f \in H$
\end{thm}
\begin{proof}
	Если $\phi \equiv 0$, то очевидно $h_\phi=0 \in H$ 
	подойдёт.

	Пусть $\phi \not\equiv \Rightarrow \operatorname{Ker}
	\phi \neq H$ и $\operatorname{Ker} \phi$ --- замкнутое
	подпространство в $H$ в силу непрерывности $\phi$.
	Тогда по теореме Рисса об ортогональности дополнения
	замкнутого подпространств в гильбертово пр-ве,
	имеем равенство
	\[
		\operatorname{Ker}\phi \oplus (\operatorname{Ker}\phi)^\perp = H
	.\]
	Т.\:к. $\operatorname{Ker}\phi \neq H$, то 
	$\left( \operatorname{Ker}\phi \right) ^\perp \neq
	\{0\} $, т.\:е. $\exists g \in \left( \operatorname{Ker}\phi \right)^\perp \setminus \{0\}  $.
	Далее, $\forall f \in  H$ рассмотрим вектор
	\[
		f - \frac{\phi(f)}{\phi(g)}g \in H
	.\] 
	Имеем 
	\[
		\phi\left(f - \frac{\phi(f)}{\phi(g)}g\right)=
		\phi(f)- \frac{\phi(f) \phi(g)}{\phi(g)}=
		\phi(f) - \phi(f)=0
	,\]
	следовательно
	\[
		f - \frac{\phi(f)}{\phi(g)}g \in \operatorname{Ker}
		\phi \Rightarrow f - \frac{\phi(f)}{\phi(g)}\perp g
		\Rightarrow (f,\,g) - \left( 
		\frac{\phi(f)}{\phi(g)}g,\,g\right) =0,
	\]
	значит
	\[
		(f,\,g)= \frac{\phi(f)}{\phi(g)}\| g\|^2,
	\]
	т.\:е.
	\[
		\phi(f) = \left( f,\, \frac{\overline{\phi(g)}}{
		\|g \|^2}g \right) \quad \forall f \in H
	.\] 
	Итак,
	\[
		h_\phi= \frac{\overline{\phi(g)}}{\| g\|^2}g,
		\quad g \in \left( \operatorname{Ker}\phi \right) ^\perp \setminus \{0\} 
	,\]
	\[
		\phi(f) = (f,\, h_\phi) \quad f \in H
	.\]

	\emph{Единственность вектора $h_\phi$}
	
	Если $ \tilde{h}$ и $ \hat{h}\in H$ таковы, что
	\[
		\phi(t) =\left(f,\,\tilde{h}\right)=
		\left( f,\, \hat{h} \right) \quad \forall
		f \in H
	,\]
	то        получаем $    \left(f,\,\tilde{h}-\hat{h}\right)=0
	\;          \forall f \in H $. В       частности,
	для $f = \tilde{h}- \hat{h} \Rightarrow 
	 \left( \tilde{h}-\hat{h},\,\tilde{h}-
	 \hat{h}\right)=0 \Leftrightarrow \left\lVert 
  \tilde{h}- \hat{h}\right\rVert^2=0$, т.\:е.  $\tilde{h}=\hat{h}$.
\end{proof}
Возвращаемся к определению сопряжённого оператора.
Итак, в нашем распоряжении линейный непрерывный функционал
\[
	\psi: \overline{D(A)}\to \mathbb{C},
	\quad \psi\mid _{D(A)}= (Af,\,g) \quad \forall
	f \in D(A)
.\] Т.\:к. $\overline{D(A)}$ --- замкнутое под-во в $H$, то
само  $\overline{D(A)}$ явл. гильбертовым пр-вом. $\Rightarrow$ 
по Т. Рисса-Фреше $\exists! h_\psi\in \overline{D(A)}$ такой. что
\[
	\psi(f) = (f,\, h_\psi) \quad \forall f \in \overline{D(A)}
.\] 
В частнсти, для любогоо $f \in D(A)$ имеем равенство
\[
	(Af,\,g)= (f,\,h_\psi), \quad f \in D(A)
.\] 
По определению полагаем, что
\[
	A^* g = h_\psi,\quad g \in D(A^*)
.\] 
\begin{dfn}
Таким образом определён \emph{сопряжённый оператор} $A^*: D(A^*)\to H$,
при этом $\operatorname{Im}A^*\subset \overline{D(A)}$.
\end{dfn}
Заметим, что $A^* : D(A^*) \to H$
является линейным оператором. Действительно,
$\forall g_{1,\,2} \in D(A^*) \; \forall \alpha_{1,\,2} \in 
 \mathbb{C}$ имеем
 \[
\left\{
\begin{aligned}
	(Af,\, g_1) = (f,\, A^* g_1)\\
	(Af, g_2) =(f,\,A^* g_2)
\end{aligned}
\right.
\quad \forall f \in D(A)
 .\] 
 Т.\:к. $\alpha_1 g_1+\alpha_2 g_2 \in D(A^*)$ --- под-во в $H$,
 то  $\exists! A^* (\alpha_1 g_1 +\alpha_2 g_2) \in \overline{D(A)}$ 
такой, что
\[
	(Af,\, \alpha_1 g_1+\alpha_2 g_2)=(f,\, A^*(\alpha_1 g_1+
	\alpha_2 g_2)) \quad \forall f \in D(A)
.\] 
Но мы также имеем
\[
	\overline{\alpha_1} (Af,\, g_1)+ \overline{\alpha_2}(Af,\, g_2)=
	(A f,\,\alpha_1 g_1+\alpha_2 g_2)
 ,\]
а также
\[
	\overline{\alpha_1}(Af,\, g_1) +\overline{\alpha_2} (Af,\,g_2)= \overline{\alpha_1} (f,\,
	A^* g_1)+ \overline{\alpha_2} (f,\, A^* g_2)=
	(f,\,\alpha_1 A^* g_1+ \alpha_2 A^* g_2)
.\]
Следовательно, получаем:
\[
	\left(f,\, \underbrace{\alpha_1 A^* g_1 +\alpha_2 A^* g_2}_{\in 
	\overline{D(A)}}\right)=
	(Af,\, \alpha_1 g_1 +\alpha_2 g_2)=
	\left( f,\, \underbrace{A^*(\alpha_1 g_1 +\alpha_2 g_2}_{\in 
	\overline{D(A)}} \right) 
,\]
значит в силу единственности вектора из $\overline{D(A)}$, реализующего
это равенство $\forall f \in D(A)$
\[
	A^*(\alpha_1 g_1+\alpha_2 g_2) = \alpha_1 A^* g_1+
	\alpha_2 A^* g_2 !
\] 
\begin{dfn}
	Линейный оператор $A: D(A) \to  H$ называется
	\emph{самосопряжённым}, если
	\[
		D(A^*)=D(A) \text{ и } A^*=A
	\]
	на их общей области определения
	$D(A)=D(A^*)$.
\end{dfn}
Т.\:о. для исследования св-ва ССО требуется прежде всего иметь
равенство \underline{$D(A)=D(A^*)$}, и $\forall f \in  D(A)$ иметь
$Af= A^* f$.
 \begin{dfn}
	 Линейный оператор  $A : D(A)\to  H$ называется \emph{симметричным},
	 если 
	 \[
		 \underbrace{(Af,\, g) =(f,\,Ag) \quad f,\,g \in D(A)}_{\Downarrow}
	 \]
\[
	\forall g \in D(A) \Rightarrow g \in D(A^*),
\]
при этом $A^* g = P_{\overline{D(A)}}A g$, где
$P_{\overline{D(A)}}: H \to \overline{D(A)}$ --- ортопроектор
на $\overline{D(A)}$.
\end{dfn}
Таким образом, $A : D(A) \to  H$ симметричен, если и только если
\[
\left\{
\begin{aligned}
	D(A) \subset D(A^*)\\,
	A^*\mid _{D(A)}= P_{\overline{D(A)}}A.
\end{aligned}
\right.
\] 
Если $\overline{D(A)}=H$, т.\:е. $A $ --- плотно определённый
оператор, тогда $P_{\overline{D(A)}}=I$ --- тождественный
оператор в $H$, и в случае $\overline{D(A)}=H$ свойство
симметрии оператора $A : D(A) \to H$ равносильно условиям
\[
\left\{
\begin{aligned}
	D(A) \subset D(A^*),\\
	A^*\mid _{D(A)}=A,
\end{aligned}
\right.
\] 
эти условия часто пишут коротко  \underline{$A \subset A^*$}.
\begin{center}
	\emph{Пример симметричного несамосопряжённого оператора}
\end{center}
Пусть $H= \mathbb{L}_2 [0,\,1]$,
\[
	A= i \frac{d}{dx} : D(A) \to \mathbb{L}_2[0,\,1],
\]
где
\[
	D(A)= \left\{ f \in W^{1,\,2}[0,\,2]\mid f(0)=f(1)=0 \right\} 
.\]
Тогда $\forall f,\,g \in D(A)$ имеем
\begin{multline*}
	(Af,\,g) = \int\limits_{0}^{1} i f'(x) \overline{g(x)}dx=
	=\underbrace{i f(x) \overline{g(x)}\mid _0^1}_{=0}-\int\limits_{0}^{1} i f(x) \overline{g'(x)}dx  =\\=
	\int\limits_{0}^{1} f(x) \overline{i g'(x)}dx=
	(f,\,Ag),
\end{multline*} 
т.\:е. $A$ --- симметричный оператор. Но очевидно, что
$W^{1,\,2}[0,\,1]\subset D(A^*)$, так как $\forall g \in 
 W^{1,\,2}[0,\,1]$ имеем
\begin{multline*}
	\forall f \in D(A) \hookrightarrow (Af,\,g)=
	\int\limits_{0}^{1} i f'(x) \overline{g(x)}dx=\\=
	\underbrace{i f(x) \overline{g(x)}\mid _{0}^1}_{\substack{=0\\ \text{т.\:к. }f(0)=f(1)=0}}+
	\int\limits_{0}^{1} f(x) \overline{i g'(x)}dx=
	(f,\,i g')
 ,\end{multline*} 
 значит $\forall f \in  D(A)$ получаем $(Af,\,g)=(f,\,i g')$ ---
 непрерывно по  $f \in D(A)$, для любой $g \in W^{1,\,2}[0,\,2]$ 
\[
	W^{1,\,2}[0,\,1] \subset D(A^*) \text{ и }
	\forall g \in W^{1,\,2}[0,\,1]
\]
выполнено
\[
A^* g = i g'= i \frac{d}{dx} g
.\] 
Полагаем, что на самом деле имеет место и обратное вложение
\[
	D(A^*) \subset W^{1,\,2}[0,\,1]
.\] 
Берём $\forall g \in D(A^*)$, тогда
для $h =A^*g$ имеем
\begin{align*}
	(Af&,\,g)=(f,\,h) \quad \forall f \in D(A)\\
	   &\veq\\
	\int\limits_{0}^{1} i & f' \overline{g} dx=
	\int\limits_{0}^{1} f \overline{h}dx 
.\end{align*}
Определим функцию
\[
	\psi(x) = \int\limits_{0}^{x} h(t) dt, \quad x \in [0,\,1] 
.\] 
Тогда $\psi \in W^{1,\,2}[0,\,1]$ и $\psi(0)=0$, при этом
$\psi'=h$ для п.\:в. $x \in [0,\,1]$, следовательно
\[
\int\limits_{0}^{1} f \overline{h}dx= \int\limits_{0}^{1} f
\overline{\psi'}dx= \underbrace{f(x) \overline{\psi(x)}\mid _{0}^1}_{
=0 \text{ т.\:к. } f(0)=f(1)=0}- \int\limits_{0}^{1} 
f'(x) \overline{\psi(x)}dx
.\] 
Следовательно $\forall f \in D(A)$ имеем
\[
\int\limits_{0}^{1} i f' \overline{g} = - \int\limits_{0}^{1}
f'\overline{\psi},
\]
т.\:е.
\[
	\int\limits_{0}^{1} f'\left(i \overline{g} +\overline{\psi}\right)dx = 0 
.\] 
Рассмотрим подпространство
\[
	\left\{ f' \mid f \in D(A) \right\} =M
.\]
\[
	\forall f \in D(A) \Rightarrow \int\limits_{0}^{1} f'dx=
	\underbrace{f(1)}_{=0}-\underbrace{f(0)}_{=0}=0,
\]
т.\:е. имеем вложение
\[
	M=\left\{ f'\mid f \in D(A) \right\} \subset \left( 
	\operatorname{Lin}1\right) ^\perp
.\] 
Наоборот, если $\phi\in \left( \operatorname{Lin}1 \right) ^\perp$,
т.\:е. 
\[
	\int\limits_{0}^{1}  \phi(x) dx =0, 
\]
то определив 
\[
	f(x) =\int\limits_{0}^{x} \phi(x) dt, \quad x \in [0,\,1]
	\Rightarrow f \in W^{1,\,2}[0,\,1], \quad f (0)=f(1)=0,
\]
т.\:е. $f \in D(A)$ и $f'=\phi\Rightarrow \phi \in M$, т.\:е.
получаем равенство
\begin{multline*}
	M= \left\{ f '\mid  f \in  D(A) \right\} =(\operatorname{Lin}1)^\perp \implies \int\limits_{0}^{1} f' \overline{(\psi-i g)}dx=0
	\quad f \in D(A)
	\Leftrightarrow \\
	\Leftrightarrow(f',\, \psi-i g)=0 \forall f \in D(A)
	\Leftrightarrow \psi-ig \in \left( \left( \operatorname{Lin}
	1\right) ^\perp \right) ^\perp= \operatorname{Lin}1,
\end{multline*} 
 то есть выполнено
\[
	\psi - ig\equiv \operatorname{const} \text{ п.\:в. на }
	[0,\,1]
.\] 
Но
\[
	\psi(x) = \int\limits_{0}^{x} h(t) dt \text{ для }
	h=A^* g
,\] 
т.\:е.
\begin{multline*}
	\int\limits_{0}^{x}  h dt= ig(x) + \operatorname{const}
	\implies g \in W^{1,\,2}[0,\,1]\implies\\ \implies
	h(x) = i \frac{d}{dx} g(x)= (A^* g)(x) \text{ для п.\:в. }
	x \in [0,\,1]
.\end{multline*} 
Итак, мы доказали
\[
	D(A^*)=W^{1,\,2}[0,\,1]
\]
и
\[
	A^* g = i \frac{d}{dx} g \quad \forall g \in W^{1,\,2}[0,\,1]
.\]
Видим, что $A\neq A^*$, хотя $A$ --- симметричный.
При этом $A^*$ уже не явл. симметричным.
\section*{Семинар}
A -- сим. +$\{e_n\} $ --- ортог. базис из с.в. $A$ т.\:е.  $A e_n =\lambda_n e_n$ 
\begin{enumerate}
	\item $D(A) \in  \left\{  f \in  M : \sum_{n=1}^{\infty} \lambda^2_n \frac{
		\left| \left( f,\, e_n \right)  \right| ^2}{\|e_n\|^2}< \infty \right\} =B $ 
		\[
			Af=\sum_{n=1}^{\infty} \lambda_n \frac{\left( f,\,e_n \right) }{(e_n,\,e_n)} e_n \text{ --- спектр
			разложение }A
		.\] 
	\item было $D(A^*) \subset B$ 
	\item $D(A^*) \supset B$ 
		\[
			\forall f \in B \implies f \subset D(A^*)
		.\] 
		\[
			\forall y \in  D(A) \qquad \left| (Ag,\,f) \right| \le C_f \|g\| \text{?}
		.\] 
		\begin{multline*}
			\left| \left( Ag,\, f \right)  \right| =\text{спектр. разл. }A= \left| \left( 
			\sum_{n=1}^{\infty} \lambda_n \frac{\left( g,\, e_n \right) }{\left( e_n,\,e_n \right) }e_n,f\right)  \right|= \\=
			\text{непрерывность скалярного произведения}=\\= \left| \sum_{n=1}^{\infty} \lambda_n
			\frac{(g,\, e_n)}{(e_n,\,e_n)}\left( e_n,\, f \right) \right| =
			\left| \sum_{n=1}^{\infty} \left(g,\, \lambda_n \frac{\left( f,\, e_n \right) }{(e_n,\,e_n)}e_{n}\right) \right|\\=\\
			\text{непрерывность скалярного произведения}=\\= \left| \left( g,\, \sum_{n=1}^{\infty} \lambda_n
			\frac{(f,\,e_n)}{(e_n,\,e_n)}e_n\right)  \right| \le \text{неравенство К-Б}\le\\ \le  
			\|g\|\cdot \|\sum_{n=1}^{\infty} \lambda_n \frac{(f,\,e_n)}{(e_n,\,e_n)}e_n\|=
			\|g\|\cdot \underbrace{\sqrt{\sum_{n=1}^{\infty} \lambda_n^2 \frac{\left| \left( f,\,e_n \right)  \right| ^2}{
			\|e_n\|^2}}}_{C_f}< \infty \implies D(A^*)=B
		.\end{multline*} 
		\[
			A^*f=\sum_{n=1}^{\infty} \lambda_n \frac{\left( f,\,e_n \right) }{(e_n,\,e_n)} \text{ ---
			спектральное разложение }A^*
		.\]
	\item $A^{**}=A^*$ и $D\left(A^{* *}\right)$
	\item  $\overline{A}$ --- замыкание
		\[
			D(A^*) \subset D\left(\overline{A}\right)\text{
			т.\:е. }\forall f \in D(A^*)=B \to ? \to 
			f \in  D\left(\overline{A}\right)
		.\] 
		\begin{multline*}
			f=\text{р. Ф.}= \sum_{n=1}^{\infty} \frac{(f,\,e_n)}{
			\left( e_n,\,e_n \right) }e_n
			\qquad \text{и}\qquad
			\sum_{n=1}^{\infty} \lambda_n
			\frac{(f,\,e_n)}{(e_n,\,e_n)}=A^* f\\
			\text{т.\:к.} f  \in  D\left(A^*\right)
		.\end{multline*} 
		\[
			f_N=\sum_{n=1}^{N} \frac{(f,\,e_n)}{(e_n,\,
			e_n)}e_n \in D(A)
		.\] 
		\[
			Af_N = \sum_{n=1}^{N} \lambda_N \frac{(f,\,
			e_n)}{(e_n,\,e_n)}e_n
		.\] 
		\[
			\operatorname{Gr} \ni
			\begin{pmatrix} f_N\\ Af_n \end{pmatrix} 
			\to (N\to \infty) \to 
			\begin{pmatrix} f\\
			\sum_{n=1}^{\infty} \lambda_n
		\frac{(f,\,e_n)}{(e_n,\,e_n)}e_n\end{pmatrix} =
		\begin{pmatrix} f\\ A^* f \end{pmatrix} \ni
		\operatorname{Gr} \overline{A}\implies
		f \in D\left( \overline{A} \right) ,\, A^*f=\overline{A}f
		.\] 
	\item $D(A^*)\supset D\left( \overline{A} \right) $
		\[
			\forall f \in  D\left( \overline{A} \right)
			\text{, т.\:е.} \begin{pmatrix} 
			f_N\\ Af_N\end{pmatrix} \to (N\to \infty)
			\to \begin{pmatrix} f \\
			\overline{A} f\end{pmatrix} (f_N \in D(A))
		.\] 
		\[
			\left( Af_n,\, e_k \right) =
			\text{ сим., }f  \in D(A),\,e_k \in D(A)=
			\left( f_N,\, Ae_k \right) = \text{с.в.}=
			(f_N,\, \lambda_k e_k)=\lambda_k
			\left( f_N,\,e_k \right) \to 
			(N\to \infty, \text{ непр. ск. произв.}\to 
			\lambda_k (f,\, e_k)
		.\] 
		\[
			\left( Af_N,\, e_k \right) \to 
			(N\to \infty, \text{ непр. ск. произв.})\to 
			\left( \overline{A}f,\,e_k \right) \implies
			(\overline{A}f,\,e_k)=\lambda_k
			(f,\,e_k)
		.\] 
		\[
		\overline{A}f= \text{р. Ф.}=
		\sum_{n=1}^{\infty} \frac{\left( \overline{A}f,e_k \right) }{(e_k,\,e_k)}e_k= \sum_{n=1}^{\infty} \lambda \frac{\left( 
		f,\, e_k\right) }{\left( e_k,\, e_k\right) }e_k
		\implies \|\overline{A}f\|^2 = \text{Парс.}=
		\sum_{n=1}^{\infty} \lambda_k^2 \frac{
		\left| \left( f,\,e_k \right)  \right| ^2}{\|e_k\|^2}
		< \infty
		.\] 
\end{enumerate}
$\R^1$
\begin{dfn}
	\emph{Обобщенная производная}
	\[
	f \in L_2\left[ a,\,b \right] \text{ и }
	h \in L_2 \left[ a,\,b \right] \text{ называется
	\emph{обобщённой производной} } f \text{ если}
	\] 
	\[
		\forall \alpha \in C^1 [a,\,b] \text{ и }
		\alpha(a)=\alpha(b)=0 \text{ выполнено}
	\] 
	\[
	\int\limits_{a}^{b} f \alpha' dx= -
	\int\limits_{a}^{b} h \alpha dx 
	.\] 
	Если $f \in  C^1[a,\,b]$
	\[
		\int\limits_{a}^{b} f \alpha dx= \underbrace{f \alpha  \left. \right|_a^b}_{=0}-
	\int\limits_{a}^{b} f' \alpha dx  \implies f'=h
	.\] 

\end{dfn}
\begin{dfn}
	$f \in  L_2[a,\,b]$ и $h \in  L_2 [a,\,b]$, тогда $h$ ---
	2-я об. производная, если
	 \[
	 \forall \alpha \in C^2[a,\,b] \text{ и }
	 \alpha(a)=\alpha(b)=0
	 \] 
	\[
		\int\limits_{a}^{b} f \alpha'' dx=(-1)^2
		\int\limits_{a}^{b} h \cdot \alpha dx  
	.\] 
\end{dfn}
\emph{Св-во}: Если у $f$  $\exists$ об. произв.  $F$  и  $\exists$
об. произв. у  $F$  $G$, тогда  $G$ --- вторая об. произв. для  $f$.

 \begin{dfn}
	 \emph{Пространство Соболева} $W^{k,\,2}[a,\,b]$
	 \[
		 f \in  W^{k,\,2}[a,\,b], \text{ если } f \in 
		 L_2[a,\,b] \text{ и } \exists \text{ об. произв.
		 } f
	 \]
	 \[
		 f^{(5)} \in  L_2[a,\,b] \text{ для } s=1,\,2, \ldots,\, k
	 .\] 
\end{dfn}
\emph{Св-ва}
\begin{enumerate}
	\item Теорема вложения Соболева

		Если $k> \frac{1}{2}+l$, то $W^{k,\,2}[a,\,b]\subset
		C^l [a,\,b]$ 

	\[
		k=1 \qquad W^{1,\,2}[a,\,b] \subset C[a,\,b]
	.\] 
	\[
		k=2 \qquad W^{2,\,2}[a,\,b]\subset C^1[a,\,b]
	.\] 
	\item Ф-ла интегр. по частям
		\[
			f \in  W^{1,\,2}[a,\,b] \text{ и }
			g \in W^{1,\,2}[a,\,b], \text{ то}
			\int\limits_{a}^{b}  =
			f' g dx=f \cdot g\left. \right|_a^b- \int\limits_{a}^{b}g' f dx  
		\] 
		$g'$ и $f'$ - об. произв. 
\end{enumerate}
\end{document}
