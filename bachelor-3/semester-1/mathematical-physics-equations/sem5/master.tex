\documentclass[a4paper]{article}
% Этот шаблон документа разработан в 2014 году
% Данилом Фёдоровых (danil@fedorovykh.ru) 
% для использования в курсе 
% <<Документы и презентации в \LaTeX>>, записанном НИУ ВШЭ
% для Coursera.org: http://coursera.org/course/latex .
% Исходная версия шаблона --- 
% https://www.writelatex.com/coursera/latex/5.3

% В этом документе преамбула

\usepackage{siunitx}
%%% Работа с русским языком
%\usepackage{cmap}					% поиск в PDF
%\usepackage{mathtext} 				% русские буквы в формулах
%\usepackage[T2A]{fontenc}			% кодировка
%\usepackage[utf8]{inputenc}			% кодировка исходного текста
%\usepackage[english,russian]{babel}	% локализация и переносы
%\usepackage{indentfirst}
%\frenchspacing
%
%\renewcommand{\epsilon}{\ensuremath{\varepsilon}}
%\newcommand{\phibackup}{\ensuremath{\phi}}
%\renewcommand{\phi}{\ensuremath{\varphi}}
%\renewcommand{\varphi}{\ensuremath{\phibackup}}
%\renewcommand{\kappa}{\ensuremath{\varkappa}}
%\renewcommand{\le}{\ensuremath{\leqslant}}
%\renewcommand{\leq}{\ensuremath{\leqslant}}
%\renewcommand{\ge}{\ensuremath{\geqslant}}
%\renewcommand{\geq}{\ensuremath{\geqslant}}
%\renewcommand{\emptyset}{\varnothing}
%\renewcommand{\Im}{\operatorname{Im}}
%\renewcommand{\Re}{\operatorname{Re}}


%%% Дополнительная работа с математикой
\usepackage{amsmath,amsfonts,amssymb,amsthm,mathtools} % AMS
%\usepackage{icomma} % "Умная" запятая: $0,2$ --- число, $0, 2$ --- перечисление

%% Номера формул
%\mathtoolsset{showonlyrefs=true} % Показывать номера только у тех формул, на которые есть \eqref{} в тексте.
%\usepackage{leqno} % Нумереация формул слева

%% Свои команды
\DeclareMathOperator{\sgn}{\mathop{sgn}}
\DeclareMathOperator{\sign}{\mathop{sign}}
\DeclareMathOperator*{\res}{\mathop{res}}
\DeclareMathOperator*{\tr}{\mathop{tr}}
\DeclareMathOperator*{\rot}{\mathop{rot}}
\DeclareMathOperator*{\divop}{\mathop{div}}
\DeclareMathOperator*{\grad}{\mathop{grad}}

%% Перенос знаков в формулах (по Львовскому)
\newcommand*{\hm}[1]{#1\nobreak\discretionary{}
{\hbox{$\mathsurround=0pt #1$}}{}}

%%% Работа с картинками
\usepackage{graphicx}  % Для вставки рисунков
\graphicspath{{figures/}}  % папки с картинками
\setlength\fboxsep{3pt} % Отступ рамки \fbox{} от рисунка
\setlength\fboxrule{1pt} % Толщина линий рамки \fbox{}
\usepackage{wrapfig} % Обтекание рисунков текстом

%%% Работа с таблицами
\usepackage{array,tabularx,tabulary,booktabs} % Дополнительная работа с таблицами
\usepackage{longtable}  % Длинные таблицы
\usepackage{multirow} % Слияние строк в таблице

%%% Теоремы
\theoremstyle{plain} % Это стиль по умолчанию, его можно не переопределять.
\newtheorem{thm}{Теорема}
\newtheorem*{thm*}{Теорема}
\newtheorem{prop}{Предложение}
\newtheorem*{prop*}{Предложение}
 
\theoremstyle{definition} % "Определение"
%\newtheorem{corollary}{Следствие}[theorem]
\newtheorem{dfn}{Определение}
\newtheorem*{dfn*}{Определение}
\newtheorem{prob}{Задача}
\newtheorem*{prob*}{Задача}

 
\theoremstyle{remark} % "Примечание"
\newtheorem*{sol}{Решение}
\newtheorem*{rem}{Замечание}

%%% Программирование
\usepackage{etoolbox} % логические операторы

%%% Страница
%\usepackage{extsizes} % Возможность сделать 14-й шрифт
%\usepackage{geometry} % Простой способ задавать поля
%	\geometry{top=25mm}
%	\geometry{bottom=35mm}
%	\geometry{left=35mm}
%	\geometry{right=20mm}
 
\usepackage{fancyhdr} % Колонтитулы
%	\pagestyle{fancy}
 %	\renewcommand{\headrulewidth}{0pt}  % Толщина линейки, отчеркивающей верхний колонтитул
	%\lfoot{Нижний левый}
	%\rfoot{Нижний правый}
	%\rhead{Верхний правый}
	%\chead{Верхний в центре}
	%\lhead{Верхний левый}
	%\cfoot{Нижний в центре} % По умолчанию здесь номер страницы

\usepackage{setspace} % Интерлиньяж
%\onehalfspacing % Интерлиньяж 1.5
%\doublespacing % Интерлиньяж 2
%\singlespacing % Интерлиньяж 1

\usepackage{lastpage} % Узнать, сколько всего страниц в документе.

\usepackage{soul} % Модификаторы начертания

\usepackage{hyperref}
\usepackage[usenames,dvipsnames,svgnames,table,rgb]{xcolor}
\hypersetup{				% Гиперссылки
    unicode=true,           % русские буквы в раздела PDF
    pdftitle={Заголовок},   % Заголовок
    pdfauthor={Автор},      % Автор
    pdfsubject={Тема},      % Тема
    pdfcreator={Создатель}, % Создатель
    pdfproducer={Производитель}, % Производитель
    pdfkeywords={keyword1} {key2} {key3}, % Ключевые слова
%    colorlinks=true,       	% false: ссылки в рамках; true: цветные ссылки
    %linkcolor=red,          % внутренние ссылки
    %citecolor=black,        % на библиографию
    %filecolor=magenta,      % на файлы
    %urlcolor=cyan           % на URL
}

\usepackage{csquotes} % Еще инструменты для ссылок

%\usepackage[style=apa,maxcitenames=2,backend=biber,sorting=nty]{biblatex}

\usepackage{multicol} % Несколько колонок

\usepackage{tikz} % Работа с графикой
\usepackage{pgfplots}
\usepackage{pgfplotstable}
%\usepackage{coloremoji}
\usepackage{floatrow}
\usepackage{subcaption}
\graphicspath{{figures/}}

\renewcommand\thesubfigure{\asbuk{subfigure}}
%\addbibresource{master.bib}

\usepackage{import}
\usepackage{pdfpages}
\usepackage{transparent}
\usepackage{xcolor}
\usepackage{xifthen}

\newcommand{\incfig}[2][1]{%
    \def\svgwidth{#1\columnwidth}
    \import{./figures/}{#2.pdf_tex}
}
%\usepackage{titlesec}
%\titleformat{\section}{\normalfont\Large\bfseries}{}{0pt}{}
%----------------------STANDART:
%\titleformat{\chapter}[display]
%  {\normalfont\huge\bfseries}{\chaptertitlename\ \thechapter}{20pt}{\Huge}
%\titleformat{\section}{\normalfont\Large\bfseries}{\thesection}{1em}{}
%\titleformat{\subsection}
%  {\normalfont\large\bfseries}{\thesubsection}{1em}{}
%\titleformat{\subsubsection}
%  {\normalfont\normalsize\bfseries}{\thesubsubsection}{1em}{}
%\titleformat{\paragraph}[runin]
%  {\normalfont\normalsize\bfseries}{\theparagraph}{1em}{}
%\titleformat{\subparagraph}[runin]
%  {\normalfont\normalsize\bfseries}{\thesubparagraph}{1em}{}

\pdfsuppresswarningpagegroup=1
\pgfplotsset{compat=1.16}



%\setcounter{tocdepth}{1} % only parts,chapters,sections
%\titleformat{\subsection}{\normalfont\large\bfseries}{}{0em}{}
%\titleformat{\subsubsection}{\normalfont\normalsize\bfseries}{}{0em}{}

%\newcommand{\textover}[2]{\stackrel{\mathclap{\normalfont\mbox{#2}}}{#1}}

\author{Yaroslav Drachov\\
Moscow Institute of Physics and Technology}
%\author{Драчов Ярослав\\
%Факультет общей и прикладной физики МФТИ}
\newcommand{\veq}{\mathrel{\rotatebox{90}{$=$}}}
%\newcommand{\teto}[1]{\stackrel{\mathclap{\normalfont\tiny\mbox{#1}}}{\to}}
%\renewcommand{\thesubsection}{\arabic{subsection}}

%%\setcounter{secnumdepth}{0}

\definecolor{tabblue}{RGB}{30, 119, 180}
\definecolor{taborange}{RGB}{255, 127, 15}
\definecolor{tabgreen}{RGB}{45, 160, 43}
\definecolor{tabred}{RGB}{214, 38, 40}
\definecolor{tabpurple}{RGB}{148, 103, 189}
\definecolor{tabbrown}{RGB}{140, 86, 76}
\definecolor{tabpink}{RGB}{227, 119, 193}
\definecolor{tabgray}{RGB}{127, 127, 127}
\definecolor{tabolive}{RGB}{188, 189, 33}
\definecolor{tabcyan}{RGB}{22, 190, 207}
\pgfplotscreateplotcyclelist{colorbrewer-tab}{
{tabblue},
{taborange},
{tabgreen},
{tabred},
{tabpurple},
{tabbrown},
{tabpink},
{tabgray},
{tabolive},
{tabcyan},
}
\usepackage{csvsimple}
\usepackage{extarrows}
%\renewcommand{\labelenumii}{\asbuk{enumii})}
%\renewcommand{\labelenumiv}{\Asbuk{enumiv}}
%\newcommand{\prob}[1]{\subsubsection*{#1}}
\sisetup{output-decimal-marker = {,},separate-uncertainty = true,exponent-product = \cdot}

\usepackage{braket}
\usepackage{enumerate}
\usepackage{chngcntr}
%\counterwithin*{equation}{problem}
%\usepackage{bbold}

\newtheoremstyle{hiProb}% ⟨name ⟩ 
{3pt}% ⟨Space above ⟩1 
{3pt}% ⟨Space below ⟩1
{}% ⟨Body font ⟩
{}% ⟨Indent amount ⟩2
{\bfseries}% ⟨Theorem head font⟩
{.}% ⟨Punctuation after theorem head ⟩
{.5em}% ⟨Space after theorem head ⟩3
%{\thmname{#1} \thmnote{#3}}% ⟨Theorem head spec (can be left empty, meaning ‘normal’)⟩
{\thmnote{#3}}% ⟨Theorem head spec (can be left empty, meaning ‘normal’)⟩
\theoremstyle{hiProb} % "Определение"
%\newtheorem{hiProb}{Задача}
\newtheorem{hiProb}{}
%\usepackage{mmacells}
\newcommand{\textover}[2]{\stackrel{\mathclap{\normalfont\scriptsize\mbox{#2}}}{#1}}
\usepackage{units}
\usepackage[math]{cellspace}%
\setlength\cellspacetoplimit{2pt}
\setlength\cellspacebottomlimit{2pt}

\DeclareMathAlphabet{\mathbbold}{U}{bbold}{m}{n}

\newcommand{\normord}[1]{:\mathrel{#1}:}

\title{Что то про сф. ф-ии?}
\begin{document}
	\maketitle
    $\mathbb{R}^3$

    $\Delta u = 0 $ в $D$ 

    $D$: шар, внешность шар или сф. слой
    либо $u \mid _{\partial D}= u_0$, либо $u_r \mid _{\partial D}=u_1$, либо $(\alpha u +\beta u_r)\mid _{\partial D}=u_2$
   \[
   \Delta u = u_{rr}+ \frac{2}{r} u_r+ \frac{1}{r^2}\underbrace{\left[ 
   \frac{1}{\sin \theta} \frac{\partial }{\partial \theta} \left( 
   \sin \theta \frac{\partial u}{\partial \theta} \right) +
   \frac{1}{\sin^2 \theta} \frac{\partial ^2u}{\partial \phi^2} \right]}_{\Delta _{\text{угл}} \text{ --- оператор Лапласа-Бельтрами}} 
   .\] 
   Собственные функции $\Delta_\text{угл}$ называются сферическими
   функциями.

Полиномы Лежандра:
\[
	P_{n}(t) = \frac{1}{2^n n!} \frac{d^n}{dt^n} (t^2 -1)^n \text{ --- полная ортоогональная система (базис) в } L_2[-1,\,1]
.\] 
Присоединённые полиномы Лежандра
\[
	P^m_n (t)= \left( \sqrt{1-t^2}  \right) ^m \frac{d^m}{dt^m}
	P_n(t)
\text{ --- образуют базис в } L_2[-1,\,1] \text{ при каждом }m\] 
\[
m=0,\,1,\,2,\ldots,\,n
.\] 
Сферические функции
\[
	Y_{n,\,m}(\theta,\,\phi)=
	\begin{cases}
		P^m_n (\cos \theta) \cos m\phi\\
		P_n^m(\cos\theta)\sin m \phi
	\end{cases} \subset 
\lambda_n =-n (n+1)\] 
\[
	Y_{n}(\theta,\,\phi)= \sum_{m=0}^{n} \left[ 
	a_{n,\,m}P^{m}_n(\cos \theta) \cos m \phi+
b_{n,\,m}P^m_n(\cos \theta) \sin m \phi\right] 
.\] 
\[
	\Delta_\text{угл} Y_n(\theta,\,\phi)=- n(n+1) Y_n(\theta,\,\phi)
.\] 
\[
	u(z,\,\theta,\,\phi)= \sum_{n=0}^{\infty} A_n (z) Y_n(\theta,\,\phi)
.\] 
\[
\sum_{n=0}^{\infty} A''_n Y_n+ \frac{2}{r} \sum_{n=0}^{\infty} A'_n
Y_n + \frac{1}{r^2}\sum_{n=0}^{\infty} (-n)(n+1)  A_n Y_n=0
.\] 
\[
	A''_n+\frac{2}{r} A'_n - \frac{n(n+1)}{r^2}A_n=0
.\] 
\[
	r^2 A''_n +2 r A'_n - n(n+1)A_n=0
.\] 
\[
	\lambda (\lambda-1) +2 \lambda -n(n+1)=0
.\] 
\[
	\lambda^2 +\lambda -n(n+1)=0
.\]
\[
	\lambda_{1,\,2}= \frac{-1\pm\sqrt{1+4n(n+1)} }{2}=
	\frac{-1\pm \sqrt{1+4 n^2+4n} }{2}=\frac{-1 \pm(2n+1)}{2}=
	n,\, -n-1
.\] 
\[
A_n=C_n r^n +d_n \frac{1}{r^{n+1}}
.\] 
\begin{enumerate}
	\item В сф. слое
		\begin{multline*}
		u= \sum_{n=0}^{\infty} \left[ 
		r^n Y_n (\theta,\,\phi)+\frac{1}{r^{n+1}}\tilde{Y}_n
	(\theta,\,\phi)\right] =\\=
	\sum_{n=0}^{\infty} \sum_{m=0}^{n} \left[ 
	a_{n,\,m}r^n P^m_n (\cos \theta)\cos n \phi+b_{m,\,n}
\frac{1}{r^{n+1}}P_n^m (\cos\theta) \cos m \phi 
\right.+\\+ \left.c_{m,\,n}r^n P^m_n (\cos\theta) \sin m\phi +
d_{m,\,n} \frac{1}{m+1} P^m_n (\cos \theta) \sin m \phi\right] 
		.\end{multline*} 
		\item  В шаре
\[
	u = \sum_{n=0}^{\infty} \sum_{m=0}^{n} \left[a_{n,\,m} r^n P^m_n
	(\cos\theta) \cos m \phi + C_{n,\,m}r^n P^m_n (\cos\theta)
\sin m \phi\right]
.\] 
\item Вне шара (+ огр. на $+\infty$)
 \[
u= a + \sum_{n=0}^{\infty} \sum_{m=0}^{n} \left[ b_{n,\,m}
\frac{1}{r^{n+1}}P^m_n (\cos\theta)\cos m \phi + d_{n,\,m}
\frac{1}{r^{n+1}}P^m_n (\cos \theta) \sin m \phi\right] 
.\] 
\end{enumerate}
Задача $\Delta u = 0$, $|x|<1$,  \begin{multline*}u\mid _{|x|=1}=x^2_2 x_3=r^3
 \sin^2 \theta \sin^2 \phi \cos \theta\mid _{r=1}=\frac{1}{2}
 (1-\cos 2\phi) \sin^2 \theta \cos \theta =\\= \frac{1}{2}\sin^\theta
 \cos\theta - \frac{1}{2 }\sin^2 \theta \cos\theta \cos 2\phi\end{multline*}
 \begin{multline*}
	 P_3 (t) = \frac{1}{8 \cdot 6} \left( (t^2-1)^3 \right) '''=
	 \frac{1}{48} (t^6 -3 t^4 +3t^2-1)''=
	 \frac{1}{48}(6 \cdot 5 \cdot 4 t^3 - 3 \cdot 4 \cdot 3
	 \cdot 2 t)=\\= \frac{1}{2} (5 t^2 - 3t) \to  P_3 (\cos\theta)= -\frac{1}{5} P_3 (\cos\theta) +\frac{1}{5}P_1 (\cos\theta) -\frac{1}{30}P_3^2 (\cos\theta) \cos 2 \phi
	 \frac{1}{2} (5 \cos^3 \theta -3 \cos \theta)
 .\end{multline*} 
 \[
 \frac{1}{2} \sin^2 \theta \cos\theta = \frac{1}{2}
 \cos \theta - \frac{1}{2} \cos^3 \theta=
 -\frac{1}{5} \cdot \underbrace{\frac{1}{2} (5 \cos^3 \theta- 3 \cos \theta)}_{P_3(\cos\theta)}+
 \frac{1}{5} \underbrace{\cos \theta}_{P_1(\cos\theta)}
 .\] 
 \[
	 P_1(t)= \frac{1}{2}(t^2 -1)'= t \to  P_1 (\cos\theta)=\cos \theta
 .\] 
 \[
	 P^2_3(t)= (1 - t^2)\cdot \frac{1}{2}(5t^3 - 3t)''=
	 15 (1-t^2)t\to  P^2_3 (\cos \theta)= 15 \sin^2 \theta
	 \cos \theta
 .\] 
 \[
	 u= A P_3 (\cos \theta) r^3 + B P_1 ( \cos \theta)r +
	 C P_3^2 (\cos\theta) \cos 2\phi r^3
 .\] 
 \begin{multline*}
	 u\mid _{r=1}=AP_3(\cos\theta) + B P_1 (\cos\theta)+C
	 P_3^2 (\cos\theta) \cos 2\phi =\\= -\frac{1}{5} P_3 (\cos\theta) + \frac{1}{5}P_1 (\cos\theta)- \frac{1}{30 } P_3 ^2 (\cos\theta)
	 \cos_2\phi
 .\end{multline*} 
 \[
 A=-\frac{1}{5},\, B=\frac{1}{5},\, C= - \frac{1}{30}
 .\] 
\end{document}
