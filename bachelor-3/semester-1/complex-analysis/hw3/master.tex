\documentclass[a4paper]{article}
% Этот шаблон документа разработан в 2014 году
% Данилом Фёдоровых (danil@fedorovykh.ru) 
% для использования в курсе 
% <<Документы и презентации в \LaTeX>>, записанном НИУ ВШЭ
% для Coursera.org: http://coursera.org/course/latex .
% Исходная версия шаблона --- 
% https://www.writelatex.com/coursera/latex/5.3

% В этом документе преамбула

\usepackage{siunitx}
%%% Работа с русским языком
%\usepackage{cmap}					% поиск в PDF
%\usepackage{mathtext} 				% русские буквы в формулах
%\usepackage[T2A]{fontenc}			% кодировка
%\usepackage[utf8]{inputenc}			% кодировка исходного текста
%\usepackage[english,russian]{babel}	% локализация и переносы
%\usepackage{indentfirst}
%\frenchspacing
%
%\renewcommand{\epsilon}{\ensuremath{\varepsilon}}
%\newcommand{\phibackup}{\ensuremath{\phi}}
%\renewcommand{\phi}{\ensuremath{\varphi}}
%\renewcommand{\varphi}{\ensuremath{\phibackup}}
%\renewcommand{\kappa}{\ensuremath{\varkappa}}
%\renewcommand{\le}{\ensuremath{\leqslant}}
%\renewcommand{\leq}{\ensuremath{\leqslant}}
%\renewcommand{\ge}{\ensuremath{\geqslant}}
%\renewcommand{\geq}{\ensuremath{\geqslant}}
%\renewcommand{\emptyset}{\varnothing}
%\renewcommand{\Im}{\operatorname{Im}}
%\renewcommand{\Re}{\operatorname{Re}}


%%% Дополнительная работа с математикой
\usepackage{amsmath,amsfonts,amssymb,amsthm,mathtools} % AMS
%\usepackage{icomma} % "Умная" запятая: $0,2$ --- число, $0, 2$ --- перечисление

%% Номера формул
%\mathtoolsset{showonlyrefs=true} % Показывать номера только у тех формул, на которые есть \eqref{} в тексте.
%\usepackage{leqno} % Нумереация формул слева

%% Свои команды
\DeclareMathOperator{\sgn}{\mathop{sgn}}
\DeclareMathOperator{\sign}{\mathop{sign}}
\DeclareMathOperator*{\res}{\mathop{res}}
\DeclareMathOperator*{\tr}{\mathop{tr}}
\DeclareMathOperator*{\rot}{\mathop{rot}}
\DeclareMathOperator*{\divop}{\mathop{div}}
\DeclareMathOperator*{\grad}{\mathop{grad}}

%% Перенос знаков в формулах (по Львовскому)
\newcommand*{\hm}[1]{#1\nobreak\discretionary{}
{\hbox{$\mathsurround=0pt #1$}}{}}

%%% Работа с картинками
\usepackage{graphicx}  % Для вставки рисунков
\graphicspath{{figures/}}  % папки с картинками
\setlength\fboxsep{3pt} % Отступ рамки \fbox{} от рисунка
\setlength\fboxrule{1pt} % Толщина линий рамки \fbox{}
\usepackage{wrapfig} % Обтекание рисунков текстом

%%% Работа с таблицами
\usepackage{array,tabularx,tabulary,booktabs} % Дополнительная работа с таблицами
\usepackage{longtable}  % Длинные таблицы
\usepackage{multirow} % Слияние строк в таблице

%%% Теоремы
\theoremstyle{plain} % Это стиль по умолчанию, его можно не переопределять.
\newtheorem{thm}{Теорема}
\newtheorem*{thm*}{Теорема}
\newtheorem{prop}{Предложение}
\newtheorem*{prop*}{Предложение}
 
\theoremstyle{definition} % "Определение"
%\newtheorem{corollary}{Следствие}[theorem]
\newtheorem{dfn}{Определение}
\newtheorem*{dfn*}{Определение}
\newtheorem{prob}{Задача}
\newtheorem*{prob*}{Задача}

 
\theoremstyle{remark} % "Примечание"
\newtheorem*{sol}{Решение}
\newtheorem*{rem}{Замечание}

%%% Программирование
\usepackage{etoolbox} % логические операторы

%%% Страница
%\usepackage{extsizes} % Возможность сделать 14-й шрифт
%\usepackage{geometry} % Простой способ задавать поля
%	\geometry{top=25mm}
%	\geometry{bottom=35mm}
%	\geometry{left=35mm}
%	\geometry{right=20mm}
 
\usepackage{fancyhdr} % Колонтитулы
%	\pagestyle{fancy}
 %	\renewcommand{\headrulewidth}{0pt}  % Толщина линейки, отчеркивающей верхний колонтитул
	%\lfoot{Нижний левый}
	%\rfoot{Нижний правый}
	%\rhead{Верхний правый}
	%\chead{Верхний в центре}
	%\lhead{Верхний левый}
	%\cfoot{Нижний в центре} % По умолчанию здесь номер страницы

\usepackage{setspace} % Интерлиньяж
%\onehalfspacing % Интерлиньяж 1.5
%\doublespacing % Интерлиньяж 2
%\singlespacing % Интерлиньяж 1

\usepackage{lastpage} % Узнать, сколько всего страниц в документе.

\usepackage{soul} % Модификаторы начертания

\usepackage{hyperref}
\usepackage[usenames,dvipsnames,svgnames,table,rgb]{xcolor}
\hypersetup{				% Гиперссылки
    unicode=true,           % русские буквы в раздела PDF
    pdftitle={Заголовок},   % Заголовок
    pdfauthor={Автор},      % Автор
    pdfsubject={Тема},      % Тема
    pdfcreator={Создатель}, % Создатель
    pdfproducer={Производитель}, % Производитель
    pdfkeywords={keyword1} {key2} {key3}, % Ключевые слова
%    colorlinks=true,       	% false: ссылки в рамках; true: цветные ссылки
    %linkcolor=red,          % внутренние ссылки
    %citecolor=black,        % на библиографию
    %filecolor=magenta,      % на файлы
    %urlcolor=cyan           % на URL
}

\usepackage{csquotes} % Еще инструменты для ссылок

%\usepackage[style=apa,maxcitenames=2,backend=biber,sorting=nty]{biblatex}

\usepackage{multicol} % Несколько колонок

\usepackage{tikz} % Работа с графикой
\usepackage{pgfplots}
\usepackage{pgfplotstable}
%\usepackage{coloremoji}
\usepackage{floatrow}
\usepackage{subcaption}
\graphicspath{{figures/}}

\renewcommand\thesubfigure{\asbuk{subfigure}}
%\addbibresource{master.bib}

\usepackage{import}
\usepackage{pdfpages}
\usepackage{transparent}
\usepackage{xcolor}
\usepackage{xifthen}

\newcommand{\incfig}[2][1]{%
    \def\svgwidth{#1\columnwidth}
    \import{./figures/}{#2.pdf_tex}
}
%\usepackage{titlesec}
%\titleformat{\section}{\normalfont\Large\bfseries}{}{0pt}{}
%----------------------STANDART:
%\titleformat{\chapter}[display]
%  {\normalfont\huge\bfseries}{\chaptertitlename\ \thechapter}{20pt}{\Huge}
%\titleformat{\section}{\normalfont\Large\bfseries}{\thesection}{1em}{}
%\titleformat{\subsection}
%  {\normalfont\large\bfseries}{\thesubsection}{1em}{}
%\titleformat{\subsubsection}
%  {\normalfont\normalsize\bfseries}{\thesubsubsection}{1em}{}
%\titleformat{\paragraph}[runin]
%  {\normalfont\normalsize\bfseries}{\theparagraph}{1em}{}
%\titleformat{\subparagraph}[runin]
%  {\normalfont\normalsize\bfseries}{\thesubparagraph}{1em}{}

\pdfsuppresswarningpagegroup=1
\pgfplotsset{compat=1.16}



%\setcounter{tocdepth}{1} % only parts,chapters,sections
%\titleformat{\subsection}{\normalfont\large\bfseries}{}{0em}{}
%\titleformat{\subsubsection}{\normalfont\normalsize\bfseries}{}{0em}{}

%\newcommand{\textover}[2]{\stackrel{\mathclap{\normalfont\mbox{#2}}}{#1}}

\author{Yaroslav Drachov\\
Moscow Institute of Physics and Technology}
%\author{Драчов Ярослав\\
%Факультет общей и прикладной физики МФТИ}
\newcommand{\veq}{\mathrel{\rotatebox{90}{$=$}}}
%\newcommand{\teto}[1]{\stackrel{\mathclap{\normalfont\tiny\mbox{#1}}}{\to}}
%\renewcommand{\thesubsection}{\arabic{subsection}}

%%\setcounter{secnumdepth}{0}

\definecolor{tabblue}{RGB}{30, 119, 180}
\definecolor{taborange}{RGB}{255, 127, 15}
\definecolor{tabgreen}{RGB}{45, 160, 43}
\definecolor{tabred}{RGB}{214, 38, 40}
\definecolor{tabpurple}{RGB}{148, 103, 189}
\definecolor{tabbrown}{RGB}{140, 86, 76}
\definecolor{tabpink}{RGB}{227, 119, 193}
\definecolor{tabgray}{RGB}{127, 127, 127}
\definecolor{tabolive}{RGB}{188, 189, 33}
\definecolor{tabcyan}{RGB}{22, 190, 207}
\pgfplotscreateplotcyclelist{colorbrewer-tab}{
{tabblue},
{taborange},
{tabgreen},
{tabred},
{tabpurple},
{tabbrown},
{tabpink},
{tabgray},
{tabolive},
{tabcyan},
}
\usepackage{csvsimple}
\usepackage{extarrows}
%\renewcommand{\labelenumii}{\asbuk{enumii})}
%\renewcommand{\labelenumiv}{\Asbuk{enumiv}}
%\newcommand{\prob}[1]{\subsubsection*{#1}}
\sisetup{output-decimal-marker = {,},separate-uncertainty = true,exponent-product = \cdot}

\usepackage{braket}
\usepackage{enumerate}
\usepackage{chngcntr}
%\counterwithin*{equation}{problem}
%\usepackage{bbold}

\newtheoremstyle{hiProb}% ⟨name ⟩ 
{3pt}% ⟨Space above ⟩1 
{3pt}% ⟨Space below ⟩1
{}% ⟨Body font ⟩
{}% ⟨Indent amount ⟩2
{\bfseries}% ⟨Theorem head font⟩
{.}% ⟨Punctuation after theorem head ⟩
{.5em}% ⟨Space after theorem head ⟩3
%{\thmname{#1} \thmnote{#3}}% ⟨Theorem head spec (can be left empty, meaning ‘normal’)⟩
{\thmnote{#3}}% ⟨Theorem head spec (can be left empty, meaning ‘normal’)⟩
\theoremstyle{hiProb} % "Определение"
%\newtheorem{hiProb}{Задача}
\newtheorem{hiProb}{}
%\usepackage{mmacells}
\newcommand{\textover}[2]{\stackrel{\mathclap{\normalfont\scriptsize\mbox{#2}}}{#1}}
\usepackage{units}
\usepackage[math]{cellspace}%
\setlength\cellspacetoplimit{2pt}
\setlength\cellspacebottomlimit{2pt}

\DeclareMathAlphabet{\mathbbold}{U}{bbold}{m}{n}

\newcommand{\normord}[1]{:\mathrel{#1}:}

\title{Третье задание по ТФКП}
\begin{document}
	\maketitle
\renewcommand{\thesection}{\Roman{section}.}
\section{Принцип максимума модуля}
\begin{hiProb}[Т.1]
	\label{prob:1}
	Пусть $P(z)=z^n+c_1 z^{n-1}+\ldots+c_n$. Доказать, что
	если $P(z)\not \equiv z^n$, то найдётся на единичной окружности
	точка $z_0$, в которой $|P(z_0)|>1$.
\end{hiProb}
\begin{proof}
%	Из $P(z)\not\equiv z^n$ следует, что существует
%	хотя бы один
%	$c_i\not=0$.
%	\[
%		g(z)= \frac{P(z)}{z^n}= 1+\frac{c_1}{z}+\ldots
%		+\frac{c_n}{z^n}
%	.\] 
%	\[
%		\lim_{n \to \infty} g(z)=1 \implies
%	z=\infty \text{ --- УОТ для } g (z)	.\] 
%Доопределим $g(\infty)=1$ и тогда $g(z)$ голоморфна в окрестности
%бесконечности.
%\[
%	f(\xi)= g\left( \frac{1}{\xi} \right) 
%.\] 
\[
	g(z)= \frac{P(z)}{z^n};\quad f(\xi)= g\left( \frac{1}{\xi} \right) 
.\] 
\[
	g(z)=1+ \sum_{k=1}^{n} \frac{c_k}{z^k};\quad
	f(\xi)= 1+ \sum_{k=1}^{n} c_k \xi^k
.\] 
Заметим, что $f(0)=1$. Рассмотрим $D=B_1 (0)$ --- единичный круг.
\[
	\max_{\xi \in \overline{B_1(0)}}|f(\xi)| \xlongequal[]{\substack{\text{принцип}\\ \max \text{мод.}}}
	\max_{|\xi|=1}|f(\xi)|>|f(0)|=1
.\] 
Следовательно существует $\xi_0$ такой, что $|\xi_0|=1$ и
$|f(\xi_0)|>1$. Значит
\[
	|f(\xi_0)|= \left|g \left( \frac{1}{\xi_0} \right)  \right| = \left| \frac{P\left( \frac{1}{\xi_0} \right) }{\left( 1 /\xi_0 \right) ^n} \right| =|P(z_0)|>1
,\] 
где $z_0=\frac{1}{\xi_0}$, $|z_0|=1$
\end{proof}
\begin{hiProb}[Т.2]
	Пусть $P(z)$ --- полином степени $n$ и $M(r)=
	\max\limits_{|z|=r}|P(z)|$. Докажите, что для $0<r_1<r_2$ 
	выполняется неравенство
	\[
		\frac{M(r_1)}{r_1^n}\ge \frac{M(r_2)}{r_2^n}
	.\] 
\end{hiProb}	
\begin{proof}
Аналогично Т.1 рассмотрим
\[
	f(\xi)= \xi ^n P\left( \frac{1}{\xi} \right) 
.\] 
Т.\:к. $\overline{B_{\rho_1}  (0)}\supset \overline{B_{\rho_2}(0)}$
при $0< \rho_1 < \rho_2$, то
\[
	\max_{\xi \in \overline{B_{\rho_2(0)}}}|f(\xi)|\ge 
\max_{\xi \in  \overline{B_{\rho_1}(0)}}|f(\xi)|
.\] 
Следовательно по принципу максимума модуля
\[
	\max_{|\xi|=\rho_2}|f(\xi)|\ge  \max_{|\xi|=\rho_1}|f(\xi)|
.\] 
Значит
\[
	\max_{|z| = 1 /\rho_2} \left| \frac{P(z)}{z^n} \right| 	\ge  \max_{|z|= 1 /\rho_1} \left| \frac{P(z)}{z^n} \right| 
.\]
\[
r_1= \frac{1}{\rho_2},\quad r_2= \frac{1}{\rho_1},\quad 
0<r_1<r_2
.\] 
Поэтому
\[
	\frac{M(r_1)}{r_1^n}\ge \frac{M(r_2)}{r_2^n}
.\] 
\end{proof}
\section{Конформные отображения}
\begin{hiProb}[\S 27 №4(5)]
Найти образ полуплоскости $\{z: \Re z<1\} $ при отображении
\[
	w=\frac{z-3+i}{z+1+i} 
.\] 
\end{hiProb}
\begin{sol}
\[
	w(-1-i)=\infty,\quad w(3-i)=0,\quad w(\infty)=1
.\]
Т.\:к. $-1-i \not\in \{z: \Re z=1\} $, то данная прямая
перейдёт в окружность. Точки $-1-i$ и $3-i$ симметричны относительно
прямой $\Re z=1$, поэтому после отображения $\infty$ и $0$ будут симметричны
относительно окружности --- образа этой прямой. Следовательно
$0$ --- центр окружности. Также можно заметить что бесконечная
точка принадлежит прямой и отображается в единицу, следовательно
единица будет принадлежать получившейся окружности, и, поэтому,
радиус окружности будет равен единице.
\begin{figure}[htpb]
    \centering
    \incfig{1}
    \caption{}
    \label{fig:1}
\end{figure}
\end{sol}
\begin{hiProb}[\S 27 №7(2)]
	Отыскать дробно-линейную функцию $w(z)$, удовлетворяющие
	условиям
	\[
		w(i)=0,\quad w(\infty)=1,\quad w(-i)=\infty
	.\] 
\end{hiProb}
\begin{sol}
Дробно-линейные функции можно представить в виде
\[
	w(z)= \frac{az+b}{cz+d}
.\] 
Из второго условия получаем $a=c$, из первого $b=-ia$, из
третьего $d=ic$. Тогда
\[
	w(z)= \frac{az-ia}{az+ia}= \frac{z-i}{z+i}
.\] 
\end{sol}
\begin{hiProb}[\S 27 №8]
	Найти функцию $w(z)$, конформно отображающую область
	$D$ на область $D_1$ и удовлетворяющую указанным
	условиям:
	\renewcommand{\labelenumi}{\arabic{enumi})}
	\begin{enumerate}
		\setcounter{enumi}{1}
	\item $D= \{ z: |z|<1\} $, $D_1= \{w: |w|<1\} $,
		$w(z_0)=w_0$, $\arg w'(z_0)=\alpha$ ($|z_0|<1$,
		$|w_0|<1$);
		\setcounter{enumi}{3}
	\item  $D= \{z: |\Im z|>0\} $, $D_1=\{w: |w|<1\} $,
		$w(z_0)=w_0$, $\arg w'(z_0)=\alpha$ 
		($\Im z_0>0$, $|w_0|<1$).
	\end{enumerate}
\end{hiProb}
\begin{sol}
\renewcommand{\labelenumi}{\arabic{enumi})}
\begin{enumerate}
\setcounter{enumi}{1}
\item %Поскольку при дробно-линейном отображении $w(z)$ точка
%	$z_0$ должна перейти в точку  $w_0$, то по
%	свойству дробно линейных отображений точка $\tilde{z}_0$,
%	симметричная точке $z_0$ относительно окружности
%	$|z|=1$, должна будет перейти в точку $\widetilde{w}_0$, симметричную
%	$|w|=1$.
%	Докажем, что
%	\[
%		\tilde{z}_0= \frac{1}{\overline{z}_0}
%	.\] 
%	Для этого заметим, что точки $\tilde{z}_0$ и $z_0$ лежат
%	на одном луче, выходящем из $z=0$, причём
%	\[
%		|z_0|\cdot |\tilde{z}_0|=1
%	.\] 
%Тогда, записывая $z_0$ в виде
%\[
%z_0= |z_0| \cdot e^{i\phi}
% ,\] 
% получаем
% \[
%	 \tilde{z}_0=|\tilde{z}_0|\cdot e^{i\phi}=
%	 \frac{1}{|z_0| e^{-i\phi}}= \frac{1}{\overline{z}_0}
% .\] 
% Аналогично $\widetilde{w}_0=1 / \overline{w}_0$.
%	Любое конформное отображение круга $D$ на круг $D_1$ 
%	имеет вид
%	\[
%		w(z)= \frac{z-a}{1-\overline{a}z}\cdot
%		e^{i\theta}
%	,\] 
%	где $|a|<1$, $\theta \in \mathbb{R}$.
%	В нашем случае
%	\[
%	w_0=\frac{z_0-a}{1-\overline{a}z_0}\cdot e^{i\theta},\quad
%	w'(z_0)= \frac{(1-\overline{a}z_0)+\overline{a}(z_0-a)}{
%	(1-\overline{a}z_0)^2}\cdot e^{i\theta}=
%	\frac{1-\overline{a}a}{(1-\overline{a}z_0)^2}\cdot e^{i\theta}
%	.\] 
%	А также имеем следующее равенство
%	\[
%	\frac{1}{\overline{w}_0}=\frac{1 /\overline{z}_0-a}{1-
%	\overline{a} /\overline{z}_0}\cdot e^{i \theta}=
%	.\] 
%	\[
%		|w'(z_0)|^2= \frac{1}{\left(1-\overline{a}z_0\right)
%		\left( 1-a \overline{z}_0 \right) }
%	.\] 
%	\[
%		\frac{w'(z_0)}{|w'(z_0)|}=(1-a \overline{z}_0)
%		e^{i\theta}=
%		e^{i \arg w'(z_0)}=e^{i\alpha}
%	.\] 
%	\[
%		w_0= \frac{z_0-a}{|1-\overline{a}z_0|^2}e^{i\alpha}
%	.\] 
%	И
%	\[
%		w(z)=
%	.\] 
%
%	Пусть
%	\[
%		w(z)= \frac{az+b}{cz+d}
%	.\]
%Тогда
%\[
%	w'(z_0)= \frac{a(cz_0+d)-c(az_0+b)}{(cz_0+d)^2}=
%	\frac{ad-c b}{(cz_0+d)^2}
%.\] 
%\[
%	w(z_0)= \frac{az_0+b}{cz_0+d}=w_0
%.\] 
Рассмотрим отображение из диска в диск
\[
	t(z)= \frac{z-z_0}{1- z\overline{z}_0}e^{i\alpha}
,\] 
переводящее точку $z_0$ в нуль. Для него $\arg (t'(z_0))=\alpha$.

Рассмотрим также еще одно отображение из диска в диск
\[
	w(t)= \frac{t+w_0}{1+t \overline{w}_0}
,\] 
переводящее нуль в точку $w_0$. Для него $\arg (w'(0))=0$.
Следовательно $w(t(z_0))=w_0$. А также
\[
	\arg \left( \left. \frac{d w(t(z))}{dz} \right|_{z=z_0} \right) =\alpha
.\] 
Этим мы показали, что композиция данных отображений и будет
искомой функцией $w(z)$. Упростим её
\[
	w(z)=w(t(z))= \frac{\frac{z-z_0}{1-z \overline{z}_0}e^{
	i\alpha}+w_0}{1+ \frac{z-z_0}{1-z \overline{z}_0}e^{
	i\alpha}\overline{w}_0}=
	\frac{\left(e^{i\alpha}-w_0 \overline{z}_0\right)z
	-z_0 e^{i\alpha}+w_0}{\left( e^{i\alpha}\overline{w}_0-
\overline{z}_0\right)z+1-z_0 \overline{w}_0 e^{i\alpha} }
.\] 
\setcounter{enumi}{3}
\item Рассмотрим отображение из верхней полуплоскости на
	единичный круг
	\[
		t(z)= \frac{z-z_0}{z-\overline{z}_0}e^{i\alpha}
	,\]
переводящее $z_0$ в нуль, и отображение из единичного
круга в единичный круг
\[
	w(t)=\frac{t+w_0}{1+t \overline{w}_0}
,\] переводящее нуль в $w_0$. Заметим, что
\[
	\arg w'(t(z_0))= \underbrace{w'(0)}_{=0}+
	\arg (t'(z_0))=\alpha
.\] 
Тем самым мы показали, что искомое отображение $w(z)$ будет
композицией $w(t(z))$ рассмотренных отображений, и
\[
	w(z)= \frac{(z-z_0)e^{i\alpha}+w_0(z-\overline{z}_0}{
	(z-\overline{z}_0)+(z-z_0)e^{i\alpha}\overline{w}_0}=
	\frac{(w_0+e^{i\alpha})z-z_0 e^{\alpha}-w_0 \overline{z}_0}{(1+\overline{w}_0 e^{i\alpha})z-\overline{z}_0-z_0 \overline{w}_0 e^{i\alpha}}
.\] 
\end{enumerate}
\end{sol}
\begin{hiProb}[\S 28 №5]
	Найти какие-либо функции $w(z)$, осуществляющие
	конформные отображения областей, изображённых на
	рис.~\ref{fig:6} на полуплоскость $\{w\colon  \Im  w >0\} $.
\begin{figure}[ht]
	\begin{subfigure}[b]{0.4\textwidth}
    \centering
    \incfig{2}
    \caption{}
    \label{fig:2}
\end{subfigure}\qquad
\begin{subfigure}[b]{0.4\textwidth}
    \centering
    \incfig{3}
    \label{fig:3}
    \caption{}
\end{subfigure}\\
\begin{subfigure}[b]{0.4\textwidth}
    \centering
    \incfig{4}
    \label{fig:4}
    \caption{}
\end{subfigure}\qquad
\begin{subfigure}[b]{0.4\textwidth}
    \centering
    \incfig{5}
    \label{fig:5}
    \caption{}
\end{subfigure}
\caption{}
\label{fig:6}
\end{figure}
\end{hiProb}
\begin{sol}
\renewcommand{\labelenumi}{\asbuk{enumi})}
\begin{enumerate}
	\item См. рис.~\ref{fig:7}, \ref{fig:8}, \ref{fig:9}.
\begin{figure}[htpb]
    \centering
    \incfig{7}
    \caption{}
    \label{fig:7}
\end{figure}
\begin{figure}[ht]
    \centering
    \incfig{8}
    \caption{}
    \label{fig:8}
\end{figure}
\begin{figure}[ht]
    \centering
    \incfig{9}
    \caption{}
    \label{fig:9}
\end{figure}
\item См. рис.~\ref{fig:10}, \ref{fig:11}.
\begin{figure}[ht]
    \centering
    \incfig{10}
    \caption{}
    \label{fig:10}
\end{figure}
\begin{figure}[ht]
    \centering
    \incfig{11}
    \caption{}
    \label{fig:11}
\end{figure}
\item См. рис.~\ref{fig:12}, \ref{fig:13}, \ref{fig:14}.
\begin{figure}[ht]
    \centering
    \incfig{12}
    \caption{}
    \label{fig:12}
\end{figure}
\begin{figure}[ht]
    \centering
    \incfig{13}
    \caption{}
    \label{fig:13}
\end{figure}
\begin{figure}[ht]
    \centering
    \incfig{14}
    \caption{}
    \label{fig:14}
\end{figure}
\item См. рис.~\ref{fig:15}, далее как в первом пункте данной задачи.
\begin{figure}[ht]
    \centering
    \incfig{15}
    \caption{}
    \label{fig:15}
\end{figure}
	\end{enumerate}
\end{sol}
\begin{hiProb}[\S28 №10]
	Найти какие-либо функции $w(z)$, осуществляющие
	конформные отображения областей, изображённых
	на рис.~\ref{fig:19} на полуплоскость
	$\{w\colon \Im  w>0\} $.

\begin{figure}[ht]
\begin{subfigure}[b]{0.45\textwidth}
    \centering
    \incfig{16161616161616161616161616161616}
    \caption{}
    \label{fig:16}
\end{subfigure}\qquad
\begin{subfigure}[b]{0.45\textwidth}
    \centering
    \incfig{17}
    \caption{}
    \label{fig:17}
\end{subfigure}\\
\begin{subfigure}[b]{0.9\textwidth}
    \centering
    \incfig{18}
    \caption{}
    \label{fig:18}
\end{subfigure}
\label{fig:19}
\caption{}
\end{figure}
\end{hiProb}
\begin{sol}
\renewcommand{\labelenumi}{\asbuk{enumi})}
\begin{enumerate}
\item См. рис.~\ref{fig:20}.
\begin{figure}[ht]
    \centering
    \incfig{20}
    \caption{}
    \label{fig:20}
\end{figure}
\item См. рис.~\ref{fig:21}.
\begin{figure}[ht]
    \centering
    \incfig{21}
    \caption{}
    \label{fig:21}
\end{figure}
\item См. рис.~\ref{fig:22}.
\begin{figure}[ht]
    \centering
    \incfig{22}
    \caption{}
    \label{fig:22}
\end{figure}
\end{enumerate}
\end{sol}
\begin{hiProb}[\S 28 №11]
	Найти какие либо функции $w(z)$, осуществляющие
	конформные отображения области, изображённой
	на рис.~\ref{fig:23}, на круг $\{w\colon |w|<1\} $.
\begin{figure}[ht]
    \centering
    \incfig{23}
    \caption{23}
    \label{fig:23}
\end{figure}
\end{hiProb}

\end{document}
