\documentclass[a4paper]{article}
% Этот шаблон документа разработан в 2014 году
% Данилом Фёдоровых (danil@fedorovykh.ru) 
% для использования в курсе 
% <<Документы и презентации в \LaTeX>>, записанном НИУ ВШЭ
% для Coursera.org: http://coursera.org/course/latex .
% Исходная версия шаблона --- 
% https://www.writelatex.com/coursera/latex/5.3

% В этом документе преамбула

\usepackage{siunitx}
%%% Работа с русским языком
%\usepackage{cmap}					% поиск в PDF
%\usepackage{mathtext} 				% русские буквы в формулах
%\usepackage[T2A]{fontenc}			% кодировка
%\usepackage[utf8]{inputenc}			% кодировка исходного текста
%\usepackage[english,russian]{babel}	% локализация и переносы
%\usepackage{indentfirst}
%\frenchspacing
%
%\renewcommand{\epsilon}{\ensuremath{\varepsilon}}
%\newcommand{\phibackup}{\ensuremath{\phi}}
%\renewcommand{\phi}{\ensuremath{\varphi}}
%\renewcommand{\varphi}{\ensuremath{\phibackup}}
%\renewcommand{\kappa}{\ensuremath{\varkappa}}
%\renewcommand{\le}{\ensuremath{\leqslant}}
%\renewcommand{\leq}{\ensuremath{\leqslant}}
%\renewcommand{\ge}{\ensuremath{\geqslant}}
%\renewcommand{\geq}{\ensuremath{\geqslant}}
%\renewcommand{\emptyset}{\varnothing}
%\renewcommand{\Im}{\operatorname{Im}}
%\renewcommand{\Re}{\operatorname{Re}}


%%% Дополнительная работа с математикой
\usepackage{amsmath,amsfonts,amssymb,amsthm,mathtools} % AMS
%\usepackage{icomma} % "Умная" запятая: $0,2$ --- число, $0, 2$ --- перечисление

%% Номера формул
%\mathtoolsset{showonlyrefs=true} % Показывать номера только у тех формул, на которые есть \eqref{} в тексте.
%\usepackage{leqno} % Нумереация формул слева

%% Свои команды
\DeclareMathOperator{\sgn}{\mathop{sgn}}
\DeclareMathOperator{\sign}{\mathop{sign}}
\DeclareMathOperator*{\res}{\mathop{res}}
\DeclareMathOperator*{\tr}{\mathop{tr}}
\DeclareMathOperator*{\rot}{\mathop{rot}}
\DeclareMathOperator*{\divop}{\mathop{div}}
\DeclareMathOperator*{\grad}{\mathop{grad}}

%% Перенос знаков в формулах (по Львовскому)
\newcommand*{\hm}[1]{#1\nobreak\discretionary{}
{\hbox{$\mathsurround=0pt #1$}}{}}

%%% Работа с картинками
\usepackage{graphicx}  % Для вставки рисунков
\graphicspath{{figures/}}  % папки с картинками
\setlength\fboxsep{3pt} % Отступ рамки \fbox{} от рисунка
\setlength\fboxrule{1pt} % Толщина линий рамки \fbox{}
\usepackage{wrapfig} % Обтекание рисунков текстом

%%% Работа с таблицами
\usepackage{array,tabularx,tabulary,booktabs} % Дополнительная работа с таблицами
\usepackage{longtable}  % Длинные таблицы
\usepackage{multirow} % Слияние строк в таблице

%%% Теоремы
\theoremstyle{plain} % Это стиль по умолчанию, его можно не переопределять.
\newtheorem{thm}{Теорема}
\newtheorem*{thm*}{Теорема}
\newtheorem{prop}{Предложение}
\newtheorem*{prop*}{Предложение}
 
\theoremstyle{definition} % "Определение"
%\newtheorem{corollary}{Следствие}[theorem]
\newtheorem{dfn}{Определение}
\newtheorem*{dfn*}{Определение}
\newtheorem{prob}{Задача}
\newtheorem*{prob*}{Задача}

 
\theoremstyle{remark} % "Примечание"
\newtheorem*{sol}{Решение}
\newtheorem*{rem}{Замечание}

%%% Программирование
\usepackage{etoolbox} % логические операторы

%%% Страница
%\usepackage{extsizes} % Возможность сделать 14-й шрифт
%\usepackage{geometry} % Простой способ задавать поля
%	\geometry{top=25mm}
%	\geometry{bottom=35mm}
%	\geometry{left=35mm}
%	\geometry{right=20mm}
 
\usepackage{fancyhdr} % Колонтитулы
%	\pagestyle{fancy}
 %	\renewcommand{\headrulewidth}{0pt}  % Толщина линейки, отчеркивающей верхний колонтитул
	%\lfoot{Нижний левый}
	%\rfoot{Нижний правый}
	%\rhead{Верхний правый}
	%\chead{Верхний в центре}
	%\lhead{Верхний левый}
	%\cfoot{Нижний в центре} % По умолчанию здесь номер страницы

\usepackage{setspace} % Интерлиньяж
%\onehalfspacing % Интерлиньяж 1.5
%\doublespacing % Интерлиньяж 2
%\singlespacing % Интерлиньяж 1

\usepackage{lastpage} % Узнать, сколько всего страниц в документе.

\usepackage{soul} % Модификаторы начертания

\usepackage{hyperref}
\usepackage[usenames,dvipsnames,svgnames,table,rgb]{xcolor}
\hypersetup{				% Гиперссылки
    unicode=true,           % русские буквы в раздела PDF
    pdftitle={Заголовок},   % Заголовок
    pdfauthor={Автор},      % Автор
    pdfsubject={Тема},      % Тема
    pdfcreator={Создатель}, % Создатель
    pdfproducer={Производитель}, % Производитель
    pdfkeywords={keyword1} {key2} {key3}, % Ключевые слова
%    colorlinks=true,       	% false: ссылки в рамках; true: цветные ссылки
    %linkcolor=red,          % внутренние ссылки
    %citecolor=black,        % на библиографию
    %filecolor=magenta,      % на файлы
    %urlcolor=cyan           % на URL
}

\usepackage{csquotes} % Еще инструменты для ссылок

%\usepackage[style=apa,maxcitenames=2,backend=biber,sorting=nty]{biblatex}

\usepackage{multicol} % Несколько колонок

\usepackage{tikz} % Работа с графикой
\usepackage{pgfplots}
\usepackage{pgfplotstable}
%\usepackage{coloremoji}
\usepackage{floatrow}
\usepackage{subcaption}
\graphicspath{{figures/}}

\renewcommand\thesubfigure{\asbuk{subfigure}}
%\addbibresource{master.bib}

\usepackage{import}
\usepackage{pdfpages}
\usepackage{transparent}
\usepackage{xcolor}
\usepackage{xifthen}

\newcommand{\incfig}[2][1]{%
    \def\svgwidth{#1\columnwidth}
    \import{./figures/}{#2.pdf_tex}
}
%\usepackage{titlesec}
%\titleformat{\section}{\normalfont\Large\bfseries}{}{0pt}{}
%----------------------STANDART:
%\titleformat{\chapter}[display]
%  {\normalfont\huge\bfseries}{\chaptertitlename\ \thechapter}{20pt}{\Huge}
%\titleformat{\section}{\normalfont\Large\bfseries}{\thesection}{1em}{}
%\titleformat{\subsection}
%  {\normalfont\large\bfseries}{\thesubsection}{1em}{}
%\titleformat{\subsubsection}
%  {\normalfont\normalsize\bfseries}{\thesubsubsection}{1em}{}
%\titleformat{\paragraph}[runin]
%  {\normalfont\normalsize\bfseries}{\theparagraph}{1em}{}
%\titleformat{\subparagraph}[runin]
%  {\normalfont\normalsize\bfseries}{\thesubparagraph}{1em}{}

\pdfsuppresswarningpagegroup=1
\pgfplotsset{compat=1.16}



%\setcounter{tocdepth}{1} % only parts,chapters,sections
%\titleformat{\subsection}{\normalfont\large\bfseries}{}{0em}{}
%\titleformat{\subsubsection}{\normalfont\normalsize\bfseries}{}{0em}{}

%\newcommand{\textover}[2]{\stackrel{\mathclap{\normalfont\mbox{#2}}}{#1}}

\author{Yaroslav Drachov\\
Moscow Institute of Physics and Technology}
%\author{Драчов Ярослав\\
%Факультет общей и прикладной физики МФТИ}
\newcommand{\veq}{\mathrel{\rotatebox{90}{$=$}}}
%\newcommand{\teto}[1]{\stackrel{\mathclap{\normalfont\tiny\mbox{#1}}}{\to}}
%\renewcommand{\thesubsection}{\arabic{subsection}}

%%\setcounter{secnumdepth}{0}

\definecolor{tabblue}{RGB}{30, 119, 180}
\definecolor{taborange}{RGB}{255, 127, 15}
\definecolor{tabgreen}{RGB}{45, 160, 43}
\definecolor{tabred}{RGB}{214, 38, 40}
\definecolor{tabpurple}{RGB}{148, 103, 189}
\definecolor{tabbrown}{RGB}{140, 86, 76}
\definecolor{tabpink}{RGB}{227, 119, 193}
\definecolor{tabgray}{RGB}{127, 127, 127}
\definecolor{tabolive}{RGB}{188, 189, 33}
\definecolor{tabcyan}{RGB}{22, 190, 207}
\pgfplotscreateplotcyclelist{colorbrewer-tab}{
{tabblue},
{taborange},
{tabgreen},
{tabred},
{tabpurple},
{tabbrown},
{tabpink},
{tabgray},
{tabolive},
{tabcyan},
}
\usepackage{csvsimple}
\usepackage{extarrows}
%\renewcommand{\labelenumii}{\asbuk{enumii})}
%\renewcommand{\labelenumiv}{\Asbuk{enumiv}}
%\newcommand{\prob}[1]{\subsubsection*{#1}}
\sisetup{output-decimal-marker = {,},separate-uncertainty = true,exponent-product = \cdot}

\usepackage{braket}
\usepackage{enumerate}
\usepackage{chngcntr}
%\counterwithin*{equation}{problem}
%\usepackage{bbold}

\newtheoremstyle{hiProb}% ⟨name ⟩ 
{3pt}% ⟨Space above ⟩1 
{3pt}% ⟨Space below ⟩1
{}% ⟨Body font ⟩
{}% ⟨Indent amount ⟩2
{\bfseries}% ⟨Theorem head font⟩
{.}% ⟨Punctuation after theorem head ⟩
{.5em}% ⟨Space after theorem head ⟩3
%{\thmname{#1} \thmnote{#3}}% ⟨Theorem head spec (can be left empty, meaning ‘normal’)⟩
{\thmnote{#3}}% ⟨Theorem head spec (can be left empty, meaning ‘normal’)⟩
\theoremstyle{hiProb} % "Определение"
%\newtheorem{hiProb}{Задача}
\newtheorem{hiProb}{}
%\usepackage{mmacells}
\newcommand{\textover}[2]{\stackrel{\mathclap{\normalfont\scriptsize\mbox{#2}}}{#1}}
\usepackage{units}
\usepackage[math]{cellspace}%
\setlength\cellspacetoplimit{2pt}
\setlength\cellspacebottomlimit{2pt}

\DeclareMathAlphabet{\mathbbold}{U}{bbold}{m}{n}

\newcommand{\normord}[1]{:\mathrel{#1}:}

\title{Билеты по курсу <<Теория функций комплексного
переменного>>	}
\begin{document}
	\maketitle
	\tableofcontents
\section{Комплексная дифференцируемость. Условия Коши-Римана.}
Под функцией комплексного переменного $w=f(z)$ будем понимать
отображение множества $D \subset \overline{\mathbb{C}}$ в
комплексной $z$-плоскости в множество $f(D)=G \subset \overline{
\mathbb{C}}$ комлексной $w$-плоскости. Если представить
$z= x+iy$, $w=u+iv$, то задание функции $f$ эквивалентно
определению двух вещественных функций $u(x,\,y)$ и $v(x,\,y)$
вещественных переменных $x$ и $y$, т.\:е. $w=f(z)=
г(x,\,y)+iv(x,\,y)$.
\begin{dfn}
	Будем говорить, что функция $f(z)$ имеет предел
	$A$ при $z\to a$ и писать
	\[
		\lim_{z \to a} f(z)=A
	,\]
	если для каждого $\epsilon >0$ найдётся такое
	$\delta >0$, что $|f(z)-A|<\epsilon $ при
	всех $z \in \dot{\mathcal{O}}_\delta (a)$, т.\:е.
	при $0<|z-a|<\delta$.
\end{dfn}
Заметим, что условие
\[
		\lim_{z \to a} f(z)=A
	\]
эквивалентно
\[
	\lim_{z \to a} \overline{f(z)}=\overline{A}.
	\]
Откуда
\[
	u(z) \to  \Re  A, \quad v(z)\to  \Im  A
.\] 
Обратное утверждение также верно (модуль комплексного числа
не превышает суммы модулей вещественной и мнимой части).
\begin{dfn}
	Будем говорить, что функция $f(z)$ непрерывна в т. $a$,
	если
\[\lim_{z \to a} f(z)=f(a).\]
\end{dfn}
\section{Связность. Теорема о голоморфной в области функции
с обращающейся в нуль производной.}
hi
\section{Степенный ряды и элементарный функции.}
hi
\section{Перообразная и полный дифференциал в области. Условия
независимости интеграла от формы пути.}
hi
\section{Лемма Гурса и теорема Коши для выпуклой области.}
hi
\section{Интеграл Коши и его свойства.}
hi
\section{Интегральная формула Коши для круга. Бесконечная
дифференцируемость голоморфных функций. Теорема Морера.}
hi
\section{Целые функции и теорема Лиувилля.}
hi
\section{Ряд Тейлора и теорема единственности для голоморфных
функций.}
hi
\section{Приращение аргумента вдоль кривой. Индекс и его
свойства.}
hi
\section{Общая форма теоремы Коши и интегральной формулы Коши.
Следствия для односвязной и многосвязной областей.}
hi
\section{Разложение голоморфной функции в ряд Лорана. Теорема
единственности ряда Лорана.}
hi
\section{Изолированные особые точки. Связь классификации с видом
ряда Лорана. Теорема Сохоцкого.}
hi
\section{Вычеты и формулы для их вычисления. Теорема Коши о
вычетах.}
hi
\section{Вычисление несобственных интегралов с помощью вычетов.
Лемма Жордана.}
hi
\section{Регулярные ветви логарифма и корней.}
hi
\section{Принцип аргумента. Теорема Руше. Основная теорема
алгебры.}
hi
\section{Теорема о локальной структуре отображения. Принцип
сохранения области. Однолистность и локальная однолистность.}
hi
\section{Принцип максимума модуля и лемма Шварца.}
hi
\section{Локально равномерная схоодимость и теорема Вейерштрасса.
Теорема Гурвица и её следствие для однолистных функций.}
hi
\section{Локально равномерная ограниченность и принцип
компактности.}
hi
\section{Конформность и групповое свойство дробно-линейных
преобразований. Ангармоническое отношение четырёх точек.}
hi
\section{Круговое свойство и принцип симметрии для дробно-линейных
преобразований.}
hi
\section{Элементарные конформные отображения с использованием
степенной и экспоненциальной функций. Функция Жуковского. Общий
вид конформных отображений единичного круга на себя.}
hi
\section{Теорема Римана об отображении.}
hi
\section{Аналитическое продолжение. Теорема о монодромии.}
hi
\section{Теорема о стирании разреза. Принцип симметрии
Римана-Шварца.}
hi
\section{Мероморфные функции. Теорема Миттаг-Леффлера.}
hi
\section{Гармонические функции и их связь с голоморфными
функциями. Бесконечная дифференцируемость.}
hi
\section{Принцип экстремума и теорема единственности для
гармонических функций. Конформная инвариантность.}
hi
\section{Теорема о среднем и интегральная формула Пуассона.}
hi
\section{Интеграл Пуассона и решение задачи Дирихле в круге.}
hi
\section{Бесконечные произведения голоморфных функций и их нули.}
hi
\section{Гамма-функция и её представления Гаусса и Эйлера.}
hi
\section{Метод стационарной фазы и асимптотика функции Эйри в
отрицательном направлении вещественной оси.}
hi
\section{Метод перевала и асимптотика функции Эйри в положительном
направлении вещественной оси.}
\end{document}
