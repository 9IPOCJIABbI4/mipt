\lecture{2}{Пт 11 сен 2020 17:06}{}
\lec
\[
e^z= \sum_{n=0}^{\infty} \frac{z^n}{n!}
\]
--- голом. в $\mathbb{C}$
 \[
\overline{e^z}= e 
\]
\begin{thm}[Теорема сложения]
	$\forall z_1,\, z_2 \in \mathbb{C}$ 
	\[
	e ^{z_1 + z_2}= e^{ z_1}\cdot e^{z_2}
	.\] 
\end{thm}
\[
	f(x)=e^{z}\cdot e^{a-z}
.\] 
\[
	f'(x)=e^{z}(-1) e^{a-z}+e^{z}\cdot e^{a-z}
.\] 
\[
	f(x)= \text{const} \qquad f(0)= e^a \qquad e^z e ^{a-z}=e^{a}
.\] 
\[
	a,\,z \in \mathbb{C} \qquad z=z_1 \qquad a=z_1+z_2
.\]
\[
e^{z_1} \cdot e^{z_2}= e ^{z_1 +z_2} \qquad e^{ z}\cdot e^{-z} \impliese ^{-z} = \frac{1}{e^{z}}
.\] 
\[
e^{z} \neq \text{ в } \mathbb{C} \qquad z=x+i y 
.\] 
\[
e ^{z}= e ^{ x} \cdot e ^{iy} \qquad e^{iy} \cdot \overline{e^{iy}}= e ^{iy} \cdot e  ^{-iy}=1 \implies |e ^{iy}|=1
.\] 
\[
	|e ^{z}|= e ^{\operatorname{Re} z}
.\] 
\[
	\cos z = \frac{1}{2} \left( e ^{iz}+ e ^{ -iz} \right) , \qquad \sin z = \frac{1}{2i} \left( e ^{iz}- e ^{-iz} \right) 
.\] 
\[
	\ch z= \frac{1}{2}\left( e ^{z}+e ^{-z} \right) , \sh z= \frac{1}{2}\left(e ^{z}- e^{-z}\right)
.\] 
\[
	\forall z \in \mathbb{C} \qquad e^{iz}= \cos z + i \sin z
.\] 
\[
	e^{z}= \sum_{n=0}^{\infty} \frac{z^n}{n!}\qquad (\cos z)'= -\sin z\qquad (\sin z )'= \cos z
.\] 
Периоодичность
\[
	f(z+c)= f(z) \forall z
.\] 
\[
e^{z+c}= e^{ z} \implies e ^{c}=1
.\] 
\[
c= \alpha +i \beta \qquad e ^{\alpha} \cdot e ^{i\beta}=1\implies \alpha=0
.\] 
\[
\left\{
\begin{aligned}
\cos \beta = 1 \\
\sin \alpha = 0
\end{aligned}
\right.\implies
\beta= 2 k \pi, \qquad k \in \mathbb{Z}\implies c = 2k \pi i \text{ --- период. эксп-ты}
.\] 
Логарифм
\[
	a \in  \mathbb{C} \qquad \ln a - ?
.\] 
\[
z= \ln a \qquad e ^{z}=a \qquad a\neq_0
.\] 
\[
	e^{z}=a \implies a = r e^{i \phi} \qquad \phi \in  \operatorname{Arg} \{a\} 
.\] 
\[
e ^{ x+i y}= e^{x} \cdot e ^{i y}
.\] 
\[
\left\{
\begin{aligned}
e ^{x}=r
y = \phi+ 2k \pi
\end{aligned}
\right.
.\] 
\[
x = \ln r= \ln |a| \qquad y = \phi + 2k \pi
.\] 
\[
	\forall a \neq 0 \qquad \operatorname{Ln} \{a\} = \ln |a| + \operatorname{Arg} \{a\} 
.\]
$D$-область
\[
	f(z) \qquad e^{f(z)}=z
.\] 
\[
f(z) \in \operatorname{Ln} \{z\} \qquad f \text{ --- ветвь } \operatorname{Ln} \{z\} \text{ в обл. } D
.\] 
\[
	f(z)= \ln |z| + i \operatorname{z}
.\] 
\begin{figure}[ht]
    \centering
    \incfig{3}
    \caption{}
    \label{fig:3}
\end{figure}
\[
	\ln(1+z) = \ln |1 +z|+ i \operatorname{arg} (1+z)
.\] 
\begin{figure}[ht]
    \centering
    \incfig{4}
    \caption{4}
    \label{fig:4}
\end{figure}
\[
	a ^{b} \qquad a \neq 0 \qquad a,\,b \in \mathbb{C}
.\] 
\[
i^i \qquad \{a^ b\} = e ^{b \operatorname{Ln} \{a\} } \text{ --- по определению}
.\] 
\[
	\{i^i\} =e ^{i \operatorname{Ln} \{i\} }=e ^{- \frac{\pi}{2}+2 \pi k}
.\] 
\[
	\operatorname{Ln} \{ i\} = 0 + i [\ldots]
.\] 
Комплексное интегрирование
\begin{figure}[ht]
    \centering
    \incfig{5}
    \caption{}
    \label{fig:5}
\end{figure}
\[
t \to -t \qquad \tau = -t \qquad -\beta \le  \tau \le  \alpha
.\] 
\[
\gamma \to -\gamma
.\] 
Плоская кривая
\[
	z = z(t) \qquad z'(t) = x'(t)+ i y'(t)\qquad \ldots
.\] 
\begin{figure}[ht]
    \centering
    \incfig{6}
    \caption{6}
    \label{fig:6}
\end{figure}
\[
	w = f(x) = u (z) + i v(z) , \qquad z = x +i y
.\] 
$f$ --- непр.
 \[
	 \gamma: z = z(t) , \qquad \alpha \le  t \le  \beta
.\] 
\[
t_0= \alpha< t_1 < \ldots < t_n = \beta
.\] 
\[
	\theta_k = [t_{k-1},\,t_k]
.\] 
\[
	\Delta z_k = z(t_k) - z(t_{k-1})
.\] 
\[
	\sum_{k=1}^{n} f\left( \left( x(\theta_k) \right)  \right) \Delta z_k= \sum_{k=1}^{n} \left( u (z(\theta_k))\Delta x_k +
	v ( z(\theta_k)) \Delta y_k \right) 
.\]

Св-ва интеграла
Линейность
\[
	\int\limits_{\gamma}^{} \left( a f(x) + b g (x) \right) dx= a \int\limits_{\gamma}^{} f(x) dx + b \int\limits_{\gamma}^{} g(x)
	dx
.\] 
\[
	\int\limits_{-\gamma}^{} f(x) dx= - \int\limits_{\gamma}^{} f(x) dx  
.\] 
\[
	\int\limits_{-\gamma}^{} f(z) |dz|= \int\limits_{\gamma}^{} f(x) |dx|  
.\] 
Нер-во
\[
	\left| \int\limits_{\gamma}^{} f(x) dx  \right| \le  \int\limits_{\gamma}^{} |f(x)| |dx| 
.\] 
\[
\gamma= \gamma_1 + \gamma_2
.\] 
\[
	\int\limits_{\gamma_1 +\gamma_2}^{} f(x) dx= \int\limits_{\gamma_1}^{} f(x) dx + \int\limits_{\gamma_2}^{} dx   
.\] 
$f(z)$ опр. в обл. $D$ 

Будем говорить, что $f(z)dz$ является полным дифференциалом в области $D$ если сцществует голоморфная в $D$ функция
$F(z)$ :
\[
	F'(z) = f(z) \qquad dF(z)= f(z) dz
 ,\] 
 $F$ --- первообразная $f$.
\sem

\[
	(\sin x)'=(-i \ch  i z)'= - i^2 \ch i z = \ch iz= \cos z
.\] 
\[
	\ch z= \frac{e^z+e^{-z}}{2}
.\] 
\[
	(\ch z)'=\frac{e^z - e^{-z}}{2}=\sh z
.\] 
\[
	(\sh z)'=\ch z = \left( \frac{e^z-e^{-z}}{2} \right) '=
	\frac{e^z-e^{-z}}{2}?
.\]
\[
	f(z)=(z^n)'=n z^{n-1}
.\] 
\[
z'=1
.\] 
\[
	(z^{n+1})'=(z^n z)'=(z^n)'z+z^n z'=n z^{n-1}z+z^n=(n+1)z^n
.\] 
\begin{figure}[ht]
    \centering
    \incfig{1}
    \caption{}
    \label{fig:1}
\end{figure}
\begin{dfn}
	$f$ --- $R$- дифф-ма
\end{dfn}
\begin{dfn}
	\[
	df=du+i dv
	.\] 
\end{dfn}
\begin{dfn}
	\[\frac{\partial f}{\partial x} =\frac{\partial u}{\partial x} +i \frac{\partial v}{\partial x} \]
	\[
	\frac{\partial f}{\partial y} =\frac{\partial u}{\partial y} +i \frac{\partial v}{\partial y} 
	.\] 
\end{dfn}

$df=du + idv=\left( \frac{\partial u}{\partial x} dx+
\frac{\partial u}{\partial y} dy\right) +i\left( 
\frac{\partial v}{\partial x} dx+ \frac{\partial v}{\partial y} dy\right) =\left(i \frac{\partial v}{\partial x} +\frac{\partial u}{\partial x}  \right) dx+\left(i \frac{\partial v}{\partial y} +
\frac{\partial u}{\partial y} \right) dy=\frac{\partial f}{\partial x} dx+ \frac{\partial f}{\partial y} dy$ 
\begin{thm}
	\[
	df= \frac{\partial f}{\partial x} dx+\frac{\partial f}{\partial y} dy
	.\] 
\end{thm}
\[\left\{
	\frac{\partial f}{\partial z} = \frac{1}{2}\left(\frac{\partial f}{\partial x} -i \frac{\partial f}{\partial x} \right)\\
	\frac{\partial f}{\partial z} =\frac{1}{2}\left( 
	\frac{\partial f}{\partial x} +i \frac{\partial f}{\partial y} \right) \right.
.\] 
\begin{thm}
	\[
	df=\frac{\partial f}{\partial z} dz+\frac{\partial f}{\partial z} d \overline{z}
	.\] 
\end{thm}
\begin{proof}
	\ldots
\end{proof}
\begin{thm}
	\[
		df=Adz+bd \overline{z} =A(dx+i dy)+
		B(dx-i dy)=(A+B)dx+i (A-B)dy= \frac{\partial f}{\partial x} dx+\frac{\partial f}{\partial y} dy
	.\] 
	\[
	\frac{\partial f}{\partial x} =A+B;\qquad
	\frac{\partial f}{\partial y} =i(A-B)
	.\] 
\end{thm}
\begin{dfn}
	$f=u+iv$ --- $\mathbb{C}$-дифференцируема в точке
	$z_0=x_0+iy_0$ равносильно
	\[
		f- \mathbb{R}	\text{-дифф.}\qquad
		\text{и}\qquad \frac{\partial f}{\partial \overline{z}} =0
	.\] 
\end{dfn}

\begin{thm}
	$f$ --- $\mathbb{C}$-дифф-ма в т. $z_0$ равносильно
	$\exists f'(z_0)\in \mathbb{C}$. 
	\[
		f'(z_0)= \lim_{\Delta z \to 0} \frac{f(z_0+\Delta z)-f(z_0)}{\Delta z}
	.\] 
\end{thm}
\begin{proof}
	Пусть $f=u+iv$--- $\mathbb{R}$-дифф-ма
\[
	\Delta f= \frac{\partial f}{\partial z} \Delta z +\frac{\partial f}{\partial \overline{z}} \Delta \overline{z}+o(\Delta z)
.\]\ldots
\[
\frac{\Delta f}{\Delta z}= \frac{\partial f}{\partial z}+
\frac{\partial f}{\partial \overline{z}} e^{-iz\theta}+o(1)
.\] 
\begin{enumerate}
	\item \[
			\frac{\partial f}{\partial z} =0 \implies \exists f'(z_0)= \lim_{\Delta z \to 0} \frac{\Delta f}{\Delta z}
	.\] 
	\item обратно 
		\[
			\exists f'(z_0)=\lim_{\Delta z \to 0}
			\frac{\Delta f}{\Delta z}
		.\] 
		\[
			\frac{\Delta f}{\Delta z}= f'(z_0)+o(1)
		.\] 
\ldots
\end{enumerate}
\end{proof}
\[
	f'(z)= \frac{\partial f}{\partial z} =\text{Коши-Риман}=
	\frac{\partial u}{\partial x} +i \frac{\partial v}{\partial x} 
.\]
\[
	f(z)=e^{z}=e^x\left( \cos y + i \sin y  \right) 
.\] 
\[
\left\{ 
\begin{aligned}
	 u=e^x \cos y\\ v= e^x \sin y
\end{aligned}
\right. \Leftrightarrow
\left\{
	\begin{aligned}
		u_x=v_y\\ u_y= -v_x
	\end{aligned}\ldots \right.
\] 
 Функция 
\[
f = x + 2y i
\] 
дифференцируема только в точке $(0,\,0)$ по условию К-Р.

 \begin{problem}
Доказать
\[
	\frac{1}{2i} \int\limits_{\partial D}^{} f(z)=
	\iint\limits_{D}^{}  \frac{\partial f}{\partial \overline{z}} 
	dxdy
.\] 
\[
\left\{
	\begin{aligned}
		f \text{ --- } \mathbb{R}\text{-дифф.}\\
		f \text{ --- непр. в } \overline{D}
	\end{aligned}
	\right.
.\] 
\end{problem}
\begin{proof}
\begin{figure}[ht]
    \centering
    \incfig{2}
    \caption{}
    \label{fig:2}
\end{figure}
\[
\int\limits_{\partial D}^{} Pdx+Qdy= \iint\limits_{D}^{} 
(Q_x-P_y) dxdy
.\] 
\[
	P=P(x,\,y)\qquad Q=Q(x,\,y)
.\] 
\[
P,\,Q,\,Q_x,\,P_y \text{ напр. в } \overline{D}
.\] 
\begin{multline*}
	\int\limits_{\partial D}^{} f(z)dz= \int\limits_{\partial D}^{} (u +i v)(dx+i dy)=
\int\limits_{\partial D}^{} udx - v dy +i \int\limits_{}^{} v dx+udy
=\text{Грин}= \\=i\left( i \iint\limits_{D}^{} \left( v_x+u_y \right)dxdy 
+ \iint\limits_{D}^{} (u_x-v_y)dxdy \right)=\\=
2i \iint\limits_{D}^{} \underbrace{\left[ \frac{(u_x-v_y)+i (v_x+u_y)}{2} \right]}_{\frac{\partial f}{\partial \overline{z}} } dxdy=2i \int\limits_{D}^{} \frac{\partial f}{\partial z} dxdy  
.\end{multline*} 
\end{proof}
\begin{problem}
Существует ли такая функция
\[
	f \in \mathbb{O}(U_\epsilon (0)) \qquad
	f\left(\frac{1}{n}\right)=\sin \frac{\pi n}{2},
	\qquad n \in \mathbb{N}, \qquad n\ge n_0
.\] 
\end{problem}
\begin{proof}
	Пусть такая $f$ сущ.
	\begin{enumerate}
		\item $n=2k \implies f\left(\frac{1}{n}\right)
			f\left( \frac{1}{2k} \right) =\sin \pi k =0$\ldots 
		\item \ldots
	\end{enumerate} 
	теорема о единственности
\end{proof}

