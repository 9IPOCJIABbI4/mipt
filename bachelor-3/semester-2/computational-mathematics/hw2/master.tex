\documentclass[a4paper]{article}
% Этот шаблон документа разработан в 2014 году
% Данилом Фёдоровых (danil@fedorovykh.ru) 
% для использования в курсе 
% <<Документы и презентации в \LaTeX>>, записанном НИУ ВШЭ
% для Coursera.org: http://coursera.org/course/latex .
% Исходная версия шаблона --- 
% https://www.writelatex.com/coursera/latex/5.3

% В этом документе преамбула

\usepackage{siunitx}
%%% Работа с русским языком
%\usepackage{cmap}					% поиск в PDF
%\usepackage{mathtext} 				% русские буквы в формулах
%\usepackage[T2A]{fontenc}			% кодировка
%\usepackage[utf8]{inputenc}			% кодировка исходного текста
%\usepackage[english,russian]{babel}	% локализация и переносы
%\usepackage{indentfirst}
%\frenchspacing
%
%\renewcommand{\epsilon}{\ensuremath{\varepsilon}}
%\newcommand{\phibackup}{\ensuremath{\phi}}
%\renewcommand{\phi}{\ensuremath{\varphi}}
%\renewcommand{\varphi}{\ensuremath{\phibackup}}
%\renewcommand{\kappa}{\ensuremath{\varkappa}}
%\renewcommand{\le}{\ensuremath{\leqslant}}
%\renewcommand{\leq}{\ensuremath{\leqslant}}
%\renewcommand{\ge}{\ensuremath{\geqslant}}
%\renewcommand{\geq}{\ensuremath{\geqslant}}
%\renewcommand{\emptyset}{\varnothing}
%\renewcommand{\Im}{\operatorname{Im}}
%\renewcommand{\Re}{\operatorname{Re}}


%%% Дополнительная работа с математикой
\usepackage{amsmath,amsfonts,amssymb,amsthm,mathtools} % AMS
%\usepackage{icomma} % "Умная" запятая: $0,2$ --- число, $0, 2$ --- перечисление

%% Номера формул
%\mathtoolsset{showonlyrefs=true} % Показывать номера только у тех формул, на которые есть \eqref{} в тексте.
%\usepackage{leqno} % Нумереация формул слева

%% Свои команды
\DeclareMathOperator{\sgn}{\mathop{sgn}}
\DeclareMathOperator{\sign}{\mathop{sign}}
\DeclareMathOperator*{\res}{\mathop{res}}
\DeclareMathOperator*{\tr}{\mathop{tr}}
\DeclareMathOperator*{\rot}{\mathop{rot}}
\DeclareMathOperator*{\divop}{\mathop{div}}
\DeclareMathOperator*{\grad}{\mathop{grad}}

%% Перенос знаков в формулах (по Львовскому)
\newcommand*{\hm}[1]{#1\nobreak\discretionary{}
{\hbox{$\mathsurround=0pt #1$}}{}}

%%% Работа с картинками
\usepackage{graphicx}  % Для вставки рисунков
\graphicspath{{figures/}}  % папки с картинками
\setlength\fboxsep{3pt} % Отступ рамки \fbox{} от рисунка
\setlength\fboxrule{1pt} % Толщина линий рамки \fbox{}
\usepackage{wrapfig} % Обтекание рисунков текстом

%%% Работа с таблицами
\usepackage{array,tabularx,tabulary,booktabs} % Дополнительная работа с таблицами
\usepackage{longtable}  % Длинные таблицы
\usepackage{multirow} % Слияние строк в таблице

%%% Теоремы
\theoremstyle{plain} % Это стиль по умолчанию, его можно не переопределять.
\newtheorem{thm}{Теорема}
\newtheorem*{thm*}{Теорема}
\newtheorem{prop}{Предложение}
\newtheorem*{prop*}{Предложение}
 
\theoremstyle{definition} % "Определение"
%\newtheorem{corollary}{Следствие}[theorem]
\newtheorem{dfn}{Определение}
\newtheorem*{dfn*}{Определение}
\newtheorem{prob}{Задача}
\newtheorem*{prob*}{Задача}

 
\theoremstyle{remark} % "Примечание"
\newtheorem*{sol}{Решение}
\newtheorem*{rem}{Замечание}

%%% Программирование
\usepackage{etoolbox} % логические операторы

%%% Страница
%\usepackage{extsizes} % Возможность сделать 14-й шрифт
%\usepackage{geometry} % Простой способ задавать поля
%	\geometry{top=25mm}
%	\geometry{bottom=35mm}
%	\geometry{left=35mm}
%	\geometry{right=20mm}
 
\usepackage{fancyhdr} % Колонтитулы
%	\pagestyle{fancy}
 %	\renewcommand{\headrulewidth}{0pt}  % Толщина линейки, отчеркивающей верхний колонтитул
	%\lfoot{Нижний левый}
	%\rfoot{Нижний правый}
	%\rhead{Верхний правый}
	%\chead{Верхний в центре}
	%\lhead{Верхний левый}
	%\cfoot{Нижний в центре} % По умолчанию здесь номер страницы

\usepackage{setspace} % Интерлиньяж
%\onehalfspacing % Интерлиньяж 1.5
%\doublespacing % Интерлиньяж 2
%\singlespacing % Интерлиньяж 1

\usepackage{lastpage} % Узнать, сколько всего страниц в документе.

\usepackage{soul} % Модификаторы начертания

\usepackage{hyperref}
\usepackage[usenames,dvipsnames,svgnames,table,rgb]{xcolor}
\hypersetup{				% Гиперссылки
    unicode=true,           % русские буквы в раздела PDF
    pdftitle={Заголовок},   % Заголовок
    pdfauthor={Автор},      % Автор
    pdfsubject={Тема},      % Тема
    pdfcreator={Создатель}, % Создатель
    pdfproducer={Производитель}, % Производитель
    pdfkeywords={keyword1} {key2} {key3}, % Ключевые слова
%    colorlinks=true,       	% false: ссылки в рамках; true: цветные ссылки
    %linkcolor=red,          % внутренние ссылки
    %citecolor=black,        % на библиографию
    %filecolor=magenta,      % на файлы
    %urlcolor=cyan           % на URL
}

\usepackage{csquotes} % Еще инструменты для ссылок

%\usepackage[style=apa,maxcitenames=2,backend=biber,sorting=nty]{biblatex}

\usepackage{multicol} % Несколько колонок

\usepackage{tikz} % Работа с графикой
\usepackage{pgfplots}
\usepackage{pgfplotstable}
%\usepackage{coloremoji}
\usepackage{floatrow}
\usepackage{subcaption}
\graphicspath{{figures/}}

\renewcommand\thesubfigure{\asbuk{subfigure}}
%\addbibresource{master.bib}

\usepackage{import}
\usepackage{pdfpages}
\usepackage{transparent}
\usepackage{xcolor}
\usepackage{xifthen}

\newcommand{\incfig}[2][1]{%
    \def\svgwidth{#1\columnwidth}
    \import{./figures/}{#2.pdf_tex}
}
%\usepackage{titlesec}
%\titleformat{\section}{\normalfont\Large\bfseries}{}{0pt}{}
%----------------------STANDART:
%\titleformat{\chapter}[display]
%  {\normalfont\huge\bfseries}{\chaptertitlename\ \thechapter}{20pt}{\Huge}
%\titleformat{\section}{\normalfont\Large\bfseries}{\thesection}{1em}{}
%\titleformat{\subsection}
%  {\normalfont\large\bfseries}{\thesubsection}{1em}{}
%\titleformat{\subsubsection}
%  {\normalfont\normalsize\bfseries}{\thesubsubsection}{1em}{}
%\titleformat{\paragraph}[runin]
%  {\normalfont\normalsize\bfseries}{\theparagraph}{1em}{}
%\titleformat{\subparagraph}[runin]
%  {\normalfont\normalsize\bfseries}{\thesubparagraph}{1em}{}

\pdfsuppresswarningpagegroup=1
\pgfplotsset{compat=1.16}



%\setcounter{tocdepth}{1} % only parts,chapters,sections
%\titleformat{\subsection}{\normalfont\large\bfseries}{}{0em}{}
%\titleformat{\subsubsection}{\normalfont\normalsize\bfseries}{}{0em}{}

%\newcommand{\textover}[2]{\stackrel{\mathclap{\normalfont\mbox{#2}}}{#1}}

\author{Yaroslav Drachov\\
Moscow Institute of Physics and Technology}
%\author{Драчов Ярослав\\
%Факультет общей и прикладной физики МФТИ}
\newcommand{\veq}{\mathrel{\rotatebox{90}{$=$}}}
%\newcommand{\teto}[1]{\stackrel{\mathclap{\normalfont\tiny\mbox{#1}}}{\to}}
%\renewcommand{\thesubsection}{\arabic{subsection}}

%%\setcounter{secnumdepth}{0}

\definecolor{tabblue}{RGB}{30, 119, 180}
\definecolor{taborange}{RGB}{255, 127, 15}
\definecolor{tabgreen}{RGB}{45, 160, 43}
\definecolor{tabred}{RGB}{214, 38, 40}
\definecolor{tabpurple}{RGB}{148, 103, 189}
\definecolor{tabbrown}{RGB}{140, 86, 76}
\definecolor{tabpink}{RGB}{227, 119, 193}
\definecolor{tabgray}{RGB}{127, 127, 127}
\definecolor{tabolive}{RGB}{188, 189, 33}
\definecolor{tabcyan}{RGB}{22, 190, 207}
\pgfplotscreateplotcyclelist{colorbrewer-tab}{
{tabblue},
{taborange},
{tabgreen},
{tabred},
{tabpurple},
{tabbrown},
{tabpink},
{tabgray},
{tabolive},
{tabcyan},
}
\usepackage{csvsimple}
\usepackage{extarrows}
%\renewcommand{\labelenumii}{\asbuk{enumii})}
%\renewcommand{\labelenumiv}{\Asbuk{enumiv}}
%\newcommand{\prob}[1]{\subsubsection*{#1}}
\sisetup{output-decimal-marker = {,},separate-uncertainty = true,exponent-product = \cdot}

\usepackage{braket}
\usepackage{enumerate}
\usepackage{chngcntr}
%\counterwithin*{equation}{problem}
%\usepackage{bbold}

\newtheoremstyle{hiProb}% ⟨name ⟩ 
{3pt}% ⟨Space above ⟩1 
{3pt}% ⟨Space below ⟩1
{}% ⟨Body font ⟩
{}% ⟨Indent amount ⟩2
{\bfseries}% ⟨Theorem head font⟩
{.}% ⟨Punctuation after theorem head ⟩
{.5em}% ⟨Space after theorem head ⟩3
%{\thmname{#1} \thmnote{#3}}% ⟨Theorem head spec (can be left empty, meaning ‘normal’)⟩
{\thmnote{#3}}% ⟨Theorem head spec (can be left empty, meaning ‘normal’)⟩
\theoremstyle{hiProb} % "Определение"
%\newtheorem{hiProb}{Задача}
\newtheorem{hiProb}{}
%\usepackage{mmacells}
\newcommand{\textover}[2]{\stackrel{\mathclap{\normalfont\scriptsize\mbox{#2}}}{#1}}
\usepackage{units}
\usepackage[math]{cellspace}%
\setlength\cellspacetoplimit{2pt}
\setlength\cellspacebottomlimit{2pt}

\DeclareMathAlphabet{\mathbbold}{U}{bbold}{m}{n}

\newcommand{\normord}[1]{:\mathrel{#1}:}

\title{Задание 2}
\begin{document}
	\maketitle
	\section*{Тема XII. Общая теория сходимости для уравнений в частных производных}
\begin{hiProb}[7.1]
\end{hiProb}
\begin{sol}
\begin{multline*}
-i \frac{\Psi_m^{n+1}-\Psi_m^n}{\tau}=
\xi \left( \frac{\Psi_{m+1}^{n-1}-2 \Psi_m^{n+1}+\Psi_{m-1}^{n+1}}{h^2}+ \tilde{U} \Psi^n \right) 
+\\+(1-\xi) \left( \frac{\Psi_{m+1}^{n}-2 \Psi_m^{n}+\Psi_{m-1}^{n}}{h^2}+ \tilde{U} \Psi^n \right) 
.\end{multline*} 
Сначала рассмотрим схему с $\tilde{U}=0$:
\[
-i \frac{\Psi_m^{n+1}-\Psi_m^n}{\tau}=
\xi \frac{\Psi_{m+1}^{n+1}-2 \Psi_m^{n+1}+\Psi_{m-1}^{n+1}}{h^2}+(1-\xi) \frac{\Psi_{m+1}^n -2 \Psi_m^n +\Psi_{m-1}^n}{h^2}
.\] 
Пусть $\Psi_m^n=\Psi$. Разложим в окрестности
точки $(x_m,\,t_n)$:
\begin{multline*}
	-\frac{i}{\tau} \left( \Psi_t \tau
	+\Psi_{tt} \frac{\tau^2}{2}+O\left(\tau^3\right)\right) =\\=
	\frac{\xi}{h^2} \left( 
	\Psi_t \tau +\Psi_x h + \Psi_{t t}\frac{\tau^2}{2}+\Psi_{xx} \frac{h^2}{2}+
2 \Psi_{tx} \frac{\tau h}{2}+ \Psi_{x x x}
\frac{h^3}{6} \right. + \\ + \left. \Psi_{x x t}\frac{h^2 \tau}{6}\cdot 3
+ \Psi_{x t t} \frac{h \tau^2}{6}\cdot 3+
\Psi _{t  t t} \frac{\tau^3}{6}-
2 \left( \Psi_t \tau + \Psi_{tt} \frac{\tau^2}{2}+
\Psi_{t t t} \frac{\tau^3}{6}\right)
+\Psi_t \tau \right. - \\ - \left. \Psi_x h + \Psi_{t t} \frac{\tau^2}{2}+ \Psi_{ x x} \frac{h^2}{2}- \Psi_{x t}
\frac{h \tau}{2} \cdot 2 - \Psi_{x x x} \frac{h^2}{2}
+ \Psi_{x x x} \frac{h^3}{6}\right. - \\ - \left. \Psi_x h +
\Psi_{x x} \frac{h^2}{2} -\Psi_{x x x}
\frac{h^3}{6}\right) 
.\end{multline*} 
Таким образом:
\[
	-i \Psi_t + O(\tau)= \Psi_{x x}
	+O(h^2)
.\] 
Если $\Psi$ --- решение уравнения, то 
\[
- i \frac{\partial \Psi}{\partial t}=
\frac{\partial^2 \Psi }{\partial x^2} 
.\] 
Следовательно  схема аппроксимирует исходную задачу.
Устойчивость проверим по спектральному признаку:
\[
\Psi_m^n = \lambda^n e^{i \phi m}
.\] 
\[
-i \frac{\lambda-1}{\tau}=
\frac{\xi}{h^2} \left( 
\lambda e^{i\phi}-2 \lambda+ \lambda e^{i\phi}\right) + \frac{1-\xi}{h^2}
\left( e^{i \phi}-2 -e^{-i\phi} \right) 
.\] 
\begin{multline*}
	-i \frac{\lambda-1}{\tau}= \frac{\xi \lambda \cdot 2}{h^2} \left( \cos \phi-1 \right) 
	+\frac{1-\xi}{h^2}\cdot 2 (\cos \phi-1)=\\=
	\frac{\xi (\lambda-1)}{h^2}\cdot 2(\cos \phi
	-1) + \frac{2}{h^2} (\cos \phi-1)
.\end{multline*} 
\[
	(\lambda-1) \left( 
	\frac{2 \xi (1- \cos \phi)}{h^2}-
\frac{i}{\tau}\right) =
\frac{2}{h^2} (\cos \phi -1)
.\] 
Значит
\[
	\lambda= -\frac{2(1-\cos \phi)}{h^2\left( 
	\frac{1}{h^2}\cdot 2 \xi (1-\cos  \phi) - \frac{i}{\tau}\right) }+1
.\] 
\[
	\lambda= \frac{2 (\xi-1) (1- \cos \phi)-
	\frac{ih^2}{\tau}}{2\xi(1-\cos \phi)-
\frac{i}{\tau}h^2}=
\frac{1- i \frac{\tau}{h^2}\cdot 4(1-\xi) \sin ^2 \frac{\phi}{2}}{1+ i \frac{\tau}{h^2}\cdot 4
\xi \sin ^2 \frac{\phi}{2}}
.\] 
Числитель и знаменатель последней дроби имеют равные
действительные части, поэтому $|\lambda|\le 1$, если
абсолютное значение мнимой части числителя меньше
чем знаменателя, т.\:е. $1-\xi\le \xi$, следовательно
$\xi \ge 0,5$.

При $\tilde{U}\neq 0$:
\[
-i \frac{\lambda-1}{\tau}= \frac{\xi}{h^2}
\left( \lambda e^{i \phi}-2 \lambda+\lambda
e^{-i\phi}\right) + \frac{1-\xi}{h^2}
\left( e^{i\phi}-2 + e^{-i\phi} \right) +
\xi \lambda \tilde{U}+ (1-\xi) \tilde{U}
.\] 
\[
	-i \frac{\lambda-1}{\tau}= \frac{\xi (\lambda-1)}{h^2} \cdot 2 \left( 
	\cos \phi-1\right) +
	\frac{2}{h^2} \left( \cos \phi-1 \right) 
	+\xi (\lambda-1) \tilde{U}+\tilde{U}
.\] 
\[
	(\lambda-1) \left( 
	-\frac{i}{\tau}+ \frac{2\xi}{h^2}
(1-\cos \phi)- \tilde{U} \xi\right) =
\frac{2}{h^2}(\cos \phi-1) +\tilde{U}
.\] 
\[
	\lambda= 1+ \frac{-2 (-\cos \phi+1)+
	\tilde{U} h^2}{- \frac{i}{\tau}h^2+
2 \xi(1-\cos \phi)-\tilde{U} \xi h^2}
.\]
\begin{multline*}
\lambda= \frac{-\frac{i}{\tau}h^2+
2 (\xi-1) (1-\cos \phi) + \tilde{U}h^2(1-\xi)}{
-\frac{i}{\tau}h^2 +2 \xi (1-\cos \phi)-
\tilde{U}\xi h^2}=\\=
\frac{1- \frac{\tau i}{h^2}\cdot 4(1-\xi) \sin ^2
\frac{\phi}{2}+ \frac{i \tau}{h^2}\tilde{U}h^2 (1-\xi)}{1+ i \frac{\tau}{h^2}\cdot 4 \xi \sin ^2 \frac{\phi}{2}+ i \tau  \tilde{U} \xi}
.\end{multline*} 
Видим, что  введение потенциала меняет
только мнимую часть  в $\lambda$,  но
синхронно в числителе и знаменателе, следовательно
не меняет  полученный  для случая $\tilde{U}=0$ 
результат, поэтому  схема устойчива при 
$\xi\ge 0,5$ и сходимость имеет место  при $\xi\ge 0,5$.
\end{sol}
\begin{hiProb}[7.2]
\end{hiProb}
\begin{sol}
\[
u_t+  c u_x=0
.\] 
\[
\alpha y_{m}^{n+1}+ \beta y_{m-1}^n+
\gamma y_m^n+\delta y_{m+1}^n=0
.\] 
\begin{multline*}
	\alpha\left(u+u_t \tau + \frac{1}{2} u_{t t}
	\tau^2 \right)+
	\beta\left( 
	u- u_x h + \frac{1}{2} u_{ x x}h^2\right) +
	\gamma u+\\+ \delta \left( u + u_x h+ u_{x x}\frac{1}{2} h^2 \right) + O\left( h^2 \right) +
	O\left( \tau^2 \right)=0
.\end{multline*} 
\[
\alpha+\beta+\gamma+\delta=0 \implies
\gamma=-\alpha -\beta-\delta
.\] 
\[
	\alpha u_t \tau +(\delta-\beta)u_x
	h=0\implies
	\alpha \tau =1\implies \alpha= \frac{1}{\tau},\ (\delta-\beta)h=c
.\] 
\[
\left\{
\begin{aligned}
u_{tt}+ c u_{x t}&= 0 \\
u_{t x}+ c u_{xx}&= 0 \\
\end{aligned}
\right.\implies
u_{t t}- c^2 u_{x x}=0,\ u_{t t}= c^2 u_{x x}
.\] 
\[
\alpha c^2 \tau^2+  \beta h^2+ \delta h^0
.\] 
\[
	c^2\tau+ (\beta+\delta)h^2=0
.\] 
\[
\delta-\beta= \frac{c}{h} \implies
\delta= \beta + \frac{c}{h}\implies
c^2 \tau + \left( 2\beta + \frac{c}{h} \right) h^2=0
.\] 
\[
c^2 \tau+ 2\beta h^2 +ch =0\implies
\beta=  -  \frac{c(c\tau+h)}{2h^2}= \frac{-c^2\tau
-ch }{2h^2}
.\] 
\[
\delta= \frac{c h -c^2  \tau}{2h^2},\quad
\gamma= \frac{c^2 \tau}{h^2}-\frac{1}{\tau}
.\] 
Значит
\[
\frac{y_{m+1}^{n+1}- y_{m+1}^n}{\tau}+
\frac{c}{2h} \left( y_{m+1}^n-y_{m-1}^n \right) -
\frac{c^2 \tau}{2h^2}\left( 
y_{m+1}^n-2 y_m^n+y_{m-1}^n\right) =0
.\] 
Порядок аппроксимации:  $O(\tau^2,\,h^2)$.
Исследуем  устойчивость:
\[
u_{m}^n= \lambda^n  e^{im \phi}
.\] 
\[
\frac{\lambda-1}{\tau}+
\frac{c}{2h} \left( e^{i\phi}-e^{-i\phi} \right) 
- \frac{c^2 \tau}{2h^2}\left( e^{i\phi}-2+
e^{-i\phi}\right) =0
.\] 
\[
\frac{\lambda-1}{\tau}+\frac{i c}{h}\sin \phi
+\frac{c^2 \tau}{h^2} (1-\cos \phi)=0
.\] 
\[
	\lambda=1- \frac{ic}{h}\sin \phi +\frac{c^2 \tau}{h^2}(1-\cos \phi)=0
.\] 
\[
\lambda=1- \frac{ic\tau}{h}  \sin \phi-
\frac{c^2  \tau^2}{h^2}\left( 1- \cos \phi \right) 
=0
.\] 
\[
\sigma= \frac{c  \tau}{h}
.\] 
\[
	\lambda=1-i\sigma \sin \phi-\sigma^2 (1-\cos \phi)
.\] 
$
|\lambda|^2\le 1 
$
для устойчивости, поэтому
\[
	|\lambda|^2= (1- \sigma^2(1-  \cos \phi))^2+
	\sigma^2 \sin ^2 \phi\le 1
.\] 
\[
	\sigma^2 \sin ^2\phi \le 1- \left( 
	1-\sigma^2 \left( 1- \cos \phi \right) \right) ^2= \sigma^2\left( 1-\cos \phi \right) 
	\left( 2- \sigma^2 \left( 1- \cos \phi \right)  \right) 
.\] 
\[
\frac{\sin ^2 \phi}{1- \cos \phi}\le 
2- \sigma^2 (1- \cos \phi)
.\] 
\[
\sigma^2 \le  \frac{1}{1-\cos  \phi}
\left( 2- \frac{\sin ^2 \phi}{1- \cos \phi} \right) 
=\frac{2 -2\cos \phi- \sin ^2 \phi}{2 \sin ^2 \left( \phi /2 \right) \left( 1- \cos  \phi \right) }=1
.\] 
Значит схема устойчива при 
\[
\frac{c\tau}{h}\le 1
.\] 
\end{sol}
\begin{hiProb}[7.3]
\end{hiProb}
\begin{sol}
\[
u_t+ c u_x=0
.\] 
\[
\alpha y_m^{n+1}+\beta y_m^n+\gamma
y_{m-1}^n+ \delta y_{m-2}^n=0
.\] 
\begin{multline*}
	\alpha\left( u + \tau u_t + \frac{1}{2}
	\tau^2 u_{t t}+ \frac{1}{6}\tau^3 u_{t t t}
\right)+ \beta u+ \gamma \left( 
u-u_x h+ \frac{1}{2} h^2 u_{x x} - \frac{1}{6} h^3 u_{x x x }\right) +\\+
\delta \left( u-2 h u_x +2 h^2 u_{x x}-
\frac{4}{3} h^3 u_{x x x}\right) =0
.\end{multline*} 
\[
\alpha+\beta+\gamma+\delta=0\implies\beta=
\alpha-\gamma-\delta
.\] 
\[
	\alpha c \tau + \left( \gamma+ 2\delta \right) h=0\implies \alpha= \frac{1}{\tau}
.\] 
\[
\gamma+2\delta= - \frac{c}{h}\implies \gamma=
-\frac{c}{h}-2\delta
.\] 
\[
	\alpha c^2 \tau^2 + (\gamma+4\delta)h^2=0
.\] 
\[
	c^2 \tau +\left( 
	-\frac{c}{h}-2\delta +4\delta\right) h^2=0
.\] 
\[
-\frac{c^2 \tau}{h^2}= -\frac{c}{h}+2\delta\implies
\delta= \frac{c}{2h}- \frac{c^2 \tau}{2 h^2}
.\] 
\[
\gamma= -\frac{c}{h}- \frac{c}{h}+
\frac{c^2 \tau}{h^2}= -\frac{2c}{h}+\frac{c^2\tau}{
h^2}
.\] 
\[
\beta= -\frac{1}{\tau}+ \frac{2c}{h}- \frac{c^2 \tau}{h^2}- \frac{c}{2h}+ \frac{c^2 \tau}{2h^2}
=-\frac{1}{\tau}+\frac{3c}{2h}- \frac{c^2 \tau}{2h^2}
.\] 
Откуда разностная схема:
\[
\frac{y_m^{n+1}-y_m^n}{\tau}+
\frac{c}{h} \left( \frac{3}{2} y_m^n-2 y_{m-1}^n+
\frac{1}{2}y_{m-2}^n\right) -
\frac{c^2 \tau}{2h^2} \left( y_m^n-2y_{m-1}^n
+ y_{m-2}^n\right) =0
.\] 
Порядок аппроксимации: $O\left(\tau^2+h^2\right)$.

Исследуем на устойчивость:  $u_m^n= \lambda^n e^{im\phi}$.
\[
	\frac{\lambda-1}{\tau}e^{i\phi}+\frac{c}{h}\left( 
	\frac{1}{2}e^{i\phi}- \frac{1}{2}e^{- i\phi}+e^{i\phi} +e^{-i\phi}-2\right) -
	\frac{c^2 \tau}{2h^2} \left( 
	e^{i\phi}-2+e^{-i\phi}\right) =0
.\] 
\[
\frac{\lambda-1}{\tau}e^{i\phi}+\frac{c}{h}
\left( 2\left( \cos \phi-1 \right) +i
\sin \phi\right) + \frac{c^2 \tau}{h^2}
(1- \cos \phi)=0
.\] 
\[
	\lambda= 1 -e ^{-i\phi} \left( \sigma^2(1-\cos  \phi)+ \sigma \left( 2 \left( \cos \phi-1 \right) +i \sin \phi \right)  \right) 
.\] 
Оценка этого выражения в Mathematica даёт, что
 $|\lambda|\le 1$ для любого $\phi$ при $\sigma\le 2$,
значит схема устойчива при $c\tau \le 2 h$.
\end{sol}
\begin{hiProb}[7.4]
\end{hiProb}
\begin{sol}
\[
u_t+c u_x=0
.\] 
Характеристика:
\[
	x- c t= \mathrm{const}
.\] 
Значит
\[
	u_* = u_{m}^{n+1}
.\] 

Получим значение решения в точке $x^*$ интерполяцией
по значениям в трёх узлах сетки на нижнем слое.
\begin{multline*}
	P_2 (x_*)= y_{m-2} - \frac{y_{m_1}-y_{m-2}}{h}
	(x_*-x_{m-2})+\\+
	\frac{y_m-2y_{m-1}+ y_{m-2}}{2h^2}(x_*-x_{m-2})
	(x_*-x_{m-1})=\\= y_{m-2}+ \frac{1}{h}
	\left( y_{m-1}-y_{m-2} \right) (2h-c\tau)
	+\\+ \frac{1}{2h^2}\left( 
	y_m- 2y_{m-1}+y_{m-2}\right) \left( 2h-c\tau \right) 
	(h-c\tau)=\\=
	y_{m-2}+ \left( 2-\frac{c\tau}{h} \right) \left( 
	y_{m-1}-y_{m-2}\right) +\\+
	\left( 1-\frac{c\tau}{2h} \right) \left( 
	1- \frac{c\tau}{h}\right) \left( 
y_m-2 y_{m-1}+y_{m-2}\right) =\\=
y_{m-2}+2y_{m-1}-2y_{m-2}+y_m-2y_{m-1}+y_{m-2}+\\
+\frac{c\tau}{h}\left( -y_{m-1}+ y_{m-2}-\frac{1}{2}
y_m+y_{m-1}-\frac{1}{2}y_{m-2}-y_m+2y_{m-1}-y_{m-2}\right) +\\+
\frac{c^2\tau^2}{h^2}\left( y_m-2y_{m-1}+2y_m \right) 
.\end{multline*} 
\[
	P_2(x_*)=y_m^{n+1}
.\] 
Значит
\[
\frac{y_m^{n+1}-y_m^n}{\tau}+
\frac{c}{h} \left( \frac{3}{2} y_m^n-2y_{m-1}^n+\frac{1}{2}
y_{m-2}^n\right) - \frac{c^2\tau^2}{h^2}(y_m^n-2y_{m-1}^n+
y_{m-2}^n)=0
.\] 
Результат совпадает с предыдущей задачей.
\end{sol}
\begin{hiProb}[7.8]
\end{hiProb}
\begin{sol}
\[
	\text{Ш}= \left\{ \left( x_{m-1},\,y_{k-1} \right),\,
		\left( x_{m-1},\,y_{k+1} \right) ,\,
		\left( x_{m+1},\,y_{k-1} \right) ,\,
		\left( x_{m+1},\,y_{k+1},\,
		\left( x_{m},\,y_k \right) \right) 
	\right\} 
.\] 
\[
\Delta u= u_{x x}+u_{y y}=0
.\] 
\begin{figure}[ht]
    \centering
    \incfig{1}
    \caption{К задаче 7.8}
    \label{fig:1}
\end{figure}
\[
aY_{m+1}^{k+1}+b Y_{m+1}^{k-1}+c Y_{m-1}^{k+1}+d Y_{m-1}^{k-1}
-l Y_{m}^k=0
.\] 
\begin{multline*}
	a \left( u +h u_x +h u_y+ \frac{1}{2} h^2 u_{x x}
	+h^2 u_{xy}+\frac{1}{2} h^2 u_{y y}\right) +\\+
	b \left( u+ h u_x -h u_y + \frac{1}{2}h^2 u_{x x}
	-h^2 u_{xy}+ \frac{1}{2} h^2 u_{y y}\right) +\\+
	c\left( u-h u_x +h u_y + \frac{1}{2}h^2 u_{x x}
	-h^2 u_{x y} + \frac{1}{2} h^2 u_{y y}\right) +\\+
	d\left( u- h u_x - h u_y + \frac{1}{2}
	h^2 u_{x x} + h^2 u_{x y}+ \frac{1}{2}h^2 u_{y y}\right) 
	+lu=0
.\end{multline*} 
\[
\left\{
\begin{aligned}
a+b+c+d+l&= 0 \\
a+b-c-d&= 0 \\
a-b+c-d&= 0 \\
a-b-c+d&= 0 \\
\end{aligned}
\right.\implies
\left\{
\begin{aligned}
a=b&=c= d \\
l&= -4a \\
\end{aligned}
\right.
\] 
\[
	\frac{h^2}{2} \left( a+b+c+d \right) \left( 
	u_{x x}+u_{y y}\right) =0;\quad
	\frac{h^2}{2} \left( a+b+c+d \right) =1
.\] 
\[
a=b=c=d=\frac{1}{2h^2};\quad l=-\frac{2}{h^2}
.\] 
Значит разностная схема:
\[
\frac{Y_{m+1}^{k+1}-2 Y_{m}^k+Y_{m-1}^{k-1}}{2h^2}
+ \frac{Y_{m-1}^{k+1}-2Y_{m}^k+Y_{m+1}^{k-1}}{2h^2}=0
.\] 
\end{sol}
\section*{Тема XIII. Численные методы для уравнений в
частных производных параболического типа}
\begin{hiProb}[7.3]
\end{hiProb}
\begin{sol}
\[
\frac{y_m^{n+1}-y_m^n}{\tau}=D \frac{y_{m-1}^n-2y_m^{n+1}+
y_{m+1}^n}{h^2}
\tag{*}
\label{eq:1}
.\] 
\[
	\frac{y_m^{n+1}-y_m^n}{\tau}= D \frac{y_{m-1}^n-2y_m^n
	+y_{m+1}^n}{h^2}- \frac{2D}{h^2}\left( 
y_m^{n+1}-y_m^n\right) 
.\] 
\[
	\left( y_m^{n+1}-y_m^n \right) \left( 
	\frac{1}{\tau}+ \frac{2D}{h^2}\right) =D \frac{
y_{m-1}^n-2y_m^n+y_{m+1}^n}{h^2}
.\] 
Значит
\[
\frac{y_{m}^{n+1}-y_m^n}{\tau^*}=
D \frac{y_{m-1}^n-2y_m^n+y_{m+1}^n}{h^2}
.\] 
\[
	\tau^*= \frac{\tau}{1+ 2D\tau /h^2}<\tau
.\] 
Следовательно \eqref{eq:1} эквивалентна явной схеме
с уменьшенным значением шага по времени.
\end{sol}
\begin{hiProb}[8.3]
\end{hiProb}
\begin{sol}
Ознакомился.
\end{sol}
\begin{hiProb}[9.3]
\end{hiProb}
\begin{sol}
\[
\frac{y_m^{n+1}-y_m^n}{\tau}= \frac{y_{m+1}^n-2 y_m^n+y_{m+1}^n
}{h^2}
.\] 
Разложим в окрестности $\left( t_n,\,x_m \right) $:
\[
u_m^n \equiv u
.\] 
\[
u_{m}^{n+1}=u + \tau u_t + \frac{\tau^2}{2}u_{t t}
+ \frac{\tau^3}{6} u_{ t t t}+ O\left( \tau^4 \right) 
.\] 
\[
u_{m\pm 1}^n= u \pm h u_{x} + \frac{h^2}{2}u_{x x}\pm 
\frac{h^3}{6} u_{x x x}+ \frac{h^4}{24} u_{x x x x}\pm 
\frac{h^5}{120}u_{x x x x x} + O\left( x^6 \right) 
.\] 
Значит
\[
r_{\tau h}= u_t + \frac{\tau}{2} u_{t t}+
\frac{\tau^2}{6}u_{t t t }-u_{x x}-\frac{h^2}{12}u_{x x x x}
+ O\left( \tau^2 h^4 \right) 
.\] 
Из исходного уравнения: $u_t=u_{x x}$. Следовательно
\[
\left\{
\begin{aligned}
	u_{tt}&= u_{x x t} \\
	u_{t x}&= u _{x x x} \\
	u_{t x x}&= u_{x x x x}
\end{aligned}
\right.
\implies u_{t t}=u_{u_{x x x x}}
.\] 
Поэтому
\[
	r_{\tau h}= \left( \frac{\tau}{2}u_{t t}-
	\frac{h^2}{12}\underbrace{u_{x x x x}}_{=u_{t t}}\right) +O\left( \tau^2+h^4 \right) 
.\] 
Как итог схема будет иметь порядок аппроксимации $O\left( \tau^2,\,h^4 \right) $ при условии:
\[
\frac{\tau}{2}u_{tt}=\frac{h^2}{12}u_{x x x x}\implies
\frac{h^2}{\tau}= 6
.\] 
\end{sol}
\begin{hiProb}[9.1]
\end{hiProb}
\begin{sol}
\[
	\frac{\hat{y}-y}{\tau}=
	a \xi \Lambda_{x x} \hat{y}+ a (1-\xi) \Lambda
	_{x x}y + \xi Q^{n+1}+ (1-\xi) Q^n
.\] 
\renewcommand{\labelenumi}{\asbuk{enumi})}
\begin{enumerate}
	\item \emph{Метод энергетических неравенств}.

	Представим схему в виде:
	\[
		\underbrace{(E-a \xi \tau \Lambda_{x x})}_{B}
		\frac{\hat{y}-y}{\tau} \underbrace{-a \Lambda
		_{x x}}_{A} y=0
	.\] 
	Условие устойчивости:
	\[
	B\ge \frac{\tau}{2} A
	.\] 
	\[
	E-a\xi \tau \Lambda_{x x}\ge  \frac{\tau}{2}
	\left( -a \Lambda_{x x} \right) 
	.\] 
	\[
		E\ge  -\tau \Lambda_{x x}a \left( \frac{1}{2}-
		\xi\right) 
	.\] 
	\[
		1\ge -\tau a \lambda^{(k)}\left( \frac{1}{2}-
		\xi\right) 
	,\] 
	где $\lambda^{(k)}$ --- спектр $\Lambda_{x x}$.

	Известно, что для разностного оператора второй
	производной верно:
	\[
	-\Lambda_{x x} \le \frac{4}{n^2} E \implies
	\Lambda_{x x} \ge - \frac{4}{n^2} E
	.\] 
	Следовательно
	\[
		1\ge  \tau \cdot \frac{4}{h^2} \left( \frac{1}{2}-\xi \right) a \implies
		\xi \ge  \frac{1}{2} - \frac{1}{4\sigma}
	,\]
	где $\sigma= \tau a / h^2$.
\item \emph{Спектральный метод}.

	Подставим $y=\lambda^n e^{im \phi}$:
	\[
	\frac{\lambda-1}{\tau}= a\xi \lambda
	 \frac{e^{i\phi}+ e^{-i\phi}-2}{h^2}+
	 a(1-\xi) \frac{e^{i\phi}+e^{-i\phi}-2}{h^2}
	.\] 
	Значит
	\begin{multline*}
		\lambda= 1+ \frac{1}{-\xi +\left( 
		-2\sigma +2\sigma \cos  \phi\right) ^{-1}}=
		1- \frac{1}{\xi+ \left( 4\left( \sin ^2 \frac{\phi}{2} \right) \sigma \right) ^{-1}}=\\=
		1- \frac{4\sigma \sin ^2 \frac{\phi}{2}}{
		4\xi \sigma \sin ^2 \frac{\phi}{2}+1}=
		\frac{1- (1-\xi)\cdot 4 \sigma
		\sin ^2 \frac{\phi}{2}}{1+ 4\xi \sigma
	\sin ^2 \frac{\phi}{2}}
	.\end{multline*} 
	Для устойчивости должно выполняться: $|\lambda| \le 1$.
	Тогда
	\[
	0 \le  \frac{4\sigma \sin^2 \frac{\phi}{2} }{4\xi
	\sigma \sin ^2 \frac{\phi}{2}+1}\le 2
	.\] 
	\[
	2\sigma \sin ^2 \frac{\phi}{2}\le 
	4 \sigma \xi \sin ^2 \frac{\phi}{2}+1\implies
	2\sigma \le 4\sigma \xi +1
	.\] Откуда
	\[
	\xi \ge \frac{1}{2}-\frac{1}{4\sigma};\quad
	\sigma= \frac{\tau a}{h^2}
	.\] 
	\end{enumerate}
	Результаты совпадают.
\end{sol}
\begin{hiProb}[9.2]
\end{hiProb}
\begin{sol}
Однородное уравнение теплопроводности:
\[
u_t=a u_{x x}
.\] 
Разностная схема:
\[
\frac{y_m^{n+1}-y_m^n}{\tau}=a\xi
\frac{y_{m+1}^{n+1}-2y_m^{n+1}+y_{m-1}^{n+1}}{h^2}+
a\left( 1-\xi \right) \frac{y_{m+1}^n-2y_m^n+
y_{m-1}^n}{h^2}
.\] 
Разложим в окрестности точки $\left( t_n,\,x_m \right) $:
\[
	u_{m}^n\equiv u
.\] 
\[
u_{m}^{n+1}=u+\tau u_t + \frac{\tau^2}{2} u_{t t}
+ \frac{\tau^3}{6}u_{t t t}+ O\left( \tau^4 \right) 
.\] 
\[
u_{m\pm 1}^n= u\pm h u_x + \frac{h^2}{2}u_{x x}
\pm  \frac{h^3}{6}u_{x x x}+ \frac{h^4}{24}u_{x x x x}
\pm \frac{h^5}{120}u_{x x x x x}+ O\left( h^6 \right) 
.\] 
\[
	\left( u_{x x} \right) _m^{n+1}=
	u_{x x}+\tau u_{x x t}+ \frac{\tau^2}{2}u_{x x t t}
	+ \frac{\tau^3}{6}u_{x x t t t}+
	O\left( \tau^4 \right) 
.\] 
Значит
\begin{multline*}
r_{\tau h}= u_t + \frac{\tau}{2} u_{t t}+
O\left( \tau^2 \right) +a (\xi-1) \left[ 
u_{x x}+\frac{h^2}{12} u_{x x x x}+ O\left( h^4 \right) \right] -\\-
a \xi \left[ u_{x x}+ \tau u_{x x t}+ \frac{h^2}{12}\left( 
u_{x x x x }+\tau u_{x x x x t}+ O\left(\tau^2\right)\right)+
O\left( \tau^2 \right) + O\left( h^4 \right) \right] =\\
\xlongequal[]{u_t=au_{x x}}\frac{\tau}{2}
u_{t t}+ a \left( \xi-1 \right) \frac{h^2}{12} u_{x x x x}
-a \xi \frac{h^2}{12}u_{x x x x}- a \xi \tau u_{x x t}
- a \xi \tau \frac{h^2}{12} u_{x x x x t}+
O\left( \tau^2,\,h^4 \right) =\\=
\frac{\tau}{2} u_{t t}- \frac{h^2}{12}
u_{x x x x}-a \xi \tau u_{x x t}
-a \xi \tau \frac{h^2}{12}u_{x x x x t}+
O\left( \tau^2,\,h^4 \right)
.\end{multline*} 
\[
u_t = a u_{x x}
.\] 
Поэтому
\[
\left\{
\begin{aligned}
	u_{t t}&= u_{xx t}\cdot a\\
	u_{tx}&= u_{x x x}\cdot a\\
	u_{t x x} &= u_{x x x x}\cdot a \\
\end{aligned}
\right.
\implies
\frac{1}{a^2} u_{t t}=
u_{x x x x}= u_{x x t }\frac{1}{a}
.\] 
\[
	r_{\tau h}= \left( \frac{\tau}{2}
	-\frac{h^2}{12a^2}- \frac{a}{a}
\xi \tau\right) u_{t t}+ O\left( \tau^2,\,h^4 \right) 
.\] 
Следовательно схема будет иметь порядок аппроксимации $O
\left( \tau^2,\,h^4 \right) $ при условии
\[
\frac{\tau}{2}- \frac{h^2}{12a^2}- \xi \tau=0\implies
\xi = \frac{1}{2}- \frac{h^2}{12 a^2 \tau}
.\] 
\end{sol}
\begin{hiProb}[9.8]
\end{hiProb}
\begin{sol}
\[
u_t=u_{x x}
.\] 
\[
	\frac{y_m^{n+1}-y_m^{n-1}}{2\tau}=
	\frac{y_{m+1}^n-y_m^{n+1}-y_m^{n-1}+y
	_{m-1}^n}{h^2}
.\] 
Исследуем на устойчивость: $y_{m}^n=\lambda^n e^{i m \phi}$
\[
\frac{\lambda^2-1}{2\tau}= \frac{\lambda e^{i\phi} -\lambda^2-
1+\lambda e^{-i\phi}}{h^2}
.\] 
\[
\lambda= \frac{2\tau \cos \phi \pm  \sqrt{ h^4 -2 \tau^2+2\tau^2
\cos 2\phi} }{h^2+2 \tau}
.\] 
\[
\sigma= \frac{\tau}{h^2}\implies
\lambda= \frac{2\sigma \cos \phi\pm \sqrt{1-2 \sigma
^2+2\sigma^2 \cos 2\phi} }{1+2\sigma}
.\] 
Построением графика функции $\lambda(\sigma)$ с
параметром $\phi$ нетрудно убедиться, что
условие $|\lambda|\le 1$ выполнено для любых $\phi$ и
$\sigma>0$ (что так по определению $\sigma$). Следовательно
схема безусловно устойчива.

Разложим в окрестности точки $\left( t_n,\,x_m \right) $:
\[
r_{\tau h}= u_t + \frac{\tau^2}{6}u_{t t t}-u_{x x}+
\frac{\tau^2}{h^2} u_{t t}+ O\left( \tau^2,\,h^2 \right) 
.\] 
Если $\tau /h=c= \mathrm{const}$, то
 \[
	 r_{\tau h}= -u_{x x}+u_t +c^2 u_{t t}+ O\left( 
	 \tau^2,\,h^2\right) 
.\] 
Т.\:е. аппроксимируется уравнение
\[
u_t+c^2 u_{tt}=u_{x x}
.\] 
\end{sol}
\begin{hiProb}[9.9]
\end{hiProb}
\begin{sol}
\begin{multline*}
\frac{1}{12} \frac{y_{m+1}^{n+1}-y_{m+1}^n}{\tau}
+ \frac{5}{6} \frac{y_m^{n+1}-y_{m}^n}{\tau}
\frac{1}{12} \frac{y_{m-1}^{n+1}-y_{m-1}^n}{\tau}=\\
\frac{1}{2} \frac{y_{m+1}^n-2y_m^n+y_{m-1}^n}{h^2}+
\frac{1}{2} \frac{y_{m+1}^{n+1}-2y_m^{n+1}+y_{m-1}^{n+1}}{h^2}
.\end{multline*} 
Разложим в окрестности точки $\left( t_n,\,x_m \right) $:
\begin{multline*}
	r_{\tau h}= \frac{1}{12} \left( 
	u_t+h u_{t x}+ \frac{h^2}{2}u_{t xx}
+ \frac{h^3}{6}u_{t x x x}+ O\left( h^4 \right) 
+u_t-h u_{t x} \right. + \\ + \left. \frac{h^2}{2} u_{t x x}-
\frac{h^3}{6} u_{t x x x}+
O\left( h^4 \right) + \frac{\tau}{2}\left( 
u_{t t}+ \frac{h^2}{2}u_{t t x x}\cdot 2+
\frac{h^4}{12}u_{t t x x x x}+O\left( h^6 \right) \right) \right. + \\ + \left.
\frac{\tau^2}{6} \left( u_{t t t}+
h^2 u_{t t t x x}+ \frac{h^4}{12} u_{t t t x x x x}
+O\left( h^6 \right) \right) \right) +
\frac{5}{6}\left( u_t +\frac{\tau}{2}
u_{t t}+ \frac{\tau^2}{6}u_{t t t }+O\left( \tau^3 \right) \right) 
-\\-\frac{1}{2} \left[ 
u_{x x} + \frac{h^2}{12} u_{x x x x} +O\left( 
h^4\right) + u_{x x} +\tau u_{x x t}\right. + \\ + \left.
\frac{h^2}{12} \left( u_{x x x x}+\tau
u_{x x x x t}+ \frac{\tau^2}{2}u_{x x x x t t}+
O\left( \tau^3 \right) \right) +O\left( \tau^2 \right) +
O\left( h^4 \right) \right] =\\=
u_t -u_{xx} + \frac{h^2}{12} \left( u_{t x x}-
u_{x x x x}\right) +\frac{1}{2} \tau\left( 
u_{t t}- u_{x x t}\right) + \frac{h^2 \tau}{24}
\left( u_{t t x x}-u_{x x x x t}\right) +\\
+ \tau^2 \left( 
\frac{u_{t t t}}{6}- \frac{u_{x x t t}}{4}\right) +
h^2 \tau^2 \left( \frac{u_{t t t x x}}{72}-
\frac{u_{x x x x tt}}{48}\right) +
h^4 \left( \frac{u_{t x x x x}}{144}-
\frac{u_{x x x x x x}}{360}\right) .\end{multline*} 
\[
u_t=u_{x x} \implies \left\{
\begin{aligned}
u_{t x x }&= u_{x x x x} \\
u_{t t}&= u_{x x t}\\
u_{t t x x}&= u_{x x x x t} \\
\end{aligned}
\right.
\]
Значит порядок аппроксимации: $O\left( \tau^2,\,h^4,\,
h^2 \tau^2\right) $.

Исследуем на устойчивость: $y_{m}^n= \lambda^n e^{i m \phi}$.
\[
	\frac{\left( \lambda-1 \right) e^{i\phi}}{12\tau}+
	\frac{5 \left( \lambda-1 \right) }{6\tau}+
	\frac{(\lambda-1)e^{-i\phi}}{12\tau}=
	\frac{e^{i\phi}+ e^{-i\phi}-2}{2h^2}+
	\lambda \frac{e^{i\phi}+e^{-i\phi}-2}{2h^2}
.\] 
\[
	\lambda= \frac{5h^2 -6\tau +\left( h^2 +6\tau \right) 
	\cos \phi }{5h^2 +6\tau+\left( h^2-6\tau)
\cos  \phi\right) }
.\]
\[
\sigma= \frac{\tau}{h^2}
.\] 
\[
	\lambda= \frac{5-6 \sigma +\left( 1+ \sigma\cdot 6 \right) \cos  \phi}{5 + 6 \sigma + \left( 1- 6\sigma \right) 
	\cos \phi}
.\] 
\[
	\lambda= \frac{5 + \cos \phi - 6 \sigma \left( 1- \cos \phi \right) }{5 + \cos \phi + 6 \sigma (1- \cos \phi)}
.\] 
$|\lambda|\le 1$ при любых $\phi$ и $\sigma$ (т.\:к.
$(1-\cos \phi)\ge 0\implies 6 \sigma (1-\cos \phi)\ge 0$).
Значит схема безусловно устойчива.
\end{sol}
\section*{Тема XIV. Численные методы для уравнений в
частных производных гиперболического типа}
\begin{hiProb}[8.5]
\end{hiProb}
\begin{sol}
\[
	u_t+ a u_x=0,\quad a \neq \mathrm{const}
.\] 
Ищем дисперсионное соотношение для разностного уравнения
в виде: $y_m^n= e^{\lambda(k) t_n}\cdot e^{ikx_m}=
 e^{\lambda n \tau +i kmh}$.

Подставим в схему <<правый уголок>>  1-го порядка аппроксимации:
\[
	\frac{1}{\tau}\left( e^{\lambda\tau}-1 \right) 
	+ \frac{a}{h} \left( 1-e^{-i kh} \right) =0
.\] 
Значит
\[
	\lambda \left( \tau,\,h,\,k \right) 
	= \frac{1}{\tau} \ln \left( 1- \frac{a \tau}{h}
	+\frac{a\tau}{h}e^{-ikh}\right) 
.\] 
При $a \tau /h =1$ $\lambda(k)=\lambda\left( \tau,\,h,\,
k\right) $.
Предположим, что $kh \ll 1$, $h$ --- малый параметр.

Аппроксимация тем лучше, чем меньше $k$. Следовательно
\[
	\lambda\left(\tau,\,h,\,k  \right) 
	\approx-iak- \frac{ahk^2}{2} \left( 1- \frac{a\tau}{h} \right) = \lambda(k)-\frac{k^2}{2}
	a h (1-\sigma)
.\] 
$\sigma= a \tau /h$ --- число Куранта. Частное решение:
\[
	\exp \left( ik \left( mh-an\tau \right)  \right) 
	\exp  \left( -\frac{1}{2}ak^2 \left( 
	n-a\tau\right) n\tau \right) 
.\] 
При $a<0$, $h-a\tau >0$ рассматриваемую схему нельзя
использовать. Второй сомножитель имеет порядок $\exp 
\left( k^2 |a| h t_n \right) $ --- быстро растёт при
$k \sim h^{-1}$, $h\ll 1$. Аналогично при $a>0$, $
р-a\tau<0$. При $a>0$, $h-a\tau>0$ численное
решение отличается от точного наличием затухающего множителя
для гармоник с большим $k$ (малым $\lambda$). Значит
наблюдаем сглаживание решений.

Для разностной схемы Л-В 2-го порядка аппроксимации:
\[
	\left( y_m^{n+1} -y_m^n \right) +
	\frac{\sigma}{2}\left( y_{m+1}^n-
	y_{m-1}^n\right) -
	\frac{\sigma^2}{2} \left( y_{m-1}^n-2y_m^n+
	y_{m+1}^n\right) =0
.\] 
Получаем
\[
	\lambda= \frac{1}{\tau} \ln \left( 1- i \sigma
	\sin k h - 2 \sigma^2 \sin ^2 \frac{kh}{2}\right) 
	\xlongequal[]{kh\ll 1}-ika+
	ika\cdot  \frac{k^2 h^2}{6} \left( 1- 3\sigma^2 \right) 
.\] 
Значит решения дифференциального уравнения переноса имеют
вид волн:
\[
	u\left( t,\,x \right) = \exp \left( 
	ikx + \lambda(k) t\right) = \exp \left( 
ik\left( x-at \right) \right) 
.\] 
Решения:
\begin{multline*}
	y\left( t_n,\,x_m \right) =
	\exp \left( ikx_n+ \lambda \left( \tau,\,h,\,k \right) t_n \right) =\\=
	\exp \left[ 
	ik \left\{ x_m -a \left[ 1- \frac{k^2 h^2}{6}
\left( 1-3\sigma^2 \right) \right] it_n \right\} \right] =\\
\xlongequal[]{A_k= - \frac{k^2 h^2(1-3\sigma^2)}{6}}
\exp  \left( ik \left\{ x_m -a \left[ 
1+A_k(k)\right] t_n \right\}  \right) 
.\end{multline*} 
Следовательно каждая волна со своей частотой
движется со скоростью $a_k=a (1+A_k)$, поэтому
теряется монотонность профиля $u(x)$, появляется
сеточная дисперсия.
\end{sol}
\begin{hiProb}[9.1]
\end{hiProb}
\begin{proof}
\[
u_t-u_x=0
.\] 
\[
	u(t,\,x)= e^{i\alpha t}e^{i\alpha x}
.\] 
\[
\frac{y_m^{p+1}-y_{m}^p}{\tau}-
\frac{y_{m+1}^p-y_m^p}{h}=0
\]
--- имеет решение $y_m^p=\left[ 1- \sigma +\sigma e^{i\alpha h} \right] ^p\cdot e^{i\alpha h m}$, $p= \frac{t}{\tau}$, $m=\frac{x}{h}$.
\begin{multline*}
\sigma= \frac{\tau}{h}\implies
\lambda= 1 - \frac{\tau}{h}+ \frac{\tau}{h}e^{i\alpha h}=
1- \frac{\tau}{h}\left( 1-e^{i\alpha h} \right) \sim\\\sim  
1- \frac{\tau}{h} \left[ 
1- \left( 1+i \alpha h + \frac{i^2 \alpha^2 h^2}{2}+
\ldots\right) \right] \sim \\\sim 
1-\frac{\tau}{h} \left( -i\alpha h-
\frac{\alpha^2 h^2}{2}\right) \sim 1+
i\alpha \tau + \frac{\alpha^2 h}{2}+ \ldots \to 
1+i\alpha \tau\ (h\to +0)
.\end{multline*} 
\[
	\left( 1+i\alpha \tau \right) \frac{t}{\tau}
	= \left( \left( 1+i\alpha \frac{1}{1 /\tau} \right) ^{\frac{1}{\tau}} \right) t
	\xlongrightarrow[\tau \to 0]{1 /\tau\to \infty }
	e^{i\alpha t}
.\] 
Значит на самом деле решение разностной схемы стремится
к решению дифференциальной задачи при $h\to 0$, если
дополнительно $\tau\to 0$, т.\:е. $\tau =O\left( h^n \right) $, $n\ge 1$. Наибольший  интерес представляет
случай $n=1 \Leftrightarrow \tau =c h$,  где $c$ задаёт
наклон характеристик: $c= 1 /a$, где $a$ --- коэффициент
при $u_x$ в дифференциальном уравнении. Поэтому $c=-1$,
$\sigma=-1= c\tau /h$, таким  образом случай $\sigma >1$ 
не реализуется.
\end{proof}
\begin{hiProb}[9.5]
\end{hiProb}
\begin{sol}
Отрицательная схемная вязкость:
\[
	\gamma= \frac{c h }{2}\left( 1- \frac{c\tau}{h}
	\right)<0 \implies \frac{c \tau}{h}>1
.\] 
\[
	r_{\tau h}= Lu- \frac{c h}{2}\left( 1- \frac{c\tau}{h} \right) u_{xx} + O\left( \tau^2,\,h^2 \right) 
.\] 
Следовательно если $\gamma<0$, то задача Коши ставится
некорректно, а разностная схема \emph{неустойчива} (Аристова, стр. 182).

Т.\:к. у всех схем  1-го порядка есть схемная вязкость,
то точно есть порядок аппроксимации $O(\tau,\,h)$,
но т.\:к. устойчивости нет, то \emph{сходимости нет}.

\[
c\frac{\tau}{h}>1
.\] 
Решим для схемы <<явный левый уголок>>, т.\:к. у  неё
отрицательный коэффициент вязкости:
\[
\frac{y_m^{n+1}-y_m^n}{\tau}+c \frac{y_m^n-y_{m-1}^n}{h}=0
.\] 
\[
	y_{m}^{n+1}=  -\frac{c\tau}{h} \left( y_m^n
	-y_{m-1}^n\right) + y_{m}^n
.\] 
\[
y_{m+1}^{n+1}= -  \frac{c\tau}{h}
\left(y_{m+1}^n-y_m^n \right) + y_{m+1}^n
.\] 
Пусть $\left( y_{m+1}^n -y_m^n \right) >0$, тогда
\begin{multline*}
y_{m+1}^{n+1}-y_m^{n+1} = -\frac{c \tau}{h}
\left[ y_{m+1}^n - \left( y_m^n -y_{m-1}^n \right)  \right]
+y_{m+1}^n-y_m^n=\\= \frac{c\tau}{h} \left( 
y_{m}^n-y_{m-1}^n\right) -
\left( \frac{c\tau}{h}-1 \right) \left( 
y_{m+1}^n-y_m^n\right) >0
.\end{multline*} 
\[
	\frac{c\tau}{h} \left( y_m^n-y_{m-1}^n \right) 
	>\left( \frac{c\tau}{h}-1 \right) 
	\left( y_{m+1}^n-y_m^n \right) >0
.\] 
Следовательно \emph{монотонность} есть при условии:
\[
	\left( y_m^n-y_{m-1}^n \right) >
	\frac{\left( \frac{c\tau}{h}-1 \right) }{
	c\tau /h}\left( 
y_{m+1}^n-y_m^n\right) 
\]
и нет в общем случае.
\end{sol}
\section*{Тема XV. Численные методы для уравнений
в частных производных эллиптического типа}
\begin{hiProb}[5.1]
\end{hiProb}
\begin{sol}
Ознакомился.
\end{sol}
\begin{hiProb}[7.1]
\end{hiProb}
\begin{sol}
В уравнении Лапласа нет выделенных пространствеенных направлений,
значит веса зависят только от расстояния от центра
шаблона. Поэтому схему можно представить в виде
\begin{multline*}
	u_{m,l}= \alpha \left( u_{m-1,l}+
	u_{m+1,l} \right) +\beta\left( u_{m,l-1}+
u_{m,l+1}\right)
+\\+\gamma \left( u_{m-1,l+1}+
u_{m-1,l-1}+ u_{m+1,l+1}+u_{m+1,l-1}\right) 
.\end{multline*} 
Разложим всё по Тейлору в окрестности  $u_{m,l}$ и 
приравняем коэффициенты при соответствующих степенях
$h_i$:
\begin{table}[htpb]
\centering
\caption{}
\label{tab:1}
\begin{tabular}{|Cc|Cc|Cc|Cc|}
	\hline $h^0$ & $1=2\alpha+2\beta+4\gamma$ & 
		     $h_x^4$& $\frac{\alpha}{12}u_{x x x x}$ \\
	\hline $h_{x}^2$ & $\alpha u_{x x}$ & 
			 $h^4_y$& $\frac{\beta}{12}
			 u_{y y y y}$ \\
	\hline $h_y^2$ & $\beta u_{y y}$ & 
		       $h_x^2 h_y^2$& 
		       $\gamma u_{x x y y}$\\
	\hline
\end{tabular}
\end{table}
В  силу уравнения Лапласа: $\alpha h_x^2=\beta h_y^2$.
\[
	0=\left( u_{x x}+ u_{yy} \right) _{x x}=
	u_{x x x x}+ u_{x x y y}
.\] 
\[
0= u_{y y y y}+ u_{xxyy}
.\] 
\[
	u_{x x y y}=-\frac{1}{2} \left( u_{x x x x }+
	u_{y y y y}\right) 
.\] 
\begin{multline*}
h_x^4 \cdot \frac{\alpha}{12} u_{x x x x}
+h_y^4 \frac{\beta}{12} u_{y yy y}+
h_x^2 h_y^2 \frac{\gamma}{6}
u_{x x y y}=\\=
h_x^4 \frac{\alpha}{12} u_{x x x x}
+h_y^4 \frac{\beta}{12}
u_{y y y y}-
\frac{\gamma}{2} h_x^2 h_y^2 \left( 
u_{x x x x}+u_{y y y y}\right) =\\=
\frac{h_x^2}{12} \left( \alpha h_x^2 -
6\gamma h_y^2\right) u_{x x x x}+
\frac{h_y^2}{12} \left( 
\beta h_y^2 - 6 \gamma h_x^2\right) u_{y y y y}
.\end{multline*} 
\begin{itemize}
	\item при $h_x \neq h_y$  порядок $O\left( h_x^2,\,h_y^2 \right) $ 
\item при $h_x=h_y=h$ коэффициенты определяются
	системой
\[
\left\{
\begin{aligned}
2\alpha +2\beta+4\gamma&= 1 \\
\alpha&= \beta \\
\alpha&= 6\gamma \\
\end{aligned}
\right.
\]
Порядок $O\left( h^4 \right) $.
\end{itemize}
\end{sol}
\end{document}
