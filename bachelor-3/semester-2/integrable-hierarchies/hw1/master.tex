\documentclass[a4paper]{article}
% Этот шаблон документа разработан в 2014 году
% Данилом Фёдоровых (danil@fedorovykh.ru) 
% для использования в курсе 
% <<Документы и презентации в \LaTeX>>, записанном НИУ ВШЭ
% для Coursera.org: http://coursera.org/course/latex .
% Исходная версия шаблона --- 
% https://www.writelatex.com/coursera/latex/5.3

% В этом документе преамбула

\usepackage{siunitx}
%%% Работа с русским языком
%\usepackage{cmap}					% поиск в PDF
%\usepackage{mathtext} 				% русские буквы в формулах
%\usepackage[T2A]{fontenc}			% кодировка
%\usepackage[utf8]{inputenc}			% кодировка исходного текста
%\usepackage[english,russian]{babel}	% локализация и переносы
%\usepackage{indentfirst}
%\frenchspacing
%
%\renewcommand{\epsilon}{\ensuremath{\varepsilon}}
%\newcommand{\phibackup}{\ensuremath{\phi}}
%\renewcommand{\phi}{\ensuremath{\varphi}}
%\renewcommand{\varphi}{\ensuremath{\phibackup}}
%\renewcommand{\kappa}{\ensuremath{\varkappa}}
%\renewcommand{\le}{\ensuremath{\leqslant}}
%\renewcommand{\leq}{\ensuremath{\leqslant}}
%\renewcommand{\ge}{\ensuremath{\geqslant}}
%\renewcommand{\geq}{\ensuremath{\geqslant}}
%\renewcommand{\emptyset}{\varnothing}
%\renewcommand{\Im}{\operatorname{Im}}
%\renewcommand{\Re}{\operatorname{Re}}


%%% Дополнительная работа с математикой
\usepackage{amsmath,amsfonts,amssymb,amsthm,mathtools} % AMS
%\usepackage{icomma} % "Умная" запятая: $0,2$ --- число, $0, 2$ --- перечисление

%% Номера формул
%\mathtoolsset{showonlyrefs=true} % Показывать номера только у тех формул, на которые есть \eqref{} в тексте.
%\usepackage{leqno} % Нумереация формул слева

%% Свои команды
\DeclareMathOperator{\sgn}{\mathop{sgn}}
\DeclareMathOperator{\sign}{\mathop{sign}}
\DeclareMathOperator*{\res}{\mathop{res}}
\DeclareMathOperator*{\tr}{\mathop{tr}}
\DeclareMathOperator*{\rot}{\mathop{rot}}
\DeclareMathOperator*{\divop}{\mathop{div}}
\DeclareMathOperator*{\grad}{\mathop{grad}}

%% Перенос знаков в формулах (по Львовскому)
\newcommand*{\hm}[1]{#1\nobreak\discretionary{}
{\hbox{$\mathsurround=0pt #1$}}{}}

%%% Работа с картинками
\usepackage{graphicx}  % Для вставки рисунков
\graphicspath{{figures/}}  % папки с картинками
\setlength\fboxsep{3pt} % Отступ рамки \fbox{} от рисунка
\setlength\fboxrule{1pt} % Толщина линий рамки \fbox{}
\usepackage{wrapfig} % Обтекание рисунков текстом

%%% Работа с таблицами
\usepackage{array,tabularx,tabulary,booktabs} % Дополнительная работа с таблицами
\usepackage{longtable}  % Длинные таблицы
\usepackage{multirow} % Слияние строк в таблице

%%% Теоремы
\theoremstyle{plain} % Это стиль по умолчанию, его можно не переопределять.
\newtheorem{thm}{Теорема}
\newtheorem*{thm*}{Теорема}
\newtheorem{prop}{Предложение}
\newtheorem*{prop*}{Предложение}
 
\theoremstyle{definition} % "Определение"
%\newtheorem{corollary}{Следствие}[theorem]
\newtheorem{dfn}{Определение}
\newtheorem*{dfn*}{Определение}
\newtheorem{prob}{Задача}
\newtheorem*{prob*}{Задача}

 
\theoremstyle{remark} % "Примечание"
\newtheorem*{sol}{Решение}
\newtheorem*{rem}{Замечание}

%%% Программирование
\usepackage{etoolbox} % логические операторы

%%% Страница
%\usepackage{extsizes} % Возможность сделать 14-й шрифт
%\usepackage{geometry} % Простой способ задавать поля
%	\geometry{top=25mm}
%	\geometry{bottom=35mm}
%	\geometry{left=35mm}
%	\geometry{right=20mm}
 
\usepackage{fancyhdr} % Колонтитулы
%	\pagestyle{fancy}
 %	\renewcommand{\headrulewidth}{0pt}  % Толщина линейки, отчеркивающей верхний колонтитул
	%\lfoot{Нижний левый}
	%\rfoot{Нижний правый}
	%\rhead{Верхний правый}
	%\chead{Верхний в центре}
	%\lhead{Верхний левый}
	%\cfoot{Нижний в центре} % По умолчанию здесь номер страницы

\usepackage{setspace} % Интерлиньяж
%\onehalfspacing % Интерлиньяж 1.5
%\doublespacing % Интерлиньяж 2
%\singlespacing % Интерлиньяж 1

\usepackage{lastpage} % Узнать, сколько всего страниц в документе.

\usepackage{soul} % Модификаторы начертания

\usepackage{hyperref}
\usepackage[usenames,dvipsnames,svgnames,table,rgb]{xcolor}
\hypersetup{				% Гиперссылки
    unicode=true,           % русские буквы в раздела PDF
    pdftitle={Заголовок},   % Заголовок
    pdfauthor={Автор},      % Автор
    pdfsubject={Тема},      % Тема
    pdfcreator={Создатель}, % Создатель
    pdfproducer={Производитель}, % Производитель
    pdfkeywords={keyword1} {key2} {key3}, % Ключевые слова
%    colorlinks=true,       	% false: ссылки в рамках; true: цветные ссылки
    %linkcolor=red,          % внутренние ссылки
    %citecolor=black,        % на библиографию
    %filecolor=magenta,      % на файлы
    %urlcolor=cyan           % на URL
}

\usepackage{csquotes} % Еще инструменты для ссылок

%\usepackage[style=apa,maxcitenames=2,backend=biber,sorting=nty]{biblatex}

\usepackage{multicol} % Несколько колонок

\usepackage{tikz} % Работа с графикой
\usepackage{pgfplots}
\usepackage{pgfplotstable}
%\usepackage{coloremoji}
\usepackage{floatrow}
\usepackage{subcaption}
\graphicspath{{figures/}}

\renewcommand\thesubfigure{\asbuk{subfigure}}
%\addbibresource{master.bib}

\usepackage{import}
\usepackage{pdfpages}
\usepackage{transparent}
\usepackage{xcolor}
\usepackage{xifthen}

\newcommand{\incfig}[2][1]{%
    \def\svgwidth{#1\columnwidth}
    \import{./figures/}{#2.pdf_tex}
}
%\usepackage{titlesec}
%\titleformat{\section}{\normalfont\Large\bfseries}{}{0pt}{}
%----------------------STANDART:
%\titleformat{\chapter}[display]
%  {\normalfont\huge\bfseries}{\chaptertitlename\ \thechapter}{20pt}{\Huge}
%\titleformat{\section}{\normalfont\Large\bfseries}{\thesection}{1em}{}
%\titleformat{\subsection}
%  {\normalfont\large\bfseries}{\thesubsection}{1em}{}
%\titleformat{\subsubsection}
%  {\normalfont\normalsize\bfseries}{\thesubsubsection}{1em}{}
%\titleformat{\paragraph}[runin]
%  {\normalfont\normalsize\bfseries}{\theparagraph}{1em}{}
%\titleformat{\subparagraph}[runin]
%  {\normalfont\normalsize\bfseries}{\thesubparagraph}{1em}{}

\pdfsuppresswarningpagegroup=1
\pgfplotsset{compat=1.16}



%\setcounter{tocdepth}{1} % only parts,chapters,sections
%\titleformat{\subsection}{\normalfont\large\bfseries}{}{0em}{}
%\titleformat{\subsubsection}{\normalfont\normalsize\bfseries}{}{0em}{}

%\newcommand{\textover}[2]{\stackrel{\mathclap{\normalfont\mbox{#2}}}{#1}}

\author{Yaroslav Drachov\\
Moscow Institute of Physics and Technology}
%\author{Драчов Ярослав\\
%Факультет общей и прикладной физики МФТИ}
\newcommand{\veq}{\mathrel{\rotatebox{90}{$=$}}}
%\newcommand{\teto}[1]{\stackrel{\mathclap{\normalfont\tiny\mbox{#1}}}{\to}}
%\renewcommand{\thesubsection}{\arabic{subsection}}

%%\setcounter{secnumdepth}{0}

\definecolor{tabblue}{RGB}{30, 119, 180}
\definecolor{taborange}{RGB}{255, 127, 15}
\definecolor{tabgreen}{RGB}{45, 160, 43}
\definecolor{tabred}{RGB}{214, 38, 40}
\definecolor{tabpurple}{RGB}{148, 103, 189}
\definecolor{tabbrown}{RGB}{140, 86, 76}
\definecolor{tabpink}{RGB}{227, 119, 193}
\definecolor{tabgray}{RGB}{127, 127, 127}
\definecolor{tabolive}{RGB}{188, 189, 33}
\definecolor{tabcyan}{RGB}{22, 190, 207}
\pgfplotscreateplotcyclelist{colorbrewer-tab}{
{tabblue},
{taborange},
{tabgreen},
{tabred},
{tabpurple},
{tabbrown},
{tabpink},
{tabgray},
{tabolive},
{tabcyan},
}
\usepackage{csvsimple}
\usepackage{extarrows}
%\renewcommand{\labelenumii}{\asbuk{enumii})}
%\renewcommand{\labelenumiv}{\Asbuk{enumiv}}
%\newcommand{\prob}[1]{\subsubsection*{#1}}
\sisetup{output-decimal-marker = {,},separate-uncertainty = true,exponent-product = \cdot}

\usepackage{braket}
\usepackage{enumerate}
\usepackage{chngcntr}
%\counterwithin*{equation}{problem}
%\usepackage{bbold}

\newtheoremstyle{hiProb}% ⟨name ⟩ 
{3pt}% ⟨Space above ⟩1 
{3pt}% ⟨Space below ⟩1
{}% ⟨Body font ⟩
{}% ⟨Indent amount ⟩2
{\bfseries}% ⟨Theorem head font⟩
{.}% ⟨Punctuation after theorem head ⟩
{.5em}% ⟨Space after theorem head ⟩3
%{\thmname{#1} \thmnote{#3}}% ⟨Theorem head spec (can be left empty, meaning ‘normal’)⟩
{\thmnote{#3}}% ⟨Theorem head spec (can be left empty, meaning ‘normal’)⟩
\theoremstyle{hiProb} % "Определение"
%\newtheorem{hiProb}{Задача}
\newtheorem{hiProb}{}
%\usepackage{mmacells}
\newcommand{\textover}[2]{\stackrel{\mathclap{\normalfont\scriptsize\mbox{#2}}}{#1}}
\usepackage{units}
\usepackage[math]{cellspace}%
\setlength\cellspacetoplimit{2pt}
\setlength\cellspacebottomlimit{2pt}

\DeclareMathAlphabet{\mathbbold}{U}{bbold}{m}{n}

\newcommand{\normord}[1]{:\mathrel{#1}:}

\title{Домашнее задание по интегрируемым иерархиям}
\begin{document}
	\maketitle
\begin{hiProb}[Упражнение 1.0]
\end{hiProb}
\begin{sol}
\[
l_i=\epsilon _{ijk} r_j p_k
.\] 
\begin{multline*}
\left\{ l_i,\,l_j \right\} =\sum_{k=1}^{3} 
\frac{\partial l_j}{\partial p_k} 
\frac{\partial l_i}{\partial r_k} -\frac{\partial l_i}{\partial p_k} \frac{\partial l_j}{\partial r_k}=
\epsilon _{jpl}r_p \delta_{lk}
\epsilon _{imn}\delta_{m k}p_n -\epsilon _{imn}r_m \delta_{kn}
\epsilon _{jpl}\delta_{pk}p_l =\\=
\epsilon _{jpk}r_p \epsilon _{ikm}p_m-\epsilon _{imk} r_m \epsilon _{jkl}p_l
=
(\delta_{jm}\delta_{pi}-\delta_{ji}\delta_{pm})r_pp_m-(\delta_{il}\delta_{mj}-\delta_{ij}\delta_{lm})r_mp_l
=
\\=r_ip_j-\delta_{ij}r_mp_m-r_jp_i+\delta_{ij}r_mp_m =
r_i p_j-r_j p_i
.\end{multline*} 
С другой стороны
\[
\epsilon _{ijk}l_k =\epsilon _{ijk}\epsilon _{klm}r_l p_m=
(\delta_{il}\delta_{jm}-\delta_{im}\delta_{jl})r_l p_m=
r_i p_j-r_jp_i
.\] 
Откуда следует, что
\[
\left\{ l_i,\,l_j \right\} =\epsilon _{ijk}l_k
.\] 
Далее
\begin{multline*}
A_i= \epsilon _{ijk} l_j p_k + \frac{\alpha}{r}r_i=
\epsilon _{ijk} \epsilon _{jmn}r_m p_n p_k +\frac{\alpha}{r}
r_i=
(\delta_{km}\delta_{in}-\delta_{kn}\delta_{im})
r_m p_n p_k+ \frac{\alpha}{r}r_i=
\\=(\mathbf{r},\,\mathbf{p})p_i+\left(\frac{\alpha}{r}-\mathbf{p}^2\right)r_i
.\end{multline*} 
\begin{multline*}
\left\{ l_i,\,A_j \right\} =
\sum_{k=1}^{3} \frac{\partial l_i}{\partial r_k} 
\frac{\partial A_j}{\partial p_k}-
\frac{\partial A_j}{\partial r_k} \frac{\partial l_i}{\partial p_k} =\\=
\epsilon _{inm}\delta_{nk} p_m
\left(r_k p_j+(\mathbf{r},\,\mathbf{p})\delta_{jk}-2p_k r_j\right)-
\left(p_kp_j - \frac{\alpha r_k}{r^3}r_j+\left(\frac{\alpha}{r}-\mathbf{p}^2\right) \delta_{jk} \right) 
\epsilon_{inm}r_n \delta_{m k }=\\
=\epsilon _{ikm} p_m
\left(r_k p_j+(\mathbf{r},\,\mathbf{p})\delta_{jk}\right)-
\left(p_kp_j +\left(\frac{\alpha}{r}-\mathbf{p}^2\right) \delta_{jk} \right) 
\epsilon_{ink}r_n =\\
=\epsilon _{ijm}\left((\mathbf{r},\,\mathbf{p})p_m+
\left( \frac{\alpha}{r}+\mathbf{p}^2 \right) r_m\right)=
\epsilon _{ijk}A_k
.\end{multline*} 
А также
\begin{multline*}
\left\{ A_i,\,A_j \right\} =
\sum_{k=1}^{3} \frac{\partial A_i}{\partial r_k} 
\frac{\partial A_j}{\partial p_k} -
\frac{\partial A_j}{\partial r_k} \frac{\partial A_i}{\partial p_k} =\\=
\left(p_kp_i - \frac{\alpha r_k}{r^3}r_i+\left(\frac{\alpha}{r}-\mathbf{p}^2\right) \delta_{ik} \right)
\left(r_k p_j+(\mathbf{r},\,\mathbf{p})\delta_{jk}-2p_k r_j\right)-\\-
\left(p_kp_j - \frac{\alpha r_k}{r^3}r_j+\left(\frac{\alpha}{r}-\mathbf{p}^2\right) \delta_{jk} \right)
\left(r_k p_i+(\mathbf{r},\,\mathbf{p})\delta_{ik}-2p_k r_i\right)=\\=
2(\mathbf{r},\,\mathbf{p})p_ip_j-2\mathbf{p}^2p_i r_j-
\frac{\alpha}{r}r_i p_j+
\frac{\alpha(\mathbf{r},\,\mathbf{p})}{r^3}r_i r_j+\\+
\left( \frac{\alpha}{r}-\mathbf{p}^2 \right) r_i p_j
+\left( \frac{\alpha}{r}-\mathbf{p}^2 \right) \delta_{ij}-
2\left( \frac{\alpha}{r}-\mathbf{p}^2 \right) p_i r_j-\\
-2(\mathbf{r},\,\mathbf{p})p_ip_j+2\mathbf{p}^2p_j r_i+
\frac{\alpha}{r}r_j p_i-
\frac{\alpha(\mathbf{r},\,\mathbf{p})}{r^3}r_i r_j-\\-
\left( \frac{\alpha}{r}-\mathbf{p}^2 \right) r_j p_i
-\left( \frac{\alpha}{r}-\mathbf{p}^2 \right) \delta_{ij}+
2\left( \frac{\alpha}{r}-\mathbf{p}^2 \right) p_j r_i=\\
-2\mathbf{p}^2p_i r_j-
\frac{\alpha}{r}r_i p_j+
\left( \frac{\alpha}{r}-\mathbf{p}^2 \right) r_i p_j
-
2\left( \frac{\alpha}{r}-\mathbf{p}^2 \right) p_i r_j+\\
+2\mathbf{p}^2p_j r_i+
\frac{\alpha}{r}r_j p_i-
\left( \frac{\alpha}{r}-\mathbf{p}^2 \right) r_j p_i
+
2\left( \frac{\alpha}{r}-\mathbf{p}^2 \right) p_j r_i=\\
\left(2\mathbf{p}^2-
\frac{\alpha}{r}+
\left( \frac{\alpha}{r}-\mathbf{p}^2 \right)
+
2\left( \frac{\alpha}{r}-\mathbf{p}^2 \right)\right)(r_ip_j-r_jp_i)=\\=
-\left(\mathbf{p}^2 -2 \frac{\alpha}{r} \right) (r_i p_j
-r_j p_i)=-2E\epsilon _{ijk} l_k
.\end{multline*} 
\end{sol}
\begin{hiProb}[Упражнение 1.2]
\end{hiProb}
\begin{sol}
\begin{multline*}
H_3= \frac{1}{3} \sum_{i,j,k}^{} L_{ij}L_{jk}L_{ki}=
\frac{1}{3}\sum_{i=1}^{N} L_{ii}^3+
\frac{1}{3}\sum_{j=k\neq i}^{} L_{ij}L_{jk}L_{ki}+
\frac{1}{3} \sum_{\substack{i\neq j, j\neq k,\\ k\neq i}}^{} L_{ij}L_{jk}L_{ki}
%=\\
%=
%\frac{1}{3}\sum_{i=1}^{N} L_{ii}^3+
%\frac{1}{3}\sum_{j=k\neq i}^{} L_{ij}L_{jk}L_{ki}+
%\frac{1}{3} \sum_{\substack{i<j, j\neq k,\\ k\neq i}}^{} L_{ij}L_{jk}L_{ki}
%+
%\frac{1}{3} \sum_{\substack{i>j, j\neq k,\\ k\neq i}}^{} L_{ij}L_{jk}L_{ki}=\\
%=
%\frac{1}{3}\sum_{i=1}^{N} L_{ii}^3-
%\frac{1}{3}\sum_{j=k\neq i}^{} L_{ji}L_{kj}L_{ik}+
%\frac{1}{3} \sum_{\substack{i<j, j\neq k,\\ k\neq i}}^{} L_{ij}L_{jk}L_{ki}
%+
%\frac{1}{3} \sum_{\substack{i>j, j\neq k,\\ k\neq i}}^{} L_{ij}L_{jk}L_{ki}=\\
%=
%\frac{1}{3}\sum_{i=1}^{N} L_{ii}^3-
%\frac{1}{3}\sum_{j=k\neq i}^{} L_{ji}L_{kj}L_{ik}+
%\frac{1}{3} \sum_{\substack{i<j, j\neq k,\\ k\neq i}}^{} L_{ij}L_{jk}L_{ki}
%+
%\frac{1}{3} \sum_{\substack{i>j, j\neq k,\\ k\neq i}}^{} L_{ij}L_{jk}L_{ki}=\\
%=\frac{1}{3} \sum_{i=1}^{N} p_i^3+
%\sum_{i\neq j, i\neq k}^{} \frac{\nu}{q_i-q_j}L_{jk}
%\frac{\nu}{q_k-q_i}
.\end{multline*} 
\[
\sum_{\substack{i\neq j, j\neq k,\\ k\neq i}}^{} L_{ij}L_{jk}L_{ki}=
\sum_{\substack{i>j, j\neq k,\\ k\neq i}}^{} L_{ij}L_{jk}L_{ki}+
\sum_{\substack{i<j, j\neq k,\\ k\neq i}}^{} L_{ij}L_{jk}L_{ki}
.\] 
\begin{multline*}
\sum_{\substack{i<j, j\neq k,\\ k\neq i}}^{} L_{ij}L_{jk}L_{ki}=-
\sum_{\substack{i<j, j\neq k,\\ k\neq i}}^{} L_{ji}L_{kj}L_{ik}=\\=-
\sum_{\substack{j<i, i\neq k,\\ k\neq j}}^{} L_{ij}L_{ki}L_{jk}=
-\sum_{\substack{i>j, j\neq k,\\ k\neq i}}^{} L_{ij}L_{jk}L_{ki}
.\end{multline*} 
Следовательно
\[
\sum_{\substack{i\neq j, j\neq k,\\ k\neq i}}^{} L_{ij}L_{jk}L_{ki}=0
.\] 
И
\[
H_3= \frac{1}{3} \sum_{i=1}^{N} L_{ii}^3+ \frac{1}{3}
\sum_{j=k\neq i}^{} L_{ij}L_{jk}L_{ki}=
\frac{1}{3}\sum_{i=1}^{N} p_i^3-
\nu^2 \sum_{k \neq i}^{} \frac{p_i}{(q_i-q_k)^2}
.\] 
%\begin{multline*}
%	H_3=\frac{1}{3}\sum_{i,\,j,\,k}^{} L_{ij}L_{jk} L_{ki}=\\=
%	\frac{1}{3}\sum_{i=1}^{N} \sum_{j,\,k}^{} L_{ij}L_{jk}L_{ki}=\frac{1}{3}\sum_{i=1}^{N} \left( \sum_{j=1}^{N} L_{ij}L_{jj}L_{ji}+ \sum_{j\neq k}^{} L_{ij}L_{jk}L_{ki} \right) =\\=
%	\frac{1}{3}\sum_{i,\,j}^{} L_{jj}L_{ij}L_{ji}+\frac{1}{3}
%	\sum_{i=1}^{N} \sum_{j \neq k}^{} L_{ij}L_{jk}L_{ki}=\\=
%	\sum_{i=1}^{N} L_{ii}^3+ \frac{1}{3}\sum_{i\neq j}^{N} L_{jj}L_{ij}L_{ji}+\frac{1}{3}\sum_{i=1}^{N} \left( 
%	\sum_{j<k}^{} L_{ij}L_{jk}L_{ki}+\sum_{j>k}^{} L_{ij}L_{jk}L_{ki}\right) =\\=
%	\sum_{i=1}^{N} L_{ii}^3+ \frac{1}{3}\sum_{i< j}^{N} L_{jj}L_{ij}L_{ji}+\frac{1}{3}\sum_{i> j}^{N} L_{jj}L_{ij}L_{ji}+\\+\frac{1}{3}\sum_{i=1}^{N} \left( 
%	\sum_{j<k}^{} L_{ij}L_{jk}L_{ki}+\sum_{j>k}^{} L_{ij}L_{jk}L_{ki}\right) =\\=
%	\sum_{i=1}^{N} L_{ii}^3+ \frac{1}{3}\sum_{i< j}^{N} L_{jj}L_{ij}L_{ji}+\frac{1}{3}\sum_{i> j}^{N} L_{ii}L_{ji}L_{ij}+\\+\frac{1}{3}\sum_{i=1}^{N} \left( 
%	\sum_{j<k}^{} L_{ij}L_{jk}L_{ki}+\sum_{j>k}^{} L_{ij}L_{jk}L_{ki}\right) =
%	\frac{1}{3} \sum_{i=1}^{N} L_{ii}^3+
%	\frac{1}{3} \sum_{i \neq j\neq k}^{} L_{ij}L_{jk}L_{ki}
%	=\\=\frac{1}{3}\sum_{i=1}^{N} L_{ii}^3+
%	\frac{1}{3}\sum_{i\neq j}^{} \left( \sum_{j\neq k}^{} L_{ij}L_{jk}L_{ki}+
%	\sum_{k=1}^{N} L_{kk}^3\right)
%.\end{multline*} 

\end{sol}
\begin{hiProb}[Упражнение 1.3]
\end{hiProb}
\begin{sol}
Уравнения движения:
\begin{multline*}
-\dot{p}_k= \frac{\partial H}{\partial q_k} =
2\nu^2 \sum_{i<j}^{} \frac{\delta_{ik}-\delta_{jk}}{(q_i-q_j)^3}=
2\nu^2 \left( \sum_{k<j}^{} \frac{1}{(q_k-q_j)^3}-
\sum_{i<k}^{} \frac{1}{(q_i-q_k)^3}\right) =\\=
2\nu^2 \sum_{k\neq i}^{} \frac{1}{(q_k-q_i)^3}
,\end{multline*} 
\[
\dot{q}_k= \frac{\partial H}{\partial p_k} =
p_k
.\]
По определению
\[
	M_{ij}= \delta_{ij} d_i -(1-\delta_{ij})\frac{\nu}{(q_i-q_j)^2}
.\] 
Уравнение Лакса:
\begin{multline*}
\delta_{ij}\dot{p}_i+
(\delta_{ij}-1) \frac{\nu \left(\dot{q_i}-\dot{q}_j\right)}{(q_i-q_j)^2}=\\=
\sum_{k=1}^{N} \left[\left( \delta_{ik}p_i +(1-\delta_{ik}) \frac{\nu}{q_i-q_k} \right) \left( 
\delta_{kj} d_k-(1-\delta_{kj}) \frac{\nu}{(q_k-q_j)^2}\right)-
\right.\\-
\left.\left( \delta_{ik}d_i-(1-\delta_{ik})
	\frac{\nu}{(q_i-q_k)^2}\right) \left( \delta_{kj}p_k+(1-\delta_{kj})
\frac{\nu}{q_k-q_j}\right)\right]=\\=
\delta_{ij}p_id_i-
p_i(1-\delta_{ij}) \frac{\nu}{(q_i-q_j)^2}
+d_j(1-\delta_{ij}) \frac{\nu}{q_i-q_j}-
\nu^2\sum_{k\neq i,k \neq j}^{} \frac{1}{
(q_i-q_k)(q_k-q_j)^2}-\\ 
-\delta_{ij}d_ip_i-
d_i(1-\delta_{ij}) \frac{\nu}{q_i-q_j}
+p_j(1-\delta_{ij}) \frac{\nu}{(q_i-q_j)^2}+
\nu^2 \sum_{k\neq i,k\neq j}^{} \frac{1}{(q_i-q_k)^2(q_k-q_j)}=
\\=(\delta_{ij}-1)\left((p_i-p_j) \frac{\nu}{(q_i-q_j)^2}+
\frac{\nu^2}{q_i-q_j}\left(\sum_{k\neq i}^{} \frac{1}{(q_i-q_k)^2}
-\sum_{k\neq j}^{} \frac{1}{(q_j-q_k)^2}\right)\right)-\\-
\nu^2 \sum_{k\neq i,k \neq j}^{}  
\frac{q_i+q_j-2q_k}{(q_i-q_k)^2(q_k-q_j)^2} 
.\end{multline*} 
Из последнего уравнения получаем
\[
	\dot{p}_i=2\nu^2 \sum_{k\neq i}^{} \frac{1}{(q_k-q_i)^3}
.\] 
Учтя, что
\begin{multline*}
	\sum_{k\neq i}^{} \frac{1}{(q_i-q_k)^2}-
	\sum_{k\neq j}^{} \frac{1}{(q_j-q_k)^2}=
	\frac{1}{(q_i-q_j)^2}-
	\frac{1}{(q_j-q_i)^2}+\\+
	\sum_{k\neq i,k\neq j}^{} \frac{(q_j-q_i)(q_i+q_j-2q_k)}{(q_i-q_k)^2
	(q_j-q_k)^2}=
	\sum_{k\neq i,k\neq j}^{} \frac{(q_j-q_i)(q_i+q_j-2q_k)}{(q_i-q_k)^2
	(q_j-q_k)^2}
,\end{multline*} 
также получаем, что
\[
\dot{q}_k=p_k
.\] 
%\begin{multline*}
%\delta_{ij}\dot{p}_i+
%(\delta_{ij}-1) \frac{\nu \left(\dot{q_i}-\dot{q}_j\right)}{(q_i-q_j)^2}=
%\left( \delta_{ik}p_i +(1-\delta_{ik}) \frac{\nu}{q_i-q_k} \right) \left( 
%\delta_{kj} d_k-(1-\delta_{kj}) \frac{\nu}{(q_k-q_j)^2}\right)-
%\\-
%\left( \delta_{ik}d_i+(1-\delta_{ik})
%	\frac{\nu}{(q_i-q_k)^2}\right) \left( \delta_{kj}p_k+(1-\delta_{kj})
%\frac{\nu}{q_k-q_j}\right)
%.\end{multline*}
%\begin{multline*}
%\delta_{ij}\dot{p}_i+
%(\delta_{ij}-1) \frac{\nu \left(\dot{q_i}-\dot{q}_j\right)}{(q_i-q_j)^2}=
%\left( \delta_{ik}p_i +(1-\delta_{ik}) \frac{\nu}{q_i-q_k} \right) \left(
%(\delta_{kj}-1) \frac{\nu}{(q_k-q_j)^2}\right)-
%\\-
%\left( \delta_{kj}p_k+(1-\delta_{ji})
%\frac{\nu}{q_j-q_i}\right) \frac{\partial }{\partial q_i} 
%\left((1-\delta_{ij})
%\frac{\nu}{q_i-q_j}\right) 
%.\end{multline*}
%Для $i=j$:
%\[
%\dot{L}_{ii}= L_{ij}M_{ji}-M_{ij}L_{ji}
%.\] 
\end{sol}
\end{document}
