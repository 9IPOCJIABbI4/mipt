\documentclass[a4paper]{article}
% Этот шаблон документа разработан в 2014 году
% Данилом Фёдоровых (danil@fedorovykh.ru) 
% для использования в курсе 
% <<Документы и презентации в \LaTeX>>, записанном НИУ ВШЭ
% для Coursera.org: http://coursera.org/course/latex .
% Исходная версия шаблона --- 
% https://www.writelatex.com/coursera/latex/5.3

% В этом документе преамбула

\usepackage{siunitx}
%%% Работа с русским языком
\usepackage{cmap}					% поиск в PDF
\usepackage{mathtext} 				% русские буквы в формулах
\usepackage[T2A]{fontenc}			% кодировка
\usepackage[utf8]{inputenc}			% кодировка исходного текста
\usepackage[english,russian]{babel}	% локализация и переносы
\usepackage{indentfirst}
\frenchspacing

\renewcommand{\epsilon}{\ensuremath{\varepsilon}}
\renewcommand{\phi}{\ensuremath{\varphi}}
\renewcommand{\kappa}{\ensuremath{\varkappa}}
\renewcommand{\le}{\ensuremath{\leqslant}}
\renewcommand{\leq}{\ensuremath{\leqslant}}
\renewcommand{\ge}{\ensuremath{\geqslant}}
\renewcommand{\geq}{\ensuremath{\geqslant}}
\renewcommand{\emptyset}{\varnothing}
\renewcommand{\Im}{\operatorname{Im}}
\renewcommand{\Re}{\operatorname{Re}}


%%% Дополнительная работа с математикой
\usepackage{amsmath,amsfonts,amssymb,amsthm,mathtools} % AMS
\usepackage{icomma} % "Умная" запятая: $0,2$ --- число, $0, 2$ --- перечисление

%% Номера формул
%\mathtoolsset{showonlyrefs=true} % Показывать номера только у тех формул, на которые есть \eqref{} в тексте.
%\usepackage{leqno} % Нумереация формул слева

%% Свои команды
\DeclareMathOperator{\sgn}{\mathop{sgn}}
\DeclareMathOperator{\sign}{\mathop{sign}}
\DeclareMathOperator*{\res}{\mathop{res}}
\DeclareMathOperator*{\tr}{\mathop{tr}}

%% Перенос знаков в формулах (по Львовскому)
\newcommand*{\hm}[1]{#1\nobreak\discretionary{}
{\hbox{$\mathsurround=0pt #1$}}{}}

%%% Работа с картинками
\usepackage{graphicx}  % Для вставки рисунков
\graphicspath{{figures/}}  % папки с картинками
\setlength\fboxsep{3pt} % Отступ рамки \fbox{} от рисунка
\setlength\fboxrule{1pt} % Толщина линий рамки \fbox{}
\usepackage{wrapfig} % Обтекание рисунков текстом

%%% Работа с таблицами
\usepackage{array,tabularx,tabulary,booktabs} % Дополнительная работа с таблицами
\usepackage{longtable}  % Длинные таблицы
\usepackage{multirow} % Слияние строк в таблице

%%% Теоремы
\theoremstyle{plain} % Это стиль по умолчанию, его можно не переопределять.
\newtheorem{theorem}{Теорема}
\newtheorem*{thm}{Теорема}
\newtheorem{prop}{Утверждение}
 
\theoremstyle{definition} % "Определение"
%\newtheorem{corollary}{Следствие}[theorem]
\newtheorem*{dfn}{Определение}
\newtheorem{problem}{Задача}
\newtheorem*{problem*}{Задача}

 
\theoremstyle{remark} % "Примечание"
\newtheorem*{sol}{Решение}
\newtheorem*{rem}{Замечание}

%%% Программирование
\usepackage{etoolbox} % логические операторы

%%% Страница
%\usepackage{extsizes} % Возможность сделать 14-й шрифт
%\usepackage{geometry} % Простой способ задавать поля
%	\geometry{top=25mm}
%	\geometry{bottom=35mm}
%	\geometry{left=35mm}
%	\geometry{right=20mm}
 
\usepackage{fancyhdr} % Колонтитулы
%	\pagestyle{fancy}
 %	\renewcommand{\headrulewidth}{0pt}  % Толщина линейки, отчеркивающей верхний колонтитул
	%\lfoot{Нижний левый}
	%\rfoot{Нижний правый}
	%\rhead{Верхний правый}
	%\chead{Верхний в центре}
	%\lhead{Верхний левый}
	%\cfoot{Нижний в центре} % По умолчанию здесь номер страницы

\usepackage{setspace} % Интерлиньяж
%\onehalfspacing % Интерлиньяж 1.5
%\doublespacing % Интерлиньяж 2
%\singlespacing % Интерлиньяж 1

\usepackage{lastpage} % Узнать, сколько всего страниц в документе.

\usepackage{soul} % Модификаторы начертания

\usepackage{hyperref}
%\usepackage[usenames,dvipsnames,svgnames,table,rgb]{xcolor}
\hypersetup{				% Гиперссылки
    unicode=true,           % русские буквы в раздела PDF
    pdftitle={Заголовок},   % Заголовок
    pdfauthor={Автор},      % Автор
    pdfsubject={Тема},      % Тема
    pdfcreator={Создатель}, % Создатель
    pdfproducer={Производитель}, % Производитель
    pdfkeywords={keyword1} {key2} {key3}, % Ключевые слова
    colorlinks=true,       	% false: ссылки в рамках; true: цветные ссылки
    linkcolor=red,          % внутренние ссылки
    citecolor=black,        % на библиографию
    filecolor=magenta,      % на файлы
    urlcolor=cyan           % на URL
}

\usepackage{csquotes} % Еще инструменты для ссылок

%\usepackage[style=apa,maxcitenames=2,backend=biber,sorting=nty]{biblatex}

\usepackage{multicol} % Несколько колонок

\usepackage{tikz} % Работа с графикой
\usepackage{pgfplots}
\usepackage{pgfplotstable}
%\usepackage{coloremoji}
\usepackage{floatrow}
\usepackage{subcaption}
\newcommand*{\N}{\mathbb{N}}
\newcommand*{\R}{\mathbb{R}}
\newcommand*{\K}{\mathbb{K}}
\newcommand*{\V}{\mathcal{V}}
\newcommand*{\A}{\mathcal{A}}
\newcommand*{\ii}{\mathbf{1}}
\newcommand*{\oo}{\mathbf{0}}
\newcommand*{\ba}{\mathbf{a}}
\newcommand*{\bb}{\mathbf{b}}
\newcommand*{\Q}{\mathbb{Q}}
\graphicspath{{figures/}}
%\usepackage{breqn}

\renewcommand\thesubfigure{\asbuk{subfigure}}
%\addbibresource{master.bib}

\usepackage{import}
\usepackage{pdfpages}
\usepackage{transparent}
\usepackage{xcolor}
\usepackage{xifthen}

%\newcommand{\incfig}[1]{%
%    \def\svgwidth{\columnwidth}
%    \import{./figures/}{#1.pdf_tex}
%}


\newcommand{\incfig}[2][1]{%
    \def\svgwidth{#1\columnwidth}
    \import{./figures/}{#2.pdf_tex}
}
\usepackage{titlesec}
%\titleformat{\section}{\normalfont\Large\bfseries}{}{0pt}{}
%----------------------STANDART:
%\titleformat{\chapter}[display]
%  {\normalfont\huge\bfseries}{\chaptertitlename\ \thechapter}{20pt}{\Huge}
%\titleformat{\section}{\normalfont\Large\bfseries}{\thesection}{1em}{}
%\titleformat{\subsection}
%  {\normalfont\large\bfseries}{\thesubsection}{1em}{}
%\titleformat{\subsubsection}
%  {\normalfont\normalsize\bfseries}{\thesubsubsection}{1em}{}
%\titleformat{\paragraph}[runin]
%  {\normalfont\normalsize\bfseries}{\theparagraph}{1em}{}
%\titleformat{\subparagraph}[runin]
%  {\normalfont\normalsize\bfseries}{\thesubparagraph}{1em}{}

\pdfsuppresswarningpagegroup=1
\pgfplotsset{compat=1.16}

\usepackage{xifthen}
\makeatother
%\def\@lecture{}%
%\newcommand{\lecture}[3]{
%    \ifthenelse{\isempty{#3}}{%
%        \def\@lecture{Неделя #1}%
%    }{%
%        \def\@lecture{Неделя #1: #3}%
%    }%
%    \section*{\@lecture}
%    \marginpar{\small\textsf{\mbox{#2}}}
%}
\makeatletter

%\newcommand{\lec}{\subsection{Лекция}}
%\newcommand{\sem}{\subsection{Семинар}}
%\newcommand{\hw}{\subsection{Домашняя работа}}
%\newcommand{\prob}[1]{\textbf{#1}}
%\renewcommand{\thesubsection}{}
%\renewcommand{\thesubsubsection}{}

%\setcounter{tocdepth}{1} % only parts,chapters,sections
%\titleformat{\subsection}{\normalfont\large\bfseries}{}{0em}{}
%\titleformat{\subsubsection}{\normalfont\normalsize\bfseries}{}{0em}{}

%\newcommand{\textover}[2]{\stackrel{\mathclap{\normalfont\mbox{#2}}}{#1}}

\author{Драчов Ярослав\\
Факультет общей и прикладной физики МФТИ}
\newcommand{\veq}{\mathrel{\rotatebox{90}{$=$}}}
%\newcommand{\teto}[1]{\stackrel{\mathclap{\normalfont\tiny\mbox{#1}}}{\to}}
%\renewcommand{\thesubsection}{\arabic{subsection}}

%%\setcounter{secnumdepth}{0}

\definecolor{tabblue}{RGB}{30, 119, 180}
\definecolor{taborange}{RGB}{255, 127, 15}
\definecolor{tabgreen}{RGB}{45, 160, 43}
\definecolor{tabred}{RGB}{214, 38, 40}
\definecolor{tabpurple}{RGB}{148, 103, 189}
\definecolor{tabbrown}{RGB}{140, 86, 76}
\definecolor{tabpink}{RGB}{227, 119, 193}
\definecolor{tabgray}{RGB}{127, 127, 127}
\definecolor{tabolive}{RGB}{188, 189, 33}
\definecolor{tabcyan}{RGB}{22, 190, 207}
\pgfplotscreateplotcyclelist{colorbrewer-tab}{
{tabblue},
{taborange},
{tabgreen},
{tabred},
{tabpurple},
{tabbrown},
{tabpink},
{tabgray},
{tabolive},
{tabcyan},
}
\usepackage{csvsimple}
\usepackage{extarrows}
%\renewcommand{\labelenumii}{\asbuk{enumii})}
%\renewcommand{\labelenumiv}{\Asbuk{enumiv}}
\newcommand{\prob}[1]{\subsubsection*{#1}}
\sisetup{output-decimal-marker = {,},separate-uncertainty = true,exponent-product = \cdot}

\usepackage{braket}
\usepackage{enumerate}
\usepackage{chngcntr}
%\counterwithin*{equation}{problem}
%\usepackage{bbold}

\newtheoremstyle{hiProb}% ⟨name ⟩ 
{3pt}% ⟨Space above ⟩1 
{3pt}% ⟨Space below ⟩1
{}% ⟨Body font ⟩
{}% ⟨Indent amount ⟩2
{\bfseries}% ⟨Theorem head font⟩
{.}% ⟨Punctuation after theorem head ⟩
{.5em}% ⟨Space after theorem head ⟩3
%{\thmname{#1} \thmnote{#3}}% ⟨Theorem head spec (can be left empty, meaning ‘normal’)⟩
{\thmnote{#3}}% ⟨Theorem head spec (can be left empty, meaning ‘normal’)⟩
\theoremstyle{hiProb} % "Определение"
%\newtheorem{hiProb}{Задача}
\newtheorem{hiProb}{}
\usepackage{mmacells}
\newcommand{\textover}[2]{\stackrel{\mathclap{\normalfont\scriptsize\mbox{#2}}}{#1}}
\usepackage{units}
\usepackage[math]{cellspace}%
\setlength\cellspacetoplimit{2pt}
\setlength\cellspacebottomlimit{2pt}

\title{Семинар №5}
\begin{document}
	\maketitle
	Производная Хироты
	\[
		f(x-y)g(x-y)= \sum_{i=0}^{\infty} \frac{1}{j!}\left( D_x^j f\cdot g \right) y^j
	.\] 
	\[
		D_x f\cdot g=  f(x) g(x) + \frac{\partial f}{\partial x}  y g(x) - f(x) y \frac{\partial g}{\partial x} +\ldots
	.\] 
	Уравнение КдФ в производных Хироты
	\[
		(4 D_t D_x -D_x^4) \tau \tau=0
	.\] 
	Обобщаем
	\[
		P(D_1,\, D_2,\ldots)\tau \tau=0
	.\] 
	Тривиальное решение: $\tau\equiv 1$.
	Ищем решение
\[
	\tau= 1+ \epsilon f_1 +o(\epsilon^2)
.\] 
В первом порядке по $\epsilon$ уравнение преобразуется к виду
\[
	P(\partial_1,\, \partial_2,\ldots) f_1=0
.\] 
\[
	k_i P (k_1,\,k_2,\ldots)=0
.\] 
\[
f_1= e^{k_1 x_1 + k_2 x_2 +\ldots}
.\] 
Для нескольких наборов(=солитонов?)
\[
k_1^{(1)},\ k_2^{(2)},\ldots\quad k_1^{(2)},\ k_2 ^{(2)}\ldots
\] 
\[
f_1= \sum_{j=1}^{n}  c_j e^{ k_1 ^{(j)} x_1+ k_2 ^{(j)} x_2}
.\] 
\[
	\epsilon:\quad 2P(\partial_1,\,\partial_2,\ldots)1\cdot f_1=0
.\] 
\[
	\tau=1+ C e^{2kx+2k^3 t} \text{ (КдФ)}
.\] 
Рассмотрим двухсолитонное решение
\[
n=2:\quad k_i^{(1)},\,k_i ^{(2)}
.\] 
\[
\tau= 1 +\epsilon \sum_{j=1}^{2} c_j e^{k_1 ^{(j)}x_1+ k_2 ^{(j)}x_2+
\ldots}+\epsilon^2 f_2 +o(\epsilon^2)
.\] 
\[
	\epsilon^2 :\quad P(\partial_1,\,\partial_2,\ldots)f_2+
	c_1 c_2 P(k_1 ^{(1)}-k_1 ^{(2)},\, k_2 ^{(1)}- k_2 ^{(2)},\ldots) e^{\left( k_1 ^{(1)}+ k_1 ^{(2)} \right) x_1 + \ldots}=0
.\] 
\[
	P\left( D_1,\,D_2,\,\ldots \right) \left( 
	1+\epsilon A +\epsilon^2 f_2\right)\left( 1+\epsilon A+
\epsilon^2 f_2\right) =0\ldots
\] 
\[
	f_2=- \frac{P\left( k_1 ^{(1)}-k_1 ^{(2)} \right) }{P
	\left( k_1 ^{(1)}+k_1 ^{(2)} \right) }c_1 c_2 e^{
\left( \left( k_1^{(1)}k_1 ^{(2)} \right) x_1+\ldots \right) }
.\] 
КдФ:
\[
	k^{(i)}=\left( 2k_i,\,2k_i ^3 \right) \implies
	f_2=  \frac{\left( k_1-k_2 \right) ^2}{(k_1+k_2)^2}c_1 c_2
	e^{2(k_1+k_2)x +2 \left( k_1 ^3+k_2^3 \right) t}
.\] 
Расширяем набор переменных по правилу
\[
x_1=,\quad x_3=t,\quad x_5,\ x_5\ldots
.\] 
Для $n$-солитонной системы
\[
c_1,\ldots,\, c_n\quad k_1 ,\ldots,\, k_n
.\] 
\[
	\xi(x,\,k)= \sum_{j=1}^{\infty} x_j k^j
.\] 
\[
\xi_i = 2 \sum_{j=1}^{\infty} k_i^{2j+1} x_{2j+1}=
\xi(x,\,k_i)- \xi(x,\,-k_i)
.\] 
\[
	a_{i i'}= \frac{\left( k_i - k_{i'} \right) ^2}{
	(k_i+k_{i'})^2}
.\] 
\[
I= \{1,\ldots,\,n\} 
.\] 
$n$-солитонное решение иерархии КдФ:
\[
	\tau(x_1,\,x_3,\ldots)= \sum_{J \subset I}^{\infty}
	\left( \prod_{i \in  J}^{} c_i  \right) 
	\left(  \prod_{\substack{i,i' \in J\\ i< i'}}^{}a_{i i'}  \right)  \exp \left( \sum_{i \in J}^{} \xi_i \right) 
.\] 
\begin{multline*}
n=3:\quad \tau= 1+ c_1 e^{\xi_1}+ c_2 e^{\xi_2} +c_3 e^{\xi_3}+
c_1 c_2 a_{12} e^{\xi_1+\xi_2}+ c_1 c_3 a_{13} e^{\xi_1+\xi_3}+\\+
c_2 c_3 a_{23} e^{\xi_2+\xi_3}+c_1 c_2 c_3 a_{12} a_{13} a_{23}
e^{\xi_1+\xi_2+\xi_3}
.\end{multline*} 
Уравнение КдФ:
\[
\tau= 1+c e^{2k_1 x_1+2 k_1^3 x_3}
.\]
Иерарх. КдФ:
\[
\tau=1+c_1 e^{2 \sum_{j=1}^{\infty} k_1^{2j+1}x_{2j+1}}
.\] 
\[
u=2 \frac{\partial ^2}{\partial x^2} \ln \tau=\frac{8 c k_1^2 e^{2 k_1 \left(k_1^2 t+x\right)}}{\left(c e^{2 k_1 \left(k_1^2
t+x\right)}+1\right)^2}= \frac{2e^{\frac{A}{2}}k_1^2}{\ch^2 \left( k_1 x +k_1 ^3 t + \frac{A}{2} \right) }
.\] 
\section*{Вершинные операторы}
$X$ --- не диф. оператор.
\[
	\frac{\partial }{\partial \epsilon } \tau(x_1,\,x_3,\ldots)=
	X \tau \left( x_1,\,x_3,\ldots \right) 
.\]  
\[
\tau_{n+1}=e^{\epsilon X} \tau_n
.\] 
Для параметра $k$ определим
\[
	X(k)= \exp \left( 2 \sum_{j=0}^{\infty} k^{2j+1} x_{2j+1} \right) \cdot \exp \left( -2 \sum_{j=0}^{\infty} \frac{1}{(2j+1) k^{2j+1}}\frac{\partial }{\partial _{2j+1}}  \right) 
.\] 
\[
	X(k) f\left( x_1,\,x_3,\ldots \right) =
	\exp \left(  2 \sum_{j=0}^{\infty}  k^{2j+1} x_{2j+1} \right) f\left( x_1 - \frac{2}{k},\,x_3 - \frac{2}{3k^3},\ldots \right) 
.\] 
Понадобится следующая лемма:
\[
	X(k_1) X(k_2)= \frac{(k_1-k_2)^2}{(k_1+k_2)^2} \exp
	\left( 2 \sum_{i=1}^{2} \sum_{j=0}^{\infty} k_{i}^{2j+1}x_{2j+1} \right) \exp \left(  - 2 \sum_{i=1}^{2}  \sum_{j=0}^{\infty} 
	\frac{1}{(2j+1) k_1^{2j+1}} \frac{\partial }{\partial x_{2j+1}} \right) 
.\] 
\[
	e^{c X(k)}= 1 +c X(k)
.\] 
\[
\tau_1=	e^{c X(k)}\cdot 1 \text{ --- 1-солит. реш.}
.\] 
\[
	\tau= e ^{c_1 X(k_1)}\ldots e^{c_n X(k_n)} \cdot 1 \text{ --- }n\text{ -солит. реш.}
.\] 
\[
	\left(  D_1^4 +3 D_2^2 - 4 D_1 D_3\right) \tau\tau=0
.\]
\[
	P\left( k_1,\,k_2,\,k_3 \right) = k_1^4+3 k_2^2 - 4 k_1 k_3
.\] 
\[
	\left( k_1,\,k_2,\,k_3 \right) =\left( 
	p-q,\, p^2-q^2,\,p^3-q^3\right) 
.\] 
Для наборов
\[
	k_i^{(1)},\ k_i ^{(2)}
.\] 
Получим
\[
	- \frac{D\left(k_1 ^{(1)}-k_1 ^{(2)},\ldots\right)}{P\left(k_1 ^{(1)}+
	k_1 ^{(2)},\ldots\right)}=
	\frac{(p_1-p_2)(q_1-q_2)}{(p_1-q_2)(q_1-p_2)}
.\] 
\[
	\xi_i = \sum_{j=1}^{\infty} \left( p_i^j q_i^j \right) x_j=
	\xi(x,\,p_i)- \xi(x,\,q_i)
.\] 
\[
	a_{ii'}= \frac{(p_i- p_{i'})(q_i -q_{i'})}{(p_i-
	q_{i'})(q_i- p_{i'})}
.\] 
\[
	X(p,\,q)= \exp \left(\sum_{j=1}^{\infty} (p^j-q^j)x_j\right) 
	\cdot \exp \left( - \sum_{j=1}^{\infty}  \frac{1}{j}
	\left( p^{-j}- q^{-j} \right) \frac{\partial }{\partial x_j} \right) 
.\] 
\[
	\tau= e^{c_1 X(p_1,\,q_1)} \cdots e^{c_n X\left( p_n,\,q_n \right) }\cdots 1
.\] 
\[
p^i =-q^i \implies \text{КдФ}
.\] 
\section*{Билинейное тождество}
\begin{thm}
	Для любых $x,\ x'$, $\xi:= \xi (x,\,k)$, $\xi' := \xi (x',\,k)$:
	\[
		\oint \frac{dk}{2\pi i} e^{\xi-\xi'}\tau \left( 
		x_1- \frac{1}{k},\, x_2 - \frac{1}{2k^2},\ldots\right) \tau \left( x'_1 + \frac{1}{k},\,x_2' + \frac{1}{2k^2},\ldots \right) =0
	.\] 
\end{thm}
\begin{proof}
\[
	\exp \left(  \xi \left( x_1 - \frac{1}{k},\ldots \right)  \right) = \exp \left( \sum_{j=1}^{\infty} \left( p_i ^j -q_i^j \right) 
	\left( x_j - \frac{1}{j k^j} \right) \right) =
	\frac{k-p_i}{k-q_i}e^{\xi_i}
.\] 
\end{proof}
Введём вспомогательные функции
\[
	w(x,\,k)= e^{\xi (x,\,k)}\left( 1+ \sum_{i=1}^{\infty} \frac{w_i}{k^i} \right) 
.\] 
 \[
	w^*(x,\,k)= e^{-\xi (x,\,k)}\left( 1+ \sum_{i=1}^{\infty} \frac{w^*_i}{k^i} \right) 
.\]
По теореме получаем
\[
	\oint \frac{dk}{2\pi i} w (x,\,k) w^* (x',\,k)
.\] 
\[
	\frac{\partial w}{\partial x_j} =B_j w,\quad B_j =\left( L^j \right) _{+}
.\] 
\begin{enumerate}
\item $\displaystyle  \forall Q \hookrightarrow \oint
	\frac{dk}{2\pi i} Q(w) w^*(x',\,k)=0$
\item $\displaystyle \tilde{w} (x,\,k)= e^{\xi (x,\,k)} \sum_{i=1}^{\infty} \frac{\tilde{w}_i}{k^i}$ 
	\[
		\oint \frac{dk}{ 2\pi i} \tilde{w} (x,\,k) w^*(x',\,k)=0 \implies \tilde{w}_i \equiv 0
	.\] 
	\[
		Q= \frac{\partial }{\partial x_j} - (L^j)_+
	.\] 
	\[
		Qw= \frac{\partial w}{\partial x_j}- L^j w + (L^j)_-
		w \text{  --- имеет вид 2}\implies Qw=0 \text{ и т.д.}
	.\] 
\end{enumerate}
\begin{multline*}
	0 = \oint \frac{dk}{2\pi i } \exp \left( 
	2 \sum_{j=1}^{\infty} k^j y_j\right) \tau
	\left( x_1+ y_1 - \frac{1}{k},\ldots \right) \tau\left( 
	x_1- y_1+ \frac{1}{k},\ldots\right) =\\=
	\oint \frac{dk}{2\pi i} \exp \left( 2 \sum_{i=1}^{\infty} k^i
	y_i\right) \exp \left( \sum_{j=1}^{\infty} \left( y_j- \frac{1}{j k_j} \right) D_j \right) \tau\tau
.\end{multline*} 
lknlknkljbkjb
\end{document}
