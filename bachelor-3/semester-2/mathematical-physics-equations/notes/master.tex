\documentclass[10pt,landscape,a4paper]{article}
% Этот шаблон документа разработан в 2014 году
% Данилом Фёдоровых (danil@fedorovykh.ru) 
% для использования в курсе 
% <<Документы и презентации в \LaTeX>>, записанном НИУ ВШЭ
% для Coursera.org: http://coursera.org/course/latex .
% Исходная версия шаблона --- 
% https://www.writelatex.com/coursera/latex/5.3

% В этом документе преамбула

\usepackage{siunitx}
%%% Работа с русским языком
\usepackage{cmap}					% поиск в PDF
\usepackage{mathtext} 				% русские буквы в формулах
\usepackage[T2A]{fontenc}			% кодировка
\usepackage[utf8]{inputenc}			% кодировка исходного текста
\usepackage[english,russian]{babel}	% локализация и переносы
\usepackage{indentfirst}
\frenchspacing

\renewcommand{\epsilon}{\ensuremath{\varepsilon}}
\renewcommand{\phi}{\ensuremath{\varphi}}
\renewcommand{\kappa}{\ensuremath{\varkappa}}
\renewcommand{\le}{\ensuremath{\leqslant}}
\renewcommand{\leq}{\ensuremath{\leqslant}}
\renewcommand{\ge}{\ensuremath{\geqslant}}
\renewcommand{\geq}{\ensuremath{\geqslant}}
\renewcommand{\emptyset}{\varnothing}
\renewcommand{\Im}{\operatorname{Im}}
\renewcommand{\Re}{\operatorname{Re}}


%%% Дополнительная работа с математикой
\usepackage{amsmath,amsfonts,amssymb,amsthm,mathtools} % AMS
\usepackage{icomma} % "Умная" запятая: $0,2$ --- число, $0, 2$ --- перечисление

%% Номера формул
%\mathtoolsset{showonlyrefs=true} % Показывать номера только у тех формул, на которые есть \eqref{} в тексте.
%\usepackage{leqno} % Нумереация формул слева

%% Свои команды
\DeclareMathOperator{\sgn}{\mathop{sgn}}
\DeclareMathOperator{\sign}{\mathop{sign}}
\DeclareMathOperator*{\res}{\mathop{res}}
\DeclareMathOperator*{\tr}{\mathop{tr}}

%% Перенос знаков в формулах (по Львовскому)
\newcommand*{\hm}[1]{#1\nobreak\discretionary{}
{\hbox{$\mathsurround=0pt #1$}}{}}

%%% Работа с картинками
\usepackage{graphicx}  % Для вставки рисунков
\graphicspath{{figures/}}  % папки с картинками
\setlength\fboxsep{3pt} % Отступ рамки \fbox{} от рисунка
\setlength\fboxrule{1pt} % Толщина линий рамки \fbox{}
\usepackage{wrapfig} % Обтекание рисунков текстом

%%% Работа с таблицами
\usepackage{array,tabularx,tabulary,booktabs} % Дополнительная работа с таблицами
\usepackage{longtable}  % Длинные таблицы
\usepackage{multirow} % Слияние строк в таблице

%%% Теоремы
\theoremstyle{plain} % Это стиль по умолчанию, его можно не переопределять.
\newtheorem{theorem}{Теорема}
\newtheorem*{thm}{Теорема}
\newtheorem{prop}{Утверждение}
 
\theoremstyle{definition} % "Определение"
%\newtheorem{corollary}{Следствие}[theorem]
\newtheorem*{dfn}{Определение}
\newtheorem{problem}{Задача}
\newtheorem*{problem*}{Задача}

 
\theoremstyle{remark} % "Примечание"
\newtheorem*{sol}{Решение}
\newtheorem*{rem}{Замечание}

%%% Программирование
\usepackage{etoolbox} % логические операторы

%%% Страница
%\usepackage{extsizes} % Возможность сделать 14-й шрифт
%\usepackage{geometry} % Простой способ задавать поля
%	\geometry{top=25mm}
%	\geometry{bottom=35mm}
%	\geometry{left=35mm}
%	\geometry{right=20mm}
 
\usepackage{fancyhdr} % Колонтитулы
%	\pagestyle{fancy}
 %	\renewcommand{\headrulewidth}{0pt}  % Толщина линейки, отчеркивающей верхний колонтитул
	%\lfoot{Нижний левый}
	%\rfoot{Нижний правый}
	%\rhead{Верхний правый}
	%\chead{Верхний в центре}
	%\lhead{Верхний левый}
	%\cfoot{Нижний в центре} % По умолчанию здесь номер страницы

\usepackage{setspace} % Интерлиньяж
%\onehalfspacing % Интерлиньяж 1.5
%\doublespacing % Интерлиньяж 2
%\singlespacing % Интерлиньяж 1

\usepackage{lastpage} % Узнать, сколько всего страниц в документе.

\usepackage{soul} % Модификаторы начертания

\usepackage{hyperref}
%\usepackage[usenames,dvipsnames,svgnames,table,rgb]{xcolor}
\hypersetup{				% Гиперссылки
    unicode=true,           % русские буквы в раздела PDF
    pdftitle={Заголовок},   % Заголовок
    pdfauthor={Автор},      % Автор
    pdfsubject={Тема},      % Тема
    pdfcreator={Создатель}, % Создатель
    pdfproducer={Производитель}, % Производитель
    pdfkeywords={keyword1} {key2} {key3}, % Ключевые слова
    colorlinks=true,       	% false: ссылки в рамках; true: цветные ссылки
    linkcolor=red,          % внутренние ссылки
    citecolor=black,        % на библиографию
    filecolor=magenta,      % на файлы
    urlcolor=cyan           % на URL
}

\usepackage{csquotes} % Еще инструменты для ссылок

%\usepackage[style=apa,maxcitenames=2,backend=biber,sorting=nty]{biblatex}

\usepackage{multicol} % Несколько колонок

\usepackage{tikz} % Работа с графикой
\usepackage{pgfplots}
\usepackage{pgfplotstable}
%\usepackage{coloremoji}
\usepackage{floatrow}
\usepackage{subcaption}
\newcommand*{\N}{\mathbb{N}}
\newcommand*{\R}{\mathbb{R}}
\newcommand*{\K}{\mathbb{K}}
\newcommand*{\V}{\mathcal{V}}
\newcommand*{\A}{\mathcal{A}}
\newcommand*{\ii}{\mathbf{1}}
\newcommand*{\oo}{\mathbf{0}}
\newcommand*{\ba}{\mathbf{a}}
\newcommand*{\bb}{\mathbf{b}}
\newcommand*{\Q}{\mathbb{Q}}
\graphicspath{{figures/}}
%\usepackage{breqn}

\renewcommand\thesubfigure{\asbuk{subfigure}}
%\addbibresource{master.bib}

\usepackage{import}
\usepackage{pdfpages}
\usepackage{transparent}
\usepackage{xcolor}
\usepackage{xifthen}

%\newcommand{\incfig}[1]{%
%    \def\svgwidth{\columnwidth}
%    \import{./figures/}{#1.pdf_tex}
%}


\newcommand{\incfig}[2][1]{%
    \def\svgwidth{#1\columnwidth}
    \import{./figures/}{#2.pdf_tex}
}
\usepackage{titlesec}
%\titleformat{\section}{\normalfont\Large\bfseries}{}{0pt}{}
%----------------------STANDART:
%\titleformat{\chapter}[display]
%  {\normalfont\huge\bfseries}{\chaptertitlename\ \thechapter}{20pt}{\Huge}
%\titleformat{\section}{\normalfont\Large\bfseries}{\thesection}{1em}{}
%\titleformat{\subsection}
%  {\normalfont\large\bfseries}{\thesubsection}{1em}{}
%\titleformat{\subsubsection}
%  {\normalfont\normalsize\bfseries}{\thesubsubsection}{1em}{}
%\titleformat{\paragraph}[runin]
%  {\normalfont\normalsize\bfseries}{\theparagraph}{1em}{}
%\titleformat{\subparagraph}[runin]
%  {\normalfont\normalsize\bfseries}{\thesubparagraph}{1em}{}

\pdfsuppresswarningpagegroup=1
\pgfplotsset{compat=1.16}

\usepackage{xifthen}
\makeatother
%\def\@lecture{}%
%\newcommand{\lecture}[3]{
%    \ifthenelse{\isempty{#3}}{%
%        \def\@lecture{Неделя #1}%
%    }{%
%        \def\@lecture{Неделя #1: #3}%
%    }%
%    \section*{\@lecture}
%    \marginpar{\small\textsf{\mbox{#2}}}
%}
\makeatletter

%\newcommand{\lec}{\subsection{Лекция}}
%\newcommand{\sem}{\subsection{Семинар}}
%\newcommand{\hw}{\subsection{Домашняя работа}}
%\newcommand{\prob}[1]{\textbf{#1}}
%\renewcommand{\thesubsection}{}
%\renewcommand{\thesubsubsection}{}

%\setcounter{tocdepth}{1} % only parts,chapters,sections
%\titleformat{\subsection}{\normalfont\large\bfseries}{}{0em}{}
%\titleformat{\subsubsection}{\normalfont\normalsize\bfseries}{}{0em}{}

%\newcommand{\textover}[2]{\stackrel{\mathclap{\normalfont\mbox{#2}}}{#1}}

\author{Драчов Ярослав\\
Факультет общей и прикладной физики МФТИ}
\newcommand{\veq}{\mathrel{\rotatebox{90}{$=$}}}
%\newcommand{\teto}[1]{\stackrel{\mathclap{\normalfont\tiny\mbox{#1}}}{\to}}
%\renewcommand{\thesubsection}{\arabic{subsection}}

%%\setcounter{secnumdepth}{0}

\definecolor{tabblue}{RGB}{30, 119, 180}
\definecolor{taborange}{RGB}{255, 127, 15}
\definecolor{tabgreen}{RGB}{45, 160, 43}
\definecolor{tabred}{RGB}{214, 38, 40}
\definecolor{tabpurple}{RGB}{148, 103, 189}
\definecolor{tabbrown}{RGB}{140, 86, 76}
\definecolor{tabpink}{RGB}{227, 119, 193}
\definecolor{tabgray}{RGB}{127, 127, 127}
\definecolor{tabolive}{RGB}{188, 189, 33}
\definecolor{tabcyan}{RGB}{22, 190, 207}
\pgfplotscreateplotcyclelist{colorbrewer-tab}{
{tabblue},
{taborange},
{tabgreen},
{tabred},
{tabpurple},
{tabbrown},
{tabpink},
{tabgray},
{tabolive},
{tabcyan},
}
\usepackage{csvsimple}
\usepackage{extarrows}
%\renewcommand{\labelenumii}{\asbuk{enumii})}
%\renewcommand{\labelenumiv}{\Asbuk{enumiv}}
\newcommand{\prob}[1]{\subsubsection*{#1}}
\sisetup{output-decimal-marker = {,},separate-uncertainty = true,exponent-product = \cdot}

\usepackage{braket}
\usepackage{enumerate}
\usepackage{chngcntr}
%\counterwithin*{equation}{problem}
%\usepackage{bbold}

\newtheoremstyle{hiProb}% ⟨name ⟩ 
{3pt}% ⟨Space above ⟩1 
{3pt}% ⟨Space below ⟩1
{}% ⟨Body font ⟩
{}% ⟨Indent amount ⟩2
{\bfseries}% ⟨Theorem head font⟩
{.}% ⟨Punctuation after theorem head ⟩
{.5em}% ⟨Space after theorem head ⟩3
%{\thmname{#1} \thmnote{#3}}% ⟨Theorem head spec (can be left empty, meaning ‘normal’)⟩
{\thmnote{#3}}% ⟨Theorem head spec (can be left empty, meaning ‘normal’)⟩
\theoremstyle{hiProb} % "Определение"
%\newtheorem{hiProb}{Задача}
\newtheorem{hiProb}{}
\usepackage{mmacells}
\newcommand{\textover}[2]{\stackrel{\mathclap{\normalfont\scriptsize\mbox{#2}}}{#1}}
\usepackage{units}
\usepackage[math]{cellspace}%
\setlength\cellspacetoplimit{2pt}
\setlength\cellspacebottomlimit{2pt}

\usepackage{tikz}
\usetikzlibrary{shapes,positioning,arrows,fit,calc,graphs,graphs.standard}
%\usepackage[nosf]{kpfonts}
%\usepackage[t1]{sourcesanspro}
%\usepackage[lf]{MyriadPro}
%\usepackage[lf,minionint]{MinionPro}
\usepackage{multicol}
\usepackage{wrapfig}
\usepackage[top=0mm,bottom=1mm,left=0mm,right=1mm]{geometry}
\usepackage[framemethod=tikz]{mdframed}
\usepackage{microtype}

\let\bar\overline

\definecolor{myblue}{cmyk}{1,.72,0,.38}

\def\firstcircle{(0,0) circle (1.5cm)}
\def\secondcircle{(0:2cm) circle (1.5cm)}

\colorlet{circle edge}{myblue}
\colorlet{circle area}{myblue!5}

\tikzset{filled/.style={fill=circle area, draw=circle edge, thick},
    outline/.style={draw=circle edge, thick}}

\pgfdeclarelayer{background}
\pgfsetlayers{background,main}

%\everymath\expandafter{\the\everymath \color{myblue}}
%\everydisplay\expandafter{\the\everydisplay \color{myblue}}

\renewcommand{\baselinestretch}{.8}
\pagestyle{empty}

\global\mdfdefinestyle{header}{%
linecolor=gray,linewidth=1pt,%
leftmargin=0mm,rightmargin=0mm,skipbelow=0mm,skipabove=0mm,
}

%\newcommand{\header}{
%\begin{mdframed}[style=header]
%\footnotesize
%\sffamily
%Шпаргалка~по~урматам
%\end{mdframed}
%}

\newcommand{\header}{
}

\makeatletter
\renewcommand{\section}{\@startsection{section}{1}{0mm}%
                                {.2ex}%
                                {.2ex}%x
                                {%\color{myblue}
			\sffamily\small\bfseries}}
\renewcommand{\subsection}{\@startsection{subsection}{1}{0mm}%
                                {.2ex}%
                                {.2ex}%x
                                {\sffamily\bfseries}}



\def\multi@column@out{%
   \ifnum\outputpenalty <-\@M
   \speci@ls \else
   \ifvoid\colbreak@box\else
     \mult@info\@ne{Re-adding forced
               break(s) for splitting}%
     \setbox\@cclv\vbox{%
        \unvbox\colbreak@box
        \penalty-\@Mv\unvbox\@cclv}%
   \fi
   \splittopskip\topskip
   \splitmaxdepth\maxdepth
   \dimen@\@colroom
   \divide\skip\footins\col@number
   \ifvoid\footins \else
      \leave@mult@footins
   \fi
   \let\ifshr@kingsaved\ifshr@king
   \ifvbox \@kludgeins
     \advance \dimen@ -\ht\@kludgeins
     \ifdim \wd\@kludgeins>\z@
        \shr@nkingtrue
     \fi
   \fi
   \process@cols\mult@gfirstbox{%
%%%%% START CHANGE
\ifnum\count@=\numexpr\mult@rightbox+2\relax
          \setbox\count@\vsplit\@cclv to \dimexpr \dimen@-1cm\relax
%\setbox\count@\vbox to \dimen@{\vbox to 1cm{\header}\unvbox\count@\vss}%
\else
      \setbox\count@\vsplit\@cclv to \dimen@
\fi
%%%%% END CHANGE
            \set@keptmarks
            \setbox\count@
                 \vbox to\dimen@
                  {\unvbox\count@
                   \remove@discardable@items
                   \ifshr@nking\vfill\fi}%
           }%
   \setbox\mult@rightbox
       \vsplit\@cclv to\dimen@
   \set@keptmarks
   \setbox\mult@rightbox\vbox to\dimen@
          {\unvbox\mult@rightbox
           \remove@discardable@items
           \ifshr@nking\vfill\fi}%
   \let\ifshr@king\ifshr@kingsaved
   \ifvoid\@cclv \else
       \unvbox\@cclv
       \ifnum\outputpenalty=\@M
       \else
          \penalty\outputpenalty
       \fi
       \ifvoid\footins\else
         \PackageWarning{multicol}%
          {I moved some lines to
           the next page.\MessageBreak
           Footnotes on page
           \thepage\space might be wrong}%
       \fi
       \ifnum \c@tracingmulticols>\thr@@
                    \hrule\allowbreak \fi
   \fi
   \ifx\@empty\kept@firstmark
      \let\firstmark\kept@topmark
      \let\botmark\kept@topmark
   \else
      \let\firstmark\kept@firstmark
      \let\botmark\kept@botmark
   \fi
   \let\topmark\kept@topmark
   \mult@info\tw@
        {Use kept top mark:\MessageBreak
          \meaning\kept@topmark
         \MessageBreak
         Use kept first mark:\MessageBreak
          \meaning\kept@firstmark
        \MessageBreak
         Use kept bot mark:\MessageBreak
          \meaning\kept@botmark
        \MessageBreak
         Produce first mark:\MessageBreak
          \meaning\firstmark
        \MessageBreak
        Produce bot mark:\MessageBreak
          \meaning\botmark
         \@gobbletwo}%
   \setbox\@cclv\vbox{\unvbox\partial@page
                      \page@sofar}%
   \@makecol\@outputpage
     \global\let\kept@topmark\botmark
     \global\let\kept@firstmark\@empty
     \global\let\kept@botmark\@empty
     \mult@info\tw@
        {(Re)Init top mark:\MessageBreak
         \meaning\kept@topmark
         \@gobbletwo}%
   \global\@colroom\@colht
   \global \@mparbottom \z@
   \process@deferreds
   \@whilesw\if@fcolmade\fi{\@outputpage
      \global\@colroom\@colht
      \process@deferreds}%
   \mult@info\@ne
     {Colroom:\MessageBreak
      \the\@colht\space
              after float space removed
              = \the\@colroom \@gobble}%
    \set@mult@vsize \global
  \fi}

\makeatother
\setlength{\parindent}{0pt}
\begin{document}
\small
\begin{multicols*}{5}
	\section*{Пространство $S'$}
Функции медленного роста:
\[
	|f(x)|\le C\left( 1+|x|^{2p} \right) 
.\] 
\[
	\left<\mathcal{P} \frac{1}{x-x_0},\,\phi(x) \right> =
	\operatorname{v.p.} \int\limits_{-\infty}^{\infty} \frac{\phi(x)}{x-x_0}dx 
.\] 
\[
	\mathcal{P} \frac{1}{\left( x-x_0 \right) ^n}
	=\frac{(-1)^{n-1}}{(n-1)!}
	\frac{d^{n-1}}{dx^{n-1}}
	\mathcal{P} \frac{1}{x-x_0}
.\] 
\[
	x^n\mathcal{P} \frac{1}{x^n}=1
.\] 
\begin{multline*}
	\left<\mathcal{P} \frac{1}{|x|},\phi(x) \right> =\\=
	\int\limits_{|x|<1}^{} \frac{\phi(x)-\phi(0)}{|x|}dx+
	\int\limits_{|x|\ge 1}^{} \frac{\phi(x)}{|x|}dx 
.\end{multline*} 
\[
	x \mathcal{P}\frac{1}{|x|}=\sign x
.\] 
\[
\left<\partial^\alpha f,\,\phi \right> =
(-1)^{|\alpha|}\left<f,\,\partial^\alpha \phi \right> 
.\] 
\[
	\frac{d}{dx}\ln |x|= \mathcal{P} \frac{1}{x}
.\] 
\[
	S'\left( \mathbb{R}^3 \right) :\left<\delta_{S_a} (x),\,\phi(x) \right> =
	\int\limits_{|x|=a}^{} \phi(x)dS_x 
.\] 
\section*{Преобразование Фурье}
\[
F[\phi(x)](y)=
\int\limits_{\mathbb{R}^m}^{} e^{i(x,\,y)}
\phi(x) dx
.\] 
\[
	\left<F[f(x)](y),\,\phi(y) \right> 
	=\left<f(x),\,F[\phi(y)](x) \right> 
.\] 
\[
	F^{-1}[\phi(x)](y)
	=\frac{1}{\left( 2\pi \right) ^{m}}
	\int\limits_{\mathbb{R}^m}^{} 
	e^{-\left( x,\,y \right) }\phi(x)
	dx
.\] 
\[
	\left<F^{-1}
	[f(x)](y),\,\phi(y)\right> =
	\left<f(x),\,F^{-1}[\phi(y)](x) \right> 
.\] 
\[
	F^{-1}[F[f](y)](x)=
	F\left[ F^{-1}[f](y) \right] (x)= f(x)
.\] 
\begin{multline*}
	F^{-1}[f(x)](y)=
	\frac{1}{(2\pi)^{m}}
	F\left[ f(x) \right] 
	(-y)=\\= \frac{1}{\left( 2\pi \right) ^m}
	F\left[ f(-x) \right] (y)
.\end{multline*} 
\[
	\partial^{\alpha} F[f(x)](y)
	=F\left[ \left( ix \right) ^{\alpha}
	f(x)\right] (y)
.\] 
\[
	F\left[ \partial^\alpha f(x) \right] 
	(y)= \left( -iy \right) ^\alpha
	F[f(x)](y)
.\] 
\[
	F\left[ f(x-x_0) \right] (y)
	=e^{i(y,\,x_0)}
	F[f(x)](y)
.\] 
\[
F^{-1}\left[ 
f(x-x_0)\right] (y)=
e^{-i \left( y,\,x_0 \right) }
F^{-1}\left[ f(x) \right] (y)
.\] 
\[
	F\left[ \delta(x-a) \right] (y)
	=e^{iay}
.\] 
\[
	F[\delta(x)](y)=1
.\] 
\[
	F^{-1}[1](y)=\delta(y)
.\] 
\[
	F^{-1}\left[e^{iax}\right]
	(y)=\delta(y-a)
.\] 
\begin{multline*}
	F\left[e^{iax}\right]
	(y)= 2\pi F^{-1}
	\left[ e^{-iax} \right] (y)
	=\\=2\pi \delta(y+a)
.\end{multline*} 
\[
	F[1](y)=2\pi \delta (y)
.\] 
\[
	F^{-1} \left[ \delta(x-a) \right] (y)
	=\frac{1}{2\pi} e^{-iay}
.\] 
\[
	F[\sin x](y)=
	\pi i \left( 
	\delta(y+1)-\delta(y-1)\right) 
.\] 
\[
	F[\sign (x)](y)
	=2i \mathcal{P} \frac{1}{y}
.\] 
\[
	F\left[ \mathcal{P} \frac{1}{x} \right] 
	(y)=\pi i \sign y
.\] 
\[
	F[\theta (x)](y)= i \mathcal{P} \frac{1}{y}
	+\pi \delta (y)
.\] 
\[
	F\left[ |x| \right] (y)=-2 \mathcal{P}
	\frac{1}{y^2}
.\] 
\begin{multline*}
	F\left[ \frac{1}{x+b+ia} \right] (y)
	=\\= -2\pi i \sign (a) \theta (-ay) e^{ay-iby}
.\end{multline*} 
\[
	S'\left( \mathbb{R}^3 \right) :F\left[ \frac{1}{|x|} \right] (y)=
	\frac{4\pi}{|y|^2}
.\] 
\[
	F\left[ \delta_{S_a}(x) \right] (y)=
	\frac{4\pi a }{
	|y|}\sin \left( a |y| \right) 
.\]
\[
	S'\left(\mathbb{R}^3\right):
	F\left[|x|\right](y)=
	-4\pi \Delta \left( \frac{1}{|y|^2} \right) 
.\] 
\[
	S'\left( \mathbb{R}^3 \right) :F[1](y)= (2\pi)^m \delta(y)
.\] 
\begin{multline*}
	S'\left( \mathbb{R}^m \right)\to  S'\left( 
	\mathbb{R}^n\right) :\\\left<f\left( Ax+b \right) ,\,\phi(x) \right> =\\=\left<F^{-1}[f](y),\,e^{i\left( b,\,y \right) }F[\phi(x)]\left( A^Ty \right)  \right> 
.\end{multline*} 
\begin{multline*}
	S'\left( \mathbb{R}^n \right) :
	\left<f\left( Ax+b \right) ,\,\phi(x) \right> =\\=
	\left<f(y),\, \frac{\phi\left( A^{-1}(y-b) \right) }{|\det A|} \right> 
.\end{multline*} 
\section*{Функции Грина}
%\begin{table}[h]
%\caption{}
%\label{tab:1}
%\begin{tabular}{|c|c|}
%	\hline $\frac{\partial ^2 u}{\partial t^2} 
%	-a^2 \frac{\partial ^2 u}{\partial x^2} $ &  $\frac{\theta\left( at-|x| \right) }{2a}$   \\
%	\hline  $\frac{\partial ^2 u}{\partial t^2} -a^2 \Delta u$& $\frac{\delta\left( at-|x| \right) }{4\pi a^2 t}$  \\
%	\hline 
%\end{tabular}
%\end{table}
\begin{align*}
	&\text{ЛЧ} &&\mathcal{E}_a(t,\,x) \\
	&\frac{\partial ^2 u}{\partial t^2} -
	a^2 \frac{\partial ^2 u}{\partial x^2} 
	&&\frac{\theta\left( at-|x| \right) }{2a}\\
	&\frac{\partial ^2 u}{\partial t^2} -
	a^2 \Delta u
	&& \frac{\delta\left( at-|x| \right) }{4\pia^2 t}\\
	&\Delta u-a^2 u
	&& -\frac{e^{-a|x|}}{4\pi|x|}\\
	& \Delta u
	&& -\frac{1}{4\pi |x|}\\
	& \frac{\partial u}{\partial t} 
	-a^2 \Delta u
	&& \frac{\theta(t)}{\left( 2a \sqrt{\pi} 
	t\right) ^n}e^{- \frac{|x|^2}{4a^2 t}}\\
.\end{align*}
\section*{Задачи Коши}
\begin{multline*}
	\left<\left( f\left( t,\,x \right) 
	\right) *\left( \mathcal{E}_a \left( t,\,x \right)  \right),\,\phi(t,\,x) \right> =\\=
	\lim_{R \to +\infty} 
	\left< f\left( t,\,x \right) ,\,
	\eta_{1} \left( \frac{t}{R},\,
	\frac{x}{R}\right) \right. \times \\ \times \left.
	\left<\mathcal{E}_a\left( \tau,\,y \right) 
	,\,\phi\left( t+\tau,\,x+y \right) \right> \right> 
.\end{multline*}
Формула Даламбера для
одномерного
волнового уравнения с
правой частью специального вида
\[
	\delta'(t)u_0(x)+\delta(t)u_1(x)
,\]
где $u_0(x),\ u_1(x)$ --- медленного роста:
\begin{multline*}
	u(t,\,x)= \frac{\theta(t)}{2}\left( u_0
	(x+at)+u_0(x-at)\right) 
	+\\+\frac{\theta(t)}{2a}
	\int\limits_{x-at}^{x+at}  
	u_1\left( \xi \right) d\xi
.\end{multline*} 
Формула Кирхгоффа для
трёхмерного ВУ -//-:
\begin{multline*}
	u(t,x)= \frac{\partial }{\partial t} 
	\left( \frac{\theta(t)}{4\pi a^2 t}
	\int\limits_{|z-x|=at}^{} u_0(z)d S_z \right) +\\+
	\frac{\theta(t)}{4\pi a^2 t}
	\int\limits_{|z-x|=at}^{} 
	u_1(z)d S_z
.\end{multline*} 
\[
	\left<\mathcal{E}(t,\,x),\,\phi(t,\,x) \right> =\frac{1}{4\pi a^2} \int\limits_{\mathbb{R}^3}^{} \frac{\phi\left( \frac{|x|}{a},\,x \right) }{|x|}dx 
.\] 
\[
	f(x)* \left( -\frac{1}{4\pi |x|} \right) 
	=-\frac{1}{4\pi} \int\limits_{\mathbb{R}^3}^{} \frac{f(y)}{|x-y|} dy
.\] 
\vfill\null
%\columnbreak
\end{multicols*}
 
\end{document}
