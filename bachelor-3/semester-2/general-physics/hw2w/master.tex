\documentclass[a4paper]{article}
% Этот шаблон документа разработан в 2014 году
% Данилом Фёдоровых (danil@fedorovykh.ru) 
% для использования в курсе 
% <<Документы и презентации в \LaTeX>>, записанном НИУ ВШЭ
% для Coursera.org: http://coursera.org/course/latex .
% Исходная версия шаблона --- 
% https://www.writelatex.com/coursera/latex/5.3

% В этом документе преамбула

\usepackage{siunitx}
%%% Работа с русским языком
%\usepackage{cmap}					% поиск в PDF
%\usepackage{mathtext} 				% русские буквы в формулах
%\usepackage[T2A]{fontenc}			% кодировка
%\usepackage[utf8]{inputenc}			% кодировка исходного текста
%\usepackage[english,russian]{babel}	% локализация и переносы
%\usepackage{indentfirst}
%\frenchspacing
%
%\renewcommand{\epsilon}{\ensuremath{\varepsilon}}
%\newcommand{\phibackup}{\ensuremath{\phi}}
%\renewcommand{\phi}{\ensuremath{\varphi}}
%\renewcommand{\varphi}{\ensuremath{\phibackup}}
%\renewcommand{\kappa}{\ensuremath{\varkappa}}
%\renewcommand{\le}{\ensuremath{\leqslant}}
%\renewcommand{\leq}{\ensuremath{\leqslant}}
%\renewcommand{\ge}{\ensuremath{\geqslant}}
%\renewcommand{\geq}{\ensuremath{\geqslant}}
%\renewcommand{\emptyset}{\varnothing}
%\renewcommand{\Im}{\operatorname{Im}}
%\renewcommand{\Re}{\operatorname{Re}}


%%% Дополнительная работа с математикой
\usepackage{amsmath,amsfonts,amssymb,amsthm,mathtools} % AMS
%\usepackage{icomma} % "Умная" запятая: $0,2$ --- число, $0, 2$ --- перечисление

%% Номера формул
%\mathtoolsset{showonlyrefs=true} % Показывать номера только у тех формул, на которые есть \eqref{} в тексте.
%\usepackage{leqno} % Нумереация формул слева

%% Свои команды
\DeclareMathOperator{\sgn}{\mathop{sgn}}
\DeclareMathOperator{\sign}{\mathop{sign}}
\DeclareMathOperator*{\res}{\mathop{res}}
\DeclareMathOperator*{\tr}{\mathop{tr}}
\DeclareMathOperator*{\rot}{\mathop{rot}}
\DeclareMathOperator*{\divop}{\mathop{div}}
\DeclareMathOperator*{\grad}{\mathop{grad}}

%% Перенос знаков в формулах (по Львовскому)
\newcommand*{\hm}[1]{#1\nobreak\discretionary{}
{\hbox{$\mathsurround=0pt #1$}}{}}

%%% Работа с картинками
\usepackage{graphicx}  % Для вставки рисунков
\graphicspath{{figures/}}  % папки с картинками
\setlength\fboxsep{3pt} % Отступ рамки \fbox{} от рисунка
\setlength\fboxrule{1pt} % Толщина линий рамки \fbox{}
\usepackage{wrapfig} % Обтекание рисунков текстом

%%% Работа с таблицами
\usepackage{array,tabularx,tabulary,booktabs} % Дополнительная работа с таблицами
\usepackage{longtable}  % Длинные таблицы
\usepackage{multirow} % Слияние строк в таблице

%%% Теоремы
\theoremstyle{plain} % Это стиль по умолчанию, его можно не переопределять.
\newtheorem{thm}{Теорема}
\newtheorem*{thm*}{Теорема}
\newtheorem{prop}{Предложение}
\newtheorem*{prop*}{Предложение}
 
\theoremstyle{definition} % "Определение"
%\newtheorem{corollary}{Следствие}[theorem]
\newtheorem{dfn}{Определение}
\newtheorem*{dfn*}{Определение}
\newtheorem{prob}{Задача}
\newtheorem*{prob*}{Задача}

 
\theoremstyle{remark} % "Примечание"
\newtheorem*{sol}{Решение}
\newtheorem*{rem}{Замечание}

%%% Программирование
\usepackage{etoolbox} % логические операторы

%%% Страница
%\usepackage{extsizes} % Возможность сделать 14-й шрифт
%\usepackage{geometry} % Простой способ задавать поля
%	\geometry{top=25mm}
%	\geometry{bottom=35mm}
%	\geometry{left=35mm}
%	\geometry{right=20mm}
 
\usepackage{fancyhdr} % Колонтитулы
%	\pagestyle{fancy}
 %	\renewcommand{\headrulewidth}{0pt}  % Толщина линейки, отчеркивающей верхний колонтитул
	%\lfoot{Нижний левый}
	%\rfoot{Нижний правый}
	%\rhead{Верхний правый}
	%\chead{Верхний в центре}
	%\lhead{Верхний левый}
	%\cfoot{Нижний в центре} % По умолчанию здесь номер страницы

\usepackage{setspace} % Интерлиньяж
%\onehalfspacing % Интерлиньяж 1.5
%\doublespacing % Интерлиньяж 2
%\singlespacing % Интерлиньяж 1

\usepackage{lastpage} % Узнать, сколько всего страниц в документе.

\usepackage{soul} % Модификаторы начертания

\usepackage{hyperref}
\usepackage[usenames,dvipsnames,svgnames,table,rgb]{xcolor}
\hypersetup{				% Гиперссылки
    unicode=true,           % русские буквы в раздела PDF
    pdftitle={Заголовок},   % Заголовок
    pdfauthor={Автор},      % Автор
    pdfsubject={Тема},      % Тема
    pdfcreator={Создатель}, % Создатель
    pdfproducer={Производитель}, % Производитель
    pdfkeywords={keyword1} {key2} {key3}, % Ключевые слова
%    colorlinks=true,       	% false: ссылки в рамках; true: цветные ссылки
    %linkcolor=red,          % внутренние ссылки
    %citecolor=black,        % на библиографию
    %filecolor=magenta,      % на файлы
    %urlcolor=cyan           % на URL
}

\usepackage{csquotes} % Еще инструменты для ссылок

%\usepackage[style=apa,maxcitenames=2,backend=biber,sorting=nty]{biblatex}

\usepackage{multicol} % Несколько колонок

\usepackage{tikz} % Работа с графикой
\usepackage{pgfplots}
\usepackage{pgfplotstable}
%\usepackage{coloremoji}
\usepackage{floatrow}
\usepackage{subcaption}
\graphicspath{{figures/}}

\renewcommand\thesubfigure{\asbuk{subfigure}}
%\addbibresource{master.bib}

\usepackage{import}
\usepackage{pdfpages}
\usepackage{transparent}
\usepackage{xcolor}
\usepackage{xifthen}

\newcommand{\incfig}[2][1]{%
    \def\svgwidth{#1\columnwidth}
    \import{./figures/}{#2.pdf_tex}
}
%\usepackage{titlesec}
%\titleformat{\section}{\normalfont\Large\bfseries}{}{0pt}{}
%----------------------STANDART:
%\titleformat{\chapter}[display]
%  {\normalfont\huge\bfseries}{\chaptertitlename\ \thechapter}{20pt}{\Huge}
%\titleformat{\section}{\normalfont\Large\bfseries}{\thesection}{1em}{}
%\titleformat{\subsection}
%  {\normalfont\large\bfseries}{\thesubsection}{1em}{}
%\titleformat{\subsubsection}
%  {\normalfont\normalsize\bfseries}{\thesubsubsection}{1em}{}
%\titleformat{\paragraph}[runin]
%  {\normalfont\normalsize\bfseries}{\theparagraph}{1em}{}
%\titleformat{\subparagraph}[runin]
%  {\normalfont\normalsize\bfseries}{\thesubparagraph}{1em}{}

\pdfsuppresswarningpagegroup=1
\pgfplotsset{compat=1.16}



%\setcounter{tocdepth}{1} % only parts,chapters,sections
%\titleformat{\subsection}{\normalfont\large\bfseries}{}{0em}{}
%\titleformat{\subsubsection}{\normalfont\normalsize\bfseries}{}{0em}{}

%\newcommand{\textover}[2]{\stackrel{\mathclap{\normalfont\mbox{#2}}}{#1}}

\author{Yaroslav Drachov\\
Moscow Institute of Physics and Technology}
%\author{Драчов Ярослав\\
%Факультет общей и прикладной физики МФТИ}
\newcommand{\veq}{\mathrel{\rotatebox{90}{$=$}}}
%\newcommand{\teto}[1]{\stackrel{\mathclap{\normalfont\tiny\mbox{#1}}}{\to}}
%\renewcommand{\thesubsection}{\arabic{subsection}}

%%\setcounter{secnumdepth}{0}

\definecolor{tabblue}{RGB}{30, 119, 180}
\definecolor{taborange}{RGB}{255, 127, 15}
\definecolor{tabgreen}{RGB}{45, 160, 43}
\definecolor{tabred}{RGB}{214, 38, 40}
\definecolor{tabpurple}{RGB}{148, 103, 189}
\definecolor{tabbrown}{RGB}{140, 86, 76}
\definecolor{tabpink}{RGB}{227, 119, 193}
\definecolor{tabgray}{RGB}{127, 127, 127}
\definecolor{tabolive}{RGB}{188, 189, 33}
\definecolor{tabcyan}{RGB}{22, 190, 207}
\pgfplotscreateplotcyclelist{colorbrewer-tab}{
{tabblue},
{taborange},
{tabgreen},
{tabred},
{tabpurple},
{tabbrown},
{tabpink},
{tabgray},
{tabolive},
{tabcyan},
}
\usepackage{csvsimple}
\usepackage{extarrows}
%\renewcommand{\labelenumii}{\asbuk{enumii})}
%\renewcommand{\labelenumiv}{\Asbuk{enumiv}}
%\newcommand{\prob}[1]{\subsubsection*{#1}}
\sisetup{output-decimal-marker = {,},separate-uncertainty = true,exponent-product = \cdot}

\usepackage{braket}
\usepackage{enumerate}
\usepackage{chngcntr}
%\counterwithin*{equation}{problem}
%\usepackage{bbold}

\newtheoremstyle{hiProb}% ⟨name ⟩ 
{3pt}% ⟨Space above ⟩1 
{3pt}% ⟨Space below ⟩1
{}% ⟨Body font ⟩
{}% ⟨Indent amount ⟩2
{\bfseries}% ⟨Theorem head font⟩
{.}% ⟨Punctuation after theorem head ⟩
{.5em}% ⟨Space after theorem head ⟩3
%{\thmname{#1} \thmnote{#3}}% ⟨Theorem head spec (can be left empty, meaning ‘normal’)⟩
{\thmnote{#3}}% ⟨Theorem head spec (can be left empty, meaning ‘normal’)⟩
\theoremstyle{hiProb} % "Определение"
%\newtheorem{hiProb}{Задача}
\newtheorem{hiProb}{}
%\usepackage{mmacells}
\newcommand{\textover}[2]{\stackrel{\mathclap{\normalfont\scriptsize\mbox{#2}}}{#1}}
\usepackage{units}
\usepackage[math]{cellspace}%
\setlength\cellspacetoplimit{2pt}
\setlength\cellspacebottomlimit{2pt}

\DeclareMathAlphabet{\mathbbold}{U}{bbold}{m}{n}

\newcommand{\normord}[1]{:\mathrel{#1}:}

\title{Неделя №2\\
Теплоёмкость твёрдого тела.
Модель Дебая}
\begin{document}
	\maketitle
\begin{hiProb}[2.27]	
\end{hiProb}
\begin{sol}
\end{sol}
Первая зона Бриллюэна для квадратной решётки также
имеет форму квадрата со стороной $2\pi /a$. Максимальная частота
\[
	\omega_{\max}=\omega\left( \frac{\pi}{a},\,
	\frac{\pi}{a}\right) = 2\sqrt{2} 
	\sqrt{\frac{\gamma}{M}} 
.\] 
Длинноволновой предел
\[
	\omega^2= 2 \frac{\gamma}{M} \left( 
	2- \left( 1- \left( K_x a \right) ^2 /2 \right) - \left( 1-\left( K_y a \right) ^2 /2\right) \right) 
	=\frac{\left( Ka \right) ^2 \gamma}{M}
\] 
изотропен (от направления не зависит), скорость
звука $\displaystyle  s= a \sqrt{\frac{\gamma}{M}} $.

В модели Дебая заменяем спектр линейным изотропным
$\omega =sK$ и ограничиваем его сверху так, чтобы
полное число колебаний сохранилось:
\[
	N= \frac{S \pi k_D^2}{\left( 2\pi \right) ^2}
\]
где $N$ --- число атомов, а $S$ --- площадь решётки.
\[
k_D^2= \frac{4\pi}{a^2}
.\] 
\[
k_D= \frac{2\sqrt{\pi} }{a} \approx \frac{3,545}{a}
> \frac{\pi}{a} = k_{Br}
\] 
и для частоты
\[
\omega_D = \frac{2 \sqrt{\pi}s}{a}=
2 \sqrt{\pi}  \sqrt{\frac{\gamma}{M}} > \omega_{\max}
.\] 
\begin{hiProb}[Т2-1]
\end{hiProb}
\begin{sol}
Нужно, во-первых, перейти к теплоёмкости на примитивную
ячейку. У NaCl гранецентрированная решётка с четырьмя
формульными единицами на элементарный куб.
То есть, примитивная ячейка имеет объём 1/4 элементарного куба:
 \[
C _{\text{прим}}= \frac{d^3}{4}C
.\] 
Считаем, что температура достаточно низка для применения низкотемпературного приближения:
\[
C_\text{прим} = \frac{C_{\text{Дебай}}}{N}\approx
\frac{12}{5} \pi^4 k_B \left( \frac{T}{\Theta} \right) ^3
.\] 
\[
	\Theta= T \left( \frac{48 \pi^4}{5}
	\frac{k_B}{C d^3}\right) ^{1 /3} \approx
	315 \text{ К}
.\] 
Далее ищем скорость звука \[\Theta = \frac{\hbar s}{k_B}\left( 6 \pi^2 n \right) ^{1 /3}= \frac{\hbar s}{
k_B} \left( 6 \pi^2 \frac{4}{d^3} \right) ^{1 /3}=
\frac{2 \sqrt[3]{3\pi^2}\hbar s }{d k_B},\]
при вычислении дебаевской температуры не забываем, что
кубическая ячейка не примитивная.

Откуда окончательно для усреднённой скорости звука
\[
	s= \frac{k_B \Theta d}{2 \sqrt[3]{3\pi^2} \hbar }=3,76 \cdot 10^5 \frac{\text{см}}{\text{с}}
.\] 
\end{sol}
\begin{hiProb}[2.47]
\end{hiProb}
\begin{sol}
	При нулевых граничных условиях (закреплённая
	граница, т.\:е. амплитуда колебаний на
	границе равна нулю) смещение $U$ может
	быть записано в виде
	\[
	U= A e^{i \omega_n t} \sin 
	\frac{\pi x}{L} n_x \cdot \sin \frac{\pi y}{
	L} n_y \cdot \sin \frac{\pi z}{L} n_z
	,\]
	а частота $n$-го колебания $\omega_n$ 
	равна
	\[
		\omega_n^2 = \left( 
		\frac{\pi s}{L}\right) ^2
		\left( n_x^2 +n_y^2 +n_z^2 \right) 
	.\] 
	Числа $n_x,\ n_y,\ n_z$ принимают все целочисленные значения $\ge 1$. Энергия колебаний
	\[
		\mathcal{E}=3
		\sum_{n_x,n_y,n_z}^{} \frac{
		\hbar  \omega_n}{\exp \left( 
	\frac{\hbar  \omega_n }{k_\text{Б} T}-1\right) }
	,\] 
	где коэффициент  <<3>> --- это три
	независимые поляризации. При больших
	$L$ сумма может быть заменена интегралом
	\[
		\mathcal{E}(T)=
		\int\limits_{\omega_{\min}}^{\omega_{\max}} d\omega \mathcal{D} (\omega) n(\omega, T)
		\hbar  \omega,\quad
		\text{где }  \mathcal{D}(\omega)=
		\frac{3}{2} \frac{V \omega^2}{\pi^2 s^3}
	,\]
	откуда энергия единицы объёма кластера
	\[
		u_{\text{класт}}(T)=
		\frac{\mathcal{E}}{V}=
		\frac{3}{2} \frac{\hbar }{\pi
		^2 s^3} \int\limits_{\omega_{\min}}^{\omega_{\max}} \frac{\omega^3 d \omega}{e^{\hbar \omega /k_\text{Б} T}-1}= \frac{3}{2}
		\frac{(k_\text{Б}T)^4}{\pi
		^2 \hbar ^3 s^3}
		\int\limits_{x_{\min}}^{x_{\max}} 
		\frac{x^3 dx}{e^x-1}
	.\] 
	Здесь $\omega_{\max}= \frac{\pi s}{L}\left( 
	n_x^2 +n_y^2 +n_z^2\right) _{\max}\simeq
	\frac{\pi s}{L}N$, $L=Na$. Точное
	значение коэффициента при низких температурах
	несущественно.
	С другой стороны,
	\[
		\omega_{\min}= \frac{\pi s}{L} \left( 
		n_x^2+n_y^2+n_z^2\right) _{\min}=
		\begin{cases}
			0 & \text{при } L\to 0,\\
			\frac{\pi s}{L}\sqrt{3} &
			\text{при } L<\infty.
		\end{cases}
	\] 
	Кроме того, в интеграле была сделана стандартная замена переменныых
	\[
	x_{\max}= \frac{\hbar  \omega_{\max}}{k_\text{Б}T}= \frac{\pi s \hbar }{a k_\text{Б}T}=
	\frac{\theta}{T};\quad
	x_{\min}= \frac{\hbar \omega_{\min}}{k_\text{Б}T}= \frac{\pi s \sqrt{3} }{10a k_\text{Б}T}=
	\frac{\theta}{T} \frac{\sqrt{3} }{10}
	.\] 
Таким образом, энергия единицы объёма кластера
\[
	u_{\text{класт}}= \frac{3}{2} \frac{\left( 
	k_\text{Б}T\right) ^4}{\pi^2 \hbar ^3
s^3} \int\limits_{\frac{\theta}{T}\frac{\sqrt{3} }{10}}^{\theta /T} \frac{x^3 dx}{e^x-1} 
.\] 
Количество теплоты, необходимое для нагрева единицы
объёма кластера
\[
Q_{\text{класт}}= \Delta u_{\text{класт}}=
u_{\text{класт}} \left( \frac{\theta}{30} \right) -
u_\text{класт}(0)= \frac{3}{2} \frac{
\left( k_\text{Б}\frac{\theta}{30} \right)^4 }{
\pi^2 h^3 s^3} \int\limits_{3 \sqrt{3} }^{30} 
\frac{x^3 dx}{e^x-1}
.\] 
Для единицы объёма большого тела
\[
\Delta u_\text{тела}= \frac{3}{2\pi^2 \hbar ^3 s^3}
\left( \frac{k_\text{Б} \theta}{30} \right) ^4
\int\limits_{0}^{30} \frac{x^3 dx}{e^x -1} \text{ и
 тогда }  \frac{Q_\text{класт}}{Q_\text{тела}}=
\frac{\Delta u_{\text{класт}}}{\Delta u_\text{тела}}=
\frac{\int\limits_{3 \sqrt{3} }^{30}  \frac{x^3 dx}{e^x-1}}{\int\limits_{0}^{30} \frac{x^3 dx}{e^x -1} }
.\] 
В этих интегралах верхний предел можно
заменить на бесконечность, а поскольку
$
e^{3\sqrt{3} }\simeq 180 \gg 1, 
$, то
\[
\int\limits_{3\sqrt{3} }^{30} \frac{x^3 dx}{e^x -1}
\approx \int\limits_{3\sqrt{3} }^{\infty} e^{-x}
x^3 dx =1,43;\quad
\int\limits_{0}^{\infty} \frac{x^3 dx}{e^x-1}=
\frac{\pi^4}{15} \approx 6,5
.\] 
Отсюда \[
\frac{Q_\text{класт}}{Q_\text{тела}}=\frac{1,43}{
6,5}=0,22
.\] 

Если же непосредственно подсчитать сумму всех
возможных колебаний, то ограничиваясь числами
$(n_x,\,n_y,\,n_z)$: $(1,\,1,\,1)$; $(1,\,1,\,2)$;
$(1,\,2,\,1)$; $(2,\,1,\,1)$; $(2,\,2,\,1)$;
$(2,\,1,\,2)$; $(1,\,2,\,2)$; $(1,\,1,\,3)$;
$(1,\,3,\,1)$; $(3,\,1,\,1)$ и $(2,\,2,\,2)$,
получаем
\[
\frac{Q_\text{класт}}{Q_\text{тела}}=0,13
.\] 
\end{sol}
\begin{hiProb}[2.58]
\end{hiProb}
\begin{sol}
В условии даётся температура Дебая 19 К, смысл которой
для жидкости странен. Тут важно, что температура много
меньше ротонного минимума (который около 8 К),
поэтому есть только фононы, а их спектр до энергий,
соответствующий температуре
0,5 К нашей задачи, можно считать линейным.
В жидкости есть единственная поляризация
звуковых волн (продольная).
\[
	E= \frac{V}{(2\pi)^3} \int
	\frac{\hbar \omega}{\exp \left( \frac{\hbar \omega}{k_B T} \right) -1}d^3k= \frac{V}{2\pi^2}
	\frac{(k_B T)^4}{(\hbar s)^3} \int\limits_{0}^{\infty}  \frac{x^3}{e^x -1}dx=
	V \frac{\pi^2}{30} \frac{(k_B T)^4}{(\hbar 
	s)^3}
.\] 
\[
	\frac{C}{V}= \frac{2}{15} \pi^2 k_B \left( 
	\frac{k_B T}{\hbar s}\right) ^3
\] 
и далее по формулам термодинамики
\[
	\frac{S(T_1)}{V}= \int\limits_{0}^{T_1} 
	\frac{C}{T} dT= \frac{2}{45}
	\pi^2 k_B \left( \frac{k_B T_1}{\hbar  s} \right) ^3
\]
для искомой удельной энтропии
\[
S_\text{уд} = \frac{2}{45}\frac{\pi^2 k_B}{\rho}
\left( \frac{k_B T}{\hbar s} \right) ^3 \approx
8,5 \cdot 10^3 \frac{\text{эрг}}{\text{К}\cdot
\text{г}}
.\] 
\end{sol}
\begin{hiProb}[2.75]
\end{hiProb}
\begin{sol}
	Аналогично задаче $2.74$
	наложим закреплённые граничные условия.
	От граничных условий несколько меняется
	ответ в этой задаче, но по постановке
	<<мысленного опыта>> по определению
	конечной длины цепочки, для которой
	роль квантовых колебаний
	ещё неразрушительна, закреплённые
	граничные условия кажутся даже более
	логичными.

	Тогда собственные моды колебаний в одномерном
	случае имеют вид $u_k= A_{0k} \sin (kx)
	\sin (\omega t)$, $k>0$, на одно
	состояние приходится объём $\pi /L$ в одномерном
	$k$- пространстве. Для среднего квадрата
	смещения $\left<\left<u^2 \right>  \right> =
	\frac{1}{4} \sum_{}^{} A_{0k}^2$, а из
	сравнения каждой моды с гармоническим
	осциллятором
	\[
	\frac{\hbar  \omega}{2}= E_k =
	\left<E_k \right> = 2 \left<K_k \right> =
	M \sum_{n}^{} \left< V_n^2 \right> =
	\frac{MN \omega_k^2 A_{0k}^2}{4} \text{ и }
	A_{0k}^2= \frac{2\hbar }{MN\omega_k}
	.\] 
	В одномерном случае для единственной
	поляризации, в дебаевской модели,
	ограничивая нижний предел из-за конечности
	цепочки:
	\[
	\frac{\left<\left<u^2 \right>  \right> }{a^2}
	=\frac{L}{\pi} \frac{2 \hbar }{N Ma^2}
	\int\limits_{k_{\min}}^{k_D} \frac{dk}{sk}=
	\frac{2 \hbar }{\pi s Ma} \ln \frac{k_D}{k_{\min}}
	.\] 
	Для однородной цепочки $k_D= \pi /a$, а 
	для закреплённых граничных условий
	$k_{\min}= \pi /L$.

	Окончательно
	\[
	\alpha= \frac{2\hbar }{\pi s M a}\ln \frac{L}{a}
	.\] 
	\[
	\ln \frac{L}{a}= \alpha \frac{\pi s Ma}{2\hbar } \approx 11,2
	.\] 
	\[
	L \approx 22 \text{ мкм}
	.\] 
\end{sol}
\end{document}
