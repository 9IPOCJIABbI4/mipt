\documentclass[a4paper]{article}
% Этот шаблон документа разработан в 2014 году
% Данилом Фёдоровых (danil@fedorovykh.ru) 
% для использования в курсе 
% <<Документы и презентации в \LaTeX>>, записанном НИУ ВШЭ
% для Coursera.org: http://coursera.org/course/latex .
% Исходная версия шаблона --- 
% https://www.writelatex.com/coursera/latex/5.3

% В этом документе преамбула

\usepackage{siunitx}
%%% Работа с русским языком
%\usepackage{cmap}					% поиск в PDF
%\usepackage{mathtext} 				% русские буквы в формулах
%\usepackage[T2A]{fontenc}			% кодировка
%\usepackage[utf8]{inputenc}			% кодировка исходного текста
%\usepackage[english,russian]{babel}	% локализация и переносы
%\usepackage{indentfirst}
%\frenchspacing
%
%\renewcommand{\epsilon}{\ensuremath{\varepsilon}}
%\newcommand{\phibackup}{\ensuremath{\phi}}
%\renewcommand{\phi}{\ensuremath{\varphi}}
%\renewcommand{\varphi}{\ensuremath{\phibackup}}
%\renewcommand{\kappa}{\ensuremath{\varkappa}}
%\renewcommand{\le}{\ensuremath{\leqslant}}
%\renewcommand{\leq}{\ensuremath{\leqslant}}
%\renewcommand{\ge}{\ensuremath{\geqslant}}
%\renewcommand{\geq}{\ensuremath{\geqslant}}
%\renewcommand{\emptyset}{\varnothing}
%\renewcommand{\Im}{\operatorname{Im}}
%\renewcommand{\Re}{\operatorname{Re}}


%%% Дополнительная работа с математикой
\usepackage{amsmath,amsfonts,amssymb,amsthm,mathtools} % AMS
%\usepackage{icomma} % "Умная" запятая: $0,2$ --- число, $0, 2$ --- перечисление

%% Номера формул
%\mathtoolsset{showonlyrefs=true} % Показывать номера только у тех формул, на которые есть \eqref{} в тексте.
%\usepackage{leqno} % Нумереация формул слева

%% Свои команды
\DeclareMathOperator{\sgn}{\mathop{sgn}}
\DeclareMathOperator{\sign}{\mathop{sign}}
\DeclareMathOperator*{\res}{\mathop{res}}
\DeclareMathOperator*{\tr}{\mathop{tr}}
\DeclareMathOperator*{\rot}{\mathop{rot}}
\DeclareMathOperator*{\divop}{\mathop{div}}
\DeclareMathOperator*{\grad}{\mathop{grad}}

%% Перенос знаков в формулах (по Львовскому)
\newcommand*{\hm}[1]{#1\nobreak\discretionary{}
{\hbox{$\mathsurround=0pt #1$}}{}}

%%% Работа с картинками
\usepackage{graphicx}  % Для вставки рисунков
\graphicspath{{figures/}}  % папки с картинками
\setlength\fboxsep{3pt} % Отступ рамки \fbox{} от рисунка
\setlength\fboxrule{1pt} % Толщина линий рамки \fbox{}
\usepackage{wrapfig} % Обтекание рисунков текстом

%%% Работа с таблицами
\usepackage{array,tabularx,tabulary,booktabs} % Дополнительная работа с таблицами
\usepackage{longtable}  % Длинные таблицы
\usepackage{multirow} % Слияние строк в таблице

%%% Теоремы
\theoremstyle{plain} % Это стиль по умолчанию, его можно не переопределять.
\newtheorem{thm}{Теорема}
\newtheorem*{thm*}{Теорема}
\newtheorem{prop}{Предложение}
\newtheorem*{prop*}{Предложение}
 
\theoremstyle{definition} % "Определение"
%\newtheorem{corollary}{Следствие}[theorem]
\newtheorem{dfn}{Определение}
\newtheorem*{dfn*}{Определение}
\newtheorem{prob}{Задача}
\newtheorem*{prob*}{Задача}

 
\theoremstyle{remark} % "Примечание"
\newtheorem*{sol}{Решение}
\newtheorem*{rem}{Замечание}

%%% Программирование
\usepackage{etoolbox} % логические операторы

%%% Страница
%\usepackage{extsizes} % Возможность сделать 14-й шрифт
%\usepackage{geometry} % Простой способ задавать поля
%	\geometry{top=25mm}
%	\geometry{bottom=35mm}
%	\geometry{left=35mm}
%	\geometry{right=20mm}
 
\usepackage{fancyhdr} % Колонтитулы
%	\pagestyle{fancy}
 %	\renewcommand{\headrulewidth}{0pt}  % Толщина линейки, отчеркивающей верхний колонтитул
	%\lfoot{Нижний левый}
	%\rfoot{Нижний правый}
	%\rhead{Верхний правый}
	%\chead{Верхний в центре}
	%\lhead{Верхний левый}
	%\cfoot{Нижний в центре} % По умолчанию здесь номер страницы

\usepackage{setspace} % Интерлиньяж
%\onehalfspacing % Интерлиньяж 1.5
%\doublespacing % Интерлиньяж 2
%\singlespacing % Интерлиньяж 1

\usepackage{lastpage} % Узнать, сколько всего страниц в документе.

\usepackage{soul} % Модификаторы начертания

\usepackage{hyperref}
\usepackage[usenames,dvipsnames,svgnames,table,rgb]{xcolor}
\hypersetup{				% Гиперссылки
    unicode=true,           % русские буквы в раздела PDF
    pdftitle={Заголовок},   % Заголовок
    pdfauthor={Автор},      % Автор
    pdfsubject={Тема},      % Тема
    pdfcreator={Создатель}, % Создатель
    pdfproducer={Производитель}, % Производитель
    pdfkeywords={keyword1} {key2} {key3}, % Ключевые слова
%    colorlinks=true,       	% false: ссылки в рамках; true: цветные ссылки
    %linkcolor=red,          % внутренние ссылки
    %citecolor=black,        % на библиографию
    %filecolor=magenta,      % на файлы
    %urlcolor=cyan           % на URL
}

\usepackage{csquotes} % Еще инструменты для ссылок

%\usepackage[style=apa,maxcitenames=2,backend=biber,sorting=nty]{biblatex}

\usepackage{multicol} % Несколько колонок

\usepackage{tikz} % Работа с графикой
\usepackage{pgfplots}
\usepackage{pgfplotstable}
%\usepackage{coloremoji}
\usepackage{floatrow}
\usepackage{subcaption}
\graphicspath{{figures/}}

\renewcommand\thesubfigure{\asbuk{subfigure}}
%\addbibresource{master.bib}

\usepackage{import}
\usepackage{pdfpages}
\usepackage{transparent}
\usepackage{xcolor}
\usepackage{xifthen}

\newcommand{\incfig}[2][1]{%
    \def\svgwidth{#1\columnwidth}
    \import{./figures/}{#2.pdf_tex}
}
%\usepackage{titlesec}
%\titleformat{\section}{\normalfont\Large\bfseries}{}{0pt}{}
%----------------------STANDART:
%\titleformat{\chapter}[display]
%  {\normalfont\huge\bfseries}{\chaptertitlename\ \thechapter}{20pt}{\Huge}
%\titleformat{\section}{\normalfont\Large\bfseries}{\thesection}{1em}{}
%\titleformat{\subsection}
%  {\normalfont\large\bfseries}{\thesubsection}{1em}{}
%\titleformat{\subsubsection}
%  {\normalfont\normalsize\bfseries}{\thesubsubsection}{1em}{}
%\titleformat{\paragraph}[runin]
%  {\normalfont\normalsize\bfseries}{\theparagraph}{1em}{}
%\titleformat{\subparagraph}[runin]
%  {\normalfont\normalsize\bfseries}{\thesubparagraph}{1em}{}

\pdfsuppresswarningpagegroup=1
\pgfplotsset{compat=1.16}



%\setcounter{tocdepth}{1} % only parts,chapters,sections
%\titleformat{\subsection}{\normalfont\large\bfseries}{}{0em}{}
%\titleformat{\subsubsection}{\normalfont\normalsize\bfseries}{}{0em}{}

%\newcommand{\textover}[2]{\stackrel{\mathclap{\normalfont\mbox{#2}}}{#1}}

\author{Yaroslav Drachov\\
Moscow Institute of Physics and Technology}
%\author{Драчов Ярослав\\
%Факультет общей и прикладной физики МФТИ}
\newcommand{\veq}{\mathrel{\rotatebox{90}{$=$}}}
%\newcommand{\teto}[1]{\stackrel{\mathclap{\normalfont\tiny\mbox{#1}}}{\to}}
%\renewcommand{\thesubsection}{\arabic{subsection}}

%%\setcounter{secnumdepth}{0}

\definecolor{tabblue}{RGB}{30, 119, 180}
\definecolor{taborange}{RGB}{255, 127, 15}
\definecolor{tabgreen}{RGB}{45, 160, 43}
\definecolor{tabred}{RGB}{214, 38, 40}
\definecolor{tabpurple}{RGB}{148, 103, 189}
\definecolor{tabbrown}{RGB}{140, 86, 76}
\definecolor{tabpink}{RGB}{227, 119, 193}
\definecolor{tabgray}{RGB}{127, 127, 127}
\definecolor{tabolive}{RGB}{188, 189, 33}
\definecolor{tabcyan}{RGB}{22, 190, 207}
\pgfplotscreateplotcyclelist{colorbrewer-tab}{
{tabblue},
{taborange},
{tabgreen},
{tabred},
{tabpurple},
{tabbrown},
{tabpink},
{tabgray},
{tabolive},
{tabcyan},
}
\usepackage{csvsimple}
\usepackage{extarrows}
%\renewcommand{\labelenumii}{\asbuk{enumii})}
%\renewcommand{\labelenumiv}{\Asbuk{enumiv}}
%\newcommand{\prob}[1]{\subsubsection*{#1}}
\sisetup{output-decimal-marker = {,},separate-uncertainty = true,exponent-product = \cdot}

\usepackage{braket}
\usepackage{enumerate}
\usepackage{chngcntr}
%\counterwithin*{equation}{problem}
%\usepackage{bbold}

\newtheoremstyle{hiProb}% ⟨name ⟩ 
{3pt}% ⟨Space above ⟩1 
{3pt}% ⟨Space below ⟩1
{}% ⟨Body font ⟩
{}% ⟨Indent amount ⟩2
{\bfseries}% ⟨Theorem head font⟩
{.}% ⟨Punctuation after theorem head ⟩
{.5em}% ⟨Space after theorem head ⟩3
%{\thmname{#1} \thmnote{#3}}% ⟨Theorem head spec (can be left empty, meaning ‘normal’)⟩
{\thmnote{#3}}% ⟨Theorem head spec (can be left empty, meaning ‘normal’)⟩
\theoremstyle{hiProb} % "Определение"
%\newtheorem{hiProb}{Задача}
\newtheorem{hiProb}{}
%\usepackage{mmacells}
\newcommand{\textover}[2]{\stackrel{\mathclap{\normalfont\scriptsize\mbox{#2}}}{#1}}
\usepackage{units}
\usepackage[math]{cellspace}%
\setlength\cellspacetoplimit{2pt}
\setlength\cellspacebottomlimit{2pt}

\DeclareMathAlphabet{\mathbbold}{U}{bbold}{m}{n}

\newcommand{\normord}[1]{:\mathrel{#1}:}

\title{Неделя №11\\
Контактные явления в полупроводниках}
\begin{document}
	\maketitle
\begin{hiProb}[4.16]
\end{hiProb}
\begin{sol}
\[
\sigma= \frac{n e^2 \tau}{m}
.\] 
\[
R_{s q}= \rho \frac{L}{L d}= \frac{1}{\sigma d}=
\frac{m}{(d n)e^2 \tau}= \frac{m }{n_s e^2 \tau}
,\] 
где $n_s$ --- поверхностная плотность.
\[
R_{sq}= \frac{m V_F}{n_s e^2 l}= \frac{p_F}{n_s e^2 l}=
\frac{\sqrt{2 \pi n_s}  \hbar }{n_s e^2 l}
.\] 
\[
l= \frac{\hbar }{e^2} \frac{1}{R_{sq}}
\sqrt{\frac{2\pi}{n_s}} \approx 14,5 \text{ мкм}
.\] 
\end{sol}
\begin{hiProb}[4.20]
\end{hiProb}
\begin{sol}
Температура Ферми при заданной концентрации $T_F \sim 
\frac{\hbar ^2}{2m k_B} \left( 3\pi ^2 n \right) ^{2 /3}
\sim 35$ К (для массы свободного электрона). Если
эффективная масса не слишком мала (например, для
кремния $0,19 m_0$), то при 300 К имеем классический газ
электронов.

Прыжковое движение классических электронов между узлами
можно рассмотреть как аналог броуновского движения.
Подвижность носителя связана с коэффициентом диффузии
как $\mu = eD /(k_B T)$ (вариант записи для заряженных
частиц, где подвижность традиционно связывает
дрейфовую скорость с напряжённостью поля, а не с силой).
По определению $V_{\text{др}}= \mu E$, $j= neV_\text{др}=
n e \mu E= \sigma E$, откуда $\sigma= ne \mu=ne^2 D/(k_BT)$.
Коэффициент диффузии при прыжковом движении по порядку
величины есть $a^2 \nu$ (соображения размерности и
вид уравнения диффузии $\frac{\partial n}{\partial t} =
В\Delta n$, где $\Delta$ --- оператор Лапласа). Уточняющий
множитель ищется в задаче о случайных блужданиях: на
кубической решётке прыжки происходят вдоль выбранной
координатной оси с частотой $\frac{\nu}{3}$ и 
\[
\left<\Delta x^2 \right> = \frac{\nu}{3}ta^2,
\]
где $\nu t$ --- число прыжков за время $t$. В то же
время, из теории броуновского движения,  $\left<x^2 \right> =2Dt$, откуда $D= \frac{\nu a^2}{6}$. Окончательно
\[
\sigma= \frac{ne^2 a^2 \nu}{6k_B T}= 0,009 \frac{1}{\text{Ом}\cdot \text{см}}
.\] 
\end{sol}
\begin{hiProb}[Т11-2]
\end{hiProb}
\begin{sol}
	\renewcommand{\labelenumi}{(\roman{enumi})}
\begin{enumerate}
\item Построение зонной структуры стандартно.
	Без перераспределения заряда уровень химпотенциала в InGaN находится посередине запрещённой зоны, в $n$-GaN под дном зоны проводимости, в 
	$p$-GaN  над потолком валентной зоны. Перераспределение зарядов возникает с донорных
уровней $n$-GaN в зону проводимости InGaN и из
валентной зоны InGaN на акцепторные центры $p$-GaN.
По условию <<берега>> структуры сильно легированы,
поэтому основной изгиб зон происходит внутри активного
слоя. Так как толщина активного слоя велика по
сравнению с толщиной слоя аккумуляции, то
получим картину с двумя треугольными ямами: для 
электронов на контакте с $n$-GaN и для дырок на
контакте с $p$-GaN (<<электроны --- тонут, дырки ---
всплывают>>). Скачки зон на границе по условию симметричны и равны 0,45 эВ.
\item При пропускании тока к гетероструктуре
	прикладывается внешний потенциал, со стороны
	$n$-GaN идёт поток электронов, со стороны
	$p$-GaN поток дырок. Открытие $p$-$n$ 
	перехода соответствует выравниванию
	потолков валентных зон в GaN.
	Релаксируя в активном слое носители собираются
	вблизи дна зоны проводимости и потолка
	валентной зоны. Эти же места спектра имеют
	максимальную плотность состояний. Поэтому
	при рекомбинации основной вклад будет давать
	рекомбинация электронов с дна зоныы
	проводимости InGaN и дырок с потолка валентной
	зоны InGaN. Энергия излучаемых при этом
	фотонов равна ширине запрещённой зоны 2,5 эВ,
	что соответствует длине волны 500 нм (синий свет). Можно дополнительно отметить, что в принципе
	возможна и рекомбинация неотрелаксировавших
	носителей, поэтому свет будет не
	монохроматичен. Также можно отметить, что
	так как переход происходит в кристалле, то
	закон сохранения импульса можно не учитывать
	--- в отличие от аннигиляции пары частица-античастица рекомбинация
	электрона и дырки может происходить с
	излучением одного фотона.
	Наконец, интересно отметить, что при выравнивании
	потолков валентных зон в  GaN потолок
	валентной зоны в InGaN оказывается всегда выше,
	чем в GaN, им остаётся только рекомбинировать.
	Это позволяет сконцентрировать процессы
	рекомбинации в активном слое, создать там
	неравновесную концентрацию  носителей --- то есть
	все предпосылки к созданию лазера.
\item Кремний или германий являются непрямозонными
полупроводниками, в них рекомбинация с излучением фотона
запрещена правилами отбора по импульсу.
\end{enumerate}
\end{sol}
\begin{hiProb}[Т11-3]
\end{hiProb}
\begin{sol}
Химпотенциал и металла, и полупроводника зависит от
температуры. Так как в  равновесии
электрохимпотенциал постоянен по образцу, это приведёт
к перераспределению  электронной плотности. Соответствующая разность электростатических
потенциалов на краях образца будет очевидно равна
разности химпотенциалов.

Эффект <<фотонного ветра>>, которым мы по условию
пренебрегаем --- это  увлечение электронов фононами,
всегда движущимися от горячего конца образца
к холодному. Как будет виднно из дальнейшего этот
эффект противоположен чисто электронному эффекту в
металлах и полупроводниках $n$-типа. В некоторых
случаях он оказывается основным. Но в этой задаче
он не учитывается.

Для трёхмерного металла на лекциях получается формула
\[
	\mu(T)= \mu_0 - \frac{\pi^2}{12} \frac{T^2}{\mu_0} 
\]
(здесь температура в энергетических единицах).
Химпотенциал  <<горячего>> конца металлического
стержня понизится, электроны перераспределятся к
горячему концу, электрическое поле
будет направлено от холодного к горячему концу.
\[
E= \frac{\Delta \phi}{L}= \frac{1}{e} \frac{\Delta \mu}{L}= \frac{\pi^2}{6} \frac{k_B^2 T \delta T}{\mu_0
eL}
.\] 
Для оценки можно взять энергию Ферми около 5-7 эВ (медь
--- 7 эВ, серебро --- 5,5 эВ) или 60000 К. Отсюда
$E_{\text{мет}}\sim  7\cdot 10^{-5}$ В/м. Для полупроводников $p$  и $n$  типа при  $T=0$ 
химпотенциал находится посередине между краем ближней
зоны и примесным уровнем, при росте  температуры
смещается к  центру зоны. Поэтому, в полупроводнике
$n$-типа на горячем конце химпотенциал   уменьшится,
а в полупроводнике  $p$-типа увеличится: электрическое
поле будет  разного направления. Зависимость
химпотенциала от  температуры  для $p$-типа:
\[
	\mu(T)= \frac{\epsilon_0}{2}+\frac{T}{2}
	\ln  \frac{Q}{N}
\]
(здесь температура в энергетических единицах),
откуда \[E_\text{полупров}= \frac{1}{Le} \frac{k_B \Delta T}{2} \ln \frac{Q}{N}.\] Статфактор
зоны для данной эффективной массы при 300 К $2,2 \cdot 10^{18} 1/ \text{см}^3$, логарифм равен $7,7$ и
$E_{\text{полупров}}=0,033 \text{В} / \text{м}$.
Для термоЭДС полупроводниковой термопары получим 
$U=0,066$ В.
\end{sol}
\begin{hiProb}[Т11-4]
\end{hiProb}
\begin{sol}
В двумерном газе 
\[
	2 \frac{\pi k_F^2}{(2\pi)^2}=N
.\] 
Откуда
\[
E_F= \frac{\hbar ^2 k_F^2}{2}m^*= \frac{m_0}{m^*}
\frac{\hbar ^2 \pi N}{m_0}=  5\cdot 10^{-16}\text{ эрг}
\approx 300 \text{ К}
,\]
\[
p_F= \hbar  k_F= \hbar  \sqrt{2 \pi N} \approx
2,5 \cdot 10^{-21} \text{ г}\cdot \text{см} /\text{с}
,\] 
\[
k_F= 2,5 \cdot 10^6 \ 1 / \text{см}
,\] 
\[
V_F=\frac{p_F}{m^*} \approx 4\cdot 10^7 \text{ см} /
\text{с}
.\] 
\end{sol}
\end{document}
