\documentclass[a4paper]{article}
% Этот шаблон документа разработан в 2014 году
% Данилом Фёдоровых (danil@fedorovykh.ru) 
% для использования в курсе 
% <<Документы и презентации в \LaTeX>>, записанном НИУ ВШЭ
% для Coursera.org: http://coursera.org/course/latex .
% Исходная версия шаблона --- 
% https://www.writelatex.com/coursera/latex/5.3

% В этом документе преамбула

\usepackage{siunitx}
%%% Работа с русским языком
%\usepackage{cmap}					% поиск в PDF
%\usepackage{mathtext} 				% русские буквы в формулах
%\usepackage[T2A]{fontenc}			% кодировка
%\usepackage[utf8]{inputenc}			% кодировка исходного текста
%\usepackage[english,russian]{babel}	% локализация и переносы
%\usepackage{indentfirst}
%\frenchspacing
%
%\renewcommand{\epsilon}{\ensuremath{\varepsilon}}
%\newcommand{\phibackup}{\ensuremath{\phi}}
%\renewcommand{\phi}{\ensuremath{\varphi}}
%\renewcommand{\varphi}{\ensuremath{\phibackup}}
%\renewcommand{\kappa}{\ensuremath{\varkappa}}
%\renewcommand{\le}{\ensuremath{\leqslant}}
%\renewcommand{\leq}{\ensuremath{\leqslant}}
%\renewcommand{\ge}{\ensuremath{\geqslant}}
%\renewcommand{\geq}{\ensuremath{\geqslant}}
%\renewcommand{\emptyset}{\varnothing}
%\renewcommand{\Im}{\operatorname{Im}}
%\renewcommand{\Re}{\operatorname{Re}}


%%% Дополнительная работа с математикой
\usepackage{amsmath,amsfonts,amssymb,amsthm,mathtools} % AMS
%\usepackage{icomma} % "Умная" запятая: $0,2$ --- число, $0, 2$ --- перечисление

%% Номера формул
%\mathtoolsset{showonlyrefs=true} % Показывать номера только у тех формул, на которые есть \eqref{} в тексте.
%\usepackage{leqno} % Нумереация формул слева

%% Свои команды
\DeclareMathOperator{\sgn}{\mathop{sgn}}
\DeclareMathOperator{\sign}{\mathop{sign}}
\DeclareMathOperator*{\res}{\mathop{res}}
\DeclareMathOperator*{\tr}{\mathop{tr}}
\DeclareMathOperator*{\rot}{\mathop{rot}}
\DeclareMathOperator*{\divop}{\mathop{div}}
\DeclareMathOperator*{\grad}{\mathop{grad}}

%% Перенос знаков в формулах (по Львовскому)
\newcommand*{\hm}[1]{#1\nobreak\discretionary{}
{\hbox{$\mathsurround=0pt #1$}}{}}

%%% Работа с картинками
\usepackage{graphicx}  % Для вставки рисунков
\graphicspath{{figures/}}  % папки с картинками
\setlength\fboxsep{3pt} % Отступ рамки \fbox{} от рисунка
\setlength\fboxrule{1pt} % Толщина линий рамки \fbox{}
\usepackage{wrapfig} % Обтекание рисунков текстом

%%% Работа с таблицами
\usepackage{array,tabularx,tabulary,booktabs} % Дополнительная работа с таблицами
\usepackage{longtable}  % Длинные таблицы
\usepackage{multirow} % Слияние строк в таблице

%%% Теоремы
\theoremstyle{plain} % Это стиль по умолчанию, его можно не переопределять.
\newtheorem{thm}{Теорема}
\newtheorem*{thm*}{Теорема}
\newtheorem{prop}{Предложение}
\newtheorem*{prop*}{Предложение}
 
\theoremstyle{definition} % "Определение"
%\newtheorem{corollary}{Следствие}[theorem]
\newtheorem{dfn}{Определение}
\newtheorem*{dfn*}{Определение}
\newtheorem{prob}{Задача}
\newtheorem*{prob*}{Задача}

 
\theoremstyle{remark} % "Примечание"
\newtheorem*{sol}{Решение}
\newtheorem*{rem}{Замечание}

%%% Программирование
\usepackage{etoolbox} % логические операторы

%%% Страница
%\usepackage{extsizes} % Возможность сделать 14-й шрифт
%\usepackage{geometry} % Простой способ задавать поля
%	\geometry{top=25mm}
%	\geometry{bottom=35mm}
%	\geometry{left=35mm}
%	\geometry{right=20mm}
 
\usepackage{fancyhdr} % Колонтитулы
%	\pagestyle{fancy}
 %	\renewcommand{\headrulewidth}{0pt}  % Толщина линейки, отчеркивающей верхний колонтитул
	%\lfoot{Нижний левый}
	%\rfoot{Нижний правый}
	%\rhead{Верхний правый}
	%\chead{Верхний в центре}
	%\lhead{Верхний левый}
	%\cfoot{Нижний в центре} % По умолчанию здесь номер страницы

\usepackage{setspace} % Интерлиньяж
%\onehalfspacing % Интерлиньяж 1.5
%\doublespacing % Интерлиньяж 2
%\singlespacing % Интерлиньяж 1

\usepackage{lastpage} % Узнать, сколько всего страниц в документе.

\usepackage{soul} % Модификаторы начертания

\usepackage{hyperref}
\usepackage[usenames,dvipsnames,svgnames,table,rgb]{xcolor}
\hypersetup{				% Гиперссылки
    unicode=true,           % русские буквы в раздела PDF
    pdftitle={Заголовок},   % Заголовок
    pdfauthor={Автор},      % Автор
    pdfsubject={Тема},      % Тема
    pdfcreator={Создатель}, % Создатель
    pdfproducer={Производитель}, % Производитель
    pdfkeywords={keyword1} {key2} {key3}, % Ключевые слова
%    colorlinks=true,       	% false: ссылки в рамках; true: цветные ссылки
    %linkcolor=red,          % внутренние ссылки
    %citecolor=black,        % на библиографию
    %filecolor=magenta,      % на файлы
    %urlcolor=cyan           % на URL
}

\usepackage{csquotes} % Еще инструменты для ссылок

%\usepackage[style=apa,maxcitenames=2,backend=biber,sorting=nty]{biblatex}

\usepackage{multicol} % Несколько колонок

\usepackage{tikz} % Работа с графикой
\usepackage{pgfplots}
\usepackage{pgfplotstable}
%\usepackage{coloremoji}
\usepackage{floatrow}
\usepackage{subcaption}
\graphicspath{{figures/}}

\renewcommand\thesubfigure{\asbuk{subfigure}}
%\addbibresource{master.bib}

\usepackage{import}
\usepackage{pdfpages}
\usepackage{transparent}
\usepackage{xcolor}
\usepackage{xifthen}

\newcommand{\incfig}[2][1]{%
    \def\svgwidth{#1\columnwidth}
    \import{./figures/}{#2.pdf_tex}
}
%\usepackage{titlesec}
%\titleformat{\section}{\normalfont\Large\bfseries}{}{0pt}{}
%----------------------STANDART:
%\titleformat{\chapter}[display]
%  {\normalfont\huge\bfseries}{\chaptertitlename\ \thechapter}{20pt}{\Huge}
%\titleformat{\section}{\normalfont\Large\bfseries}{\thesection}{1em}{}
%\titleformat{\subsection}
%  {\normalfont\large\bfseries}{\thesubsection}{1em}{}
%\titleformat{\subsubsection}
%  {\normalfont\normalsize\bfseries}{\thesubsubsection}{1em}{}
%\titleformat{\paragraph}[runin]
%  {\normalfont\normalsize\bfseries}{\theparagraph}{1em}{}
%\titleformat{\subparagraph}[runin]
%  {\normalfont\normalsize\bfseries}{\thesubparagraph}{1em}{}

\pdfsuppresswarningpagegroup=1
\pgfplotsset{compat=1.16}



%\setcounter{tocdepth}{1} % only parts,chapters,sections
%\titleformat{\subsection}{\normalfont\large\bfseries}{}{0em}{}
%\titleformat{\subsubsection}{\normalfont\normalsize\bfseries}{}{0em}{}

%\newcommand{\textover}[2]{\stackrel{\mathclap{\normalfont\mbox{#2}}}{#1}}

\author{Yaroslav Drachov\\
Moscow Institute of Physics and Technology}
%\author{Драчов Ярослав\\
%Факультет общей и прикладной физики МФТИ}
\newcommand{\veq}{\mathrel{\rotatebox{90}{$=$}}}
%\newcommand{\teto}[1]{\stackrel{\mathclap{\normalfont\tiny\mbox{#1}}}{\to}}
%\renewcommand{\thesubsection}{\arabic{subsection}}

%%\setcounter{secnumdepth}{0}

\definecolor{tabblue}{RGB}{30, 119, 180}
\definecolor{taborange}{RGB}{255, 127, 15}
\definecolor{tabgreen}{RGB}{45, 160, 43}
\definecolor{tabred}{RGB}{214, 38, 40}
\definecolor{tabpurple}{RGB}{148, 103, 189}
\definecolor{tabbrown}{RGB}{140, 86, 76}
\definecolor{tabpink}{RGB}{227, 119, 193}
\definecolor{tabgray}{RGB}{127, 127, 127}
\definecolor{tabolive}{RGB}{188, 189, 33}
\definecolor{tabcyan}{RGB}{22, 190, 207}
\pgfplotscreateplotcyclelist{colorbrewer-tab}{
{tabblue},
{taborange},
{tabgreen},
{tabred},
{tabpurple},
{tabbrown},
{tabpink},
{tabgray},
{tabolive},
{tabcyan},
}
\usepackage{csvsimple}
\usepackage{extarrows}
%\renewcommand{\labelenumii}{\asbuk{enumii})}
%\renewcommand{\labelenumiv}{\Asbuk{enumiv}}
%\newcommand{\prob}[1]{\subsubsection*{#1}}
\sisetup{output-decimal-marker = {,},separate-uncertainty = true,exponent-product = \cdot}

\usepackage{braket}
\usepackage{enumerate}
\usepackage{chngcntr}
%\counterwithin*{equation}{problem}
%\usepackage{bbold}

\newtheoremstyle{hiProb}% ⟨name ⟩ 
{3pt}% ⟨Space above ⟩1 
{3pt}% ⟨Space below ⟩1
{}% ⟨Body font ⟩
{}% ⟨Indent amount ⟩2
{\bfseries}% ⟨Theorem head font⟩
{.}% ⟨Punctuation after theorem head ⟩
{.5em}% ⟨Space after theorem head ⟩3
%{\thmname{#1} \thmnote{#3}}% ⟨Theorem head spec (can be left empty, meaning ‘normal’)⟩
{\thmnote{#3}}% ⟨Theorem head spec (can be left empty, meaning ‘normal’)⟩
\theoremstyle{hiProb} % "Определение"
%\newtheorem{hiProb}{Задача}
\newtheorem{hiProb}{}
%\usepackage{mmacells}
\newcommand{\textover}[2]{\stackrel{\mathclap{\normalfont\scriptsize\mbox{#2}}}{#1}}
\usepackage{units}
\usepackage[math]{cellspace}%
\setlength\cellspacetoplimit{2pt}
\setlength\cellspacebottomlimit{2pt}

\DeclareMathAlphabet{\mathbbold}{U}{bbold}{m}{n}

\newcommand{\normord}[1]{:\mathrel{#1}:}

\title{Домашняя работа по общей физике\\
Неделя 1}
\begin{document}
	\maketitle
\begin{hiProb}[0-1-1]
\end{hiProb}
\begin{sol}
Закон дисперсии упругих волн может быть записан через скорость
звука
\[
	\omega=\frac{2s}{a}\left| \sin \left( \frac{ka}{2} \right)  \right|
.\] 
Откуда
\[
	\omega_\text{max} \sim \frac{s}{a}\sim 10^{13}\text{ с}^{-1}
.\] 
\end{sol}
\begin{hiProb}[0-1-2]
\end{hiProb}
\begin{sol}
Величины базисных векторов обратной решётки задаются выражением
 \[
	 2\pi\left| \frac{\mathbf{b}\times \mathbf{c}}{\mathbf{a}\cdot\left[ \mathbf{b}\times \mathbf{c} \right] } \right| 
 ,\] 
записанным с точность до циклической перестановки
$\mathbf{a}$, $\mathbf{b}$ и $\mathbf{c}$. Т.\:к. решётка
--- кубическая, то её базисные векторы взаимно-перпендикулярны, а значит последнее выражение в точности равно $2\pi$,
а в обратной решётке сохранится взаимная перпендикулярность
базисных векторов. Нетрудно заметить, что наименьшие длины
векторов обратной решётки из геометрических соображений
будут равны
 \[
	 |\mathbf{a}|=2\pi,\quad
	 |\mathbf{a}+\mathbf{b}|=2\sqrt{2} \pi,\quad
	 |\mathbf{a}+2\mathbf{b}|=2\sqrt{5} \pi
.\] 
\end{sol}
\begin{hiProb}[2.1]
\end{hiProb}
\begin{sol}
В простой кубической решётке на объём куба $a^3$ 
приходится один шар, радиус которого $r=a /2$, а 
объём куба $(4 /3) \pi r^3$. Откуда плотность упаковки
%\begin{figure}[ht]
%    \centering
%    \incfig{2}
%    \caption{2}
%    \label{fig:2}
%\end{figure}
\[
	\frac{4}{3} \pi \frac{r^3}{(2r)^3}=\frac{\pi}{6}=
	0,523
.\] 

В случае гранецентрированной решётки шары соприкасаются
по диагонали грани, поэтому $a \cdot \sqrt{2} =4r$.
В результате, так как на куб приходится 4 шара, получаем
\[
4\cdot \frac{4}{3}\pi \frac{r^3}{a^3}=\pi \cdot
\sqrt{2} /6=0,740
.\] 
В случае объёмно-центрированной решётки шары соприкасаются
по диагонали куба. Следовательно, $a\cdot \sqrt{3} =
4r$. На куб приходится 2 шара, поэтому
\[
2\cdot \frac{4}{3} \pi \frac{r^3}{a^3}=\pi \cdot
\sqrt{3} /8=0,681
.\] 
\end{sol}
\begin{hiProb}[Т1-1]
\end{hiProb}
\begin{sol}
%\begin{figure}[ht]
%    \centering
%    \incfig{3}
%    \caption{3}
%    \label{fig:3}
%\end{figure}
\begin{figure}[ht]
    \centering
    \incfig{4}
    \caption{}
    \label{fig:4}
\end{figure}
Элементарная ячейка исходной решётки представляет собой
прямоугольную призму с отношением сторон основания $1:2$ 
и с центрированными основаниями. Объём элементарной ячейки
исходной ромбической решётки в обычном пространстве:
$V_r =2a^2 c$.

Вводим базис $\mathbf{x},\ \mathbf{y},\ \mathbf{z}$ вдоль
$a,\ b,\ c$ осей, соответственно.

Базоцентрированная решётка непримитивная --- для
построения первой зоны Бриллюэна по определению необходимо
перейти к примитивной решётке.

Переход к примитивной решётке неоднозначен, но это
нее влияет на конечный результат. Это можно сделать, например,
 заменив одну из трансляций в плоскости основания на вектор
 в центр основания: $\mathbf{b}'= (\mathbf{a}+\mathbf{b}) /2$.
 Объём примитивной ячейки $V_{r,\text{ прим}}=V_r /2=
 a^2 c$ (вдвое меньше исходной).

Вектора обратной решётки, построенные для примитивной ячейки:
\begin{align*}
	\mathbf{a}^*&= \frac{2\pi}{V_{r,\text{ прим}}}
	\mathbf{b}' \times \mathbf{c}=\frac{\pi}{a}(2\mathbf{x}-\mathbf{y})\\
	\mathbf{b}^*&= \frac{2\pi}{a}\mathbf{y} \\
	\mathbf{c}^*&= \frac{2\pi}{c}\mathbf{z}
.\end{align*}
Отсюда сразу объём элементарной ячейки для обратной решётки,
по определению равный объёму первой зоны Бриллюэна,
равен 
\[
	V_k=V_{1\text{з.Бр.}}= \mathbf{a}^*\cdot\left[ 
	\mathbf{b}^* \times \mathbf{c}\right] =
	\frac{(2\pi)^3}{a^2 c}= \frac{\left( 2\pi \right) ^3}{
	V_{r \text{, прим}}}=2 \frac{(2\pi)^3}{V_r}
.\] 
Множитель 2 здесь связан с тем, что объём базоцентрированной
ромбической элементарной ячейки вдвое больше объёма примитивной ячейки, для примитивной ячейки всегда верно соотношение
$V_{\text{1з.Бр.}}=(2\pi)^3 /V_\text{прим}$.

Так как $\mathbf{c}^*\perp \mathbf{a}^*,\ \mathbf{b}^*$,
то элементарная ячейка обратной решётки, построенная на
векторах трансляции обратной решётки, будет иметь вид
прямоугольной призмы.

Для построения первой зоны Бриллюэна пользуемся (по определению) построением ячейки Вигнера-Зейтца. Построение в направлении оси $Z$ тривиально: границы первой зоны Бриллюэна лежат
на расстоянии $\pm c^* /2$ от плоскости $XY$: это положеия
серединных перпендикуляров к соседним вдоль направления
$Z$ узлам. В рассматриваемом случае из-за ортогональности 
$\mathbf{c}^*\perp \mathbf{a}^*,\ \mathbf{b}^*$ серединные
перпендикуляры к другим узлам вне плоскости $XY$ 
(смещённым на трансляции типа $\mathbf{a}^*,\ \mathbf{b}^*$)
не будут <<срезать>> углы ячейки  Вигнера-Зейтца. В этом
можно убедиться либо геометрическим анализом, либо
если заметить, что объём первой зоны Бриллюэна (найденнный
ранее) равен (из ортогональности) произведению высоты
призмы $|\mathbf{c}^*|$ на площадь основания $\left| \mathbf{a}^*\times \mathbf{b}^* \right| $, так как площадь
этого основания будет равно по построению площади
сечения ячейки Вигнера-Зейтца плоскостью $XY$, то
никаких углов этой призмы в направленими вектора $\mathbf{c}^*$ 
<<срезать>> не надо.
\begin{figure}[ht]
    \centering
    \incfig{5}
    \caption{К решению задачи Т1-1. Обратная решётка в плоскости
    $XY$ и проекция первой зоны Бриллюэна на эту плоскость. Красные
    кружки --- обратная решётка для примитивной элементарной
    ячейки. Синие кружки --- дополнительные узлы обратной
    решётки при построении непримитивной элементарной ячейки. Показаны
    вектора обратной решётки для построения по примитивной ячейке.
    Выделена первая зона Бриллюэна, построенная как ячейка Вигнера-Зейтца.}
    \label{fig:5}
\end{figure}
Интерес представляет построение сечения первой зоны Бриллюэна в плоскости
$XY$. Построение представлено на рис.~\ref{fig:5}. Первая зона
Бриллюэна для исходно <<прямоугольной>> структуры имеет вид прямоугольной
призмы с основанием в форме неправильного шестиугольника.
%
%	Имеем три взаимно-перпендикулярных вектора --- $\mathbf{a}$, $\mathbf{b}$ и $\mathbf{c}$,
%	на которых строится решётка. Тройкой примитивных трансляций 
%	выберем $\mathbf{c}$, $\mathbf{b}$, $(\mathbf{a}+\mathbf{b}) /2$.
%	Для них найдём
%	 \[
%		 \mathbf{a}^*=
%		 2\pi \frac{\mathbf{b}\times \mathbf{c}}{
%		 (\mathbf{a}+\mathbf{b}) /2 \cdot
%	 \left[ \mathbf{b}\times \mathbf{c} \right] }=
%	 \frac{4\pi}{a^2}\mathbf{a}
%	,\] 
%	\[
%		\mathbf{b}^*=2\pi
%		\frac{\mathbf{c}\times(\mathbf{a}+\mathbf{b}) /2}{
%		 (\mathbf{a}+\mathbf{b}) /2 \cdot
%	 \left[ \mathbf{b}\times \mathbf{c} \right] }=
%	 2\pi\left( \frac{\mathbf{b}}{b^2}+
%	 \frac{\mathbf{a}}{a^2}\right) 
%	,\] 
%\[
%	\mathbf{c}^*= 2\pi \frac{(\mathbf{a}+\mathbf{b})
%	/2 \times \mathbf{b}}{(\mathbf{a}+\mathbf{b})
%/2 \cdot \left[ \mathbf{b}\times \mathbf{c} \right] }=
%\frac{2\pi}{c^2}\mathbf{c}
%.\] 
\end{sol}
\begin{hiProb}[2.16]
\end{hiProb}
\begin{sol}
Периодическое граничное условие означает, что, если дополнить нашу
цепочку атомов <<нулевым атомом>>, то смещение атомов с $i=0$ и 
$i=N$ одинаково. Фактически, это означает <<сворачивание>> цепочки в кольцо,
так что $N$-ный атом взаимодействует теперь с первым.

Предполагая, что смещения атомов в такой волне $u_n=A e^{i\left( K
an-\omega t\right) }$, получаем из граничного условия $u_0=u_N$.
Это даёт условие $KaN=2\pi p$, где $p$ --- целое и $a$ --- расстояние
между атомами в цепочке, которое выделяет $N$ разрешённых
значений вида 
 \[
K=0;\ \frac{2\pi}{Na};\ 2\cdot\frac{2\pi}{Na};\
3\cdot\frac{2\pi}{Na};\ldots;\ K=\frac{2\pi (N-1)}{Na}
\]
(в этой задаче удобно не сводить все волновые векторы в первую
зону Бриллюэна).

Смещения атомов в такой волне $u_n=A e^{i\left( Kan-\omega t \right) }$, мгновенные скорости $v_n=-i\omega A e^{i\left( Kan-\omega t \right) }$. Полный импульс такой цепочки
\[
	P= \sum_{n=1}^{N} p_n= -i\omega MA e^{i\omega t} \sum_{n=1}^{N} e^{iKan}= -i\omega MA e^{i\left( Ka-\omega t \right) }
	\frac{1-e^{iKaN}}{1-e^{iKa}}
\] 
(выполнено суммирование геометрической прогрессии).
С учётом граничных условий числитель дроби всегда нулевой.
Отдельным оказывается случай $K=0$. В этом случае в дроби
получается неопределённость типа $\frac{0}{0}$, $\lim_{K \to 0} 
\frac{1-e^{iKaN}}{1-e^{iKa}}=N$. Однако и частота акустических
фононов с $K=0$ оказывается нулевой. Но можно заметить,
что однородное $(K=0)$ и постоянное $(\omega=0)$ колебание
по своему смыслу есть движение всех атомов цепочки с постоянной
скоростью $u_n=Vt$. Другими словами, нулевая частота означает
отсутствие возвращающей силы при таких колебаниях и уравнения
динамики в модели <<шариков и пружинок>> принимают вид
$\frac{d^2u_n}{dt^2}=0$ с решением типа $u_n=Vt$. В этом
случае полный импульс цепочки получается $P=MNV$.
\end{sol}
\end{document}
