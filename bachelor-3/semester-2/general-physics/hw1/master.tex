\documentclass[a4paper]{article}
% Этот шаблон документа разработан в 2014 году
% Данилом Фёдоровых (danil@fedorovykh.ru) 
% для использования в курсе 
% <<Документы и презентации в \LaTeX>>, записанном НИУ ВШЭ
% для Coursera.org: http://coursera.org/course/latex .
% Исходная версия шаблона --- 
% https://www.writelatex.com/coursera/latex/5.3

% В этом документе преамбула

\usepackage{siunitx}
%%% Работа с русским языком
%\usepackage{cmap}					% поиск в PDF
%\usepackage{mathtext} 				% русские буквы в формулах
%\usepackage[T2A]{fontenc}			% кодировка
%\usepackage[utf8]{inputenc}			% кодировка исходного текста
%\usepackage[english,russian]{babel}	% локализация и переносы
%\usepackage{indentfirst}
%\frenchspacing
%
%\renewcommand{\epsilon}{\ensuremath{\varepsilon}}
%\newcommand{\phibackup}{\ensuremath{\phi}}
%\renewcommand{\phi}{\ensuremath{\varphi}}
%\renewcommand{\varphi}{\ensuremath{\phibackup}}
%\renewcommand{\kappa}{\ensuremath{\varkappa}}
%\renewcommand{\le}{\ensuremath{\leqslant}}
%\renewcommand{\leq}{\ensuremath{\leqslant}}
%\renewcommand{\ge}{\ensuremath{\geqslant}}
%\renewcommand{\geq}{\ensuremath{\geqslant}}
%\renewcommand{\emptyset}{\varnothing}
%\renewcommand{\Im}{\operatorname{Im}}
%\renewcommand{\Re}{\operatorname{Re}}


%%% Дополнительная работа с математикой
\usepackage{amsmath,amsfonts,amssymb,amsthm,mathtools} % AMS
%\usepackage{icomma} % "Умная" запятая: $0,2$ --- число, $0, 2$ --- перечисление

%% Номера формул
%\mathtoolsset{showonlyrefs=true} % Показывать номера только у тех формул, на которые есть \eqref{} в тексте.
%\usepackage{leqno} % Нумереация формул слева

%% Свои команды
\DeclareMathOperator{\sgn}{\mathop{sgn}}
\DeclareMathOperator{\sign}{\mathop{sign}}
\DeclareMathOperator*{\res}{\mathop{res}}
\DeclareMathOperator*{\tr}{\mathop{tr}}
\DeclareMathOperator*{\rot}{\mathop{rot}}
\DeclareMathOperator*{\divop}{\mathop{div}}
\DeclareMathOperator*{\grad}{\mathop{grad}}

%% Перенос знаков в формулах (по Львовскому)
\newcommand*{\hm}[1]{#1\nobreak\discretionary{}
{\hbox{$\mathsurround=0pt #1$}}{}}

%%% Работа с картинками
\usepackage{graphicx}  % Для вставки рисунков
\graphicspath{{figures/}}  % папки с картинками
\setlength\fboxsep{3pt} % Отступ рамки \fbox{} от рисунка
\setlength\fboxrule{1pt} % Толщина линий рамки \fbox{}
\usepackage{wrapfig} % Обтекание рисунков текстом

%%% Работа с таблицами
\usepackage{array,tabularx,tabulary,booktabs} % Дополнительная работа с таблицами
\usepackage{longtable}  % Длинные таблицы
\usepackage{multirow} % Слияние строк в таблице

%%% Теоремы
\theoremstyle{plain} % Это стиль по умолчанию, его можно не переопределять.
\newtheorem{thm}{Теорема}
\newtheorem*{thm*}{Теорема}
\newtheorem{prop}{Предложение}
\newtheorem*{prop*}{Предложение}
 
\theoremstyle{definition} % "Определение"
%\newtheorem{corollary}{Следствие}[theorem]
\newtheorem{dfn}{Определение}
\newtheorem*{dfn*}{Определение}
\newtheorem{prob}{Задача}
\newtheorem*{prob*}{Задача}

 
\theoremstyle{remark} % "Примечание"
\newtheorem*{sol}{Решение}
\newtheorem*{rem}{Замечание}

%%% Программирование
\usepackage{etoolbox} % логические операторы

%%% Страница
%\usepackage{extsizes} % Возможность сделать 14-й шрифт
%\usepackage{geometry} % Простой способ задавать поля
%	\geometry{top=25mm}
%	\geometry{bottom=35mm}
%	\geometry{left=35mm}
%	\geometry{right=20mm}
 
\usepackage{fancyhdr} % Колонтитулы
%	\pagestyle{fancy}
 %	\renewcommand{\headrulewidth}{0pt}  % Толщина линейки, отчеркивающей верхний колонтитул
	%\lfoot{Нижний левый}
	%\rfoot{Нижний правый}
	%\rhead{Верхний правый}
	%\chead{Верхний в центре}
	%\lhead{Верхний левый}
	%\cfoot{Нижний в центре} % По умолчанию здесь номер страницы

\usepackage{setspace} % Интерлиньяж
%\onehalfspacing % Интерлиньяж 1.5
%\doublespacing % Интерлиньяж 2
%\singlespacing % Интерлиньяж 1

\usepackage{lastpage} % Узнать, сколько всего страниц в документе.

\usepackage{soul} % Модификаторы начертания

\usepackage{hyperref}
\usepackage[usenames,dvipsnames,svgnames,table,rgb]{xcolor}
\hypersetup{				% Гиперссылки
    unicode=true,           % русские буквы в раздела PDF
    pdftitle={Заголовок},   % Заголовок
    pdfauthor={Автор},      % Автор
    pdfsubject={Тема},      % Тема
    pdfcreator={Создатель}, % Создатель
    pdfproducer={Производитель}, % Производитель
    pdfkeywords={keyword1} {key2} {key3}, % Ключевые слова
%    colorlinks=true,       	% false: ссылки в рамках; true: цветные ссылки
    %linkcolor=red,          % внутренние ссылки
    %citecolor=black,        % на библиографию
    %filecolor=magenta,      % на файлы
    %urlcolor=cyan           % на URL
}

\usepackage{csquotes} % Еще инструменты для ссылок

%\usepackage[style=apa,maxcitenames=2,backend=biber,sorting=nty]{biblatex}

\usepackage{multicol} % Несколько колонок

\usepackage{tikz} % Работа с графикой
\usepackage{pgfplots}
\usepackage{pgfplotstable}
%\usepackage{coloremoji}
\usepackage{floatrow}
\usepackage{subcaption}
\graphicspath{{figures/}}

\renewcommand\thesubfigure{\asbuk{subfigure}}
%\addbibresource{master.bib}

\usepackage{import}
\usepackage{pdfpages}
\usepackage{transparent}
\usepackage{xcolor}
\usepackage{xifthen}

\newcommand{\incfig}[2][1]{%
    \def\svgwidth{#1\columnwidth}
    \import{./figures/}{#2.pdf_tex}
}
%\usepackage{titlesec}
%\titleformat{\section}{\normalfont\Large\bfseries}{}{0pt}{}
%----------------------STANDART:
%\titleformat{\chapter}[display]
%  {\normalfont\huge\bfseries}{\chaptertitlename\ \thechapter}{20pt}{\Huge}
%\titleformat{\section}{\normalfont\Large\bfseries}{\thesection}{1em}{}
%\titleformat{\subsection}
%  {\normalfont\large\bfseries}{\thesubsection}{1em}{}
%\titleformat{\subsubsection}
%  {\normalfont\normalsize\bfseries}{\thesubsubsection}{1em}{}
%\titleformat{\paragraph}[runin]
%  {\normalfont\normalsize\bfseries}{\theparagraph}{1em}{}
%\titleformat{\subparagraph}[runin]
%  {\normalfont\normalsize\bfseries}{\thesubparagraph}{1em}{}

\pdfsuppresswarningpagegroup=1
\pgfplotsset{compat=1.16}



%\setcounter{tocdepth}{1} % only parts,chapters,sections
%\titleformat{\subsection}{\normalfont\large\bfseries}{}{0em}{}
%\titleformat{\subsubsection}{\normalfont\normalsize\bfseries}{}{0em}{}

%\newcommand{\textover}[2]{\stackrel{\mathclap{\normalfont\mbox{#2}}}{#1}}

\author{Yaroslav Drachov\\
Moscow Institute of Physics and Technology}
%\author{Драчов Ярослав\\
%Факультет общей и прикладной физики МФТИ}
\newcommand{\veq}{\mathrel{\rotatebox{90}{$=$}}}
%\newcommand{\teto}[1]{\stackrel{\mathclap{\normalfont\tiny\mbox{#1}}}{\to}}
%\renewcommand{\thesubsection}{\arabic{subsection}}

%%\setcounter{secnumdepth}{0}

\definecolor{tabblue}{RGB}{30, 119, 180}
\definecolor{taborange}{RGB}{255, 127, 15}
\definecolor{tabgreen}{RGB}{45, 160, 43}
\definecolor{tabred}{RGB}{214, 38, 40}
\definecolor{tabpurple}{RGB}{148, 103, 189}
\definecolor{tabbrown}{RGB}{140, 86, 76}
\definecolor{tabpink}{RGB}{227, 119, 193}
\definecolor{tabgray}{RGB}{127, 127, 127}
\definecolor{tabolive}{RGB}{188, 189, 33}
\definecolor{tabcyan}{RGB}{22, 190, 207}
\pgfplotscreateplotcyclelist{colorbrewer-tab}{
{tabblue},
{taborange},
{tabgreen},
{tabred},
{tabpurple},
{tabbrown},
{tabpink},
{tabgray},
{tabolive},
{tabcyan},
}
\usepackage{csvsimple}
\usepackage{extarrows}
%\renewcommand{\labelenumii}{\asbuk{enumii})}
%\renewcommand{\labelenumiv}{\Asbuk{enumiv}}
%\newcommand{\prob}[1]{\subsubsection*{#1}}
\sisetup{output-decimal-marker = {,},separate-uncertainty = true,exponent-product = \cdot}

\usepackage{braket}
\usepackage{enumerate}
\usepackage{chngcntr}
%\counterwithin*{equation}{problem}
%\usepackage{bbold}

\newtheoremstyle{hiProb}% ⟨name ⟩ 
{3pt}% ⟨Space above ⟩1 
{3pt}% ⟨Space below ⟩1
{}% ⟨Body font ⟩
{}% ⟨Indent amount ⟩2
{\bfseries}% ⟨Theorem head font⟩
{.}% ⟨Punctuation after theorem head ⟩
{.5em}% ⟨Space after theorem head ⟩3
%{\thmname{#1} \thmnote{#3}}% ⟨Theorem head spec (can be left empty, meaning ‘normal’)⟩
{\thmnote{#3}}% ⟨Theorem head spec (can be left empty, meaning ‘normal’)⟩
\theoremstyle{hiProb} % "Определение"
%\newtheorem{hiProb}{Задача}
\newtheorem{hiProb}{}
%\usepackage{mmacells}
\newcommand{\textover}[2]{\stackrel{\mathclap{\normalfont\scriptsize\mbox{#2}}}{#1}}
\usepackage{units}
\usepackage[math]{cellspace}%
\setlength\cellspacetoplimit{2pt}
\setlength\cellspacebottomlimit{2pt}

\DeclareMathAlphabet{\mathbbold}{U}{bbold}{m}{n}

\newcommand{\normord}[1]{:\mathrel{#1}:}

\title{Домашняя работа по общей физике}
\begin{document}
	\maketitle
\section{Структура и колебания
кристаллической решётки, фононы}
%\begin{hiProb}[0-1-1]
%\end{hiProb}
%\begin{sol}
%Закон дисперсии упругих волн может быть записан через скорость
%звука
%\[
%	\omega=\frac{2s}{a}\left| \sin \left( \frac{ka}{2} \right)  \right|
%.\] 
%Откуда
%\[
%	\omega_\text{max} \sim \frac{s}{a}\sim 10^{13}\text{ с}^{-1}
%.\] 
%\end{sol}
%\begin{hiProb}[0-1-2]
%\end{hiProb}
%\begin{sol}
%Величины базисных векторов обратной решётки задаются выражением
% \[
%	 2\pi\left| \frac{\mathbf{b}\times \mathbf{c}}{\mathbf{a}\cdot\left[ \mathbf{b}\times \mathbf{c} \right] } \right| 
% ,\] 
%записанным с точность до циклической перестановки
%$\mathbf{a}$, $\mathbf{b}$ и $\mathbf{c}$. Т.\:к. решётка
%--- кубическая, то её базисные векторы взаимно-перпендикулярны, а значит последнее выражение в точности равно $2\pi$,
%а в обратной решётке сохранится взаимная перпендикулярность
%базисных векторов. Нетрудно заметить, что наименьшие длины
%векторов обратной решётки из геометрических соображений
%будут равны
% \[
%	 |\mathbf{a}|=2\pi,\quad
%	 |\mathbf{a}+\mathbf{b}|=2\sqrt{2} \pi,\quad
%	 |\mathbf{a}+2\mathbf{b}|=2\sqrt{5} \pi
%.\] 
%\end{sol}
%\begin{hiProb}[2.1]
%\end{hiProb}
%\begin{sol}
%В простой кубической решётке на объём куба $a^3$ 
%приходится один шар, радиус которого $r=a /2$, а 
%объём куба $(4 /3) \pi r^3$. Откуда плотность упаковки
%%\begin{figure}[ht]
%%    \centering
%%    \incfig{2}
%%    \caption{2}
%%    \label{fig:2}
%%\end{figure}
%\[
%	\frac{4}{3} \pi \frac{r^3}{(2r)^3}=\frac{\pi}{6}=
%	0,523
%.\] 
%
%В случае гранецентрированной решётки шары соприкасаются
%по диагонали грани, поэтому $a \cdot \sqrt{2} =4r$.
%В результате, так как на куб приходится 4 шара, получаем
%\[
%4\cdot \frac{4}{3}\pi \frac{r^3}{a^3}=\pi \cdot
%\sqrt{2} /6=0,740
%.\] 
%В случае объёмно-центрированной решётки шары соприкасаются
%по диагонали куба. Следовательно, $a\cdot \sqrt{3} =
%4r$. На куб приходится 2 шара, поэтому
%\[
%2\cdot \frac{4}{3} \pi \frac{r^3}{a^3}=\pi \cdot
%\sqrt{3} /8=0,681
%.\] 
%\end{sol}
%\begin{hiProb}[Т1-1]
%\end{hiProb}
%\begin{sol}
%%\begin{figure}[ht]
%%    \centering
%%    \incfig{3}
%%    \caption{3}
%%    \label{fig:3}
%%\end{figure}
%\begin{figure}[ht]
%    \centering
%    \incfig{4}
%    \caption{}
%    \label{fig:4}
%\end{figure}
%Элементарная ячейка исходной решётки представляет собой
%прямоугольную призму с отношением сторон основания $1:2$ 
%и с центрированными основаниями. Объём элементарной ячейки
%исходной ромбической решётки в обычном пространстве:
%$V_r =2a^2 c$.
%
%Вводим базис $\mathbf{x},\ \mathbf{y},\ \mathbf{z}$ вдоль
%$a,\ b,\ c$ осей, соответственно.
%
%Базоцентрированная решётка непримитивная --- для
%построения первой зоны Бриллюэна по определению необходимо
%перейти к примитивной решётке.
%
%Переход к примитивной решётке неоднозначен, но это
%нее влияет на конечный результат. Это можно сделать, например,
% заменив одну из трансляций в плоскости основания на вектор
% в центр основания: $\mathbf{b}'= (\mathbf{a}+\mathbf{b}) /2$.
% Объём примитивной ячейки $V_{r,\text{ прим}}=V_r /2=
% a^2 c$ (вдвое меньше исходной).
%
%Вектора обратной решётки, построенные для примитивной ячейки:
%\begin{align*}
%	\mathbf{a}^*&= \frac{2\pi}{V_{r,\text{ прим}}}
%	\mathbf{b}' \times \mathbf{c}=\frac{\pi}{a}(2\mathbf{x}-\mathbf{y})\\
%	\mathbf{b}^*&= \frac{2\pi}{a}\mathbf{y} \\
%	\mathbf{c}^*&= \frac{2\pi}{c}\mathbf{z}
%.\end{align*}
%Отсюда сразу объём элементарной ячейки для обратной решётки,
%по определению равный объёму первой зоны Бриллюэна,
%равен 
%\[
%	V_k=V_{1\text{з.Бр.}}= \mathbf{a}^*\cdot\left[ 
%	\mathbf{b}^* \times \mathbf{c}\right] =
%	\frac{(2\pi)^3}{a^2 c}= \frac{\left( 2\pi \right) ^3}{
%	V_{r \text{, прим}}}=2 \frac{(2\pi)^3}{V_r}
%.\] 
%Множитель 2 здесь связан с тем, что объём базоцентрированной
%ромбической элементарной ячейки вдвое больше объёма примитивной ячейки, для примитивной ячейки всегда верно соотношение
%$V_{\text{1з.Бр.}}=(2\pi)^3 /V_\text{прим}$.
%
%Так как $\mathbf{c}^*\perp \mathbf{a}^*,\ \mathbf{b}^*$,
%то элементарная ячейка обратной решётки, построенная на
%векторах трансляции обратной решётки, будет иметь вид
%прямоугольной призмы.
%
%Для построения первой зоны Бриллюэна пользуемся (по определению) построением ячейки Вигнера-Зейтца. Построение в направлении оси $Z$ тривиально: границы первой зоны Бриллюэна лежат
%на расстоянии $\pm c^* /2$ от плоскости $XY$: это положеия
%серединных перпендикуляров к соседним вдоль направления
%$Z$ узлам. В рассматриваемом случае из-за ортогональности 
%$\mathbf{c}^*\perp \mathbf{a}^*,\ \mathbf{b}^*$ серединные
%перпендикуляры к другим узлам вне плоскости $XY$ 
%(смещённым на трансляции типа $\mathbf{a}^*,\ \mathbf{b}^*$)
%не будут <<срезать>> углы ячейки  Вигнера-Зейтца. В этом
%можно убедиться либо геометрическим анализом, либо
%если заметить, что объём первой зоны Бриллюэна (найденнный
%ранее) равен (из ортогональности) произведению высоты
%призмы $|\mathbf{c}^*|$ на площадь основания $\left| \mathbf{a}^*\times \mathbf{b}^* \right| $, так как площадь
%этого основания будет равно по построению площади
%сечения ячейки Вигнера-Зейтца плоскостью $XY$, то
%никаких углов этой призмы в направленими вектора $\mathbf{c}^*$ 
%<<срезать>> не надо.
%\begin{figure}[ht]
%    \centering
%    \incfig{5}
%    \caption{К решению задачи Т1-1. Обратная решётка в плоскости
%    $XY$ и проекция первой зоны Бриллюэна на эту плоскость. Красные
%    кружки --- обратная решётка для примитивной элементарной
%    ячейки. Синие кружки --- дополнительные узлы обратной
%    решётки при построении непримитивной элементарной ячейки. Показаны
%    вектора обратной решётки для построения по примитивной ячейке.
%    Выделена первая зона Бриллюэна, построенная как ячейка Вигнера-Зейтца.}
%    \label{fig:5}
%\end{figure}
%Интерес представляет построение сечения первой зоны Бриллюэна в плоскости
%$XY$. Построение представлено на рис.~\ref{fig:5}. Первая зона
%Бриллюэна для исходно <<прямоугольной>> структуры имеет вид прямоугольной
%призмы с основанием в форме неправильного шестиугольника.
%%
%%	Имеем три взаимно-перпендикулярных вектора --- $\mathbf{a}$, $\mathbf{b}$ и $\mathbf{c}$,
%%	на которых строится решётка. Тройкой примитивных трансляций 
%%	выберем $\mathbf{c}$, $\mathbf{b}$, $(\mathbf{a}+\mathbf{b}) /2$.
%%	Для них найдём
%%	 \[
%%		 \mathbf{a}^*=
%%		 2\pi \frac{\mathbf{b}\times \mathbf{c}}{
%%		 (\mathbf{a}+\mathbf{b}) /2 \cdot
%%	 \left[ \mathbf{b}\times \mathbf{c} \right] }=
%%	 \frac{4\pi}{a^2}\mathbf{a}
%%	,\] 
%%	\[
%%		\mathbf{b}^*=2\pi
%%		\frac{\mathbf{c}\times(\mathbf{a}+\mathbf{b}) /2}{
%%		 (\mathbf{a}+\mathbf{b}) /2 \cdot
%%	 \left[ \mathbf{b}\times \mathbf{c} \right] }=
%%	 2\pi\left( \frac{\mathbf{b}}{b^2}+
%%	 \frac{\mathbf{a}}{a^2}\right) 
%%	,\] 
%%\[
%%	\mathbf{c}^*= 2\pi \frac{(\mathbf{a}+\mathbf{b})
%%	/2 \times \mathbf{b}}{(\mathbf{a}+\mathbf{b})
%%/2 \cdot \left[ \mathbf{b}\times \mathbf{c} \right] }=
%%\frac{2\pi}{c^2}\mathbf{c}
%%.\] 
%\end{sol}
%\begin{hiProb}[2.16]
%\end{hiProb}
%\begin{sol}
%Периодическое граничное условие означает, что, если дополнить нашу
%цепочку атомов <<нулевым атомом>>, то смещение атомов с $i=0$ и 
%$i=N$ одинаково. Фактически, это означает <<сворачивание>> цепочки в кольцо,
%так что $N$-ный атом взаимодействует теперь с первым.
%
%Предполагая, что смещения атомов в такой волне $u_n=A e^{i\left( K
%an-\omega t\right) }$, получаем из граничного условия $u_0=u_N$.
%Это даёт условие $KaN=2\pi p$, где $p$ --- целое и $a$ --- расстояние
%между атомами в цепочке, которое выделяет $N$ разрешённых
%значений вида 
% \[
%K=0;\ \frac{2\pi}{Na};\ 2\cdot\frac{2\pi}{Na};\
%3\cdot\frac{2\pi}{Na};\ldots;\ K=\frac{2\pi (N-1)}{Na}
%\]
%(в этой задаче удобно не сводить все волновые векторы в первую
%зону Бриллюэна).
%
%Смещения атомов в такой волне $u_n=A e^{i\left( Kan-\omega t \right) }$, мгновенные скорости $v_n=-i\omega A e^{i\left( Kan-\omega t \right) }$. Полный импульс такой цепочки
%\[
%	P= \sum_{n=1}^{N} p_n= -i\omega MA e^{i\omega t} \sum_{n=1}^{N} e^{iKan}= -i\omega MA e^{i\left( Ka-\omega t \right) }
%	\frac{1-e^{iKaN}}{1-e^{iKa}}
%\] 
%(выполнено суммирование геометрической прогрессии).
%С учётом граничных условий числитель дроби всегда нулевой.
%Отдельным оказывается случай $K=0$. В этом случае в дроби
%получается неопределённость типа $\frac{0}{0}$, $\lim_{K \to 0} 
%\frac{1-e^{iKaN}}{1-e^{iKa}}=N$. Однако и частота акустических
%фононов с $K=0$ оказывается нулевой. Но можно заметить,
%что однородное $(K=0)$ и постоянное $(\omega=0)$ колебание
%по своему смыслу есть движение всех атомов цепочки с постоянной
%скоростью $u_n=Vt$. Другими словами, нулевая частота означает
%отсутствие возвращающей силы при таких колебаниях и уравнения
%динамики в модели <<шариков и пружинок>> принимают вид
%$\frac{d^2u_n}{dt^2}=0$ с решением типа $u_n=Vt$. В этом
%случае полный импульс цепочки получается $P=MNV$.
%\end{sol}
\begin{hiProb}[2.20]
\end{hiProb}
\begin{sol}
\[
\omega = \frac{2s}{a} \left| \sin \frac{ka}{2} \right| 
.\] 
\[
\Delta \phi= k \Delta x= 10 ka= \frac{\pi}{2}\implies
 ka= \frac{\pi}{20} \ll 1
.\]
\[
	\omega \approx  \frac{2s}{a} \frac{\pi }{40} 
	\approx 1,05 \cdot 10^12 \text{ с}^{-1}
.\] 
\end{sol}
\begin{hiProb}[2.62]
\end{hiProb}
\begin{sol}
Закон сохранения энергии: $\omega=\omega'+\Omega$ 
(здесь и далее прописные буквы для фононов, простые
--- для фотонов, штрихованные для рассеянного фотона).

Закон сохранения импульса:
\[
	\mathbf{k}= \mathbf{k}'+ \mathbf{K}
.\] 
Так как фонон рождается в жидкости, периодичности нет
и вектор обратной решётки в закон сохранения квазиимпульса в принципе вводить не надо. Впрочем, можно
показать, что и в кристалле при рассеянии видимого
света могут рождаться только фононы с малыми
волновыми векторами.

При рассеянии на 90 градусо
\[
k=K_z,\quad k'=-K
.\] 
\[
	\frac{\Omega^2}{s^2}= K^2=k^2+k'^2= \left( 
	\frac{n}{c}\right) ^2 \left( 
\omega^2 +(\omega')^2\right) \approx
2\omega^2 \left( \frac{n}{c} \right) ^2
.\] 
\[
\frac{1}{R}= \frac{\omega - \omega'}{\omega}=
\frac{\Omega}{\omega} \approx \frac{\sqrt{2} n s}{c}
\approx 0,94\cdot 10^{-5}
.\] 
\end{sol}
\begin{hiProb}[2.77]
\end{hiProb}
\begin{sol}
Предлагается считать граничные условия для фононов
закреплёнными, при этом условие формирования стоячих
волн в кристалле в форме прямоугольного параллелепипеда
$\sin \left( k_x L_x \right) \sin \left( k_y L_y \right) \sin \left( k_z L_z \right) =0$. Минимальное
значение волнового вектора для возбуждаемой 
стоячей волны $k_\text{min} \simeq \frac{\pi}{L}$, где
$L$ --- характерный линейный размер (считаем
форму кристалла кубиком).

Частота такого фонона (который оказывается длинноволновым)
\[
\omega= k_{\text{min}} s
.\] 
Её надо приравнять по закону сохранения разнице энергии
спиновых подуровней. Так как по условию парамагнетик
в $s$-состоянии, то $g=2$.
\[
L \simeq \frac{\pi}{k_{\text{min}}}
=\frac{\pi s \hbar  }{\hbar \omega} = \frac{
\pi s \hbar }{g \mu_{B} B}\approx 0,57 \text{ мкм}
.\] 
\end{sol}
\begin{hiProb}[2.72]
\end{hiProb}
\begin{sol}
Мы знаем, что решения для колебания в кристалле
имеют вид (для цепочки)
\[
\omega^2 =\omega^2_{\max} \sin^2 \frac{ka}{2}
\] 
для заданной частоты получается синус больше единицы, что
соответствует комплексному волновому вектору ---
то есть волне, амплитуда которой зависит от координаты.
\[
\sin \frac{ka}{2}= 1,001
.\] 
\[
e^{ika /2}-e^{-ika /2}=2,002 i
.\] 
\[
	\left( e^{ika /2} \right) ^2 -
	2,002 i e^{ika /2}-1=0
.\] 
\[
	e^{ika /2}=1,001 i \pm \sqrt{- (1,001)^2+1} 
	=\left( 1,001 \pm -0,045 \right) i=
	\begin{cases}
		1,046i\\
		0,956i
	\end{cases}
.\] 
Подставляем $ka= \pi +i \delta$.
\[
e^{-\delta a /2}= \begin{cases}
	1,046\\
	0,956
\end{cases}
.\] 
\[
\delta_1 a=-0,090
.\] 
\[
\delta_2 a=0,090
.\] 
Получается два решения, одно соответствует волне,
растущей с ростом $X$, другое --- убывающей с ростом
$X$. Решения отличаются только знаком, они
равны по модулю.

Выбор решения зависит от того, слева или справа от иона,
на который идёт воздействие, находится интересующий
нас ион. В одном случае надо выбирать решение,
затухающее налево, в другом --- затухающее направо.
Очевидно, что так и должно быть --- затухающие
колебания симметрично распространяются в обе стороны.

Соответствен, для искомого отношения амплитуд
\[
\frac{u_{100}}{u_0}= e^{-100 \cdot 0,090}=1,23
\cdot 10^{-4}
.\] 
\end{sol}
\section{Теплоёмкость твёрдого тела.
Модель Дебая}
\begin{hiProb}[2.27]	
\end{hiProb}
\begin{sol}

\end{sol}
Первая зона Бриллюэна для квадратной решётки также
имеет форму квадрата со стороной $2\pi /a$. Максимальная частота
\[
	\omega_{\max}=\omega\left( \frac{\pi}{a},\,
	\frac{\pi}{a}\right) = 2\sqrt{2} 
	\sqrt{\frac{\gamma}{M}} 
.\] 
Длинноволновой предел
\[
	\omega^2= 2 \frac{\gamma}{M} \left( 
	2- \left( 1- \left( K_x a \right) ^2 /2 \right) - \left( 1-\left( K_y a \right) ^2 /2\right) \right) 
	=\frac{\left( Ka \right) ^2 \gamma}{M}
\] 
изотропен (от направления не зависит), скорость
звука $\displaystyle  s= a \sqrt{\frac{\gamma}{M}} $.

В модели Дебая заменяем спектр линейным изотропным
$\omega =sK$ и ограничиваем его сверху так, чтобы
полное число колебаний сохранилось:
\[
	N= \frac{S \pi k_D^2}{\left( 2\pi \right) ^2}
\]
где $N$ --- число атомов, а $S$ --- площадь решётки.
\[
k_D^2= \frac{4\pi}{a^2}
.\] 
\[
k_D= \frac{2\sqrt{\pi} }{a} \approx \frac{3,545}{a}
> \frac{\pi}{a} = k_{Br}
\] 
и для частоты
\[
\omega_D = \frac{2 \sqrt{\pi}s}{a}=
2 \sqrt{\pi}  \sqrt{\frac{\gamma}{M}} > \omega_{\max}
.\] 
\begin{hiProb}[Т2-1]
\end{hiProb}
\begin{sol}
Нужно, во-первых, перейти к теплоёмкости на примитивную
ячейку. У NaCl гранецентрированная решётка с четырьмя
формульными единицами на элементарный куб.
То есть, примитивная ячейка имеет объём 1/4 элементарного куба:
 \[
C _{\text{прим}}= \frac{d^3}{4}C
.\] 
Считаем, что температура достаточно низка для применения низкотемпературного приближения:
\[
C_\text{прим} = \frac{C_{\text{Дебай}}}{N}\approx
\frac{12}{5} \pi^4 k_B \left( \frac{T}{\Theta} \right) ^3
.\] 
\[
	\Theta= T \left( \frac{48 \pi^4}{5}
	\frac{k_B}{C d^3}\right) ^{1 /3} \approx
	315 \text{ К}
.\] 
Далее ищем скорость звука \[\Theta = \frac{\hbar s}{k_B}\left( 6 \pi^2 n \right) ^{1 /3}= \frac{\hbar s}{
k_B} \left( 6 \pi^2 \frac{4}{d^3} \right) ^{1 /3}=
\frac{2 \sqrt[3]{3\pi^2}\hbar s }{d k_B},\]
при вычислении дебаевской температуры не забываем, что
кубическая ячейка не примитивная.

Откуда окончательно для усреднённой скорости звука
\[
	s= \frac{k_B \Theta d}{2 \sqrt[3]{3\pi^2} \hbar }=3,76 \cdot 10^5 \frac{\text{см}}{\text{с}}
.\] 
\end{sol}
\begin{hiProb}[2.47]
\end{hiProb}
\begin{sol}
	При нулевых граничных условиях (закреплённая
	граница, т.\:е. амплитуда колебаний на
	границе равна нулю) смещение $U$ может
	быть записано в виде
	\[
	U= A e^{i \omega_n t} \sin 
	\frac{\pi x}{L} n_x \cdot \sin \frac{\pi y}{
	L} n_y \cdot \sin \frac{\pi z}{L} n_z
	,\]
	а частота $n$-го колебания $\omega_n$ 
	равна
	\[
		\omega_n^2 = \left( 
		\frac{\pi s}{L}\right) ^2
		\left( n_x^2 +n_y^2 +n_z^2 \right) 
	.\] 
	Числа $n_x,\ n_y,\ n_z$ принимают все целочисленные значения $\ge 1$. Энергия колебаний
	\[
		\mathcal{E}=3
		\sum_{n_x,n_y,n_z}^{} \frac{
		\hbar  \omega_n}{\exp \left( 
	\frac{\hbar  \omega_n }{k_\text{Б} T}-1\right) }
	,\] 
	где коэффициент  <<3>> --- это три
	независимые поляризации. При больших
	$L$ сумма может быть заменена интегралом
	\[
		\mathcal{E}(T)=
		\int\limits_{\omega_{\min}}^{\omega_{\max}} d\omega \mathcal{D} (\omega) n(\omega, T)
		\hbar  \omega,\quad
		\text{где }  \mathcal{D}(\omega)=
		\frac{3}{2} \frac{V \omega^2}{\pi^2 s^3}
	,\]
	откуда энергия единицы объёма кластера
	\[
		u_{\text{класт}}(T)=
		\frac{\mathcal{E}}{V}=
		\frac{3}{2} \frac{\hbar }{\pi
		^2 s^3} \int\limits_{\omega_{\min}}^{\omega_{\max}} \frac{\omega^3 d \omega}{e^{\hbar \omega /k_\text{Б} T}-1}= \frac{3}{2}
		\frac{(k_\text{Б}T)^4}{\pi
		^2 \hbar ^3 s^3}
		\int\limits_{x_{\min}}^{x_{\max}} 
		\frac{x^3 dx}{e^x-1}
	.\] 
	Здесь $\omega_{\max}= \frac{\pi s}{L}\left( 
	n_x^2 +n_y^2 +n_z^2\right) _{\max}\simeq
	\frac{\pi s}{L}N$, $L=Na$. Точное
	значение коэффициента при низких температурах
	несущественно.
	С другой стороны,
	\[
		\omega_{\min}= \frac{\pi s}{L} \left( 
		n_x^2+n_y^2+n_z^2\right) _{\min}=
		\begin{cases}
			0 & \text{при } L\to 0,\\
			\frac{\pi s}{L}\sqrt{3} &
			\text{при } L<\infty.
		\end{cases}
	\] 
	Кроме того, в интеграле была сделана стандартная замена переменныых
	\[
	x_{\max}= \frac{\hbar  \omega_{\max}}{k_\text{Б}T}= \frac{\pi s \hbar }{a k_\text{Б}T}=
	\frac{\theta}{T};\quad
	x_{\min}= \frac{\hbar \omega_{\min}}{k_\text{Б}T}= \frac{\pi s \sqrt{3} }{10a k_\text{Б}T}=
	\frac{\theta}{T} \frac{\sqrt{3} }{10}
	.\] 
Таким образом, энергия единицы объёма кластера
\[
	u_{\text{класт}}= \frac{3}{2} \frac{\left( 
	k_\text{Б}T\right) ^4}{\pi^2 \hbar ^3
s^3} \int\limits_{\frac{\theta}{T}\frac{\sqrt{3} }{10}}^{\theta /T} \frac{x^3 dx}{e^x-1} 
.\] 
Количество теплоты, необходимое для нагрева единицы
объёма кластера
\[
Q_{\text{класт}}= \Delta u_{\text{класт}}=
u_{\text{класт}} \left( \frac{\theta}{30} \right) -
u_\text{класт}(0)= \frac{3}{2} \frac{
\left( k_\text{Б}\frac{\theta}{30} \right)^4 }{
\pi^2 h^3 s^3} \int\limits_{3 \sqrt{3} }^{30} 
\frac{x^3 dx}{e^x-1}
.\] 
Для единицы объёма большого тела
\[
\Delta u_\text{тела}= \frac{3}{2\pi^2 \hbar ^3 s^3}
\left( \frac{k_\text{Б} \theta}{30} \right) ^4
\int\limits_{0}^{30} \frac{x^3 dx}{e^x -1} \text{ и
 тогда }  \frac{Q_\text{класт}}{Q_\text{тела}}=
\frac{\Delta u_{\text{класт}}}{\Delta u_\text{тела}}=
\frac{\int\limits_{3 \sqrt{3} }^{30}  \frac{x^3 dx}{e^x-1}}{\int\limits_{0}^{30} \frac{x^3 dx}{e^x -1} }
.\] 
В этих интегралах верхний предел можно
заменить на бесконечность, а поскольку
$
e^{3\sqrt{3} }\simeq 180 \gg 1, 
$, то
\[
\int\limits_{3\sqrt{3} }^{30} \frac{x^3 dx}{e^x -1}
\approx \int\limits_{3\sqrt{3} }^{\infty} e^{-x}
x^3 dx =1,43;\quad
\int\limits_{0}^{\infty} \frac{x^3 dx}{e^x-1}=
\frac{\pi^4}{15} \approx 6,5
.\] 
Отсюда \[
\frac{Q_\text{класт}}{Q_\text{тела}}=\frac{1,43}{
6,5}=0,22
.\] 

Если же непосредственно подсчитать сумму всех
возможных колебаний, то ограничиваясь числами
$(n_x,\,n_y,\,n_z)$: $(1,\,1,\,1)$; $(1,\,1,\,2)$;
$(1,\,2,\,1)$; $(2,\,1,\,1)$; $(2,\,2,\,1)$;
$(2,\,1,\,2)$; $(1,\,2,\,2)$; $(1,\,1,\,3)$;
$(1,\,3,\,1)$; $(3,\,1,\,1)$ и $(2,\,2,\,2)$,
получаем
\[
\frac{Q_\text{класт}}{Q_\text{тела}}=0,13
.\] 
\end{sol}
\begin{hiProb}[2.58]
\end{hiProb}
\begin{sol}
В условии даётся температура Дебая 19 К, смысл которой
для жидкости странен. Тут важно, что температура много
меньше ротонного минимума (который около 8 К),
поэтому есть только фононы, а их спектр до энергий,
соответствующий температуре
0,5 К нашей задачи, можно считать линейным.
В жидкости есть единственная поляризация
звуковых волн (продольная).
\[
	E= \frac{V}{(2\pi)^3} \int
	\frac{\hbar \omega}{\exp \left( \frac{\hbar \omega}{k_B T} \right) -1}d^3k= \frac{V}{2\pi^2}
	\frac{(k_B T)^4}{(\hbar s)^3} \int\limits_{0}^{\infty}  \frac{x^3}{e^x -1}dx=
	V \frac{\pi^2}{30} \frac{(k_B T)^4}{(\hbar 
	s)^3}
.\] 
\[
	\frac{C}{V}= \frac{2}{15} \pi^2 k_B \left( 
	\frac{k_B T}{\hbar s}\right) ^3
\] 
и далее по формулам термодинамики
\[
	\frac{S(T_1)}{V}= \int\limits_{0}^{T_1} 
	\frac{C}{T} dT= \frac{2}{45}
	\pi^2 k_B \left( \frac{k_B T_1}{\hbar  s} \right) ^3
\]
для искомой удельной энтропии
\[
S_\text{уд} = \frac{2}{45}\frac{\pi^2 k_B}{\rho}
\left( \frac{k_B T}{\hbar s} \right) ^3 \approx
8,5 \cdot 10^3 \frac{\text{эрг}}{\text{К}\cdot
\text{г}}
.\] 
\end{sol}
\begin{hiProb}[2.75]
\end{hiProb}
\begin{sol}
	Аналогично задаче $2.74$
	наложим закреплённые граничные условия.
	От граничных условий несколько меняется
	ответ в этой задаче, но по постановке
	<<мысленного опыта>> по определению
	конечной длины цепочки, для которой
	роль квантовых колебаний
	ещё неразрушительна, закреплённые
	граничные условия кажутся даже более
	логичными.

	Тогда собственные моды колебаний в одномерном
	случае имеют вид $u_k= A_{0k} \sin (kx)
	\sin (\omega t)$, $k>0$, на одно
	состояние приходится объём $\pi /L$ в одномерном
	$k$- пространстве. Для среднего квадрата
	смещения $\left<\left<u^2 \right>  \right> =
	\frac{1}{4} \sum_{}^{} A_{0k}^2$, а из
	сравнения каждой моды с гармоническим
	осциллятором
	\[
	\frac{\hbar  \omega}{2}= E_k =
	\left<E_k \right> = 2 \left<K_k \right> =
	M \sum_{n}^{} \left< V_n^2 \right> =
	\frac{MN \omega_k^2 A_{0k}^2}{4} \text{ и }
	A_{0k}^2= \frac{2\hbar }{MN\omega_k}
	.\] 
	В одномерном случае для единственной
	поляризации, в дебаевской модели,
	ограничивая нижний предел из-за конечности
	цепочки:
	\[
	\frac{\left<\left<u^2 \right>  \right> }{a^2}
	=\frac{L}{\pi} \frac{2 \hbar }{N Ma^2}
	\int\limits_{k_{\min}}^{k_D} \frac{dk}{sk}=
	\frac{2 \hbar }{\pi s Ma} \ln \frac{k_D}{k_{\min}}
	.\] 
	Для однородной цепочки $k_D= \pi /a$, а 
	для закреплённых граничных условий
	$k_{\min}= \pi /L$.

	Окончательно
	\[
	\alpha= \frac{2\hbar }{\pi s M a}\ln \frac{L}{a}
	.\] 
	\[
	\ln \frac{L}{a}= \alpha \frac{\pi s Ma}{2\hbar } \approx 11,2
	.\] 
	\[
	L \approx 22 \text{ мкм}
	.\] 
\end{sol}
\section{Электронный ферми-газ}
\begin{hiProb}[3.44]
\end{hiProb}
\begin{sol}
В обоих случаях отдаётся в зону проводимости
по одному электрону на атом, примитивные 
ячейки содержат единственный ион, так что
числа электронов проводимости, примитивных
ячеек и атомов совпадают.

Для электронной теплоёмкости пользуемся ферми-газовой
моделью. Теплоёмкость электронов линейна по температуре
при $T \ll E_F$, то есть вплоть до температуры
плавления металла:
\[
	C_{el}= \frac{k_B N T m}{\hbar ^2}
	\left( \frac{\pi}{3n} \right) ^{2 /3}=
	\frac{\pi^2 k_B ^2 NT}{2E_F}
.\] 
Теплоёмкость ферми-газа выводится на лекции.

Фононная теплоёмкость при высоких температурах $(T>\theta)$ равна $C_{ph}= 3 N k_B$, что в 
$\sim \frac{E_F}{(k_B T)}\gg 1$ раз больше
электронной. Значит сравниваются теплоёмкости
при низких температурах.

При низких температурах для фононной теплоёмкости
есть формула Дебая:
\[
	C_{ph}= \frac{12 \pi^4 N k_B}{5} \left( \frac{T}{\theta} \right) ^3
.\] 
Приравнивая, получаем
\[
T^2= \frac{5 k_B \Theta^3}{24 E_F \pi^2}
.\] 
После подстановки численных значений, получаем
ответ: 3,3 К для меди и 1,5 К для натрия.
Полученные числа оправдывают применение дебаевского
приближения.
\end{sol}
\begin{hiProb}[3.53]
\end{hiProb}
\begin{sol}
	В равновесии (когда пластины соединили)
	установится такое распределение заряда, что
	$e\phi +\mu = \mathrm{const}$ $(e<0)$,
	то есть выравнивается электрохимпотенциал.
	Электронам выгодно понижать свою энергию,
	переходя из натрия в медь (поверхность Ферми
	меди ниже по энергии), но такие переходы
	нарушают электронейтральность и
	возникает задерживающая разность
	потенциалов. При этом массивные металлические
	образцы можно считать в равновесии
	эквипотенциальными.

	Значит разность электрических потенциалов
	равна $A_{\mathrm{Cu}}-A_{\mathrm{Na}} /e$,
	чтобы обеспечить эту разность потенциалов,
	перетёк заряд $C(A_{\mathrm{Cu}}-A_{\operatorname{Na}}) /e=2,2 \cdot 10^{-2} \text{ Кл}=1,38
	\cdot 10^7 e$ (зарядов электрона),
	что составляет $5,2 \cdot 10^{-16}$ 
	от общего числа электронов в образце.
\end{sol}
\begin{hiProb}[3.59]
\end{hiProb}
\begin{sol}
Пусть плотность состояний на уровне Ферми для
каждого из направлений спина равна $D'$.
В модели с квадратичным законом дисперсии
\[
	D'=\frac{dN}{dE}= \frac{V 4 \pi k_F^2 df /(2\pi)^3}{
	\hbar ^2 k_F dk/ m^*} =\frac{1}{2}
	\frac{m^* p_F}{\pi^2 \hbar ^3}= \frac{1}{2}
	\frac{m^*}{\hbar ^2} \sqrt[3]{\frac{3n}{\pi^4}} 
,\]
или, по-другому, \[D'= \frac{3n}{4 E_F}.\] Здесь $n$ ---
полная концентрация электронов, $D'$ 
вдвое меньше  полной плотности состояний.

Условие, что поле мало означает, что изменение
распределения электронов мало. Тогда можно
считать, что плотность состояний не изменилась при
приложении поля. Считаем магнетизм чисто спиновым:
$g=2$, магнитный момент каждого  электрона
равен боровскому магнетону и может быть направлен
либо по полю, либо против поля.

Это значит, что электронов, магнитный момент
которых направлен по полю (энергия которых понижается),
стало в единице объёма больше на $D' \mu_B H$,
а электронов, магнитный момент которых направлен
против поля (энергия которых повышается) ---
меньше на ту же самую величину $D' \mu_B H$. Естественно, полное число электронов сохранилось (система
осталась электронейтральной).

Таким образом
 \[
\frac{\delta n}{n}= \frac{2 D' \mu_B H}{n}=
\frac{3\mu_B H}{2E_F}
.\] 
Для поля 10 Тл $\frac{3}{2}\mu_B H \sim 1$ мэВ,
то есть в реальных лабораторных полях в электронном
газе в типичном металле ($E_F \sim 1$ эВ) перераспределяется ничтожная доля электронов проводимости.

Магнитный момент без поля был равен 0, а в поле
стал равен  \[M=\mu_B \delta n= H \mu_B^2 \frac{m^*}{\hbar ^2}
\sqrt[3]{\frac{3n}{\pi^4}} .\]
Поскольку для каждой проекции спина
\[
\mu_B H= \delta E= \frac{p_F \delta p}{m}
,\] 
то
\[
\frac{\delta p}{p_F}=2 m^* \frac{\mu_B H}{p_F^2}=
\frac{\mu_B H}{E_F}
.\] 
Для восприимчивости:
\[
\chi= M /H = \mu_B^2 \frac{m^*}{\hbar ^2}
\sqrt[3]{\frac{3n}{\pi^4}} =5,2 \cdot 10^{-7}
.\] 
\end{sol}
\begin{hiProb}[3.87]
\end{hiProb}
\begin{sol}
Для фононной теплоёмкости одномерной цепочки
при низких температурах (на единицу длины)
\[
E= 2 \int\limits_{0}^{\infty} \frac{\hbar k s}{
\exp \left( \frac{\hbar  k s}{T} \right) -1}
\frac{dk}{2\pi}= \frac{1}{\pi}
\frac{T^2}{\hbar  s} \int\limits_{0}^{\infty} 
\frac{xdx}{e^x -1}= \frac{\pi}{6} \frac{T^2}{\hbar s}
\]
двойка учитывает движение фононов в обе
стороны, поляризация для одномерной системы
единственная. Для теплоёмкости, возвращая
в запись постоянную Больцмана,
\[
C_{ph}= \frac{\pi k_B^2 T}{3\hbar  s}
.\] 
Скорость Ферми по определению $v_F = \left( \frac{dE}{dp} \right) _{E_F}$ независимо от вида спектра.
Для электронной теплоёмкости
\[
	C= \frac{\pi^2}{3} D(E_F) T
\]
плотность состояний в одномерном случае,
но не делая явных предположений о спектре.
\[
D= \frac{dN}{dE}= \frac{dN}{dp} \cdot \frac{dp}{dE}=
2 \cdot 2 \cdot  \frac{1}{\hbar }\frac{L}{2\pi}
\cdot \frac{1}{v_F}
\]
(учтён спиновый множитель 2 и множитель 2,
учитывающий движение электронов в обе стороны),
откуда электронная теплоёмкость на единицу длины с
постоянной Больцмана
\[
C_{el}= \frac{2\pi}{3} \frac{k_B^2 T}{\hbar v_F}
.\] 
Соответственно,
\[
\frac{C_{el}}{C_{ph}}= \frac{2s}{v_F} \sim 2\cdot 
10^{-2}
.\] 
\end{sol}
\begin{hiProb}[3.61]
\end{hiProb}
\begin{sol}
У ферми-жидкости два свойства:
\begin{enumerate}
\item Концентрация и фермиевский импульс в ней
	соответствуют полному числу частиц
\item  Вблизи Ферми-поверхности наблюдается
	перенормированная (изменённая) масса,
	соответствующая изменённой плотности
	состояний. Именно эта масса и определяет
	все термодинамические свойства.
\end{enumerate}
Спин ядра гелия-3 равен 1/2, электронный спин полностью заполненной $s$-оболочки нулевой,
поэтому полный спин атома равен 1/2. Атом гелия-3
является ферми-частицей и низкотемпературные свойства
гелия-3 --- это свойства жидкости. Тогда можно
использовать результат для теплоёмкости
электронного газа:
\[
C_V= \frac{k_B n T m^*}{\hbar ^2}
\left( \frac{\pi}{3n} \right) ^{2 /3}=
\frac{m^*}{m} \frac{k_B (nm)T}{\hbar ^2}
\left( \frac{\pi}{3n} \right) ^{2 /3}
.\] 
Напомним, что эта формула --- на единицу объёма.
Воспользуемся тем, что $nm=\rho$, где $m$ --- масса
атома He-3. Подставляя, и помня, что в условии ---
молярная теплоёмкость, а нужна для вычислений
теплоёмкость в единице объёма  $C_V= C_\mu \rho
/(m N_a)$, получаем после очевидных
преобразований
\[
	\frac{m^*}{m}=\frac{C_\mu \hbar ^2 \rho}{
	N_a k_B^2 T\rho m} \left( \frac{3\rho}{\pi m} \right) ^{2 /3}=2,15
.\] 
\end{sol}
\begin{hiProb}[3.28]
\end{hiProb}
\begin{sol}
Поскольку в звезде присутствует положительный фон,
который в точности равен отрицательному, то мы
пренебрегаем взаимодействием и рассматриваем
только кинетическую энергию электронов.

Будем считать закон дисперсии ультрарелятивистским:
$\epsilon =pc$, поскольку для большинства состояний
$\epsilon \gg mc^2$.

Полная концентрация электронов:
\[
	n= 2\cdot  \frac{4}{3} \pi p_F^3\cdot \frac{1}{(2\pi \hbar )^3}
.\] 
Фермиевский импульс такой же как и в обычном электронном газе (он определяется только концентрацией
и не зависит от закона дисперсии):
\[
	p_F= \hbar  (3\pi^2 n)^{1 /3}
.\] 

Полная энергия электронов в единице объёма:
\[
	E=2\cdot \frac{1}{(2\pi \hbar )^3}
	\int\limits_{0}^{p_F} cp \cdot 4\pi
	p^2 dp= 2\pi c p_F^4 \cdot 
	\frac{1}{(2\pi \hbar )^3}=0,25c \hbar 
	(9\pi n^2)^{2 /3}
.\] 
Полная энергия всего газа
\[
	E_{\text{полн}}= E V= \frac{(9\pi)^{2 /3}}{4}
	c \hbar  N^{4 /3} \frac{1}{V^{1 /3}}
.\] 
\[
	P=-\frac{dE_{\text{полн}}}{dV}=\frac{(9\pi)^{2 /3}}{12} c \hbar  n^{4 /3},\quad \text{или }
	pV^{4 /3}=\mathrm{const}
.\] 
\end{sol}
\section{Зонная структура}
\begin{hiProb}[3.38]
\end{hiProb}
\begin{sol}
По определению групповой скорости
\[
V=\frac{\partial E}{\partial p} 
.\] 
\[
	V_y= \frac{\epsilon_0 a}{\hbar } \sin \left( \frac{p_y a}{\hbar } \right) =0=V_z,\quad
	V_x= \frac{\epsilon_0 a}{\hbar }\sin
	\frac{5\pi}{6}= \frac{\epsilon_0 a}{2\hbar }
.\] 
Ускорение
\[
\frac{dV}{dt}= \frac{d}{dt} \frac{\partial E}{\partial p} =
\lim_{\Delta t \to 0} \frac{\left. \frac{\partial E}{\partial p}  \right|_{p=p+ \frac{dp}{dt}\Delta t}-
	\left. \frac{\partial E}{\partial p}  \right|_{p=p}}{\Delta t}
.\] 
По определению
\[
\frac{dp_x}{dt}= \frac{dp_z}{dt}=0,\quad
\frac{dp_y}{dt}= \frac{He}{c} \frac{\epsilon_0 a}{
2\hbar }
.\] 
\[
\frac{dV_y}{dt}=\frac{He}{c} \frac{\epsilon_0^2 a^3}{2\hbar ^3}
,\]
соответственно
\[
\frac{dV_x}{dt}= \frac{dV_z}{dt}=0
.\] 
\end{sol}
\begin{hiProb}[3.85]
\end{hiProb}
\begin{sol}
Ответом в задаче будет
\[
E= \frac{2\pi \hbar  c}{\lambda_0}+E_F=E_F+4,61 \text{ эВ}
,\] 
так как красной границе фотоэффекта соответствует
<<выбивание>> электрона с уровня Ферми.

ГЦК-решётка означает четыре атома на элементарный
куб. У серебра один электрон проводимости на атом.
Это значит, что на элементарный куб приходится
четыре электрона. Концентрация электронов $n=4 /a^3$,
для импульса Ферми пользуемся известным выражением
\[
	k_F= \sqrt[3]{3\pi^2 n} = \frac{\sqrt[3]{12 \pi^2} }{a}
.\]
Откуда имеем
\[
p_F= \hbar  k_F=1,26 \cdot  10^{-24} \frac{\text{кг}\cdot \text{м}}{\text{с}}
\]
и соответственно
\[
E_F= \frac{p_F^2}{2m}=5,44 \text{ эВ}
.\] 
Добавляя расстояние от уровня Ферми до уровня свободного электрона 4,61 эВ, получим ответ $E=10,05$ эВ.
\end{sol}
\begin{hiProb}[3.57]
\end{hiProb}
\begin{sol}
Во втором металле закон дисперсии в той же системе
координат имеет вид:
\[
E_{\operatorname{II}}= \frac{p_x^2}{2m_y^*}+
\frac{p_y^2}{2m_x^*}+ \frac{p_z^2}{2m_z^*}
\] 
(из-за поворота локальных осей кристалла движение
вдоль оси $X$ теперь характеризуется эффективной
массой $m_y$ и наоборот).

Перейдём к вычислению закона преломления. Пусть в
области I был электрон с импульсом  $(p_x,\,p_y,\,0)$,
после прохождения границы в области II он
перейдёт в состояние с импульсом  $(p_{x \mathrm{I}},\,
p_{y \mathrm{I}},\,0)$. Поскольку испускания или
поглощения фононов и других неупругих процессов нет,
энергия сохраняется, то есть:
\[
\frac{p_x^2}{m^*_x}+ \frac{p_y^2}{m^*_y}=
\frac{p_{x\mathrm{I}}^2}{m_y^*}+
\frac{p_{y\mathrm{I}}^2}{m_x^*}
.\] 
Другая сохраняющаяся величина это $p_x=p_{x \mathrm{I}}$. Данный закон сохранения квазиимпульса связан с
тем, что система трансляционно ивнвариантна вдоль
направлений $Ox$ и $Oz$.
Соответственно
\[
	p_{y\mathrm{I}}=\sqrt{
	p_x^2 \left( 1- \frac{m_x^*}{m_y^*} \right) +
p_y^2 \frac{m_x^*}{m_y^*}} 
.\] 
Данные выражения для импульсов решают задачу.

Найдём закон преломления. Пусть электрон падал
под углом $\alpha$ (отсчитываем от нормали к поверхности раздела, то есть от оси $Y$),  $\tg \alpha =
 p_x /p_y$, а вылетел под углом $\beta$, 
 $\tg \beta =p_x /p_{y\mathrm{I}}$. Тогда,
выражая всё через $p_x$ и подставляя в закон
сохранения энергии, имеем:
\[
\frac{1}{m^*_x}+ \frac{\ctg^2 \alpha}{m^*_y}=
\frac{1}{m^*_y}+\frac{\ctg^2 \beta}{m^*_{x}}
.\] 
\[
\frac{\ctg^2 \alpha -1}{m^*_{y}}= \frac{\ctg ^2 \beta-1}{m^*_x}
.\] 
\[
\frac{1}{m^*_y} \frac{\cos 2\alpha}{\sin^2 \alpha}=
\frac{1}{m^*_x} \frac{\cos 2 \beta}{\sin^2 \beta }
.\] 
Откуда
\[
\frac{\sin \alpha}{\sin \beta}\sqrt{\frac{\cos 2\beta}{\cos 2\alpha}} =\sqrt{\frac{m_x^*}{m_y^*}} 
.\] 
Асимптотика при малых углах
\[
	\frac{\alpha}{\beta}= \sqrt{\frac{m^*_x}{
	m^*_y}} 
.\] 
\end{sol}
\begin{hiProb}[4.54]
\end{hiProb}
\begin{sol}
Основное свойство кристаллов --- их трансляционная
симметрия. Поэтому для описания кристалла можно
воспользоваться следующим приёмом: взять кристалл
конечных размеров и транслировать его для
получения кристаллов произвольных размеров. При
этом размер этого кристалла (т.\:н. основная
область) должен быть выбран так, чтобы сохранилась
трансляционная симметрия, т.\:е. на основных
направлениях трансляции укладывалось целое
число элементарных (примитивных) ячеек. При
этом набег фазы волновой функции электрона вдоль
соответствующего направления будет кратен
$2\pi$. Это приводит к дискретным значениям проекций
квазиимпульса в зоне Бриллюэна, определяемым
размерами основной области кристалла. Поскольку
размер зоны Бриллюэна однозначно определяется
размеров примитивной ячейки, число разрешённых
значений квазиимпульса в зоне Бриллюэна равно
числу примитивных ячеек. Если кристалл графена
состоит из $N$ атомов, а в примитивной ячейке
графена содержится два атома, то число разрешённых
значений будет равно $N /2$. Каждому значению
квазиимпульса, как следует из закона дисперсии,
соответствует два уровня энергии: один в валентной
зоне, другой --- в зоне проводимости. Поэтому число
мест для электронов с учётом спинового вырождения
будет равно $2\cdot 2\cdot N /2 =2N$. Из них
$N$ мест будет в валентной зоне и $N$ --- в зоне
проводимости. Т.\:к. на один атом приходится по
одному <<свободному>> электрону, то всего делокализованных
электронов будет $N$. При нулевой температуре электроны
занимают наинизшие энергетические состояния, поэтому
вследствие симметричного характера спектра
вся валентная зона будет заполнена, а зона проводимости
будет пустой. Таким образом для указанного вида
спектра химический потенциал графена, отсчитываемый
от уровня энергии электрона в атоме, равен нулю
и уравнение для определения формы ферми-поверхности
имеет вид
\[
1+4 \cos p_x a \cos \frac{p_x a}{\sqrt{3} }+
4 \cos ^2 \frac{p_y a}{\sqrt{3} }=0
.\] 
Т.\:к. $\displaystyle \cos \frac{p_y a}{\sqrt{3} }
 \neq 0$, то уравнение можно переписать так:
\[
-4 \cos  p_x a = 4 \cos \frac{p_y a}{\sqrt{3} }+
\frac{1}{\cos \frac{p_y a}{\sqrt{3} } }
.\] 
Согласно неравенству о среднем арифметическом и
среднем геометрическом, правая часть уравнения
$\ge 4$ при $\displaystyle \cos \frac{p_y a}{\sqrt{3} }>0$, либо $\le -4$ при $\displaystyle \cos \frac{p_y
a}{\sqrt{3} }<0$.

Поскольку левая часть уравнения ограничена $-4 \le 
4 \cos p_x a\le 4$, то равенство возможно
только если одновременно
\[
\left\{
\begin{aligned}
\cos p_x a &= 1 \\
\cos \frac{p_y a}{\sqrt{3} }&= -\frac{1}{2} \\
\end{aligned}
\right.
\text{ или }
\left\{
\begin{aligned}
\cos p_x a&= -1 \\
\cos \frac{p_y a}{\sqrt{3} }=\frac{1}{2},
\end{aligned}
\right.
\]
откуда
\[
\left\{
\begin{aligned}
p_x a&= 2 \pi m \\
\frac{p_y a}{\sqrt{3} }&= \pm \frac{2}{3}
\pi+2 \pi s\\
\end{aligned}
\right.
\text{ или }
\left\{
\begin{aligned}
	p_x a&= \pi +2\pi n\\
	\frac{p_y a}{\sqrt{3} }&=  \pm \frac{1}{3}
	\pi +2 \pi l.\\
\end{aligned}
\right.
\] 

Подставляя данные, получаем, что
\[
\left\{
\begin{aligned}
p_x &= \frac{4\pi \hbar }{3b}m \\
p_y &= \pm \frac{4\pi \hbar }{3 \sqrt{3} b}+
\frac{4\pi \hbar }{\sqrt{3} b}s\\
\end{aligned}
\right.
\text{ или}
\left\{
\begin{aligned}
	p_x&= \frac{2\pi \hbar }{3b}+ \frac{4\pi \hbar }{3b}
m\\
	p_y &= \pm  \frac{2\pi \hbar }{3\sqrt{3} b}+
\frac{4\pi \hbar }{\sqrt{3} b}l.
\end{aligned}
\right.
\] 

Отбирая только те значения квазиимпульса, которые
попадают в первую зону Бриллюэна, получаем уравнение
прямых линий в $\mathbf{p}$-пространстве:
 \[
\left\{
\begin{aligned}
p_x&= 0 \\
p_y &= \pm \frac{4\pi \hbar }{3\sqrt{3} b} \\
\end{aligned}
\right.
\text{ или }
\left\{
\begin{aligned}
p_x &= \pm  \frac{2\pi \hbar }{3b} \\
p_y &= \pm \frac{2\pi \hbar }{3\sqrt{3} b}. \\
\end{aligned}
\right.
\] 

Пересечение указанных линий даёт вершины шестиугольника.
Таким образом, поверхность Ферми графена состоит
из шести точек и её площадь равна нулю.

Как видно из закона дисперсии, для любого направления
в обратном пространстве кроме диагоналей шестиугольника, на границе зоны Бриллюэна возникает щель 
(запрещённая зона) в спектре электронов. Она
максимальна в середине сторон шестиугольника и
уменьшается в стороны вершин, где исчезает вовсе
(зона проводимости соприкасается с валентной
зоной). С точки зрения своих электронных свойств,
графен можно считать либо бесщелевым полупроводником,
либо полуметаллом с нулевым перекрытием зон.
\end{sol}
\begin{hiProb}[Т4-2]
\end{hiProb}
\begin{sol}
\begin{figure}[ht]
    \centering
    \incfig{6}
    \caption{Схема заполнения электронных состояний до (слева)
    и после (справа) димеризации цепочек}
    \label{fig:6}
\end{figure}
В исходной цепочке один электрон на примитивную ячейку и первая
зона Бриллюэна одномерной цепочки оказывается заполнена ровно на половину --- поэтому это и будет проводником (металлом). После димеризации объём, занимаемый
электронами в $k$-пространстве не меняется, а первая
зона Бриллюэна уменьшается вдвое. Изменение
периодичности кристалла приводит к появлению
дополнительного вклада в создаваемый им потенциал
с периодом $2a$, этот вклад мал в силу малости смещения ионов и может быть учтён в рамках приближения слабой
связи.

Точный расчёт задачей не требуется, но результат
применения приближения слабой связи очевиден ---
на границе новой зоны Бриллюэна образуется
разрыв спектра и запрещённая зона. При этом
вблизи границы энергия электронов из нижней
ветви станет ещё немного ниже. В результате
суммарная кинетическая энергия электронов понизится.
Таким образом, в этом фазовом переходе выигрыш
в энергии возникает за счёт уменьшения
суммарной кинетической энергии электронов и
оказывается, что в одномерном случае
он всегда больше проигрыша в упругой энергии при
деформации (проигрыш $\propto \delta^2$, выигрыш
$\propto \delta^2 \ln \delta$).

После димеризации нижняя энергетическая зона
окажется заполнена полностью и система станет
диэлектриком.
\end{sol}
\begin{hiProb}[Т4-3]
\end{hiProb}
\begin{sol}
Для простой кубической решётки обратная решётка
также простая кубическая с периодом $2\pi /a$.
Фермиевский волновой вектор $k_F = \sqrt[3]{3\pi^2 n} \approx 3,09 /a< \pi /a$, поэтому ферми-сфера
полностью умещается в первой зоне Бриллюэна.
В силу слабости взаимодействия (по условию)
искажением ферми-поверхности от идеальной
сферической формы пренебрегаем.

В силу слабости взаимодействия электронов с периодическим потенциалом кристалла изменение спектра можно
считать слабым, запрещённые зоны узкими и за
исключением непосредственной окрестности границ
зоны Бриллюэна можно считать спектр квадратичным
\[
	E(\mathbf{k})= \frac{\hbar ^2 k^2}{2m}
.\] 
Импульс кванта света в УФ-диапазоне много меньше
фермиевского, поэтому в рамках приведённой зонной
схемы переход происходит <<вертикально>> ---
без изменения импульса электрона. Мы можем добавить
вектор обратной решётки, чтобы перейти к расширенной
или периодической зонной схемам. Необходимо после
такого перехода <<вверх>> на $\hbar \omega$ и
<<вбок>> на $\mathbf{G}$ попасть на квадратичный
спектр с минимальным приростом энергии. Очевидно,
что минимальному изменению энергии будет
соответствовать смещение на вектор обратной
решётки минимальной длины, например, $\left( 2\pi /a;\, 0;\, 0 \right) $ из точки в $k$-пространстве
$\left( -k_F;\,0;\,0 \right) $. Так как $k_F$ 
всё же близко к границе зоны Бриллюэна, то и
оттранслированная точка оказывается близка к
границе:
\[
\frac{2\pi}{a} -k_F \approx \frac{3.19
}{a}
.\] 
Отсюда искомая энергия кванта
\begin{multline*}
	\hbar  \omega= \frac{\hbar ^2}{2m} \left( 
	\left( \frac{2\pi}{a}-k_F \right) ^2-
k_F^2\right) = \frac{\hbar ^2 k_F^2}{2m} \left( 
\frac{(2\pi)^2}{\left( 3\pi ^2 \right) ^{2 /3}}-
2 \frac{2\pi}{\left( 3\pi^2 \right) ^{1 /3}}\right) 
\approx \\
\approx 0,068 E_F= 0,2 \text{ эВ }
.\end{multline*} 
Это энергия кванта ИК-диапазона, она существенно
меньше типичных значений работы выхода для металла.
\end{sol}
\section{Кинетические и электрические явления
в твёрдых телах и металлах}
\begin{hiProb}[2.65]
\end{hiProb}
\begin{sol}
Для фононов можно использовать оценочную формулу
для газовой кинетической теории 
\[
\kappa_{ph}= \frac{1}{3} C s\Lambda
.\] 
При низких температурах $C \propto T^3$.
Рассеяние фононов может происходить на других
фононах, краях образца, дефектах. Время
рассеяния на других фононах будет обратно
пропорционально их концентрации $(T^3)$. Следовательно,
данный процесс будет вымерзать при понижении температуры, что и происходит ниже 7 К, согласно условию
задачи. По-видимому также образцы содержат мало
дефектов, раз теплопроводность зависит от толщины
образца.

По аналогии с кнудсеновским течением газа длину
свободного пробега  $\Lambda$ надо принять
равной расстоянию между границами кристалла. Тогда
увеличение толщины кристалла в 4 раза увеличит
и $ \Lambda$ и $\kappa$ тоже в 4 раза.
\end{sol}
\begin{hiProb}[3.77]
\end{hiProb}
\begin{sol}
При $T > \Theta$ можно для каждого атома
применить теорему о равнораспределении энергии
по степеням свободы, считая, что каждый атом находится
в изотропном гармоническом потенциале, описываемом
<<жёсткостью>> $k$:
\[
\frac{3}{2} k_B T= k \frac{\overline{\xi^2}}{2}=
k \frac{\overline{x^2 +y^2+z^2}}{2}
.\] 
Коэффициент упругости <<отдельной пружинки>>
$k$ можно оценить, зная модуль Юнга. Действительно,
если мы растягиваем весь кристалл с относительным
удлинением $\epsilon $, то мы  просто растягиваем
$n_{at}=a^{-3}$ <<пружинок>> в единице объёма
на величину $a\epsilon $. Приравняем две
упругие энергии:
\[
	E \frac{\epsilon ^2}{2}=k n_{at} \frac{(a\epsilon )^2}{2}
.\] 
Отсюда $k=Ea$. Соответственно 
\[
\overline{\xi^2}= \frac{3k_B T}{Ea}
.\] 
Длина свободного пробега
\[
\Lambda = \frac{1}{n_{at}\sigma}= \frac{1}{n_{at}\pi
\overline{\xi^2}}= \frac{Ea^4}{3\pi k_B T}= 2\cdot 
10^{-6} \text{ см}
.\] 
Из этого решения следует, что
 \[
\rho\equiv \frac{m}{ne^2 \tau}=m \frac{v_F}{ne^2 \Lambda} \propto T
,\]
что наблюдается в большинстве металлов.

Ответ отличается от того, что в задачнике заменой
$n_e$ на $n_{at}$ и цифрой 3 в знаменателе,
возникающей из-за равнораспределения.
\end{sol}
\begin{hiProb}[3.79]
\end{hiProb}
\begin{sol}
Характерное время $t$, через которое пластинка
 <<почувствует>> изменение температуры, оценивается
как $d^2 /D$, где $D$ --- коэффициент диффузии.
Это время также известно, как время <<выравнивания>>.

Коэффициент диффузии есть отношение коэффициентов
теплопроводности кристалла $\kappa$ к его
теплоёмкости $C$. При комнатных температурах
теплоёмкость кристалла практически равна решёточной
$C_{\text{реш}}$ и согласно закону Дюлонга-Пти
$C=C_{\text{реш}}=3nk_\text{Б}T$, где $n$ ---
плотность атомов. Таким образом
\[
D= \frac{\kappa}{C_\text{реш}}=\frac{\kappa}{n k_\text{Б}}
.\] 

Коэффициент теплопроводности $\kappa$ для переноса
тепла в газе со средней
скоростью частиц $v$ и длиной свободного пробега
$\Lambda$ равен
\[
\kappa =\frac{1}{3} C_V \Lambda v
,\] 
где $C_V$ --- теплоёмкость единицы объёма газа.
При комнатных температурах в большинстве металлов
почти весь тепловой поток переносят электроны.
В применении к электронному газу в качестве $v$ 
разумно взять $v_F$, а 
\[
C_V=C_\text{эл}= \frac{\pi^2}{2} n k_\text{Б}^2
\frac{T}{\epsilon _F}= \frac{\pi^2 n k_\text{Б}^2
T}{m_e v_F} \Lambda
.\] 

Таким образом, коэффициент диффузии
\[
D= \frac{\kappa}{C_\text{реш}}= \frac{\pi^2}{9}
\frac{k_\text{Б} T \Lambda}{m_e v_F}
,\] 
откуда искомое время
\[
t \simeq \frac{d^2}{D} \approx
\frac{9}{\pi^2} \frac{d^2 m_e v_F}{k_\text{Б} T
\Lambda}\simeq 2\cdot 10^{-2} \text{ с}
,\] 
где $\displaystyle v_F \simeq \frac{\hbar }{m_e a} \left( 3\pi^2 \right) ^{1 /3}\sim 10^{8} \text{ см} / \text{с}$ 
(рассчитано для постоянной решётки $a \simeq
3$~\AA).
\end{sol}
\begin{hiProb}[3.80]
\end{hiProb}
\begin{sol}
В стержне можно считать что тепловое равновесие
поперёк устанавливается мгновенно по
сравнению с тепловым равновесием вдоль и перенос
тепла подчиняется одномерному уравнению теплопроводности:
\[
c \frac{\partial T}{\partial t} = -\kappa
\frac{\partial^2 t}{\partial x^2}
.\] 

Отсюда из размерности сразу понятно, какое будет
время установления:
\[
\tau \sim \frac{cL^2}{\kappa}
.\] 

Поскольку температура много больше дебаевской,
применима классическая теория теплоёмкости:
$c=3R \rho /\mu$ (здесь речь идёт о теплоёмкости
на единицу объёма, $R$ --- газовая постоянная,
$\mu=64$ г/моль --- молярная масса меди).

Имеем ответ
\[
\tau= 3R \frac{\rho L^2}{\kappa \mu}= 92 \text{ с}
.\] 
\end{sol}
\begin{hiProb}[3.88]
\end{hiProb}
\begin{sol}
	Квазиклассическое описание (импульс
	как квантовое число) и понятие длины
	свободного пробега применимы только 
	когда на длине свободного пробега
	укладывается несколько длин волн
	де-Бройля. Минимальная металлическая
	проводимость реализуется, когда
	длина свободного пробега равна $2\pi /k_F$.

	Подставляем в формулу Друде:
	\[
	\sigma_{\min}= \frac{ne^2 \tau}{m}=
	\frac{ne^2 \cdot 2\pi \hbar }{p_F v_F m}=
	\frac{ne^2 \cdot 2\pi}{\hbar  k_F^2}=
	\frac{2\pi ne^2}{\hbar \left( 3\pi
	^2 n\right) ^{2 /3}}=
	2n^{1 /3}(9\pi) ^{-1 /3}\frac{e^2}{\hbar }
	.\] 
	С точностью до численного множителя порядка
	1 это совпадает с ответом задачника
	и даёт $2,8\cdot 10^{-4}\text{ Ом}\cdot\text{см}$.
\end{sol}
\section{Объёмные полупроводники}
\begin{hiProb}[4.7]
\end{hiProb}
\begin{sol}
Договоримся отсчитывать химпотенциал от дна зоны
проводимости. Раз полупроводник собственный ---
будет электронейтральность и число электронов
будет равно числу дырок: $n_e=n_h$.

Когда  $\Delta \gg T$, уровень химпотенциала
лежит внутри щели и числа заполнения для
электронов и дырок $\left<n \right> \ll 1$, так
что можно пренебречь 1 в знаменателе распределения
Ферми. Это даёт:
\[
	n_h= 2 \left( \frac{m_h T}{2\pi \hbar ^2} \right) ^{3 /2} e^{(\mu +\Delta) /T}
.\] 
\[
	n_e =2 \left( \frac{m_e T}{2\pi \hbar ^2} \right) ^{3/2} e^{-\mu /T}
.\] 
Приравнивая две концентрации и сокращая, получаем:
\[
	\left( \frac{m_e}{m_h} \right) ^{3/2}
	=e^{(2\mu +\Delta) /T}
.\] 
Это даёт ответ:
\[
\mu=-\frac{\Delta}{2} +0,75 T \ln \frac{m_h}{m_e}
.\] 
Обсудим этот ответ. При низких температурах химпотенциал находится посредине щели. Это объясняется тем,
что электронные и дырочные возбуждения могут родиться
только парами и являются равноневыгодными. При
повышении температуры, если масса дырок больше,
то химпотенциал едет в электронную сторону,
если больше масса электронов --- наоборот. Это
связано с тем, что чем большее масса, тем больше
плотность состояний, и чтобы удовлетворить
равенству концентраций, необходимо, при прочих
равных условиях, сдвинуть химпотенциал дальше
от носителей с большей массой.

Надо заметить, что, при наличии сколь угодно
малого количества легирующей примеси
одного типа, ситуация при низких температурах изменится: химпотенциал при $T=0$ окажется посередине
между примесным уровнем и соответствующей зоной
(зоной проводимости для донорной примеси и валентной
зоной для акцепторной).
\end{sol}
\begin{hiProb}[Т6-1]
\end{hiProb}
\begin{sol}
Так как длина волны видимого света много больше
межатомного расстояния, то импульс такого фотона
много меньше бриллюэновского. Это означает, что
при поглощении фотона видимого света электрон
в кристалле переходит между разрешёнными состояниями
<<вертикальнно>> --- практически без изменения своего
квазиимпульса.

Соответственно, в непрямозонном полупроводнике
при поглощении фотона переход
электрона с потолка валентной зоны на дно зоны
проводимости возможен только с поглощением
или излучением дополнительного фонона, такие
процессы менее вероятны, чем прямые
переходы из неэкстремального положения в валентной
зоне в неэкстремальное положение в зоне проводимости.
Процесс с поглощением фонона дополнительно
запрещён низкими температурами (фононов мало).

Для поиска минимальной энергии фотона, с которой
такие вертикальные переходы становятся
разрешёнными, рассмотрим разность энергий
электрона в валентной зоне и зоне проводимости вдоль
прямой, соединяющей экстремумы в $k$-пространстве.
Ноль отсчёта импульса поместим на потолок валентной
зоны, если $\delta$ --- расстояние в $k$-пространстве
между экстремумами и $\xi$ --- координата точки, то
\[
	E(\xi) = \frac{\hbar ^2 \left( \delta
	-\xi \right) ^2}{2m}+\Delta +\frac{\hbar ^2 \xi^2}{2m}
.\] 
Ищем минимум, он достигается при $\xi =\delta /2$ и
равен 
\[
E_{\min}= \Delta + \frac{\hbar ^2 \delta^2}{4m}
.\] 
Откуда
\[
	\delta^2= \frac{4m}{\hbar ^2} (E-\Delta)
.\] 
\end{sol}
\begin{hiProb}[4.50]
\end{hiProb}
\begin{sol}
Вопрос этой задачи можно переформулировать так:
при каких концентрациях примеси легирование нельзя
считать слабым? При повышении
концентрации примеси электроны начинают 
чувствовать не только потенциал своего донора,
но и соседних. Это становится существенным, когда
характерное расстояние между примесями становится
меньше удвоенного боровского радиуса. При этом
мы будем подразумевать, что длина экранирования
будет большой по сравнению с межпримесным
расстоянием, чтоб считать потенциал взаимодействия
электрона и примеси кулоновским.

Поиску боровского радиуса в InSb была посвящена задача 4.2. Ответ
там примерно 60 нм $\left( a_B^*= \frac{\epsilon 
\hbar ^2}{m^* e^2}=a_B \epsilon  m /m^* \right) $.
Приравнивая $n_\text{donors}= \left( 2a_B \right) ^{-3}$ находим искомую концентрацию $n= 5,8 \cdot 
10^{14}\text{см}^{-3}$.

По полупроводниковым меркам это очень маленькая
концентрация: концентрация атомов в твёрдом
теле $\sim 10^{23} \text{см}^{-3}$, то есть
речь идёт об относительной концентрации примесей
на уровне $10^{-8}$. В кремнии, например, из-за
большей эффективной массы эта величина будет
на 4 порядка больше. Данный ответ показывает
насколько сложно работать с узкозонными полупроводниками
(в которых малая эффективная масса): даже
маленькое количество примеси может привести к
образованию примесной зоны.
\end{sol}
\begin{hiProb}[4.12]
\end{hiProb}
\begin{sol}
	Из-за электронейтральности концентрации (а значит и $p_F$) электронов ($n_e$) и дырок
	$(n_h)$ равны между собой. Будем для
простоты считать температуру нулевой. Тогда
перекрытие зон равно сумме энергии Ферми электроннов
(отсчитанной от дна зоны проводимости) и
дырок (отсчитанной от потолка валентной зоны).

Пользуясь тем, что $p_F =\hbar (3\pi^2 n)^{1 /3}$,
запишем
\[
	\frac{1}{2} \hbar ^2 \left( 3\pi^2
	n \right) ^{2 /3} \left( 
\frac{1}{m_e}+ \frac{1}{m_h}\right) =\Delta E
.\] 
Отсюда напрямую получаем
\[
	n= \left( 2\Delta E \frac{m_e m_h}{m_e+m_h} \right) ^{3 /2} \frac{1}{3\pi^2 \hbar ^2}=
	9,3 \cdot 10^{16} \text{ см}^{-3}
.\] 
Энергии Ферми делят перекрытие зон обратно
пропорционально массам, соответственно $E_{F,e}=
0,015$ эВ, $E_{F,h}=0,025$ эВ.
\end{sol}
\end{document}
