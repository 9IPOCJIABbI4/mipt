\documentclass[a4paper]{article}
% Этот шаблон документа разработан в 2014 году
% Данилом Фёдоровых (danil@fedorovykh.ru) 
% для использования в курсе 
% <<Документы и презентации в \LaTeX>>, записанном НИУ ВШЭ
% для Coursera.org: http://coursera.org/course/latex .
% Исходная версия шаблона --- 
% https://www.writelatex.com/coursera/latex/5.3

% В этом документе преамбула

\usepackage{siunitx}
%%% Работа с русским языком
%\usepackage{cmap}					% поиск в PDF
%\usepackage{mathtext} 				% русские буквы в формулах
%\usepackage[T2A]{fontenc}			% кодировка
%\usepackage[utf8]{inputenc}			% кодировка исходного текста
%\usepackage[english,russian]{babel}	% локализация и переносы
%\usepackage{indentfirst}
%\frenchspacing
%
%\renewcommand{\epsilon}{\ensuremath{\varepsilon}}
%\newcommand{\phibackup}{\ensuremath{\phi}}
%\renewcommand{\phi}{\ensuremath{\varphi}}
%\renewcommand{\varphi}{\ensuremath{\phibackup}}
%\renewcommand{\kappa}{\ensuremath{\varkappa}}
%\renewcommand{\le}{\ensuremath{\leqslant}}
%\renewcommand{\leq}{\ensuremath{\leqslant}}
%\renewcommand{\ge}{\ensuremath{\geqslant}}
%\renewcommand{\geq}{\ensuremath{\geqslant}}
%\renewcommand{\emptyset}{\varnothing}
%\renewcommand{\Im}{\operatorname{Im}}
%\renewcommand{\Re}{\operatorname{Re}}


%%% Дополнительная работа с математикой
\usepackage{amsmath,amsfonts,amssymb,amsthm,mathtools} % AMS
%\usepackage{icomma} % "Умная" запятая: $0,2$ --- число, $0, 2$ --- перечисление

%% Номера формул
%\mathtoolsset{showonlyrefs=true} % Показывать номера только у тех формул, на которые есть \eqref{} в тексте.
%\usepackage{leqno} % Нумереация формул слева

%% Свои команды
\DeclareMathOperator{\sgn}{\mathop{sgn}}
\DeclareMathOperator{\sign}{\mathop{sign}}
\DeclareMathOperator*{\res}{\mathop{res}}
\DeclareMathOperator*{\tr}{\mathop{tr}}
\DeclareMathOperator*{\rot}{\mathop{rot}}
\DeclareMathOperator*{\divop}{\mathop{div}}
\DeclareMathOperator*{\grad}{\mathop{grad}}

%% Перенос знаков в формулах (по Львовскому)
\newcommand*{\hm}[1]{#1\nobreak\discretionary{}
{\hbox{$\mathsurround=0pt #1$}}{}}

%%% Работа с картинками
\usepackage{graphicx}  % Для вставки рисунков
\graphicspath{{figures/}}  % папки с картинками
\setlength\fboxsep{3pt} % Отступ рамки \fbox{} от рисунка
\setlength\fboxrule{1pt} % Толщина линий рамки \fbox{}
\usepackage{wrapfig} % Обтекание рисунков текстом

%%% Работа с таблицами
\usepackage{array,tabularx,tabulary,booktabs} % Дополнительная работа с таблицами
\usepackage{longtable}  % Длинные таблицы
\usepackage{multirow} % Слияние строк в таблице

%%% Теоремы
\theoremstyle{plain} % Это стиль по умолчанию, его можно не переопределять.
\newtheorem{thm}{Теорема}
\newtheorem*{thm*}{Теорема}
\newtheorem{prop}{Предложение}
\newtheorem*{prop*}{Предложение}
 
\theoremstyle{definition} % "Определение"
%\newtheorem{corollary}{Следствие}[theorem]
\newtheorem{dfn}{Определение}
\newtheorem*{dfn*}{Определение}
\newtheorem{prob}{Задача}
\newtheorem*{prob*}{Задача}

 
\theoremstyle{remark} % "Примечание"
\newtheorem*{sol}{Решение}
\newtheorem*{rem}{Замечание}

%%% Программирование
\usepackage{etoolbox} % логические операторы

%%% Страница
%\usepackage{extsizes} % Возможность сделать 14-й шрифт
%\usepackage{geometry} % Простой способ задавать поля
%	\geometry{top=25mm}
%	\geometry{bottom=35mm}
%	\geometry{left=35mm}
%	\geometry{right=20mm}
 
\usepackage{fancyhdr} % Колонтитулы
%	\pagestyle{fancy}
 %	\renewcommand{\headrulewidth}{0pt}  % Толщина линейки, отчеркивающей верхний колонтитул
	%\lfoot{Нижний левый}
	%\rfoot{Нижний правый}
	%\rhead{Верхний правый}
	%\chead{Верхний в центре}
	%\lhead{Верхний левый}
	%\cfoot{Нижний в центре} % По умолчанию здесь номер страницы

\usepackage{setspace} % Интерлиньяж
%\onehalfspacing % Интерлиньяж 1.5
%\doublespacing % Интерлиньяж 2
%\singlespacing % Интерлиньяж 1

\usepackage{lastpage} % Узнать, сколько всего страниц в документе.

\usepackage{soul} % Модификаторы начертания

\usepackage{hyperref}
\usepackage[usenames,dvipsnames,svgnames,table,rgb]{xcolor}
\hypersetup{				% Гиперссылки
    unicode=true,           % русские буквы в раздела PDF
    pdftitle={Заголовок},   % Заголовок
    pdfauthor={Автор},      % Автор
    pdfsubject={Тема},      % Тема
    pdfcreator={Создатель}, % Создатель
    pdfproducer={Производитель}, % Производитель
    pdfkeywords={keyword1} {key2} {key3}, % Ключевые слова
%    colorlinks=true,       	% false: ссылки в рамках; true: цветные ссылки
    %linkcolor=red,          % внутренние ссылки
    %citecolor=black,        % на библиографию
    %filecolor=magenta,      % на файлы
    %urlcolor=cyan           % на URL
}

\usepackage{csquotes} % Еще инструменты для ссылок

%\usepackage[style=apa,maxcitenames=2,backend=biber,sorting=nty]{biblatex}

\usepackage{multicol} % Несколько колонок

\usepackage{tikz} % Работа с графикой
\usepackage{pgfplots}
\usepackage{pgfplotstable}
%\usepackage{coloremoji}
\usepackage{floatrow}
\usepackage{subcaption}
\graphicspath{{figures/}}

\renewcommand\thesubfigure{\asbuk{subfigure}}
%\addbibresource{master.bib}

\usepackage{import}
\usepackage{pdfpages}
\usepackage{transparent}
\usepackage{xcolor}
\usepackage{xifthen}

\newcommand{\incfig}[2][1]{%
    \def\svgwidth{#1\columnwidth}
    \import{./figures/}{#2.pdf_tex}
}
%\usepackage{titlesec}
%\titleformat{\section}{\normalfont\Large\bfseries}{}{0pt}{}
%----------------------STANDART:
%\titleformat{\chapter}[display]
%  {\normalfont\huge\bfseries}{\chaptertitlename\ \thechapter}{20pt}{\Huge}
%\titleformat{\section}{\normalfont\Large\bfseries}{\thesection}{1em}{}
%\titleformat{\subsection}
%  {\normalfont\large\bfseries}{\thesubsection}{1em}{}
%\titleformat{\subsubsection}
%  {\normalfont\normalsize\bfseries}{\thesubsubsection}{1em}{}
%\titleformat{\paragraph}[runin]
%  {\normalfont\normalsize\bfseries}{\theparagraph}{1em}{}
%\titleformat{\subparagraph}[runin]
%  {\normalfont\normalsize\bfseries}{\thesubparagraph}{1em}{}

\pdfsuppresswarningpagegroup=1
\pgfplotsset{compat=1.16}



%\setcounter{tocdepth}{1} % only parts,chapters,sections
%\titleformat{\subsection}{\normalfont\large\bfseries}{}{0em}{}
%\titleformat{\subsubsection}{\normalfont\normalsize\bfseries}{}{0em}{}

%\newcommand{\textover}[2]{\stackrel{\mathclap{\normalfont\mbox{#2}}}{#1}}

\author{Yaroslav Drachov\\
Moscow Institute of Physics and Technology}
%\author{Драчов Ярослав\\
%Факультет общей и прикладной физики МФТИ}
\newcommand{\veq}{\mathrel{\rotatebox{90}{$=$}}}
%\newcommand{\teto}[1]{\stackrel{\mathclap{\normalfont\tiny\mbox{#1}}}{\to}}
%\renewcommand{\thesubsection}{\arabic{subsection}}

%%\setcounter{secnumdepth}{0}

\definecolor{tabblue}{RGB}{30, 119, 180}
\definecolor{taborange}{RGB}{255, 127, 15}
\definecolor{tabgreen}{RGB}{45, 160, 43}
\definecolor{tabred}{RGB}{214, 38, 40}
\definecolor{tabpurple}{RGB}{148, 103, 189}
\definecolor{tabbrown}{RGB}{140, 86, 76}
\definecolor{tabpink}{RGB}{227, 119, 193}
\definecolor{tabgray}{RGB}{127, 127, 127}
\definecolor{tabolive}{RGB}{188, 189, 33}
\definecolor{tabcyan}{RGB}{22, 190, 207}
\pgfplotscreateplotcyclelist{colorbrewer-tab}{
{tabblue},
{taborange},
{tabgreen},
{tabred},
{tabpurple},
{tabbrown},
{tabpink},
{tabgray},
{tabolive},
{tabcyan},
}
\usepackage{csvsimple}
\usepackage{extarrows}
%\renewcommand{\labelenumii}{\asbuk{enumii})}
%\renewcommand{\labelenumiv}{\Asbuk{enumiv}}
%\newcommand{\prob}[1]{\subsubsection*{#1}}
\sisetup{output-decimal-marker = {,},separate-uncertainty = true,exponent-product = \cdot}

\usepackage{braket}
\usepackage{enumerate}
\usepackage{chngcntr}
%\counterwithin*{equation}{problem}
%\usepackage{bbold}

\newtheoremstyle{hiProb}% ⟨name ⟩ 
{3pt}% ⟨Space above ⟩1 
{3pt}% ⟨Space below ⟩1
{}% ⟨Body font ⟩
{}% ⟨Indent amount ⟩2
{\bfseries}% ⟨Theorem head font⟩
{.}% ⟨Punctuation after theorem head ⟩
{.5em}% ⟨Space after theorem head ⟩3
%{\thmname{#1} \thmnote{#3}}% ⟨Theorem head spec (can be left empty, meaning ‘normal’)⟩
{\thmnote{#3}}% ⟨Theorem head spec (can be left empty, meaning ‘normal’)⟩
\theoremstyle{hiProb} % "Определение"
%\newtheorem{hiProb}{Задача}
\newtheorem{hiProb}{}
%\usepackage{mmacells}
\newcommand{\textover}[2]{\stackrel{\mathclap{\normalfont\scriptsize\mbox{#2}}}{#1}}
\usepackage{units}
\usepackage[math]{cellspace}%
\setlength\cellspacetoplimit{2pt}
\setlength\cellspacebottomlimit{2pt}

\DeclareMathAlphabet{\mathbbold}{U}{bbold}{m}{n}

\newcommand{\normord}[1]{:\mathrel{#1}:}

\title{Неделя №3\\
Электронный ферми-газ}
\begin{document}
	\maketitle
\begin{hiProb}[3.44]
\end{hiProb}
\begin{sol}
В обоих случаях отдаётся в зону проводимости
по одному электрону на атом, примитивные 
ячейки содержат единственный ион, так что
числа электронов проводимости, примитивных
ячеек и атомов совпадают.

Для электронной теплоёмкости пользуемся ферми-газовой
моделью. Теплоёмкость электронов линейна по температуре
при $T \ll E_F$, то есть вплоть до температуры
плавления металла:
\[
	C_{el}= \frac{k_B N T m}{\hbar ^2}
	\left( \frac{\pi}{3n} \right) ^{2 /3}=
	\frac{\pi^2 k_B ^2 NT}{2E_F}
.\] 
Теплоёмкость ферми-газа выводится на лекции.

Фононная теплоёмкость при высоких температурах $(T>\theta)$ равна $C_{ph}= 3 N k_B$, что в 
$\sim \frac{E_F}{(k_B T)}\gg 1$ раз больше
электронной. Значит сравниваются теплоёмкости
при низких температурах.

При низких температурах для фононной теплоёмкости
есть формула Дебая:
\[
	C_{ph}= \frac{12 \pi^4 N k_B}{5} \left( \frac{T}{\theta} \right) ^3
.\] 
Приравнивая, получаем
\[
T^2= \frac{5 k_B \Theta^3}{24 E_F \pi^2}
.\] 
После подстановки численных значений, получаем
ответ: 3,3 К для меди и 1,5 К для натрия.
Полученные числа оправдывают применение дебаевского
приближения.
\end{sol}
\begin{hiProb}[3.53]
\end{hiProb}
\begin{sol}
	В равновесии (когда пластины соединили)
	установится такое распределение заряда, что
	$e\phi +\mu = \mathrm{const}$ $(e<0)$,
	то есть выравнивается электрохимпотенциал.
	Электронам выгодно понижать свою энергию,
	переходя из натрия в медь (поверхность Ферми
	меди ниже по энергии), но такие переходы
	нарушают электронейтральность и
	возникает задерживающая разность
	потенциалов. При этом массивные металлические
	образцы можно считать в равновесии
	эквипотенциальными.

	Значит разность электрических потенциалов
	равна $A_{\mathrm{Cu}}-A_{\mathrm{Na}} /e$,
	чтобы обеспечить эту разность потенциалов,
	перетёк заряд $C(A_{\mathrm{Cu}}-A_{\operatorname{Na}}) /e=2,2 \cdot 10^{-2} \text{ Кл}=1,38
	\cdot 10^7 e$ (зарядов электрона),
	что составляет $5,2 \cdot 10^{-16}$ 
	от общего числа электронов в образце.
\end{sol}
\begin{hiProb}[3.59]
\end{hiProb}
\begin{sol}
Пусть плотность состояний на уровне Ферми для
каждого из направлений спина равна $D'$.
В модели с квадратичным законом дисперсии
\[
	D'=\frac{dN}{dE}= \frac{V 4 \pi k_F^2 df /(2\pi)^3}{
	\hbar ^2 k_F dk/ m^*} =\frac{1}{2}
	\frac{m^* p_F}{\pi^2 \hbar ^3}= \frac{1}{2}
	\frac{m^*}{\hbar ^2} \sqrt[3]{\frac{3n}{\pi^4}} 
,\]
или, по-другому, \[D'= \frac{3n}{4 E_F}.\] Здесь $n$ ---
полная концентрация электронов, $D'$ 
вдвое меньше  полной плотности состояний.

Условие, что поле мало означает, что изменение
распределения электронов мало. Тогда можно
считать, что плотность состояний не изменилась при
приложении поля. Считаем магнетизм чисто спиновым:
$g=2$, магнитный момент каждого  электрона
равен боровскому магнетону и может быть направлен
либо по полю, либо против поля.

Это значит, что электронов, магнитный момент
которых направлен по полю (энергия которых понижается),
стало в единице объёма больше на $D' \mu_B H$,
а электронов, магнитный момент которых направлен
против поля (энергия которых повышается) ---
меньше на ту же самую величину $D' \mu_B H$. Естественно, полное число электронов сохранилось (система
осталась электронейтральной).

Таким образом
 \[
\frac{\delta n}{n}= \frac{2 D' \mu_B H}{n}=
\frac{3\mu_B H}{2E_F}
.\] 
Для поля 10 Тл $\frac{3}{2}\mu_B H \sim 1$ мэВ,
то есть в реальных лабораторных полях в электронном
газе в типичном металле ($E_F \sim 1$ эВ) перераспределяется ничтожная доля электронов проводимости.

Магнитный момент без поля был равен 0, а в поле
стал равен  \[M=\mu_B \delta n= H \mu_B^2 \frac{m^*}{\hbar ^2}
\sqrt[3]{\frac{3n}{\pi^4}} .\]
Поскольку для каждой проекции спина
\[
\mu_B H= \delta E= \frac{p_F \delta p}{m}
,\] 
то
\[
\frac{\delta p}{p_F}=2 m^* \frac{\mu_B H}{p_F^2}=
\frac{\mu_B H}{E_F}
.\] 
Для восприимчивости:
\[
\chi= M /H = \mu_B^2 \frac{m^*}{\hbar ^2}
\sqrt[3]{\frac{3n}{\pi^4}} =5,2 \cdot 10^{-7}
.\] 
\end{sol}
\begin{hiProb}[3.87]
\end{hiProb}
\begin{sol}
Для фононной теплоёмкости одномерной цепочки
при низких температурах (на единицу длины)
\[
E= 2 \int\limits_{0}^{\infty} \frac{\hbar k s}{
\exp \left( \frac{\hbar  k s}{T} \right) -1}
\frac{dk}{2\pi}= \frac{1}{\pi}
\frac{T^2}{\hbar  s} \int\limits_{0}^{\infty} 
\frac{xdx}{e^x -1}= \frac{\pi}{6} \frac{T^2}{\hbar s}
\]
двойка учитывает движение фононов в обе
стороны, поляризация для одномерной системы
единственная. Для теплоёмкости, возвращая
в запись постоянную Больцмана,
\[
C_{ph}= \frac{\pi k_B^2 T}{3\hbar  s}
.\] 
Скорость Ферми по определению $v_F = \left( \frac{dE}{dp} \right) _{E_F}$ независимо от вида спектра.
Для электронной теплоёмкости
\[
	C= \frac{\pi^2}{3} D(E_F) T
\]
плотность состояний в одномерном случае,
но не делая явных предположений о спектре.
\[
D= \frac{dN}{dE}= \frac{dN}{dp} \cdot \frac{dp}{dE}=
2 \cdot 2 \cdot  \frac{1}{\hbar }\frac{L}{2\pi}
\cdot \frac{1}{v_F}
\]
(учтён спиновый множитель 2 и множитель 2,
учитывающий движение электронов в обе стороны),
откуда электронная теплоёмкость на единицу длины с
постоянной Больцмана
\[
C_{el}= \frac{2\pi}{3} \frac{k_B^2 T}{\hbar v_F}
.\] 
Соответственно,
\[
\frac{C_{el}}{C_{ph}}= \frac{2s}{v_F} \sim 2\cdot 
10^{-2}
.\] 
\end{sol}
\begin{hiProb}[3.61]
\end{hiProb}
\begin{sol}
У ферми-жидкости два свойства:
\begin{enumerate}
\item Концентрация и фермиевский импульс в ней
	соответствуют полному числу частиц
\item  Вблизи Ферми-поверхности наблюдается
	перенормированная (изменённая) масса,
	соответствующая изменённой плотности
	состояний. Именно эта масса и определяет
	все термодинамические свойства.
\end{enumerate}
Спин ядра гелия-3 равен 1/2, электронный спин полностью заполненной $s$-оболочки нулевой,
поэтому полный спин атома равен 1/2. Атом гелия-3
является ферми-частицей и низкотемпературные свойства
гелия-3 --- это свойства жидкости. Тогда можно
использовать результат для теплоёмкости
электронного газа:
\[
C_V= \frac{k_B n T m^*}{\hbar ^2}
\left( \frac{\pi}{3n} \right) ^{2 /3}=
\frac{m^*}{m} \frac{k_B (nm)T}{\hbar ^2}
\left( \frac{\pi}{3n} \right) ^{2 /3}
.\] 
Напомним, что эта формула --- на единицу объёма.
Воспользуемся тем, что $nm=\rho$, где $m$ --- масса
атома He-3. Подставляя, и помня, что в условии ---
молярная теплоёмкость, а нужна для вычислений
теплоёмкость в единице объёма  $C_V= C_\mu \rho
/(m N_a)$, получаем после очевидных
преобразований
\[
	\frac{m^*}{m}=\frac{C_\mu \hbar ^2 \rho}{
	N_a k_B^2 T\rho m} \left( \frac{3\rho}{\pi m} \right) ^{2 /3}=2,15
.\] 
\end{sol}
\begin{hiProb}[3.28]
\end{hiProb}
\begin{sol}
Поскольку в звезде присутствует положительный фон,
который в точности равен отрицательному, то мы
пренебрегаем взаимодействием и рассматриваем
только кинетическую энергию электронов.

Будем считать закон дисперсии ультрарелятивистским:
$\epsilon =pc$, поскольку для большинства состояний
$\epsilon \gg mc^2$.

Полная концентрация электронов:
\[
	n= 2\cdot  \frac{4}{3} \pi p_F^3\cdot \frac{1}{(2\pi \hbar )^3}
.\] 
Фермиевский импульс такой же как и в обычном электронном газе (он определяется только концентрацией
и не зависит от закона дисперсии):
\[
	p_F= \hbar  (3\pi^2 n)^{1 /3}
.\] 

Полная энергия электронов в единице объёма:
\[
	E=2\cdot \frac{1}{(2\pi \hbar )^3}
	\int\limits_{0}^{p_F} cp \cdot 4\pi
	p^2 dp= 2\pi c p_F^4 \cdot 
	\frac{1}{(2\pi \hbar )^3}=0,25c \hbar 
	(9\pi n^2)^{2 /3}
.\] 
Полная энергия всего газа
\[
	E_{\text{полн}}= E V= \frac{(9\pi)^{2 /3}}{4}
	c \hbar  N^{4 /3} \frac{1}{V^{1 /3}}
.\] 
\[
	P=-\frac{dE_{\text{полн}}}{dV}=\frac{(9\pi)^{2 /3}}{12} c \hbar  n^{4 /3},\quad \text{или }
	pV^{4 /3}=\mathrm{const}
.\] 
\end{sol}
\end{document}
