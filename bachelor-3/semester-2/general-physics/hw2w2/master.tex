\documentclass[a4paper]{article}
% Этот шаблон документа разработан в 2014 году
% Данилом Фёдоровых (danil@fedorovykh.ru) 
% для использования в курсе 
% <<Документы и презентации в \LaTeX>>, записанном НИУ ВШЭ
% для Coursera.org: http://coursera.org/course/latex .
% Исходная версия шаблона --- 
% https://www.writelatex.com/coursera/latex/5.3

% В этом документе преамбула

\usepackage{siunitx}
%%% Работа с русским языком
%\usepackage{cmap}					% поиск в PDF
%\usepackage{mathtext} 				% русские буквы в формулах
%\usepackage[T2A]{fontenc}			% кодировка
%\usepackage[utf8]{inputenc}			% кодировка исходного текста
%\usepackage[english,russian]{babel}	% локализация и переносы
%\usepackage{indentfirst}
%\frenchspacing
%
%\renewcommand{\epsilon}{\ensuremath{\varepsilon}}
%\newcommand{\phibackup}{\ensuremath{\phi}}
%\renewcommand{\phi}{\ensuremath{\varphi}}
%\renewcommand{\varphi}{\ensuremath{\phibackup}}
%\renewcommand{\kappa}{\ensuremath{\varkappa}}
%\renewcommand{\le}{\ensuremath{\leqslant}}
%\renewcommand{\leq}{\ensuremath{\leqslant}}
%\renewcommand{\ge}{\ensuremath{\geqslant}}
%\renewcommand{\geq}{\ensuremath{\geqslant}}
%\renewcommand{\emptyset}{\varnothing}
%\renewcommand{\Im}{\operatorname{Im}}
%\renewcommand{\Re}{\operatorname{Re}}


%%% Дополнительная работа с математикой
\usepackage{amsmath,amsfonts,amssymb,amsthm,mathtools} % AMS
%\usepackage{icomma} % "Умная" запятая: $0,2$ --- число, $0, 2$ --- перечисление

%% Номера формул
%\mathtoolsset{showonlyrefs=true} % Показывать номера только у тех формул, на которые есть \eqref{} в тексте.
%\usepackage{leqno} % Нумереация формул слева

%% Свои команды
\DeclareMathOperator{\sgn}{\mathop{sgn}}
\DeclareMathOperator{\sign}{\mathop{sign}}
\DeclareMathOperator*{\res}{\mathop{res}}
\DeclareMathOperator*{\tr}{\mathop{tr}}
\DeclareMathOperator*{\rot}{\mathop{rot}}
\DeclareMathOperator*{\divop}{\mathop{div}}
\DeclareMathOperator*{\grad}{\mathop{grad}}

%% Перенос знаков в формулах (по Львовскому)
\newcommand*{\hm}[1]{#1\nobreak\discretionary{}
{\hbox{$\mathsurround=0pt #1$}}{}}

%%% Работа с картинками
\usepackage{graphicx}  % Для вставки рисунков
\graphicspath{{figures/}}  % папки с картинками
\setlength\fboxsep{3pt} % Отступ рамки \fbox{} от рисунка
\setlength\fboxrule{1pt} % Толщина линий рамки \fbox{}
\usepackage{wrapfig} % Обтекание рисунков текстом

%%% Работа с таблицами
\usepackage{array,tabularx,tabulary,booktabs} % Дополнительная работа с таблицами
\usepackage{longtable}  % Длинные таблицы
\usepackage{multirow} % Слияние строк в таблице

%%% Теоремы
\theoremstyle{plain} % Это стиль по умолчанию, его можно не переопределять.
\newtheorem{thm}{Теорема}
\newtheorem*{thm*}{Теорема}
\newtheorem{prop}{Предложение}
\newtheorem*{prop*}{Предложение}
 
\theoremstyle{definition} % "Определение"
%\newtheorem{corollary}{Следствие}[theorem]
\newtheorem{dfn}{Определение}
\newtheorem*{dfn*}{Определение}
\newtheorem{prob}{Задача}
\newtheorem*{prob*}{Задача}

 
\theoremstyle{remark} % "Примечание"
\newtheorem*{sol}{Решение}
\newtheorem*{rem}{Замечание}

%%% Программирование
\usepackage{etoolbox} % логические операторы

%%% Страница
%\usepackage{extsizes} % Возможность сделать 14-й шрифт
%\usepackage{geometry} % Простой способ задавать поля
%	\geometry{top=25mm}
%	\geometry{bottom=35mm}
%	\geometry{left=35mm}
%	\geometry{right=20mm}
 
\usepackage{fancyhdr} % Колонтитулы
%	\pagestyle{fancy}
 %	\renewcommand{\headrulewidth}{0pt}  % Толщина линейки, отчеркивающей верхний колонтитул
	%\lfoot{Нижний левый}
	%\rfoot{Нижний правый}
	%\rhead{Верхний правый}
	%\chead{Верхний в центре}
	%\lhead{Верхний левый}
	%\cfoot{Нижний в центре} % По умолчанию здесь номер страницы

\usepackage{setspace} % Интерлиньяж
%\onehalfspacing % Интерлиньяж 1.5
%\doublespacing % Интерлиньяж 2
%\singlespacing % Интерлиньяж 1

\usepackage{lastpage} % Узнать, сколько всего страниц в документе.

\usepackage{soul} % Модификаторы начертания

\usepackage{hyperref}
\usepackage[usenames,dvipsnames,svgnames,table,rgb]{xcolor}
\hypersetup{				% Гиперссылки
    unicode=true,           % русские буквы в раздела PDF
    pdftitle={Заголовок},   % Заголовок
    pdfauthor={Автор},      % Автор
    pdfsubject={Тема},      % Тема
    pdfcreator={Создатель}, % Создатель
    pdfproducer={Производитель}, % Производитель
    pdfkeywords={keyword1} {key2} {key3}, % Ключевые слова
%    colorlinks=true,       	% false: ссылки в рамках; true: цветные ссылки
    %linkcolor=red,          % внутренние ссылки
    %citecolor=black,        % на библиографию
    %filecolor=magenta,      % на файлы
    %urlcolor=cyan           % на URL
}

\usepackage{csquotes} % Еще инструменты для ссылок

%\usepackage[style=apa,maxcitenames=2,backend=biber,sorting=nty]{biblatex}

\usepackage{multicol} % Несколько колонок

\usepackage{tikz} % Работа с графикой
\usepackage{pgfplots}
\usepackage{pgfplotstable}
%\usepackage{coloremoji}
\usepackage{floatrow}
\usepackage{subcaption}
\graphicspath{{figures/}}

\renewcommand\thesubfigure{\asbuk{subfigure}}
%\addbibresource{master.bib}

\usepackage{import}
\usepackage{pdfpages}
\usepackage{transparent}
\usepackage{xcolor}
\usepackage{xifthen}

\newcommand{\incfig}[2][1]{%
    \def\svgwidth{#1\columnwidth}
    \import{./figures/}{#2.pdf_tex}
}
%\usepackage{titlesec}
%\titleformat{\section}{\normalfont\Large\bfseries}{}{0pt}{}
%----------------------STANDART:
%\titleformat{\chapter}[display]
%  {\normalfont\huge\bfseries}{\chaptertitlename\ \thechapter}{20pt}{\Huge}
%\titleformat{\section}{\normalfont\Large\bfseries}{\thesection}{1em}{}
%\titleformat{\subsection}
%  {\normalfont\large\bfseries}{\thesubsection}{1em}{}
%\titleformat{\subsubsection}
%  {\normalfont\normalsize\bfseries}{\thesubsubsection}{1em}{}
%\titleformat{\paragraph}[runin]
%  {\normalfont\normalsize\bfseries}{\theparagraph}{1em}{}
%\titleformat{\subparagraph}[runin]
%  {\normalfont\normalsize\bfseries}{\thesubparagraph}{1em}{}

\pdfsuppresswarningpagegroup=1
\pgfplotsset{compat=1.16}



%\setcounter{tocdepth}{1} % only parts,chapters,sections
%\titleformat{\subsection}{\normalfont\large\bfseries}{}{0em}{}
%\titleformat{\subsubsection}{\normalfont\normalsize\bfseries}{}{0em}{}

%\newcommand{\textover}[2]{\stackrel{\mathclap{\normalfont\mbox{#2}}}{#1}}

\author{Yaroslav Drachov\\
Moscow Institute of Physics and Technology}
%\author{Драчов Ярослав\\
%Факультет общей и прикладной физики МФТИ}
\newcommand{\veq}{\mathrel{\rotatebox{90}{$=$}}}
%\newcommand{\teto}[1]{\stackrel{\mathclap{\normalfont\tiny\mbox{#1}}}{\to}}
%\renewcommand{\thesubsection}{\arabic{subsection}}

%%\setcounter{secnumdepth}{0}

\definecolor{tabblue}{RGB}{30, 119, 180}
\definecolor{taborange}{RGB}{255, 127, 15}
\definecolor{tabgreen}{RGB}{45, 160, 43}
\definecolor{tabred}{RGB}{214, 38, 40}
\definecolor{tabpurple}{RGB}{148, 103, 189}
\definecolor{tabbrown}{RGB}{140, 86, 76}
\definecolor{tabpink}{RGB}{227, 119, 193}
\definecolor{tabgray}{RGB}{127, 127, 127}
\definecolor{tabolive}{RGB}{188, 189, 33}
\definecolor{tabcyan}{RGB}{22, 190, 207}
\pgfplotscreateplotcyclelist{colorbrewer-tab}{
{tabblue},
{taborange},
{tabgreen},
{tabred},
{tabpurple},
{tabbrown},
{tabpink},
{tabgray},
{tabolive},
{tabcyan},
}
\usepackage{csvsimple}
\usepackage{extarrows}
%\renewcommand{\labelenumii}{\asbuk{enumii})}
%\renewcommand{\labelenumiv}{\Asbuk{enumiv}}
%\newcommand{\prob}[1]{\subsubsection*{#1}}
\sisetup{output-decimal-marker = {,},separate-uncertainty = true,exponent-product = \cdot}

\usepackage{braket}
\usepackage{enumerate}
\usepackage{chngcntr}
%\counterwithin*{equation}{problem}
%\usepackage{bbold}

\newtheoremstyle{hiProb}% ⟨name ⟩ 
{3pt}% ⟨Space above ⟩1 
{3pt}% ⟨Space below ⟩1
{}% ⟨Body font ⟩
{}% ⟨Indent amount ⟩2
{\bfseries}% ⟨Theorem head font⟩
{.}% ⟨Punctuation after theorem head ⟩
{.5em}% ⟨Space after theorem head ⟩3
%{\thmname{#1} \thmnote{#3}}% ⟨Theorem head spec (can be left empty, meaning ‘normal’)⟩
{\thmnote{#3}}% ⟨Theorem head spec (can be left empty, meaning ‘normal’)⟩
\theoremstyle{hiProb} % "Определение"
%\newtheorem{hiProb}{Задача}
\newtheorem{hiProb}{}
%\usepackage{mmacells}
\newcommand{\textover}[2]{\stackrel{\mathclap{\normalfont\scriptsize\mbox{#2}}}{#1}}
\usepackage{units}
\usepackage[math]{cellspace}%
\setlength\cellspacetoplimit{2pt}
\setlength\cellspacebottomlimit{2pt}

\DeclareMathAlphabet{\mathbbold}{U}{bbold}{m}{n}

\newcommand{\normord}[1]{:\mathrel{#1}:}

\title{Неделя №9\\
Электродинамика сверхпроводников. Основы микроскопики сверхпроводников}
\begin{document}
	\maketitle
\begin{hiProb}[5.4]
\end{hiProb}
\begin{sol}
Поскольку радиус кольца много больше расстояния до
сверхпроводника, локально участок кольца можно считать прямым.
Тогда возникает задача о взаимодействии прямолинейного тонкого
провода со сверхпроводником. Магнитное поле <<обтекает>>
сверхпроводник, поэтому на поверхности оно должно
иметь касательное направление. Это означает, что
распределение поля можно описать с помощью метода зеркальных
изображений, дополнив настоящий прямой провод зеркальным,
ток по которому течёт в противоположную сторону. Сила,
действующая между двумя параллельными проводами, расположенными
на расстоянии $R$, равна
\[
F= \frac{2 I_1 I_2 L}{Rc^2}
\] 
для участков длины $L$. В нашем случае полная длина
проводов есть $L=2\pi r$, расстояние есть $R=2h$, токи
$I_1=-I_2=J$ отталкиваются. Это отталкивание должно
превысить силу тяжести. Это происходит при $F=mg$, откуда
\[
J= \sqrt{\frac{mghc^2}{2\pi r}} \approx 6,49 \cdot 10^{10}
\text{ ед. СГСЭ}\approx 21,6 \text{ А}
.\] 
\end{sol}
\begin{hiProb}[5.7]
\end{hiProb}
\begin{sol}
Мейснеровские токи текут по поверхности цилиндра, поэтому
распределение токов и полей такое же, как в соленоиде
(по сравнению с радиусом цилиндра толщиной поверхностного
слоя, в котором текут токи, можно пренебречь, т.\:к.
она имеет порядок глубины проникновения $\lambda$,
которая в обычных сверхпроводниках $\sim 10^{-5}\ldots 10^{-6}$ 
см). Внутри поле $\mathbf{B}$ должно быть равно нулю, поэтому
\[
\frac{4\pi}{c} \frac{I}{h}=H
\] 
где $I$ --- ток сверхпроводящих электронов (точнее, куперовских
пар). Если $N$ куперовских пар (с зарядом $2e$ и массой
$2m$) движутся со скоростью $v$, то период их обращения
вокруг цилиндра $T= 2\pi R /\nu$, момент импульса $L=2RmNv$,
ток $I=2e N/T$. Отсюда $L$ можно выразить через $I$ и 
затем через  $H$:
\[
L= \frac{mcHhR^2 }{2e}
.\] 
Куперовские пары движутся без трения по поверхности цилиндра.
Когда цилиндр переводят в нормальное состояние, электроны
и решётки начинают двигаться как целое, поэтому
запасённый парами момент импульса становится моментом
импульса всего цилиндра, и нить закручивается. Момент инерции
цилиндра равен $J=M R^2 /2$, а энергия $E=L^2 /2J$.

Эта энергия пойдёт на совершение работы по закручиванию
нити.

Работа, совершаемая моментом силы $M$ при изменении
угла на $d\phi$ есть $dA= Md\phi$, а по определению
модуля кручения $M=\alpha\phi$. Приравнивая работу при
закручивании нити на угол $\phi$ исходной энергии вращения,
получаем
\[
\frac{L^2}{2J}= \frac{\alpha \phi^2}{2}
\]
откуда в результате
\[
\phi= \frac{mcHhR}{e\sqrt{2\alpha M} }\approx
4,5 \cdot 10^{-5}\text{ рад}
.\] 
\end{sol}
\begin{hiProb}[Т9-4]
\end{hiProb}
\begin{sol}
При $H=H_{c1}$ поле ещё не проникло в сверхпроводник,
поэтому $B_1=H+4\pi M=0$, т.\:е. $4\pi M=-H_{c1}$.
При $H=1,25H_{c 1}$ получаем $B_2= H+\frac{1}{2} 4\pi
M=300$ Гс. Плотность вихрей $n=B_2 /\Phi_0$, вихри
образуют треугольную решётку со стороной $a$. Площадь
равностороннего треугольника со стороной $a$ равна $S=
 a^2 \sqrt{3} /4$, при этом в среднем на треугольник
 приходится $ 1 /2$ вихря (т.\:к. треугольник образуют
3 вихря, а каждый вихрь принадлежит 6 треугольникам).
Другими словами, примитивной ячейкой треугольной
решётки является ромб, построенный на двух сторонах
треугольника: его площадь вдвое больше площади
треугольника. В таком ромбе содержится один вихрь.

В результате $\displaystyle n= \frac{2}{\sqrt{3} a^2}$, поэтому
\[
a= \sqrt{\frac{2}{\sqrt{3} } \frac{\Phi_0}{B_2}} 
\approx 2,8 \cdot 10^{-5}\text{ см}
.\] 
\end{sol}
\begin{hiProb}[9.5]
\end{hiProb}
\begin{sol}
Направление тока выберем за ось $y$, тогда поле направлено
по $z$, $\mathbf{h}= \left(0,\,0,\,h(x)\right)$.
Важно понимать, что ток не может быть однородно распределён
по толщине плёнки --- его плотность должна зависеть от
$x$, т.\:е. $\mathbf{j}_s= \left( 0,\,j_s(x),\,0 \right) $.
В то же время из симметрии ясно, что $j_s(x)$ ---
чётная функция, а  $h(x)$ --- нечётная.

Возьмём прямоугольное сечение плёнки плоскостью $xz$ (т.\:е.
перпендикулярно току), имеющее единичную длину вдоль
$z$. Применяя уравнения Максвелла и формулу Стокса, находим
 \[
	 - h\frac{d}{2}
	 +h \frac{d}{2} = \frac{4\pi}{c}I \implies
	 h \frac{d}{2} = -\frac{2\pi I}{c}
.\] 
С учётом этого граничного условия общее решение
уравнения Лондонов даёт
\[
	h(x)= - \frac{2\pi I}{c} \frac{\sh \left(x /\lambda\right)}{
	\sh \left( d /(2\lambda) \right) }
,\]
а из уравнений Максвелла получаем
\[
	j_s(x)= -\frac{c}{4\pi} \frac{dh}{dx}= \frac{I}{2\lambda} \frac{\ch \left( x/\lambda \right) }{
	\sh  \left( d /(2\lambda) \right) }
.\] 
\end{sol}
\begin{hiProb}[Т9-6]
\end{hiProb}
\begin{sol}
Введём обозначение
\[
\lambda^2= \frac{mc^2}{4\pi n c^2}
.\] 
Полный ток:
\[
	I= \iint j(z) dy dz= d \cdot  j_0 \int e^{-\frac{z}{\lambda}}dz=\lambda d j_0
.\] 
\[
L^K= \frac{2c^2}{J^2} \int n \frac{mu^2}{2} dV= \frac{2c^2}{I^2} \frac{m}{ne^2} \frac{4\pi}{4\pi} \int j^2 dV=
\frac{4\pi \lambda^2 }{\lambda^2 d^2 j_0^2}d^2 j_0^2 \int\limits_{0}^{\infty} e^{-\frac{2z}{\lambda}}dz= 2\pi \lambda 
.\] 
\[
L_\text{внутр}^M= \frac{2c^2}{I^2} \frac{1}{8\pi}
\left( \frac{4\pi}{c} \frac{I}{d} \right) ^2 d^2 \int\limits_{0}^{\infty} e^{-\frac{2z}{\lambda}}dz=2\pi\lambda 
.\] 
\[
L^K=L_{\text{внутр}}^M=2\pi \lambda
.\] 
\end{sol}
\begin{hiProb}[Т9-7]
\end{hiProb}
\begin{sol}
\renewcommand{\labelenumi}{\asbuk{enumi})}
\begin{enumerate}
\item Поскольку $d \ll \lambda$, ток распределён по
	плёнке однородно. Обобщённое второее
	уравнение Лондонов:
	\[
		\mathbf{j}_s= \frac{1}{c\Lambda} \left( 
		\frac{\Phi_0}{2\pi} \nabla 
	\theta - \mathbf{A}\right) 
	.\] 
	Интегрируем по замкнутому круговому контуру
	радиуса $R$, получаем
	\[
	\Phi= \Phi_0 n - \frac{2\pi m c R}{n_s e^2}j_s
	.\] 
	С другой стороны, из уравнений Максвелла следует, что
	магнитное поле внутри цилиндра связано с поверхностным
	током формулой
	\[
	H= \frac{4\pi}{c} j_s d
	,\] 
поэтому поток внутри цилиндра
\[
\Phi= \frac{4\pi^2}{c}j_s d R^2
.\] 
Выражая отсюда $j_s$ и подставляя, получаем
\[
	\Phi= \Phi_0 n \left( 
	1+ \frac{2\lambda^2}{R d}\right) ^{-1}
.\] 
\item 
	Важно, что ток распределён по плёнке однородно.
	Тогда интегрируя уравнение Максвелла по прямоугольному
	контуру, одна сторона которого идёт вдоль внешней
	поверхности плёнки, а другая --- внутри плёнки,
	мы получим линейный закон нарастания магнитного поля при
	удалении от внешней поверхности плёнки (рис.~\ref{fig:1}).
\begin{figure}[ht]
    \centering
    \incfig{1}
    \caption{}
    \label{fig:1}
\end{figure}
\end{enumerate}
\end{sol}
\end{document}
