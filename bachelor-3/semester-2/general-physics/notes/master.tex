\documentclass[10pt,landscape,a4paper]{article}
% Этот шаблон документа разработан в 2014 году
% Данилом Фёдоровых (danil@fedorovykh.ru) 
% для использования в курсе 
% <<Документы и презентации в \LaTeX>>, записанном НИУ ВШЭ
% для Coursera.org: http://coursera.org/course/latex .
% Исходная версия шаблона --- 
% https://www.writelatex.com/coursera/latex/5.3

% В этом документе преамбула

\usepackage{siunitx}
%%% Работа с русским языком
\usepackage{cmap}					% поиск в PDF
\usepackage{mathtext} 				% русские буквы в формулах
\usepackage[T2A]{fontenc}			% кодировка
\usepackage[utf8]{inputenc}			% кодировка исходного текста
\usepackage[english,russian]{babel}	% локализация и переносы
\usepackage{indentfirst}
\frenchspacing

\renewcommand{\epsilon}{\ensuremath{\varepsilon}}
\renewcommand{\phi}{\ensuremath{\varphi}}
\renewcommand{\kappa}{\ensuremath{\varkappa}}
\renewcommand{\le}{\ensuremath{\leqslant}}
\renewcommand{\leq}{\ensuremath{\leqslant}}
\renewcommand{\ge}{\ensuremath{\geqslant}}
\renewcommand{\geq}{\ensuremath{\geqslant}}
\renewcommand{\emptyset}{\varnothing}
\renewcommand{\Im}{\operatorname{Im}}
\renewcommand{\Re}{\operatorname{Re}}


%%% Дополнительная работа с математикой
\usepackage{amsmath,amsfonts,amssymb,amsthm,mathtools} % AMS
\usepackage{icomma} % "Умная" запятая: $0,2$ --- число, $0, 2$ --- перечисление

%% Номера формул
%\mathtoolsset{showonlyrefs=true} % Показывать номера только у тех формул, на которые есть \eqref{} в тексте.
%\usepackage{leqno} % Нумереация формул слева

%% Свои команды
\DeclareMathOperator{\sgn}{\mathop{sgn}}
\DeclareMathOperator{\sign}{\mathop{sign}}
\DeclareMathOperator*{\res}{\mathop{res}}
\DeclareMathOperator*{\tr}{\mathop{tr}}

%% Перенос знаков в формулах (по Львовскому)
\newcommand*{\hm}[1]{#1\nobreak\discretionary{}
{\hbox{$\mathsurround=0pt #1$}}{}}

%%% Работа с картинками
\usepackage{graphicx}  % Для вставки рисунков
\graphicspath{{figures/}}  % папки с картинками
\setlength\fboxsep{3pt} % Отступ рамки \fbox{} от рисунка
\setlength\fboxrule{1pt} % Толщина линий рамки \fbox{}
\usepackage{wrapfig} % Обтекание рисунков текстом

%%% Работа с таблицами
\usepackage{array,tabularx,tabulary,booktabs} % Дополнительная работа с таблицами
\usepackage{longtable}  % Длинные таблицы
\usepackage{multirow} % Слияние строк в таблице

%%% Теоремы
\theoremstyle{plain} % Это стиль по умолчанию, его можно не переопределять.
\newtheorem{theorem}{Теорема}
\newtheorem*{thm}{Теорема}
\newtheorem{prop}{Утверждение}
 
\theoremstyle{definition} % "Определение"
%\newtheorem{corollary}{Следствие}[theorem]
\newtheorem*{dfn}{Определение}
\newtheorem{problem}{Задача}
\newtheorem*{problem*}{Задача}

 
\theoremstyle{remark} % "Примечание"
\newtheorem*{sol}{Решение}
\newtheorem*{rem}{Замечание}

%%% Программирование
\usepackage{etoolbox} % логические операторы

%%% Страница
%\usepackage{extsizes} % Возможность сделать 14-й шрифт
%\usepackage{geometry} % Простой способ задавать поля
%	\geometry{top=25mm}
%	\geometry{bottom=35mm}
%	\geometry{left=35mm}
%	\geometry{right=20mm}
 
\usepackage{fancyhdr} % Колонтитулы
%	\pagestyle{fancy}
 %	\renewcommand{\headrulewidth}{0pt}  % Толщина линейки, отчеркивающей верхний колонтитул
	%\lfoot{Нижний левый}
	%\rfoot{Нижний правый}
	%\rhead{Верхний правый}
	%\chead{Верхний в центре}
	%\lhead{Верхний левый}
	%\cfoot{Нижний в центре} % По умолчанию здесь номер страницы

\usepackage{setspace} % Интерлиньяж
%\onehalfspacing % Интерлиньяж 1.5
%\doublespacing % Интерлиньяж 2
%\singlespacing % Интерлиньяж 1

\usepackage{lastpage} % Узнать, сколько всего страниц в документе.

\usepackage{soul} % Модификаторы начертания

\usepackage{hyperref}
%\usepackage[usenames,dvipsnames,svgnames,table,rgb]{xcolor}
\hypersetup{				% Гиперссылки
    unicode=true,           % русские буквы в раздела PDF
    pdftitle={Заголовок},   % Заголовок
    pdfauthor={Автор},      % Автор
    pdfsubject={Тема},      % Тема
    pdfcreator={Создатель}, % Создатель
    pdfproducer={Производитель}, % Производитель
    pdfkeywords={keyword1} {key2} {key3}, % Ключевые слова
    colorlinks=true,       	% false: ссылки в рамках; true: цветные ссылки
    linkcolor=red,          % внутренние ссылки
    citecolor=black,        % на библиографию
    filecolor=magenta,      % на файлы
    urlcolor=cyan           % на URL
}

\usepackage{csquotes} % Еще инструменты для ссылок

%\usepackage[style=apa,maxcitenames=2,backend=biber,sorting=nty]{biblatex}

\usepackage{multicol} % Несколько колонок

\usepackage{tikz} % Работа с графикой
\usepackage{pgfplots}
\usepackage{pgfplotstable}
%\usepackage{coloremoji}
\usepackage{floatrow}
\usepackage{subcaption}
\newcommand*{\N}{\mathbb{N}}
\newcommand*{\R}{\mathbb{R}}
\newcommand*{\K}{\mathbb{K}}
\newcommand*{\V}{\mathcal{V}}
\newcommand*{\A}{\mathcal{A}}
\newcommand*{\ii}{\mathbf{1}}
\newcommand*{\oo}{\mathbf{0}}
\newcommand*{\ba}{\mathbf{a}}
\newcommand*{\bb}{\mathbf{b}}
\newcommand*{\Q}{\mathbb{Q}}
\graphicspath{{figures/}}
%\usepackage{breqn}

\renewcommand\thesubfigure{\asbuk{subfigure}}
%\addbibresource{master.bib}

\usepackage{import}
\usepackage{pdfpages}
\usepackage{transparent}
\usepackage{xcolor}
\usepackage{xifthen}

%\newcommand{\incfig}[1]{%
%    \def\svgwidth{\columnwidth}
%    \import{./figures/}{#1.pdf_tex}
%}


\newcommand{\incfig}[2][1]{%
    \def\svgwidth{#1\columnwidth}
    \import{./figures/}{#2.pdf_tex}
}
\usepackage{titlesec}
%\titleformat{\section}{\normalfont\Large\bfseries}{}{0pt}{}
%----------------------STANDART:
%\titleformat{\chapter}[display]
%  {\normalfont\huge\bfseries}{\chaptertitlename\ \thechapter}{20pt}{\Huge}
%\titleformat{\section}{\normalfont\Large\bfseries}{\thesection}{1em}{}
%\titleformat{\subsection}
%  {\normalfont\large\bfseries}{\thesubsection}{1em}{}
%\titleformat{\subsubsection}
%  {\normalfont\normalsize\bfseries}{\thesubsubsection}{1em}{}
%\titleformat{\paragraph}[runin]
%  {\normalfont\normalsize\bfseries}{\theparagraph}{1em}{}
%\titleformat{\subparagraph}[runin]
%  {\normalfont\normalsize\bfseries}{\thesubparagraph}{1em}{}

\pdfsuppresswarningpagegroup=1
\pgfplotsset{compat=1.16}

\usepackage{xifthen}
\makeatother
%\def\@lecture{}%
%\newcommand{\lecture}[3]{
%    \ifthenelse{\isempty{#3}}{%
%        \def\@lecture{Неделя #1}%
%    }{%
%        \def\@lecture{Неделя #1: #3}%
%    }%
%    \section*{\@lecture}
%    \marginpar{\small\textsf{\mbox{#2}}}
%}
\makeatletter

%\newcommand{\lec}{\subsection{Лекция}}
%\newcommand{\sem}{\subsection{Семинар}}
%\newcommand{\hw}{\subsection{Домашняя работа}}
%\newcommand{\prob}[1]{\textbf{#1}}
%\renewcommand{\thesubsection}{}
%\renewcommand{\thesubsubsection}{}

%\setcounter{tocdepth}{1} % only parts,chapters,sections
%\titleformat{\subsection}{\normalfont\large\bfseries}{}{0em}{}
%\titleformat{\subsubsection}{\normalfont\normalsize\bfseries}{}{0em}{}

%\newcommand{\textover}[2]{\stackrel{\mathclap{\normalfont\mbox{#2}}}{#1}}

\author{Драчов Ярослав\\
Факультет общей и прикладной физики МФТИ}
\newcommand{\veq}{\mathrel{\rotatebox{90}{$=$}}}
%\newcommand{\teto}[1]{\stackrel{\mathclap{\normalfont\tiny\mbox{#1}}}{\to}}
%\renewcommand{\thesubsection}{\arabic{subsection}}

%%\setcounter{secnumdepth}{0}

\definecolor{tabblue}{RGB}{30, 119, 180}
\definecolor{taborange}{RGB}{255, 127, 15}
\definecolor{tabgreen}{RGB}{45, 160, 43}
\definecolor{tabred}{RGB}{214, 38, 40}
\definecolor{tabpurple}{RGB}{148, 103, 189}
\definecolor{tabbrown}{RGB}{140, 86, 76}
\definecolor{tabpink}{RGB}{227, 119, 193}
\definecolor{tabgray}{RGB}{127, 127, 127}
\definecolor{tabolive}{RGB}{188, 189, 33}
\definecolor{tabcyan}{RGB}{22, 190, 207}
\pgfplotscreateplotcyclelist{colorbrewer-tab}{
{tabblue},
{taborange},
{tabgreen},
{tabred},
{tabpurple},
{tabbrown},
{tabpink},
{tabgray},
{tabolive},
{tabcyan},
}
\usepackage{csvsimple}
\usepackage{extarrows}
%\renewcommand{\labelenumii}{\asbuk{enumii})}
%\renewcommand{\labelenumiv}{\Asbuk{enumiv}}
\newcommand{\prob}[1]{\subsubsection*{#1}}
\sisetup{output-decimal-marker = {,},separate-uncertainty = true,exponent-product = \cdot}

\usepackage{braket}
\usepackage{enumerate}
\usepackage{chngcntr}
%\counterwithin*{equation}{problem}
%\usepackage{bbold}

\newtheoremstyle{hiProb}% ⟨name ⟩ 
{3pt}% ⟨Space above ⟩1 
{3pt}% ⟨Space below ⟩1
{}% ⟨Body font ⟩
{}% ⟨Indent amount ⟩2
{\bfseries}% ⟨Theorem head font⟩
{.}% ⟨Punctuation after theorem head ⟩
{.5em}% ⟨Space after theorem head ⟩3
%{\thmname{#1} \thmnote{#3}}% ⟨Theorem head spec (can be left empty, meaning ‘normal’)⟩
{\thmnote{#3}}% ⟨Theorem head spec (can be left empty, meaning ‘normal’)⟩
\theoremstyle{hiProb} % "Определение"
%\newtheorem{hiProb}{Задача}
\newtheorem{hiProb}{}
\usepackage{mmacells}
\newcommand{\textover}[2]{\stackrel{\mathclap{\normalfont\scriptsize\mbox{#2}}}{#1}}
\usepackage{units}
\usepackage[math]{cellspace}%
\setlength\cellspacetoplimit{2pt}
\setlength\cellspacebottomlimit{2pt}

\usepackage{tikz}
\usetikzlibrary{shapes,positioning,arrows,fit,calc,graphs,graphs.standard}
%\usepackage[nosf]{kpfonts}
%\usepackage[t1]{sourcesanspro}
%\usepackage[lf]{MyriadPro}
%\usepackage[lf,minionint]{MinionPro}
\usepackage{multicol}
\usepackage{wrapfig}
\usepackage[top=0mm,bottom=1mm,left=0mm,right=1mm]{geometry}
\usepackage[framemethod=tikz]{mdframed}
\usepackage{microtype}

\let\bar\overline

\definecolor{myblue}{cmyk}{1,.72,0,.38}

\def\firstcircle{(0,0) circle (1.5cm)}
\def\secondcircle{(0:2cm) circle (1.5cm)}

\colorlet{circle edge}{myblue}
\colorlet{circle area}{myblue!5}

\tikzset{filled/.style={fill=circle area, draw=circle edge, thick},
    outline/.style={draw=circle edge, thick}}

\pgfdeclarelayer{background}
\pgfsetlayers{background,main}

\everymath\expandafter{\the\everymath \color{myblue}}
\everydisplay\expandafter{\the\everydisplay \color{myblue}}

\renewcommand{\baselinestretch}{.8}
\pagestyle{empty}

\global\mdfdefinestyle{header}{%
linecolor=gray,linewidth=1pt,%
leftmargin=0mm,rightmargin=0mm,skipbelow=0mm,skipabove=0mm,
}

\newcommand{\header}{
\begin{mdframed}[style=header]
\footnotesize
\sffamily
Шпаргалка~по~общесосу
\end{mdframed}
}

\makeatletter
\renewcommand{\section}{\@startsection{section}{1}{0mm}%
                                {.2ex}%
                                {.2ex}%x
                                {\color{myblue}\sffamily\small\bfseries}}
\renewcommand{\subsection}{\@startsection{subsection}{1}{0mm}%
                                {.2ex}%
                                {.2ex}%x
                                {\sffamily\bfseries}}



\def\multi@column@out{%
   \ifnum\outputpenalty <-\@M
   \speci@ls \else
   \ifvoid\colbreak@box\else
     \mult@info\@ne{Re-adding forced
               break(s) for splitting}%
     \setbox\@cclv\vbox{%
        \unvbox\colbreak@box
        \penalty-\@Mv\unvbox\@cclv}%
   \fi
   \splittopskip\topskip
   \splitmaxdepth\maxdepth
   \dimen@\@colroom
   \divide\skip\footins\col@number
   \ifvoid\footins \else
      \leave@mult@footins
   \fi
   \let\ifshr@kingsaved\ifshr@king
   \ifvbox \@kludgeins
     \advance \dimen@ -\ht\@kludgeins
     \ifdim \wd\@kludgeins>\z@
        \shr@nkingtrue
     \fi
   \fi
   \process@cols\mult@gfirstbox{%
%%%%% START CHANGE
\ifnum\count@=\numexpr\mult@rightbox+2\relax
          \setbox\count@\vsplit\@cclv to \dimexpr \dimen@-1cm\relax
\setbox\count@\vbox to \dimen@{\vbox to 1cm{\header}\unvbox\count@\vss}%
\else
      \setbox\count@\vsplit\@cclv to \dimen@
\fi
%%%%% END CHANGE
            \set@keptmarks
            \setbox\count@
                 \vbox to\dimen@
                  {\unvbox\count@
                   \remove@discardable@items
                   \ifshr@nking\vfill\fi}%
           }%
   \setbox\mult@rightbox
       \vsplit\@cclv to\dimen@
   \set@keptmarks
   \setbox\mult@rightbox\vbox to\dimen@
          {\unvbox\mult@rightbox
           \remove@discardable@items
           \ifshr@nking\vfill\fi}%
   \let\ifshr@king\ifshr@kingsaved
   \ifvoid\@cclv \else
       \unvbox\@cclv
       \ifnum\outputpenalty=\@M
       \else
          \penalty\outputpenalty
       \fi
       \ifvoid\footins\else
         \PackageWarning{multicol}%
          {I moved some lines to
           the next page.\MessageBreak
           Footnotes on page
           \thepage\space might be wrong}%
       \fi
       \ifnum \c@tracingmulticols>\thr@@
                    \hrule\allowbreak \fi
   \fi
   \ifx\@empty\kept@firstmark
      \let\firstmark\kept@topmark
      \let\botmark\kept@topmark
   \else
      \let\firstmark\kept@firstmark
      \let\botmark\kept@botmark
   \fi
   \let\topmark\kept@topmark
   \mult@info\tw@
        {Use kept top mark:\MessageBreak
          \meaning\kept@topmark
         \MessageBreak
         Use kept first mark:\MessageBreak
          \meaning\kept@firstmark
        \MessageBreak
         Use kept bot mark:\MessageBreak
          \meaning\kept@botmark
        \MessageBreak
         Produce first mark:\MessageBreak
          \meaning\firstmark
        \MessageBreak
        Produce bot mark:\MessageBreak
          \meaning\botmark
         \@gobbletwo}%
   \setbox\@cclv\vbox{\unvbox\partial@page
                      \page@sofar}%
   \@makecol\@outputpage
     \global\let\kept@topmark\botmark
     \global\let\kept@firstmark\@empty
     \global\let\kept@botmark\@empty
     \mult@info\tw@
        {(Re)Init top mark:\MessageBreak
         \meaning\kept@topmark
         \@gobbletwo}%
   \global\@colroom\@colht
   \global \@mparbottom \z@
   \process@deferreds
   \@whilesw\if@fcolmade\fi{\@outputpage
      \global\@colroom\@colht
      \process@deferreds}%
   \mult@info\@ne
     {Colroom:\MessageBreak
      \the\@colht\space
              after float space removed
              = \the\@colroom \@gobble}%
    \set@mult@vsize \global
  \fi}

\makeatother
\setlength{\parindent}{0pt}
\title{Шпаргалка к общесосу}
\begin{document}
\small
\begin{multicols*}{5}

\begin{hiProb}[0-9-1]
Оценить максимальный ток, который может протекать через индиевую проволочку диаметром $d=100 \text{ мкм}$, не разрушая сверхпроводимость. Критическое поле для индия при $T=0$ равно $H_c=280\text{ Э}$.
\end{hiProb}
\begin{sol}
Применим теорему о циркуляции для магнитного
поля для контура, опоясывающего провод:
\begin{multline*}
2\pi r H_c= \frac{4\pi}{c}J_c,\quad
J_c= \frac{cr}{2}H_c=\\= \frac{dc}{4} H_c\approx 21\cdot 10^{10} \text{ ед. СГС}=7 \text{ А}
.\end{multline*} 
\end{sol}
\begin{hiProb}[0-9-2]
	На кварцевый стержень диаметром  $d=10 \text{ мкм}$ напылён слой свинца ($T_c=7,2\text{ К}$), толщина которого много больше лондоновской глубины проникновения. Оценить, чему должна быть равна напряженность магнитного поля, приложенного параллельного оси стержня при $T>T_c$, чтобы при переходе в сверхпроводящее состояние был захвачен магнитный поток, равный кванту потока $\Phi_0$.
\end{hiProb}
\begin{sol}
Квант магнитного потока равен
$
\Phi_0= \pi \hbar c /e
.$ 
Захваченный магнитный поток равен величине $BS$, где $S$ 
--- площадь сечения: 
$
S= \pi d^2 /4.$ 
\[
B=
\frac{\Phi_0}{S}= \frac{4\pi \hbar c}{\pi e d^2}=
\frac{4\hbar c}{ed^2}\approx 0,9 \cdot 10^{-9}\text{ Гс}
.\] 
\end{sol}
\begin{hiProb}[0-10-1]
При приложении какого напряжения начнёт течь ток в туннельном контакте между сверхпроводящим свинцом ($T_c=7,2\text{ К}$) и нормальным металлом? Температура $T\ll T_c$.
\end{hiProb}
\begin{sol}
Щель в спектре возбуждений равна $\Delta \approx 1,5 k_B T_c$. Пороговое напряжение равно $U= \Delta /e$. Получаем,
что $U= 1,5k_B T_c /e\approx 928\text{ мкВ}$.
\end{sol}
\begin{hiProb}[0-10-2]
При каком напряжении на джозефсоновском контакте в нём будут генерироваться электромагнитные волны с длиной волны $\lambda=3\text{ см}$?
\end{hiProb}
\begin{sol}
Связь частоты осцилляции и напряжения: $\omega=2eU /\hbar $.
Тогда для длины волны имеем, что
\[
\lambda= \frac{2\pi c}{\omega}= \frac{2\pi \hbar c}{2eU}=
\frac{hc}{eU} 
.\] 
\[
U=\frac{hc}{e\lambda}\approx
40 \text{ мкВ}
.\] 
\end{sol}
\begin{hiProb}[0-11-1]
Чему равен уровень химпотенциала в полупроводнике с легированием акцепторными примесями при $T=0$? Считать энергию «потолка» валентной зоны равной нулю, ширина запрещенной зоны $\Delta$, расстояние от примесного уровня до «дна» зоны проводимости $\delta$.
\end{hiProb}
\begin{sol}
При нулевой температуре при наличии акцепторных
примесей уровень химпотенциала находится между
потолком валентной зоны и акцепторным уровнем.
Акцепторный уровень: $\Delta-\delta$. Получаем,
что $\mu = (\Delta-\delta) /2$.
\end{sol}
\begin{hiProb}[0-11-2]
Найти потенциал плоской границы контакта полупроводников на $p$-$n$-переходе при $T=0$, если концентрация донорных примесей в 10 раз больше, чем концентрация акцепторных. Оба полупроводника получены легированием кремния ($\Delta=1,1\text{ эВ}$), примесные уровни близки к
\end{hiProb}
\begin{sol}
Пусть $N_d$ и $N_a$ --- концентрации доноров и
акцепторов. Пусть $x=0$ есть граница контакта, $x<0$
есть проводник типа $(n)$, $x>0$ есть проводник
типа $(p)$. Тогда химический потенциал монотонно убывает
с увеличением $x$. Тогда существуют такие точки $h_d$ и
$h_a$, что при $-h_d<x<0$ донорные уровни выше
химического потенциала, а при $0<x<h_a$ акцепторные
уровни ниже химического потенциала. В этих областях
соответствующие ионы полностью ионизированы.
Из условия электронейтральности: $N_d h_d=N_a h_a$.
Если  $\epsilon $ --- диэлектрическая проницаемость
среды, то имеем уравнения Пуассона в областях:
\[
	\epsilon \phi''(x)=4\pi e\cdot 
	\begin{cases}
		N_d & \text{при } x  \in \left( -h_d,\,0 \right) ,\\
		N_a & \text{при } x  \in \left( 0,\,h_a \right) .\\
	\end{cases}
\] 
При $x < -h_d$ считаем, что $\phi(x)=0$.
\[
	\phi(x)=%\begin{cases}
		- \frac{4\pi e N_d}{2\epsilon }(x+h_d)^2
		%& \text{при } x \in \left( -h_d, 0 \right) ,\\
		%\frac{1}{\epsilon }
		%\left( 2\pi e N_a \left( x-h_a \right) ^2-
		%2\pi e N_a h_a^2 -2\pi e N_d h_d^2\right) 
		%& \text{при } x \in \left( 
		%0,\,h_a\right) .
		%\end{cases}
	\] 
	при $x \in \left( -h_d,\,0 \right) $.
	\begin{multline*}
		\phi(x)= \frac{1}{\epsilon }
		\left( 2\pi e N_a \left( x-h_a \right) ^2\right. - \\ - \left.
		2\pi e N_a h_a^2-
	2\pi e N_d h_d^2\right) 
	\end{multline*}
	при $x \in \left( 0,\,h_a \right) $
	Получаем, что $\phi(0)= - 2\pi e N_d h_d^2 /\epsilon $. С другой стороны
\[
	\phi(h_a)= -\frac{2\pi e}{\epsilon }
	\left( N_a h_a^2+N_d h_d^2 \right) 			
\]
и $\phi(h_d)=0$. С учётом того, что
$N_a h_a= N_d h_d$, получаем, что
 \[
	 \phi(0)= -\frac{N_a}{N_a+N_d}\left( 
	 \phi(h_d)-\phi(h_a)\right) 
\]
есть выражение для потенциала границы.
При этом
\[
	\phi \left( h_d \right) -
	\phi\left( h_a \right) =
	\frac{1}{e} \left( \mu_d- \mu_a \right) 
	\approx
	\frac{\Delta}{e}
,\] 
где учтено, что химические потенциалы близки к
границам зоны.
Получаем потенциал границы:
\[
	\phi(0)= - \frac{1}{1+ N_d /N_a}
	\frac{\Delta}{e}= -0,1 \text{ В}
.\] 
\end{sol}
\begin{hiProb}[0-12-1]
При изучении двумерного электронного газа в графене или
гетероструктурах часто используют дополнительный электрод («затвор»),
который создаёт перпендикулярное плоскости газа электрическое поле,
притягивающее или выталкивающее электроны из внешней цепи. Это
позволяет регулировать концентрацию электронов. В опыте такого типа
при приложении электрического поля перпендикулярно плоскости
графена энергия Ферми электронов оказалась равна $E_F=0,1\text{ эВ}$. Найти
поверхностную плотность электронов в графене в этом опыте. Спектр электронов вблизи точки Дирака $E\left( \mathbf{k}-\mathbf{k}_0 \right) =\hbar c^* \left| \mathbf{k}-\mathbf{k}_0 \right| $, $c^*=108 \text{ см} /\text{с}$.
\end{hiProb}
\begin{sol}
Фермиевский вектор в двумерном случае равен $k_F=\sqrt{2\pi n} $, где $n$ --- плотность электронов.
\[
	E_F= c^* k_F \hbar,\quad n= \frac{k_F^2}{2\pi}
.\] 
Получаем, что
 \[
n= \frac{E_F^2}{2c^{*2}\pi\hbar ^2}\approx 3,6\cdot 10^{11}
\text{ см}^{-2}
.\] 
\end{sol}
\begin{hiProb}[0-12-2]
Вычислить энергетическое расстояние между нижними уровнями размерного квантования для двумерного электронного газа в Si и GaAs в приближении прямоугольной потенциальной ямы с бесконечно высокими потенциальными стенками в направлении, нормальном к плоскости двумерного электронного газа. Эффективная масса электрона $m^*$ в GaAs и Si составляет 0.067 и 0.19 масс свободного электрона соответственно. Ширину ямы принять равной $d=20\text{ нм}$. Сравнить полученное значение с характерными температурами – комнатной (300 К), температурой жидкого азота (77 К), температурой жидкого гелия (4.2 К).
\end{hiProb}
\begin{sol}
?
\end{sol}
\begin{hiProb}[0-13-1]
Оценить, при какой температуре диполь-дипольное взаимодействие атомных магнитных моментов могло бы привести к возникновению упорядоченной магнитной структуры. Для оценки принять магнитный момент атомов равным $\mu_B$, расстояние между атомами $d=4\text{ \AA}$.
\end{hiProb}
\begin{sol}
Диполь-дипольное взаимодействие может привести к
возникновению упорядоченной магнитной структуры,
если энергия этого взаимодействия будет больше
тепловой энергии. Значит, $\Delta E >k_B T$. Энергию
дипольного взаимодействия оценим как
$\Delta E \simeq \mu_B^2 /\alpha^3$. Получаем оценку
для температуры
$T<\mu_B^2 / \left(k_B \alpha^3\right) \approx 0,01 \text{ К}$.
\end{sol}
\begin{hiProb}[0-13-2]
	Спектр длинноволновых элементарных возбуждений ферромагнетика (спиновых волн) $\omega(k)=Ak^2$. Определить зависимость от температуры вклада спиновых волн в теплоёмкость трёхмерного ферромагнетика.
\end{hiProb}
\begin{sol}
Запишем плотность энергии для магнонов:
\[
	\rho = \int\limits_{0}^{+\infty} n(\omega) 
	\hbar \omega \frac{d^3k}{\left( 2\pi \right) ^3}
,\] 
где
\[
	n= \frac{1}{\exp \left( \frac{\hbar \omega}{k_B T} \right) -1}
\]
--- функция распределения.
\begin{multline*}
	\int\limits_{0}^{+\infty} n(\omega)
	\hbar \omega \frac{d^3k}{\left( 2\pi \right) ^3}=\\=
	\int\limits_{0}^{+\infty}  
	\frac{A \hbar  k^2}{\exp \left( \frac{A\hbar k^2}{k_B T} \right) -1} \frac{4\pi k^2 dk}{\left( 2\pi \right) ^3}=\\= \left( \frac{k_B T}{A\hbar } \right) ^{5 /2}
	\frac{4\pi \hbar }{\left( 2\pi \right) ^3}
	\int\limits_{0}^{+\infty} 
	\frac{x^4 dx}{\exp \left(x^2\right)-1}
	\propto\\ \propto T^{5 /2}
.\end{multline*} 
Теплоёмкость равна $\displaystyle C=\frac{d\rho}{dT}
 \propto T^{3 /2}$. Получаем зависимость теплоёмкости
 от температуры: $C \propto T^{3 /2}$.
\end{sol}
\end{multicols*}
\end{document}
