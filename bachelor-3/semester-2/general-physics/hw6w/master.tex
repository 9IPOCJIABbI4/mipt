\documentclass[a4paper]{article}
% Этот шаблон документа разработан в 2014 году
% Данилом Фёдоровых (danil@fedorovykh.ru) 
% для использования в курсе 
% <<Документы и презентации в \LaTeX>>, записанном НИУ ВШЭ
% для Coursera.org: http://coursera.org/course/latex .
% Исходная версия шаблона --- 
% https://www.writelatex.com/coursera/latex/5.3

% В этом документе преамбула

\usepackage{siunitx}
%%% Работа с русским языком
%\usepackage{cmap}					% поиск в PDF
%\usepackage{mathtext} 				% русские буквы в формулах
%\usepackage[T2A]{fontenc}			% кодировка
%\usepackage[utf8]{inputenc}			% кодировка исходного текста
%\usepackage[english,russian]{babel}	% локализация и переносы
%\usepackage{indentfirst}
%\frenchspacing
%
%\renewcommand{\epsilon}{\ensuremath{\varepsilon}}
%\newcommand{\phibackup}{\ensuremath{\phi}}
%\renewcommand{\phi}{\ensuremath{\varphi}}
%\renewcommand{\varphi}{\ensuremath{\phibackup}}
%\renewcommand{\kappa}{\ensuremath{\varkappa}}
%\renewcommand{\le}{\ensuremath{\leqslant}}
%\renewcommand{\leq}{\ensuremath{\leqslant}}
%\renewcommand{\ge}{\ensuremath{\geqslant}}
%\renewcommand{\geq}{\ensuremath{\geqslant}}
%\renewcommand{\emptyset}{\varnothing}
%\renewcommand{\Im}{\operatorname{Im}}
%\renewcommand{\Re}{\operatorname{Re}}


%%% Дополнительная работа с математикой
\usepackage{amsmath,amsfonts,amssymb,amsthm,mathtools} % AMS
%\usepackage{icomma} % "Умная" запятая: $0,2$ --- число, $0, 2$ --- перечисление

%% Номера формул
%\mathtoolsset{showonlyrefs=true} % Показывать номера только у тех формул, на которые есть \eqref{} в тексте.
%\usepackage{leqno} % Нумереация формул слева

%% Свои команды
\DeclareMathOperator{\sgn}{\mathop{sgn}}
\DeclareMathOperator{\sign}{\mathop{sign}}
\DeclareMathOperator*{\res}{\mathop{res}}
\DeclareMathOperator*{\tr}{\mathop{tr}}
\DeclareMathOperator*{\rot}{\mathop{rot}}
\DeclareMathOperator*{\divop}{\mathop{div}}
\DeclareMathOperator*{\grad}{\mathop{grad}}

%% Перенос знаков в формулах (по Львовскому)
\newcommand*{\hm}[1]{#1\nobreak\discretionary{}
{\hbox{$\mathsurround=0pt #1$}}{}}

%%% Работа с картинками
\usepackage{graphicx}  % Для вставки рисунков
\graphicspath{{figures/}}  % папки с картинками
\setlength\fboxsep{3pt} % Отступ рамки \fbox{} от рисунка
\setlength\fboxrule{1pt} % Толщина линий рамки \fbox{}
\usepackage{wrapfig} % Обтекание рисунков текстом

%%% Работа с таблицами
\usepackage{array,tabularx,tabulary,booktabs} % Дополнительная работа с таблицами
\usepackage{longtable}  % Длинные таблицы
\usepackage{multirow} % Слияние строк в таблице

%%% Теоремы
\theoremstyle{plain} % Это стиль по умолчанию, его можно не переопределять.
\newtheorem{thm}{Теорема}
\newtheorem*{thm*}{Теорема}
\newtheorem{prop}{Предложение}
\newtheorem*{prop*}{Предложение}
 
\theoremstyle{definition} % "Определение"
%\newtheorem{corollary}{Следствие}[theorem]
\newtheorem{dfn}{Определение}
\newtheorem*{dfn*}{Определение}
\newtheorem{prob}{Задача}
\newtheorem*{prob*}{Задача}

 
\theoremstyle{remark} % "Примечание"
\newtheorem*{sol}{Решение}
\newtheorem*{rem}{Замечание}

%%% Программирование
\usepackage{etoolbox} % логические операторы

%%% Страница
%\usepackage{extsizes} % Возможность сделать 14-й шрифт
%\usepackage{geometry} % Простой способ задавать поля
%	\geometry{top=25mm}
%	\geometry{bottom=35mm}
%	\geometry{left=35mm}
%	\geometry{right=20mm}
 
\usepackage{fancyhdr} % Колонтитулы
%	\pagestyle{fancy}
 %	\renewcommand{\headrulewidth}{0pt}  % Толщина линейки, отчеркивающей верхний колонтитул
	%\lfoot{Нижний левый}
	%\rfoot{Нижний правый}
	%\rhead{Верхний правый}
	%\chead{Верхний в центре}
	%\lhead{Верхний левый}
	%\cfoot{Нижний в центре} % По умолчанию здесь номер страницы

\usepackage{setspace} % Интерлиньяж
%\onehalfspacing % Интерлиньяж 1.5
%\doublespacing % Интерлиньяж 2
%\singlespacing % Интерлиньяж 1

\usepackage{lastpage} % Узнать, сколько всего страниц в документе.

\usepackage{soul} % Модификаторы начертания

\usepackage{hyperref}
\usepackage[usenames,dvipsnames,svgnames,table,rgb]{xcolor}
\hypersetup{				% Гиперссылки
    unicode=true,           % русские буквы в раздела PDF
    pdftitle={Заголовок},   % Заголовок
    pdfauthor={Автор},      % Автор
    pdfsubject={Тема},      % Тема
    pdfcreator={Создатель}, % Создатель
    pdfproducer={Производитель}, % Производитель
    pdfkeywords={keyword1} {key2} {key3}, % Ключевые слова
%    colorlinks=true,       	% false: ссылки в рамках; true: цветные ссылки
    %linkcolor=red,          % внутренние ссылки
    %citecolor=black,        % на библиографию
    %filecolor=magenta,      % на файлы
    %urlcolor=cyan           % на URL
}

\usepackage{csquotes} % Еще инструменты для ссылок

%\usepackage[style=apa,maxcitenames=2,backend=biber,sorting=nty]{biblatex}

\usepackage{multicol} % Несколько колонок

\usepackage{tikz} % Работа с графикой
\usepackage{pgfplots}
\usepackage{pgfplotstable}
%\usepackage{coloremoji}
\usepackage{floatrow}
\usepackage{subcaption}
\graphicspath{{figures/}}

\renewcommand\thesubfigure{\asbuk{subfigure}}
%\addbibresource{master.bib}

\usepackage{import}
\usepackage{pdfpages}
\usepackage{transparent}
\usepackage{xcolor}
\usepackage{xifthen}

\newcommand{\incfig}[2][1]{%
    \def\svgwidth{#1\columnwidth}
    \import{./figures/}{#2.pdf_tex}
}
%\usepackage{titlesec}
%\titleformat{\section}{\normalfont\Large\bfseries}{}{0pt}{}
%----------------------STANDART:
%\titleformat{\chapter}[display]
%  {\normalfont\huge\bfseries}{\chaptertitlename\ \thechapter}{20pt}{\Huge}
%\titleformat{\section}{\normalfont\Large\bfseries}{\thesection}{1em}{}
%\titleformat{\subsection}
%  {\normalfont\large\bfseries}{\thesubsection}{1em}{}
%\titleformat{\subsubsection}
%  {\normalfont\normalsize\bfseries}{\thesubsubsection}{1em}{}
%\titleformat{\paragraph}[runin]
%  {\normalfont\normalsize\bfseries}{\theparagraph}{1em}{}
%\titleformat{\subparagraph}[runin]
%  {\normalfont\normalsize\bfseries}{\thesubparagraph}{1em}{}

\pdfsuppresswarningpagegroup=1
\pgfplotsset{compat=1.16}



%\setcounter{tocdepth}{1} % only parts,chapters,sections
%\titleformat{\subsection}{\normalfont\large\bfseries}{}{0em}{}
%\titleformat{\subsubsection}{\normalfont\normalsize\bfseries}{}{0em}{}

%\newcommand{\textover}[2]{\stackrel{\mathclap{\normalfont\mbox{#2}}}{#1}}

\author{Yaroslav Drachov\\
Moscow Institute of Physics and Technology}
%\author{Драчов Ярослав\\
%Факультет общей и прикладной физики МФТИ}
\newcommand{\veq}{\mathrel{\rotatebox{90}{$=$}}}
%\newcommand{\teto}[1]{\stackrel{\mathclap{\normalfont\tiny\mbox{#1}}}{\to}}
%\renewcommand{\thesubsection}{\arabic{subsection}}

%%\setcounter{secnumdepth}{0}

\definecolor{tabblue}{RGB}{30, 119, 180}
\definecolor{taborange}{RGB}{255, 127, 15}
\definecolor{tabgreen}{RGB}{45, 160, 43}
\definecolor{tabred}{RGB}{214, 38, 40}
\definecolor{tabpurple}{RGB}{148, 103, 189}
\definecolor{tabbrown}{RGB}{140, 86, 76}
\definecolor{tabpink}{RGB}{227, 119, 193}
\definecolor{tabgray}{RGB}{127, 127, 127}
\definecolor{tabolive}{RGB}{188, 189, 33}
\definecolor{tabcyan}{RGB}{22, 190, 207}
\pgfplotscreateplotcyclelist{colorbrewer-tab}{
{tabblue},
{taborange},
{tabgreen},
{tabred},
{tabpurple},
{tabbrown},
{tabpink},
{tabgray},
{tabolive},
{tabcyan},
}
\usepackage{csvsimple}
\usepackage{extarrows}
%\renewcommand{\labelenumii}{\asbuk{enumii})}
%\renewcommand{\labelenumiv}{\Asbuk{enumiv}}
%\newcommand{\prob}[1]{\subsubsection*{#1}}
\sisetup{output-decimal-marker = {,},separate-uncertainty = true,exponent-product = \cdot}

\usepackage{braket}
\usepackage{enumerate}
\usepackage{chngcntr}
%\counterwithin*{equation}{problem}
%\usepackage{bbold}

\newtheoremstyle{hiProb}% ⟨name ⟩ 
{3pt}% ⟨Space above ⟩1 
{3pt}% ⟨Space below ⟩1
{}% ⟨Body font ⟩
{}% ⟨Indent amount ⟩2
{\bfseries}% ⟨Theorem head font⟩
{.}% ⟨Punctuation after theorem head ⟩
{.5em}% ⟨Space after theorem head ⟩3
%{\thmname{#1} \thmnote{#3}}% ⟨Theorem head spec (can be left empty, meaning ‘normal’)⟩
{\thmnote{#3}}% ⟨Theorem head spec (can be left empty, meaning ‘normal’)⟩
\theoremstyle{hiProb} % "Определение"
%\newtheorem{hiProb}{Задача}
\newtheorem{hiProb}{}
%\usepackage{mmacells}
\newcommand{\textover}[2]{\stackrel{\mathclap{\normalfont\scriptsize\mbox{#2}}}{#1}}
\usepackage{units}
\usepackage[math]{cellspace}%
\setlength\cellspacetoplimit{2pt}
\setlength\cellspacebottomlimit{2pt}

\DeclareMathAlphabet{\mathbbold}{U}{bbold}{m}{n}

\newcommand{\normord}[1]{:\mathrel{#1}:}

\title{Неделя №6\\
Объёмные полупроводники}
\begin{document}
	\maketitle
\begin{hiProb}[4.7]
\end{hiProb}
\begin{sol}
Договоримся отсчитывать химпотенциал от дна зоны
проводимости. Раз полупроводник собственный ---
будет электронейтральность и число электронов
будет равно числу дырок: $n_e=n_h$.

Когда  $\Delta \gg T$, уровень химпотенциала
лежит внутри щели и числа заполнения для
электронов и дырок $\left<n \right> \ll 1$, так
что можно пренебречь 1 в знаменателе распределения
Ферми. Это даёт:
\[
	n_h= 2 \left( \frac{m_h T}{2\pi \hbar ^2} \right) ^{3 /2} e^{(\mu +\Delta) /T}
.\] 
\[
	n_e =2 \left( \frac{m_e T}{2\pi \hbar ^2} \right) ^{3/2} e^{-\mu /T}
.\] 
Приравнивая две концентрации и сокращая, получаем:
\[
	\left( \frac{m_e}{m_h} \right) ^{3/2}
	=e^{(2\mu +\Delta) /T}
.\] 
Это даёт ответ:
\[
\mu=-\frac{\Delta}{2} +0,75 T \ln \frac{m_h}{m_e}
.\] 
Обсудим этот ответ. При низких температурах химпотенциал находится посредине щели. Это объясняется тем,
что электронные и дырочные возбуждения могут родиться
только парами и являются равноневыгодными. При
повышении температуры, если масса дырок больше,
то химпотенциал едет в электронную сторону,
если больше масса электронов --- наоборот. Это
связано с тем, что чем большее масса, тем больше
плотность состояний, и чтобы удовлетворить
равенству концентраций, необходимо, при прочих
равных условиях, сдвинуть химпотенциал дальше
от носителей с большей массой.

Надо заметить, что, при наличии сколь угодно
малого количества легирующей примеси
одного типа, ситуация при низких температурах изменится: химпотенциал при $T=0$ окажется посередине
между примесным уровнем и соответствующей зоной
(зоной проводимости для донорной примеси и валентной
зоной для акцепторной).
\end{sol}
\begin{hiProb}[Т6-1]
\end{hiProb}
\begin{sol}
Так как длина волны видимого света много больше
межатомного расстояния, то импульс такого фотона
много меньше бриллюэновского. Это означает, что
при поглощении фотона видимого света электрон
в кристалле переходит между разрешёнными состояниями
<<вертикальнно>> --- практически без изменения своего
квазиимпульса.

Соответственно, в непрямозонном полупроводнике
при поглощении фотона переход
электрона с потолка валентной зоны на дно зоны
проводимости возможен только с поглощением
или излучением дополнительного фонона, такие
процессы менее вероятны, чем прямые
переходы из неэкстремального положения в валентной
зоне в неэкстремальное положение в зоне проводимости.
Процесс с поглощением фонона дополнительно
запрещён низкими температурами (фононов мало).

Для поиска минимальной энергии фотона, с которой
такие вертикальные переходы становятся
разрешёнными, рассмотрим разность энергий
электрона в валентной зоне и зоне проводимости вдоль
прямой, соединяющей экстремумы в $k$-пространстве.
Ноль отсчёта импульса поместим на потолок валентной
зоны, если $\delta$ --- расстояние в $k$-пространстве
между экстремумами и $\xi$ --- координата точки, то
\[
	E(\xi) = \frac{\hbar ^2 \left( \delta
	-\xi \right) ^2}{2m}+\Delta +\frac{\hbar ^2 \xi^2}{2m}
.\] 
Ищем минимум, он достигается при $\xi =\delta /2$ и
равен 
\[
E_{\min}= \Delta + \frac{\hbar ^2 \delta^2}{4m}
.\] 
Откуда
\[
	\delta^2= \frac{4m}{\hbar ^2} (E-\Delta)
.\] 
\end{sol}
\begin{hiProb}[4.50]
\end{hiProb}
\begin{sol}
Вопрос этой задачи можно переформулировать так:
при каких концентрациях примеси легирование нельзя
считать слабым? При повышении
концентрации примеси электроны начинают 
чувствовать не только потенциал своего донора,
но и соседних. Это становится существенным, когда
характерное расстояние между примесями становится
меньше удвоенного боровского радиуса. При этом
мы будем подразумевать, что длина экранирования
будет большой по сравнению с межпримесным
расстоянием, чтоб считать потенциал взаимодействия
электрона и примеси кулоновским.

Поиску боровского радиуса в InSb была посвящена задача 4.2. Ответ
там примерно 60 нм $\left( a_B^*= \frac{\epsilon 
\hbar ^2}{m^* e^2}=a_B \epsilon  m /m^* \right) $.
Приравнивая $n_\text{donors}= \left( 2a_B \right) ^{-3}$ находим искомую концентрацию $n= 5,8 \cdot 
10^{14}\text{см}^{-3}$.

По полупроводниковым меркам это очень маленькая
концентрация: концентрация атомов в твёрдом
теле $\sim 10^{23} \text{см}^{-3}$, то есть
речь идёт об относительной концентрации примесей
на уровне $10^{-8}$. В кремнии, например, из-за
большей эффективной массы эта величина будет
на 4 порядка больше. Данный ответ показывает
насколько сложно работать с узкозонными полупроводниками
(в которых малая эффективная масса): даже
маленькое количество примеси может привести к
образованию примесной зоны.
\end{sol}
\begin{hiProb}[4.12]
\end{hiProb}
\begin{sol}
	Из-за электронейтральности концентрации (а значит и $p_F$) электронов ($n_e$) и дырок
	$(n_h)$ равны между собой. Будем для
простоты считать температуру нулевой. Тогда
перекрытие зон равно сумме энергии Ферми электроннов
(отсчитанной от дна зоны проводимости) и
дырок (отсчитанной от потолка валентной зоны).

Пользуясь тем, что $p_F =\hbar (3\pi^2 n)^{1 /3}$,
запишем
\[
	\frac{1}{2} \hbar ^2 \left( 3\pi^2
	n \right) ^{2 /3} \left( 
\frac{1}{m_e}+ \frac{1}{m_h}\right) =\Delta E
.\] 
Отсюда напрямую получаем
\[
	n= \left( 2\Delta E \frac{m_e m_h}{m_e+m_h} \right) ^{3 /2} \frac{1}{3\pi^2 \hbar ^2}=
	9,3 \cdot 10^{16} \text{ см}^{-3}
.\] 
Энергии Ферми делят перекрытие зон обратно
пропорционально массам, соответственно $E_{F,e}=
0,015$ эВ, $E_{F,h}=0,025$ эВ.
\end{sol}
\end{document}
