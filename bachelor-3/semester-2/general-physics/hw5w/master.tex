\documentclass[a4paper]{article}
% Этот шаблон документа разработан в 2014 году
% Данилом Фёдоровых (danil@fedorovykh.ru) 
% для использования в курсе 
% <<Документы и презентации в \LaTeX>>, записанном НИУ ВШЭ
% для Coursera.org: http://coursera.org/course/latex .
% Исходная версия шаблона --- 
% https://www.writelatex.com/coursera/latex/5.3

% В этом документе преамбула

\usepackage{siunitx}
%%% Работа с русским языком
\usepackage{cmap}					% поиск в PDF
\usepackage{mathtext} 				% русские буквы в формулах
\usepackage[T2A]{fontenc}			% кодировка
\usepackage[utf8]{inputenc}			% кодировка исходного текста
\usepackage[english,russian]{babel}	% локализация и переносы
\usepackage{indentfirst}
\frenchspacing

\renewcommand{\epsilon}{\ensuremath{\varepsilon}}
\renewcommand{\phi}{\ensuremath{\varphi}}
\renewcommand{\kappa}{\ensuremath{\varkappa}}
\renewcommand{\le}{\ensuremath{\leqslant}}
\renewcommand{\leq}{\ensuremath{\leqslant}}
\renewcommand{\ge}{\ensuremath{\geqslant}}
\renewcommand{\geq}{\ensuremath{\geqslant}}
\renewcommand{\emptyset}{\varnothing}
\renewcommand{\Im}{\operatorname{Im}}
\renewcommand{\Re}{\operatorname{Re}}


%%% Дополнительная работа с математикой
\usepackage{amsmath,amsfonts,amssymb,amsthm,mathtools} % AMS
\usepackage{icomma} % "Умная" запятая: $0,2$ --- число, $0, 2$ --- перечисление

%% Номера формул
%\mathtoolsset{showonlyrefs=true} % Показывать номера только у тех формул, на которые есть \eqref{} в тексте.
%\usepackage{leqno} % Нумереация формул слева

%% Свои команды
\DeclareMathOperator{\sgn}{\mathop{sgn}}
\DeclareMathOperator{\sign}{\mathop{sign}}
\DeclareMathOperator*{\res}{\mathop{res}}
\DeclareMathOperator*{\tr}{\mathop{tr}}

%% Перенос знаков в формулах (по Львовскому)
\newcommand*{\hm}[1]{#1\nobreak\discretionary{}
{\hbox{$\mathsurround=0pt #1$}}{}}

%%% Работа с картинками
\usepackage{graphicx}  % Для вставки рисунков
\graphicspath{{figures/}}  % папки с картинками
\setlength\fboxsep{3pt} % Отступ рамки \fbox{} от рисунка
\setlength\fboxrule{1pt} % Толщина линий рамки \fbox{}
\usepackage{wrapfig} % Обтекание рисунков текстом

%%% Работа с таблицами
\usepackage{array,tabularx,tabulary,booktabs} % Дополнительная работа с таблицами
\usepackage{longtable}  % Длинные таблицы
\usepackage{multirow} % Слияние строк в таблице

%%% Теоремы
\theoremstyle{plain} % Это стиль по умолчанию, его можно не переопределять.
\newtheorem{theorem}{Теорема}
\newtheorem*{thm}{Теорема}
\newtheorem{prop}{Утверждение}
 
\theoremstyle{definition} % "Определение"
%\newtheorem{corollary}{Следствие}[theorem]
\newtheorem*{dfn}{Определение}
\newtheorem{problem}{Задача}
\newtheorem*{problem*}{Задача}

 
\theoremstyle{remark} % "Примечание"
\newtheorem*{sol}{Решение}
\newtheorem*{rem}{Замечание}

%%% Программирование
\usepackage{etoolbox} % логические операторы

%%% Страница
%\usepackage{extsizes} % Возможность сделать 14-й шрифт
%\usepackage{geometry} % Простой способ задавать поля
%	\geometry{top=25mm}
%	\geometry{bottom=35mm}
%	\geometry{left=35mm}
%	\geometry{right=20mm}
 
\usepackage{fancyhdr} % Колонтитулы
%	\pagestyle{fancy}
 %	\renewcommand{\headrulewidth}{0pt}  % Толщина линейки, отчеркивающей верхний колонтитул
	%\lfoot{Нижний левый}
	%\rfoot{Нижний правый}
	%\rhead{Верхний правый}
	%\chead{Верхний в центре}
	%\lhead{Верхний левый}
	%\cfoot{Нижний в центре} % По умолчанию здесь номер страницы

\usepackage{setspace} % Интерлиньяж
%\onehalfspacing % Интерлиньяж 1.5
%\doublespacing % Интерлиньяж 2
%\singlespacing % Интерлиньяж 1

\usepackage{lastpage} % Узнать, сколько всего страниц в документе.

\usepackage{soul} % Модификаторы начертания

\usepackage{hyperref}
%\usepackage[usenames,dvipsnames,svgnames,table,rgb]{xcolor}
\hypersetup{				% Гиперссылки
    unicode=true,           % русские буквы в раздела PDF
    pdftitle={Заголовок},   % Заголовок
    pdfauthor={Автор},      % Автор
    pdfsubject={Тема},      % Тема
    pdfcreator={Создатель}, % Создатель
    pdfproducer={Производитель}, % Производитель
    pdfkeywords={keyword1} {key2} {key3}, % Ключевые слова
    colorlinks=true,       	% false: ссылки в рамках; true: цветные ссылки
    linkcolor=red,          % внутренние ссылки
    citecolor=black,        % на библиографию
    filecolor=magenta,      % на файлы
    urlcolor=cyan           % на URL
}

\usepackage{csquotes} % Еще инструменты для ссылок

%\usepackage[style=apa,maxcitenames=2,backend=biber,sorting=nty]{biblatex}

\usepackage{multicol} % Несколько колонок

\usepackage{tikz} % Работа с графикой
\usepackage{pgfplots}
\usepackage{pgfplotstable}
%\usepackage{coloremoji}
\usepackage{floatrow}
\usepackage{subcaption}
\newcommand*{\N}{\mathbb{N}}
\newcommand*{\R}{\mathbb{R}}
\newcommand*{\K}{\mathbb{K}}
\newcommand*{\V}{\mathcal{V}}
\newcommand*{\A}{\mathcal{A}}
\newcommand*{\ii}{\mathbf{1}}
\newcommand*{\oo}{\mathbf{0}}
\newcommand*{\ba}{\mathbf{a}}
\newcommand*{\bb}{\mathbf{b}}
\newcommand*{\Q}{\mathbb{Q}}
\graphicspath{{figures/}}
%\usepackage{breqn}

\renewcommand\thesubfigure{\asbuk{subfigure}}
%\addbibresource{master.bib}

\usepackage{import}
\usepackage{pdfpages}
\usepackage{transparent}
\usepackage{xcolor}
\usepackage{xifthen}

%\newcommand{\incfig}[1]{%
%    \def\svgwidth{\columnwidth}
%    \import{./figures/}{#1.pdf_tex}
%}


\newcommand{\incfig}[2][1]{%
    \def\svgwidth{#1\columnwidth}
    \import{./figures/}{#2.pdf_tex}
}
\usepackage{titlesec}
%\titleformat{\section}{\normalfont\Large\bfseries}{}{0pt}{}
%----------------------STANDART:
%\titleformat{\chapter}[display]
%  {\normalfont\huge\bfseries}{\chaptertitlename\ \thechapter}{20pt}{\Huge}
%\titleformat{\section}{\normalfont\Large\bfseries}{\thesection}{1em}{}
%\titleformat{\subsection}
%  {\normalfont\large\bfseries}{\thesubsection}{1em}{}
%\titleformat{\subsubsection}
%  {\normalfont\normalsize\bfseries}{\thesubsubsection}{1em}{}
%\titleformat{\paragraph}[runin]
%  {\normalfont\normalsize\bfseries}{\theparagraph}{1em}{}
%\titleformat{\subparagraph}[runin]
%  {\normalfont\normalsize\bfseries}{\thesubparagraph}{1em}{}

\pdfsuppresswarningpagegroup=1
\pgfplotsset{compat=1.16}

\usepackage{xifthen}
\makeatother
%\def\@lecture{}%
%\newcommand{\lecture}[3]{
%    \ifthenelse{\isempty{#3}}{%
%        \def\@lecture{Неделя #1}%
%    }{%
%        \def\@lecture{Неделя #1: #3}%
%    }%
%    \section*{\@lecture}
%    \marginpar{\small\textsf{\mbox{#2}}}
%}
\makeatletter

%\newcommand{\lec}{\subsection{Лекция}}
%\newcommand{\sem}{\subsection{Семинар}}
%\newcommand{\hw}{\subsection{Домашняя работа}}
%\newcommand{\prob}[1]{\textbf{#1}}
%\renewcommand{\thesubsection}{}
%\renewcommand{\thesubsubsection}{}

%\setcounter{tocdepth}{1} % only parts,chapters,sections
%\titleformat{\subsection}{\normalfont\large\bfseries}{}{0em}{}
%\titleformat{\subsubsection}{\normalfont\normalsize\bfseries}{}{0em}{}

%\newcommand{\textover}[2]{\stackrel{\mathclap{\normalfont\mbox{#2}}}{#1}}

\author{Драчов Ярослав\\
Факультет общей и прикладной физики МФТИ}
\newcommand{\veq}{\mathrel{\rotatebox{90}{$=$}}}
%\newcommand{\teto}[1]{\stackrel{\mathclap{\normalfont\tiny\mbox{#1}}}{\to}}
%\renewcommand{\thesubsection}{\arabic{subsection}}

%%\setcounter{secnumdepth}{0}

\definecolor{tabblue}{RGB}{30, 119, 180}
\definecolor{taborange}{RGB}{255, 127, 15}
\definecolor{tabgreen}{RGB}{45, 160, 43}
\definecolor{tabred}{RGB}{214, 38, 40}
\definecolor{tabpurple}{RGB}{148, 103, 189}
\definecolor{tabbrown}{RGB}{140, 86, 76}
\definecolor{tabpink}{RGB}{227, 119, 193}
\definecolor{tabgray}{RGB}{127, 127, 127}
\definecolor{tabolive}{RGB}{188, 189, 33}
\definecolor{tabcyan}{RGB}{22, 190, 207}
\pgfplotscreateplotcyclelist{colorbrewer-tab}{
{tabblue},
{taborange},
{tabgreen},
{tabred},
{tabpurple},
{tabbrown},
{tabpink},
{tabgray},
{tabolive},
{tabcyan},
}
\usepackage{csvsimple}
\usepackage{extarrows}
%\renewcommand{\labelenumii}{\asbuk{enumii})}
%\renewcommand{\labelenumiv}{\Asbuk{enumiv}}
\newcommand{\prob}[1]{\subsubsection*{#1}}
\sisetup{output-decimal-marker = {,},separate-uncertainty = true,exponent-product = \cdot}

\usepackage{braket}
\usepackage{enumerate}
\usepackage{chngcntr}
%\counterwithin*{equation}{problem}
%\usepackage{bbold}

\newtheoremstyle{hiProb}% ⟨name ⟩ 
{3pt}% ⟨Space above ⟩1 
{3pt}% ⟨Space below ⟩1
{}% ⟨Body font ⟩
{}% ⟨Indent amount ⟩2
{\bfseries}% ⟨Theorem head font⟩
{.}% ⟨Punctuation after theorem head ⟩
{.5em}% ⟨Space after theorem head ⟩3
%{\thmname{#1} \thmnote{#3}}% ⟨Theorem head spec (can be left empty, meaning ‘normal’)⟩
{\thmnote{#3}}% ⟨Theorem head spec (can be left empty, meaning ‘normal’)⟩
\theoremstyle{hiProb} % "Определение"
%\newtheorem{hiProb}{Задача}
\newtheorem{hiProb}{}
\usepackage{mmacells}
\newcommand{\textover}[2]{\stackrel{\mathclap{\normalfont\scriptsize\mbox{#2}}}{#1}}
\usepackage{units}
\usepackage[math]{cellspace}%
\setlength\cellspacetoplimit{2pt}
\setlength\cellspacebottomlimit{2pt}

\title{Неделя №5\\
Кинетические и электрические явления
в твёрдых телах и металлах}
\begin{document}
	\maketitle
\begin{hiProb}[2.65]
\end{hiProb}
\begin{sol}
Для фононов можно использовать оценочную формулу
для газовой кинетической теории 
\[
\kappa_{ph}= \frac{1}{3} C s\Lambda
.\] 
При низких температурах $C \propto T^3$.
Рассеяние фононов может происходить на других
фононах, краях образца, дефектах. Время
рассеяния на других фононах будет обратно
пропорционально их концентрации $(T^3)$. Следовательно,
данный процесс будет вымерзать при понижении температуры, что и происходит ниже 7 К, согласно условию
задачи. По-видимому также образцы содержат мало
дефектов, раз теплопроводность зависит от толщины
образца.

По аналогии с кнудсеновским течением газа длину
свободного пробега  $\Lambda$ надо принять
равной расстоянию между границами кристалла. Тогда
увеличение толщины кристалла в 4 раза увеличит
и $ \Lambda$ и $\kappa$ тоже в 4 раза.
\end{sol}
\begin{hiProb}[3.77]
\end{hiProb}
\begin{sol}
При $T > \Theta$ можно для каждого атома
применить теорему о равнораспределении энергии
по степеням свободы, считая, что каждый атом находится
в изотропном гармоническом потенциале, описываемом
<<жёсткостью>> $k$:
\[
\frac{3}{2} k_B T= k \frac{\overline{\xi^2}}{2}=
k \frac{\overline{x^2 +y^2+z^2}}{2}
.\] 
Коэффициент упругости <<отдельной пружинки>>
$k$ можно оценить, зная модуль Юнга. Действительно,
если мы растягиваем весь кристалл с относительным
удлинением $\epsilon $, то мы  просто растягиваем
$n_{at}=a^{-3}$ <<пружинок>> в единице объёма
на величину $a\epsilon $. Приравняем две
упругие энергии:
\[
	E \frac{\epsilon ^2}{2}=k n_{at} \frac{(a\epsilon )^2}{2}
.\] 
Отсюда $k=Ea$. Соответственно 
\[
\overline{\xi^2}= \frac{3k_B T}{Ea}
.\] 
Длина свободного пробега
\[
\Lambda = \frac{1}{n_{at}\sigma}= \frac{1}{n_{at}\pi
\overline{\xi^2}}= \frac{Ea^4}{3\pi k_B T}= 2\cdot 
10^{-6} \text{ см}
.\] 
Из этого решения следует, что
 \[
\rho\equiv \frac{m}{ne^2 \tau}=m \frac{v_F}{ne^2 \Lambda} \propto T
,\]
что наблюдается в большинстве металлов.

Ответ отличается от того, что в задачнике заменой
$n_e$ на $n_{at}$ и цифрой 3 в знаменателе,
возникающей из-за равнораспределения.
\end{sol}
\begin{hiProb}[3.79]
\end{hiProb}
\begin{sol}
Характерное время $t$, через которое пластинка
 <<почувствует>> изменение температуры, оценивается
как $d^2 /D$, где $D$ --- коэффициент диффузии.
Это время также известно, как время <<выравнивания>>.

Коэффициент диффузии есть отношение коэффициентов
теплопроводности кристалла $\kappa$ к его
теплоёмкости $C$. При комнатных температурах
теплоёмкость кристалла практически равна решёточной
$C_{\text{реш}}$ и согласно закону Дюлонга-Пти
$C=C_{\text{реш}}=3nk_\text{Б}T$, где $n$ ---
плотность атомов. Таким образом
\[
D= \frac{\kappa}{C_\text{реш}}=\frac{\kappa}{n k_\text{Б}}
.\] 

Коэффициент теплопроводности $\kappa$ для переноса
тепла в газе со средней
скоростью частиц $v$ и длиной свободного пробега
$\Lambda$ равен
\[
\kappa =\frac{1}{3} C_V \Lambda v
,\] 
где $C_V$ --- теплоёмкость единицы объёма газа.
При комнатных температурах в большинстве металлов
почти весь тепловой поток переносят электроны.
В применении к электронному газу в качестве $v$ 
разумно взять $v_F$, а 
\[
C_V=C_\text{эл}= \frac{\pi^2}{2} n k_\text{Б}^2
\frac{T}{\epsilon _F}= \frac{\pi^2 n k_\text{Б}^2
T}{m_e v_F} \Lambda
.\] 

Таким образом, коэффициент диффузии
\[
D= \frac{\kappa}{C_\text{реш}}= \frac{\pi^2}{9}
\frac{k_\text{Б} T \Lambda}{m_e v_F}
,\] 
откуда искомое время
\[
t \simeq \frac{d^2}{D} \approx
\frac{9}{\pi^2} \frac{d^2 m_e v_F}{k_\text{Б} T
\Lambda}\simeq 2\cdot 10^{-2} \text{ с}
,\] 
где $\displaystyle v_F \simeq \frac{\hbar }{m_e a} \left( 3\pi^2 \right) ^{1 /3}\sim 10^{8} \text{ см} / \text{с}$ 
(рассчитано для постоянной решётки $a \simeq
3$~\AA).
\end{sol}
\begin{hiProb}[3.80]
\end{hiProb}
\begin{sol}
В стержне можно считать что тепловое равновесие
поперёк устанавливается мгновенно по
сравнению с тепловым равновесием вдоль и перенос
тепла подчиняется одномерному уравнению теплопроводности:
\[
c \frac{\partial T}{\partial t} = -\kappa
\frac{\partial^2 t}{\partial x^2}
.\] 

Отсюда из размерности сразу понятно, какое будет
время установления:
\[
\tau \sim \frac{cL^2}{\kappa}
.\] 

Поскольку температура много больше дебаевской,
применима классическая теория теплоёмкости:
$c=3R \rho /\mu$ (здесь речь идёт о теплоёмкости
на единицу объёма, $R$ --- газовая постоянная,
$\mu=64$ г/моль --- молярная масса меди).

Имеем ответ
\[
\tau= 3R \frac{\rho L^2}{\kappa \mu}= 92 \text{ с}
.\] 
\end{sol}
\begin{hiProb}[3.88]
\end{hiProb}
\begin{sol}
	Квазиклассическое описание (импульс
	как квантовое число) и понятие длины
	свободного пробега применимы только 
	когда на длине свободного пробега
	укладывается несколько длин волн
	де-Бройля. Минимальная металлическая
	проводимость реализуется, когда
	длина свободного пробега равна $2\pi /k_F$.

	Подставляем в формулу Друде:
	\[
	\sigma_{\min}= \frac{ne^2 \tau}{m}=
	\frac{ne^2 \cdot 2\pi \hbar }{p_F v_F m}=
	\frac{ne^2 \cdot 2\pi}{\hbar  k_F^2}=
	\frac{2\pi ne^2}{\hbar \left( 3\pi
	^2 n\right) ^{2 /3}}=
	2n^{1 /3}(9\pi) ^{-1 /3}\frac{e^2}{\hbar }
	.\] 
	С точностью до численного множителя порядка
	1 это совпадает с ответом задачника
	и даёт $2,8\cdot 10^{-4}\text{ Ом}\cdot\text{см}$.
\end{sol}
\end{document}
