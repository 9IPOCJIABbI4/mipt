\documentclass[a4paper]{article}
% Этот шаблон документа разработан в 2014 году
% Данилом Фёдоровых (danil@fedorovykh.ru) 
% для использования в курсе 
% <<Документы и презентации в \LaTeX>>, записанном НИУ ВШЭ
% для Coursera.org: http://coursera.org/course/latex .
% Исходная версия шаблона --- 
% https://www.writelatex.com/coursera/latex/5.3

% В этом документе преамбула

\usepackage{siunitx}
%%% Работа с русским языком
\usepackage{cmap}					% поиск в PDF
\usepackage{mathtext} 				% русские буквы в формулах
\usepackage[T2A]{fontenc}			% кодировка
\usepackage[utf8]{inputenc}			% кодировка исходного текста
\usepackage[english,russian]{babel}	% локализация и переносы
\usepackage{indentfirst}
\frenchspacing

\renewcommand{\epsilon}{\ensuremath{\varepsilon}}
\renewcommand{\phi}{\ensuremath{\varphi}}
\renewcommand{\kappa}{\ensuremath{\varkappa}}
\renewcommand{\le}{\ensuremath{\leqslant}}
\renewcommand{\leq}{\ensuremath{\leqslant}}
\renewcommand{\ge}{\ensuremath{\geqslant}}
\renewcommand{\geq}{\ensuremath{\geqslant}}
\renewcommand{\emptyset}{\varnothing}
\renewcommand{\Im}{\operatorname{Im}}
\renewcommand{\Re}{\operatorname{Re}}


%%% Дополнительная работа с математикой
\usepackage{amsmath,amsfonts,amssymb,amsthm,mathtools} % AMS
\usepackage{icomma} % "Умная" запятая: $0,2$ --- число, $0, 2$ --- перечисление

%% Номера формул
%\mathtoolsset{showonlyrefs=true} % Показывать номера только у тех формул, на которые есть \eqref{} в тексте.
%\usepackage{leqno} % Нумереация формул слева

%% Свои команды
\DeclareMathOperator{\sgn}{\mathop{sgn}}
\DeclareMathOperator{\sign}{\mathop{sign}}
\DeclareMathOperator*{\res}{\mathop{res}}
\DeclareMathOperator*{\tr}{\mathop{tr}}

%% Перенос знаков в формулах (по Львовскому)
\newcommand*{\hm}[1]{#1\nobreak\discretionary{}
{\hbox{$\mathsurround=0pt #1$}}{}}

%%% Работа с картинками
\usepackage{graphicx}  % Для вставки рисунков
\graphicspath{{figures/}}  % папки с картинками
\setlength\fboxsep{3pt} % Отступ рамки \fbox{} от рисунка
\setlength\fboxrule{1pt} % Толщина линий рамки \fbox{}
\usepackage{wrapfig} % Обтекание рисунков текстом

%%% Работа с таблицами
\usepackage{array,tabularx,tabulary,booktabs} % Дополнительная работа с таблицами
\usepackage{longtable}  % Длинные таблицы
\usepackage{multirow} % Слияние строк в таблице

%%% Теоремы
\theoremstyle{plain} % Это стиль по умолчанию, его можно не переопределять.
\newtheorem{theorem}{Теорема}
\newtheorem*{thm}{Теорема}
\newtheorem{prop}{Утверждение}
 
\theoremstyle{definition} % "Определение"
%\newtheorem{corollary}{Следствие}[theorem]
\newtheorem*{dfn}{Определение}
\newtheorem{problem}{Задача}
\newtheorem*{problem*}{Задача}

 
\theoremstyle{remark} % "Примечание"
\newtheorem*{sol}{Решение}
\newtheorem*{rem}{Замечание}

%%% Программирование
\usepackage{etoolbox} % логические операторы

%%% Страница
%\usepackage{extsizes} % Возможность сделать 14-й шрифт
%\usepackage{geometry} % Простой способ задавать поля
%	\geometry{top=25mm}
%	\geometry{bottom=35mm}
%	\geometry{left=35mm}
%	\geometry{right=20mm}
 
\usepackage{fancyhdr} % Колонтитулы
%	\pagestyle{fancy}
 %	\renewcommand{\headrulewidth}{0pt}  % Толщина линейки, отчеркивающей верхний колонтитул
	%\lfoot{Нижний левый}
	%\rfoot{Нижний правый}
	%\rhead{Верхний правый}
	%\chead{Верхний в центре}
	%\lhead{Верхний левый}
	%\cfoot{Нижний в центре} % По умолчанию здесь номер страницы

\usepackage{setspace} % Интерлиньяж
%\onehalfspacing % Интерлиньяж 1.5
%\doublespacing % Интерлиньяж 2
%\singlespacing % Интерлиньяж 1

\usepackage{lastpage} % Узнать, сколько всего страниц в документе.

\usepackage{soul} % Модификаторы начертания

\usepackage{hyperref}
%\usepackage[usenames,dvipsnames,svgnames,table,rgb]{xcolor}
\hypersetup{				% Гиперссылки
    unicode=true,           % русские буквы в раздела PDF
    pdftitle={Заголовок},   % Заголовок
    pdfauthor={Автор},      % Автор
    pdfsubject={Тема},      % Тема
    pdfcreator={Создатель}, % Создатель
    pdfproducer={Производитель}, % Производитель
    pdfkeywords={keyword1} {key2} {key3}, % Ключевые слова
    colorlinks=true,       	% false: ссылки в рамках; true: цветные ссылки
    linkcolor=red,          % внутренние ссылки
    citecolor=black,        % на библиографию
    filecolor=magenta,      % на файлы
    urlcolor=cyan           % на URL
}

\usepackage{csquotes} % Еще инструменты для ссылок

%\usepackage[style=apa,maxcitenames=2,backend=biber,sorting=nty]{biblatex}

\usepackage{multicol} % Несколько колонок

\usepackage{tikz} % Работа с графикой
\usepackage{pgfplots}
\usepackage{pgfplotstable}
%\usepackage{coloremoji}
\usepackage{floatrow}
\usepackage{subcaption}
\newcommand*{\N}{\mathbb{N}}
\newcommand*{\R}{\mathbb{R}}
\newcommand*{\K}{\mathbb{K}}
\newcommand*{\V}{\mathcal{V}}
\newcommand*{\A}{\mathcal{A}}
\newcommand*{\ii}{\mathbf{1}}
\newcommand*{\oo}{\mathbf{0}}
\newcommand*{\ba}{\mathbf{a}}
\newcommand*{\bb}{\mathbf{b}}
\newcommand*{\Q}{\mathbb{Q}}
\graphicspath{{figures/}}
%\usepackage{breqn}

\renewcommand\thesubfigure{\asbuk{subfigure}}
%\addbibresource{master.bib}

\usepackage{import}
\usepackage{pdfpages}
\usepackage{transparent}
\usepackage{xcolor}
\usepackage{xifthen}

%\newcommand{\incfig}[1]{%
%    \def\svgwidth{\columnwidth}
%    \import{./figures/}{#1.pdf_tex}
%}


\newcommand{\incfig}[2][1]{%
    \def\svgwidth{#1\columnwidth}
    \import{./figures/}{#2.pdf_tex}
}
\usepackage{titlesec}
%\titleformat{\section}{\normalfont\Large\bfseries}{}{0pt}{}
%----------------------STANDART:
%\titleformat{\chapter}[display]
%  {\normalfont\huge\bfseries}{\chaptertitlename\ \thechapter}{20pt}{\Huge}
%\titleformat{\section}{\normalfont\Large\bfseries}{\thesection}{1em}{}
%\titleformat{\subsection}
%  {\normalfont\large\bfseries}{\thesubsection}{1em}{}
%\titleformat{\subsubsection}
%  {\normalfont\normalsize\bfseries}{\thesubsubsection}{1em}{}
%\titleformat{\paragraph}[runin]
%  {\normalfont\normalsize\bfseries}{\theparagraph}{1em}{}
%\titleformat{\subparagraph}[runin]
%  {\normalfont\normalsize\bfseries}{\thesubparagraph}{1em}{}

\pdfsuppresswarningpagegroup=1
\pgfplotsset{compat=1.16}

\usepackage{xifthen}
\makeatother
%\def\@lecture{}%
%\newcommand{\lecture}[3]{
%    \ifthenelse{\isempty{#3}}{%
%        \def\@lecture{Неделя #1}%
%    }{%
%        \def\@lecture{Неделя #1: #3}%
%    }%
%    \section*{\@lecture}
%    \marginpar{\small\textsf{\mbox{#2}}}
%}
\makeatletter

%\newcommand{\lec}{\subsection{Лекция}}
%\newcommand{\sem}{\subsection{Семинар}}
%\newcommand{\hw}{\subsection{Домашняя работа}}
%\newcommand{\prob}[1]{\textbf{#1}}
%\renewcommand{\thesubsection}{}
%\renewcommand{\thesubsubsection}{}

%\setcounter{tocdepth}{1} % only parts,chapters,sections
%\titleformat{\subsection}{\normalfont\large\bfseries}{}{0em}{}
%\titleformat{\subsubsection}{\normalfont\normalsize\bfseries}{}{0em}{}

%\newcommand{\textover}[2]{\stackrel{\mathclap{\normalfont\mbox{#2}}}{#1}}

\author{Драчов Ярослав\\
Факультет общей и прикладной физики МФТИ}
\newcommand{\veq}{\mathrel{\rotatebox{90}{$=$}}}
%\newcommand{\teto}[1]{\stackrel{\mathclap{\normalfont\tiny\mbox{#1}}}{\to}}
%\renewcommand{\thesubsection}{\arabic{subsection}}

%%\setcounter{secnumdepth}{0}

\definecolor{tabblue}{RGB}{30, 119, 180}
\definecolor{taborange}{RGB}{255, 127, 15}
\definecolor{tabgreen}{RGB}{45, 160, 43}
\definecolor{tabred}{RGB}{214, 38, 40}
\definecolor{tabpurple}{RGB}{148, 103, 189}
\definecolor{tabbrown}{RGB}{140, 86, 76}
\definecolor{tabpink}{RGB}{227, 119, 193}
\definecolor{tabgray}{RGB}{127, 127, 127}
\definecolor{tabolive}{RGB}{188, 189, 33}
\definecolor{tabcyan}{RGB}{22, 190, 207}
\pgfplotscreateplotcyclelist{colorbrewer-tab}{
{tabblue},
{taborange},
{tabgreen},
{tabred},
{tabpurple},
{tabbrown},
{tabpink},
{tabgray},
{tabolive},
{tabcyan},
}
\usepackage{csvsimple}
\usepackage{extarrows}
%\renewcommand{\labelenumii}{\asbuk{enumii})}
%\renewcommand{\labelenumiv}{\Asbuk{enumiv}}
\newcommand{\prob}[1]{\subsubsection*{#1}}
\sisetup{output-decimal-marker = {,},separate-uncertainty = true,exponent-product = \cdot}

\usepackage{braket}
\usepackage{enumerate}
\usepackage{chngcntr}
%\counterwithin*{equation}{problem}
%\usepackage{bbold}

\newtheoremstyle{hiProb}% ⟨name ⟩ 
{3pt}% ⟨Space above ⟩1 
{3pt}% ⟨Space below ⟩1
{}% ⟨Body font ⟩
{}% ⟨Indent amount ⟩2
{\bfseries}% ⟨Theorem head font⟩
{.}% ⟨Punctuation after theorem head ⟩
{.5em}% ⟨Space after theorem head ⟩3
%{\thmname{#1} \thmnote{#3}}% ⟨Theorem head spec (can be left empty, meaning ‘normal’)⟩
{\thmnote{#3}}% ⟨Theorem head spec (can be left empty, meaning ‘normal’)⟩
\theoremstyle{hiProb} % "Определение"
%\newtheorem{hiProb}{Задача}
\newtheorem{hiProb}{}
\usepackage{mmacells}
\newcommand{\textover}[2]{\stackrel{\mathclap{\normalfont\scriptsize\mbox{#2}}}{#1}}
\usepackage{units}
\usepackage[math]{cellspace}%
\setlength\cellspacetoplimit{2pt}
\setlength\cellspacebottomlimit{2pt}

\title{Неделя №10\\
Энергетические диаграммы для квазичастичного тока в контактах сверхпроводников. Эффект Джозефсона}
\begin{document}
	\maketitle
\begin{hiProb}[Т10-4]
\end{hiProb}
\begin{sol}
\begin{figure}[ht]
    \centering
    \incfig{1}
    \caption{}
    \label{fig:1}
\end{figure}
Ток возникнет при $eV=\Delta_0=1,76 k_B T_c$, поэтому
\[
V= \frac{1.76 k_B T_c}{e}= 0,18 \text{ мВ}
.\] 
\end{sol}
\begin{hiProb}[Т10-5]
\end{hiProb}
\begin{sol}
Предположим сначала, что $I_{c 1}< I_{c 2}$. Тогда первый контакт ---
это <<узкое место>> и бездиссипативный ток через систему во всяком
случае не может быть больше, чем $I_{c 1}$. Докажем, что
он может быть равен $I_{c 1}$. Чтобы это было так, требуется
$\phi_1 = \pi /2$, и условие равенства токов через два
контакта даёт $I_{c 1} = I_{c 2} \sin \phi_2$, поэтому такая
ситуация имеет место при
\[
	\cos \phi_c = \cos \left( \frac{\pi}{2}+ \phi_2 \right) 
	= - \sin \phi_2= - \frac{I_{c 1}}{I_{c 2}},\quad
	\left| \cos \phi_c \right| <1
.\]
Итак, при такой разности фаз достигается критический ток $I_{c 1}$ 
через систему.

Учитывая произвольное соотношение между $I_{c 1}$ и $I_{c 2}$,
записываем ответ следующим образом:
\[
	I_{c}= \min \left( I_{c 1},\, I_{c 2} \right) ,\quad
	\cos \phi_c= - \frac{\min \left( I_{c 1},\,I_{c 2} \right) }{
	\max \left( I_{c 1},\,I_{c 2} \right) }
.\] 
\end{sol}
\begin{hiProb}[Т10-6]
\end{hiProb}
\begin{sol}
Напряжение на джозефсоновском контакте при токе, большем критического,
связано со скоростью изменения фазы $\displaystyle \frac{d\phi}{dt}=
2 \frac{eV}{\hbar }$, при этом часть тока течёт по резистивному
каналу, и в режиме постоянного заданного тока $I=I_c \sin \phi+ V /R=
 I_c \sin \phi + \frac{\hbar }{2 e R} \frac{d \phi}{dt}$.
Решением данного дифференциального уравнения будет
\[
	\phi(t)= 2 \arctg  \left[ 
	\frac{I_c}{I} - \frac{\sqrt{I^2- I_c^2} }{I}\tg \left( 
\frac{C_1 -t}{2}\sqrt{ \left( \frac{2eR}{\hbar } \right) ^2 \left( I^2-
I_c^2\right) } \right) \right] 
.\] 
Выбор константы произволен --- это начало отсчёта времени, полагаем
$C_1=0$.
Далее, для интересующего нас напряжения
\[
V= \frac{\hbar }{2e} \phi'= \frac{\hbar }{2e}
\frac{2}{1 + \left[ \frac{I_c}{I}+ \frac{\sqrt{I^2 -I_c^2} }{I}
\tg  \left( \frac{\omega t}{2} \right) \right] ^2} \frac{\sqrt{
I^2-I_c^2} }{I} \frac{1}{\cos  ^2 \left( \frac{\omega t}{2} \right) 
} \frac{\omega}{2}
,\] 
где
\[
\omega = \frac{2eR}{\hbar } I \sqrt{I^2 -I_c^2} 
.\] 
Раскрывая квадрат в знаменателе и пользуясь формулами  двойных
углов и основным тригонометрическим тождеством:
\begin{multline*}
	V= \frac{R(I^2 -I_c^2)}{I \cos ^2 \frac{\omega t}{2} + \frac{I_c^2}{I}\cos ^2 \frac{\omega t}{2} +
	\frac{I_c \sqrt{I^2 -I_c^2} }{I}\sin  \omega t+
\frac{I^2-I_c^2}{I}\sin ^2 \frac{\omega t}{2}}=\\=
\frac{R (I^2- I_c^2)}{I+ \frac{I_c^2}{I}\cos  \omega t+
\frac{I_c \sqrt{I^2-I_c^2} }{I}\sin \omega t}
.\end{multline*}
С использованием формулы для синуса суммы это преобразуется в
компактный вид
\[
	V=\frac{R \left( I^2- I_c^2 \right) }{I+
	I_c \sin \left( \omega t +\xi \right) }
,\] 
фаза $\xi$ зависит от токов, но постоянна в условиях этого 
опыта и опять равносильна выбору момента нуля отсчёта времени.
Это приводит к ответу:
\[
	V(t)= R \frac{I^2 -I_c^2}{I+I_c \cos \omega t}
.\]
При вычислении среднего
\[
	\overline{V}=R I_c \overline{\frac{(I /I_c)^2-1}{(I /I_c)+
	\cos \omega t}}= RI_c \int\limits_{0}^{2\pi} 
	\frac{(I /I_c)^2 -1}{I/I_c +\cos  x}dx = R I_c
	\frac{1}{\pi} 
	\int\limits_{0}^{\pi} \frac{(I /I_c)^2 -1}{I /I_c + \cos x}
	dx
.\] 
Интеграл табличный, получаем
\[
	\overline{V}= R I_c \frac{1}{2\pi} \frac{2\left( 
	(I /I_c)^2-1\right) }{\sqrt{(I /I_c)^2 -1} }\pi=
	\frac{\hbar  \omega}{2e}
.\] 
\end{sol}
\begin{hiProb}[Т10-7]
\end{hiProb}
\begin{sol}
Через <<рукава >> СКВИДа текут токи $I_a=I_c \sin \phi_a$,
$I_b= I_c \sin \phi_b$. Это можно представить,
как сумму текущего симметрично по <<рукавам>> транспортного
тока $I_{tr }= (I_a +I_b) /2$ и кольцевого тока $I_{loop}= (I_a+I_b) /2$. Этот кольцевой ток приводит к частичному экранированию
 внешнего магнитного потока. Скачки фаз на джозефсоновских
контактах будут определяться полным потоком.

Система уравнений СКВИДа при заданном полном токе $I$ принимает вид
\[
\left\{
\begin{aligned}
\phi_a- \phi_b &= 2\pi \frac{\Phi}{\Phi_0} ,\\
\Phi&=\Phi_{ext}-L I_{loop}= \Phi_{ext}-L(I_a-I_b) /2 ,\\
I&=2I_{t r}= I_a+I_b .\\
\end{aligned}
\right.
\] 
Здесь может возникнуть вопрос о выборе знака во втором уравнении,
как мы увидим далее знак в ответ не входит, поэтому этот вопрос
не важен. Вводим переменные $\Delta \phi= \phi_a- \phi_b$,
$\phi_0= \left( \phi_a+ \phi_b \right) /2$ и подставляем в первое
уравнение выражение для полного потока:
\[
\Delta \phi= 2\pi \frac{\Phi_{ext}}{\Phi_0}-\pi
\frac{LI}{\Phi_0}\left( \sin \phi_a-\sin \phi_b \right) =
2\pi \frac{\Phi _{ext}}{\Phi_0}- \pi \beta \cos  \phi_0
\sin  \frac{\Delta \phi}{2}
.\] 
\[
	I=I_c (\sin \phi_a+ \sin \phi_b) 2I_c \sin 
	\phi_0 \cos  \frac{\Delta \phi}{2}
.\] 
Первое из уравнений определяет разность скачков фаз в зависимости
от внешнего потока. Максимизируя второе по $\phi_0$, получим
максимальный бездиссипативный ток.

Для указанных в условии потоков:
\begin{enumerate}
	\item $\Phi_{ext}=0$ задаёт $\Delta \phi =0$,
		откуда максимальный бездиссипативный ток
		$I_{\max}= 2I_c$ 
	\item $\Phi_{ext}= \frac{\Phi_0}{2}$ в отсутствие 
		индуктивности давало бы $\Delta \phi=\pi$, первая
		малая поправка по $\beta$ очевидно будет
$\Delta \phi= \pi\beta \cos \phi_0$. Тогда в выражении для
полного бездиссипативноого тока
\[
\cos  \frac{\Delta \phi}{2}\approx
\cos  \left( \frac{\pi}{2} - \frac{\pi \beta}{2}
\cos \phi_0 \right) = \sin 
\left( \frac{\pi\beta}{2} \cos \phi_0 \right) \approx
\frac{\pi \beta}{2} \cos  \phi_0
\] 
и для полного тока имеем $I=I_c \sin 2\phi_0 \frac{\pi \beta}{2}$.
Соответственно, $I_{\max} (\Phi_0 /2)= \frac{\pi\beta}{2}I_c$.
Ответ бы не изменился при другом выборе знака в поправке к полному
потоку, так как знак можно изменить подбором фазы
$\phi_0$. 
\end{enumerate}
Для отношения максимальных бездиссипативных токов получаем
\[
	\frac{I_{\max}(\Phi_0 /2)}{I_{\max}(0)}= \frac{\pi \beta}{2}
	\approx 0,15
.\] 
\end{sol}
\end{document}
