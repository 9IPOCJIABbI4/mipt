\documentclass[a4paper]{article}
% Этот шаблон документа разработан в 2014 году
% Данилом Фёдоровых (danil@fedorovykh.ru) 
% для использования в курсе 
% <<Документы и презентации в \LaTeX>>, записанном НИУ ВШЭ
% для Coursera.org: http://coursera.org/course/latex .
% Исходная версия шаблона --- 
% https://www.writelatex.com/coursera/latex/5.3

% В этом документе преамбула

\usepackage{siunitx}
%%% Работа с русским языком
\usepackage{cmap}					% поиск в PDF
\usepackage{mathtext} 				% русские буквы в формулах
\usepackage[T2A]{fontenc}			% кодировка
\usepackage[utf8]{inputenc}			% кодировка исходного текста
\usepackage[english,russian]{babel}	% локализация и переносы
\usepackage{indentfirst}
\frenchspacing

\renewcommand{\epsilon}{\ensuremath{\varepsilon}}
\renewcommand{\phi}{\ensuremath{\varphi}}
\renewcommand{\kappa}{\ensuremath{\varkappa}}
\renewcommand{\le}{\ensuremath{\leqslant}}
\renewcommand{\leq}{\ensuremath{\leqslant}}
\renewcommand{\ge}{\ensuremath{\geqslant}}
\renewcommand{\geq}{\ensuremath{\geqslant}}
\renewcommand{\emptyset}{\varnothing}
\renewcommand{\Im}{\operatorname{Im}}
\renewcommand{\Re}{\operatorname{Re}}


%%% Дополнительная работа с математикой
\usepackage{amsmath,amsfonts,amssymb,amsthm,mathtools} % AMS
\usepackage{icomma} % "Умная" запятая: $0,2$ --- число, $0, 2$ --- перечисление

%% Номера формул
%\mathtoolsset{showonlyrefs=true} % Показывать номера только у тех формул, на которые есть \eqref{} в тексте.
%\usepackage{leqno} % Нумереация формул слева

%% Свои команды
\DeclareMathOperator{\sgn}{\mathop{sgn}}
\DeclareMathOperator{\sign}{\mathop{sign}}
\DeclareMathOperator*{\res}{\mathop{res}}
\DeclareMathOperator*{\tr}{\mathop{tr}}

%% Перенос знаков в формулах (по Львовскому)
\newcommand*{\hm}[1]{#1\nobreak\discretionary{}
{\hbox{$\mathsurround=0pt #1$}}{}}

%%% Работа с картинками
\usepackage{graphicx}  % Для вставки рисунков
\graphicspath{{figures/}}  % папки с картинками
\setlength\fboxsep{3pt} % Отступ рамки \fbox{} от рисунка
\setlength\fboxrule{1pt} % Толщина линий рамки \fbox{}
\usepackage{wrapfig} % Обтекание рисунков текстом

%%% Работа с таблицами
\usepackage{array,tabularx,tabulary,booktabs} % Дополнительная работа с таблицами
\usepackage{longtable}  % Длинные таблицы
\usepackage{multirow} % Слияние строк в таблице

%%% Теоремы
\theoremstyle{plain} % Это стиль по умолчанию, его можно не переопределять.
\newtheorem{theorem}{Теорема}
\newtheorem*{thm}{Теорема}
\newtheorem{prop}{Утверждение}
 
\theoremstyle{definition} % "Определение"
%\newtheorem{corollary}{Следствие}[theorem]
\newtheorem*{dfn}{Определение}
\newtheorem{problem}{Задача}
\newtheorem*{problem*}{Задача}

 
\theoremstyle{remark} % "Примечание"
\newtheorem*{sol}{Решение}
\newtheorem*{rem}{Замечание}

%%% Программирование
\usepackage{etoolbox} % логические операторы

%%% Страница
%\usepackage{extsizes} % Возможность сделать 14-й шрифт
%\usepackage{geometry} % Простой способ задавать поля
%	\geometry{top=25mm}
%	\geometry{bottom=35mm}
%	\geometry{left=35mm}
%	\geometry{right=20mm}
 
\usepackage{fancyhdr} % Колонтитулы
%	\pagestyle{fancy}
 %	\renewcommand{\headrulewidth}{0pt}  % Толщина линейки, отчеркивающей верхний колонтитул
	%\lfoot{Нижний левый}
	%\rfoot{Нижний правый}
	%\rhead{Верхний правый}
	%\chead{Верхний в центре}
	%\lhead{Верхний левый}
	%\cfoot{Нижний в центре} % По умолчанию здесь номер страницы

\usepackage{setspace} % Интерлиньяж
%\onehalfspacing % Интерлиньяж 1.5
%\doublespacing % Интерлиньяж 2
%\singlespacing % Интерлиньяж 1

\usepackage{lastpage} % Узнать, сколько всего страниц в документе.

\usepackage{soul} % Модификаторы начертания

\usepackage{hyperref}
%\usepackage[usenames,dvipsnames,svgnames,table,rgb]{xcolor}
\hypersetup{				% Гиперссылки
    unicode=true,           % русские буквы в раздела PDF
    pdftitle={Заголовок},   % Заголовок
    pdfauthor={Автор},      % Автор
    pdfsubject={Тема},      % Тема
    pdfcreator={Создатель}, % Создатель
    pdfproducer={Производитель}, % Производитель
    pdfkeywords={keyword1} {key2} {key3}, % Ключевые слова
    colorlinks=true,       	% false: ссылки в рамках; true: цветные ссылки
    linkcolor=red,          % внутренние ссылки
    citecolor=black,        % на библиографию
    filecolor=magenta,      % на файлы
    urlcolor=cyan           % на URL
}

\usepackage{csquotes} % Еще инструменты для ссылок

%\usepackage[style=apa,maxcitenames=2,backend=biber,sorting=nty]{biblatex}

\usepackage{multicol} % Несколько колонок

\usepackage{tikz} % Работа с графикой
\usepackage{pgfplots}
\usepackage{pgfplotstable}
%\usepackage{coloremoji}
\usepackage{floatrow}
\usepackage{subcaption}
\newcommand*{\N}{\mathbb{N}}
\newcommand*{\R}{\mathbb{R}}
\newcommand*{\K}{\mathbb{K}}
\newcommand*{\V}{\mathcal{V}}
\newcommand*{\A}{\mathcal{A}}
\newcommand*{\ii}{\mathbf{1}}
\newcommand*{\oo}{\mathbf{0}}
\newcommand*{\ba}{\mathbf{a}}
\newcommand*{\bb}{\mathbf{b}}
\newcommand*{\Q}{\mathbb{Q}}
\graphicspath{{figures/}}
%\usepackage{breqn}

\renewcommand\thesubfigure{\asbuk{subfigure}}
%\addbibresource{master.bib}

\usepackage{import}
\usepackage{pdfpages}
\usepackage{transparent}
\usepackage{xcolor}
\usepackage{xifthen}

%\newcommand{\incfig}[1]{%
%    \def\svgwidth{\columnwidth}
%    \import{./figures/}{#1.pdf_tex}
%}


\newcommand{\incfig}[2][1]{%
    \def\svgwidth{#1\columnwidth}
    \import{./figures/}{#2.pdf_tex}
}
\usepackage{titlesec}
%\titleformat{\section}{\normalfont\Large\bfseries}{}{0pt}{}
%----------------------STANDART:
%\titleformat{\chapter}[display]
%  {\normalfont\huge\bfseries}{\chaptertitlename\ \thechapter}{20pt}{\Huge}
%\titleformat{\section}{\normalfont\Large\bfseries}{\thesection}{1em}{}
%\titleformat{\subsection}
%  {\normalfont\large\bfseries}{\thesubsection}{1em}{}
%\titleformat{\subsubsection}
%  {\normalfont\normalsize\bfseries}{\thesubsubsection}{1em}{}
%\titleformat{\paragraph}[runin]
%  {\normalfont\normalsize\bfseries}{\theparagraph}{1em}{}
%\titleformat{\subparagraph}[runin]
%  {\normalfont\normalsize\bfseries}{\thesubparagraph}{1em}{}

\pdfsuppresswarningpagegroup=1
\pgfplotsset{compat=1.16}

\usepackage{xifthen}
\makeatother
%\def\@lecture{}%
%\newcommand{\lecture}[3]{
%    \ifthenelse{\isempty{#3}}{%
%        \def\@lecture{Неделя #1}%
%    }{%
%        \def\@lecture{Неделя #1: #3}%
%    }%
%    \section*{\@lecture}
%    \marginpar{\small\textsf{\mbox{#2}}}
%}
\makeatletter

%\newcommand{\lec}{\subsection{Лекция}}
%\newcommand{\sem}{\subsection{Семинар}}
%\newcommand{\hw}{\subsection{Домашняя работа}}
%\newcommand{\prob}[1]{\textbf{#1}}
%\renewcommand{\thesubsection}{}
%\renewcommand{\thesubsubsection}{}

%\setcounter{tocdepth}{1} % only parts,chapters,sections
%\titleformat{\subsection}{\normalfont\large\bfseries}{}{0em}{}
%\titleformat{\subsubsection}{\normalfont\normalsize\bfseries}{}{0em}{}

%\newcommand{\textover}[2]{\stackrel{\mathclap{\normalfont\mbox{#2}}}{#1}}

\author{Драчов Ярослав\\
Факультет общей и прикладной физики МФТИ}
\newcommand{\veq}{\mathrel{\rotatebox{90}{$=$}}}
%\newcommand{\teto}[1]{\stackrel{\mathclap{\normalfont\tiny\mbox{#1}}}{\to}}
%\renewcommand{\thesubsection}{\arabic{subsection}}

%%\setcounter{secnumdepth}{0}

\definecolor{tabblue}{RGB}{30, 119, 180}
\definecolor{taborange}{RGB}{255, 127, 15}
\definecolor{tabgreen}{RGB}{45, 160, 43}
\definecolor{tabred}{RGB}{214, 38, 40}
\definecolor{tabpurple}{RGB}{148, 103, 189}
\definecolor{tabbrown}{RGB}{140, 86, 76}
\definecolor{tabpink}{RGB}{227, 119, 193}
\definecolor{tabgray}{RGB}{127, 127, 127}
\definecolor{tabolive}{RGB}{188, 189, 33}
\definecolor{tabcyan}{RGB}{22, 190, 207}
\pgfplotscreateplotcyclelist{colorbrewer-tab}{
{tabblue},
{taborange},
{tabgreen},
{tabred},
{tabpurple},
{tabbrown},
{tabpink},
{tabgray},
{tabolive},
{tabcyan},
}
\usepackage{csvsimple}
\usepackage{extarrows}
%\renewcommand{\labelenumii}{\asbuk{enumii})}
%\renewcommand{\labelenumiv}{\Asbuk{enumiv}}
\newcommand{\prob}[1]{\subsubsection*{#1}}
\sisetup{output-decimal-marker = {,},separate-uncertainty = true,exponent-product = \cdot}

\usepackage{braket}
\usepackage{enumerate}
\usepackage{chngcntr}
%\counterwithin*{equation}{problem}
%\usepackage{bbold}

\newtheoremstyle{hiProb}% ⟨name ⟩ 
{3pt}% ⟨Space above ⟩1 
{3pt}% ⟨Space below ⟩1
{}% ⟨Body font ⟩
{}% ⟨Indent amount ⟩2
{\bfseries}% ⟨Theorem head font⟩
{.}% ⟨Punctuation after theorem head ⟩
{.5em}% ⟨Space after theorem head ⟩3
%{\thmname{#1} \thmnote{#3}}% ⟨Theorem head spec (can be left empty, meaning ‘normal’)⟩
{\thmnote{#3}}% ⟨Theorem head spec (can be left empty, meaning ‘normal’)⟩
\theoremstyle{hiProb} % "Определение"
%\newtheorem{hiProb}{Задача}
\newtheorem{hiProb}{}
\usepackage{mmacells}
\newcommand{\textover}[2]{\stackrel{\mathclap{\normalfont\scriptsize\mbox{#2}}}{#1}}
\usepackage{units}
\usepackage[math]{cellspace}%
\setlength\cellspacetoplimit{2pt}
\setlength\cellspacebottomlimit{2pt}

\title{Неделя №4\\
Зонная структура}
\begin{document}
	\maketitle
\begin{hiProb}[3.38]
\end{hiProb}
\begin{sol}
По определению групповой скорости
\[
V=\frac{\partial E}{\partial p} 
.\] 
\[
	V_y= \frac{\epsilon_0 a}{\hbar } \sin \left( \frac{p_y a}{\hbar } \right) =0=V_z,\quad
	V_x= \frac{\epsilon_0 a}{\hbar }\sin
	\frac{5\pi}{6}= \frac{\epsilon_0 a}{2\hbar }
.\] 
Ускорение
\[
\frac{dV}{dt}= \frac{d}{dt} \frac{\partial E}{\partial p} =
\lim_{\Delta t \to 0} \frac{\left. \frac{\partial E}{\partial p}  \right|_{p=p+ \frac{dp}{dt}\Delta t}-
	\left. \frac{\partial E}{\partial p}  \right|_{p=p}}{\Delta t}
.\] 
По определению
\[
\frac{dp_x}{dt}= \frac{dp_z}{dt}=0,\quad
\frac{dp_y}{dt}= \frac{He}{c} \frac{\epsilon_0 a}{
2\hbar }
.\] 
\[
\frac{dV_y}{dt}=\frac{He}{c} \frac{\epsilon_0^2 a^3}{2\hbar ^3}
,\]
соответственно
\[
\frac{dV_x}{dt}= \frac{dV_z}{dt}=0
.\] 
\end{sol}
\begin{hiProb}[3.85]
\end{hiProb}
\begin{sol}
Ответом в задаче будет
\[
E= \frac{2\pi \hbar  c}{\lambda_0}+E_F=E_F+4,61 \text{ эВ}
,\] 
так как красной границе фотоэффекта соответствует
<<выбивание>> электрона с уровня Ферми.

ГЦК-решётка означает четыре атома на элементарный
куб. У серебра один электрон проводимости на атом.
Это значит, что на элементарный куб приходится
четыре электрона. Концентрация электронов $n=4 /a^3$,
для импульса Ферми пользуемся известным выражением
\[
	k_F= \sqrt[3]{3\pi^2 n} = \frac{\sqrt[3]{12 \pi^2} }{a}
.\]
Откуда имеем
\[
p_F= \hbar  k_F=1,26 \cdot  10^{-24} \frac{\text{кг}\cdot \text{м}}{\text{с}}
\]
и соответственно
\[
E_F= \frac{p_F^2}{2m}=5,44 \text{ эВ}
.\] 
Добавляя расстояние от уровня Ферми до уровня свободного электрона 4,61 эВ, получим ответ $E=10,05$ эВ.
\end{sol}
\begin{hiProb}[3.57]
\end{hiProb}
\begin{sol}
Во втором металле закон дисперсии в той же системе
координат имеет вид:
\[
E_{\operatorname{II}}= \frac{p_x^2}{2m_y^*}+
\frac{p_y^2}{2m_x^*}+ \frac{p_z^2}{2m_z^*}
\] 
(из-за поворота локальных осей кристалла движение
вдоль оси $X$ теперь характеризуется эффективной
массой $m_y$ и наоборот).

Перейдём к вычислению закона преломления. Пусть в
области I был электрон с импульсом  $(p_x,\,p_y,\,0)$,
после прохождения границы в области II он
перейдёт в состояние с импульсом  $(p_{x \mathrm{I}},\,
p_{y \mathrm{I}},\,0)$. Поскольку испускания или
поглощения фононов и других неупругих процессов нет,
энергия сохраняется, то есть:
\[
\frac{p_x^2}{m^*_x}+ \frac{p_y^2}{m^*_y}=
\frac{p_{x\mathrm{I}}^2}{m_y^*}+
\frac{p_{y\mathrm{I}}^2}{m_x^*}
.\] 
Другая сохраняющаяся величина это $p_x=p_{x \mathrm{I}}$. Данный закон сохранения квазиимпульса связан с
тем, что система трансляционно ивнвариантна вдоль
направлений $Ox$ и $Oz$.
Соответственно
\[
	p_{y\mathrm{I}}=\sqrt{
	p_x^2 \left( 1- \frac{m_x^*}{m_y^*} \right) +
p_y^2 \frac{m_x^*}{m_y^*}} 
.\] 
Данные выражения для импульсов решают задачу.

Найдём закон преломления. Пусть электрон падал
под углом $\alpha$ (отсчитываем от нормали к поверхности раздела, то есть от оси $Y$),  $\tg \alpha =
 p_x /p_y$, а вылетел под углом $\beta$, 
 $\tg \beta =p_x /p_{y\mathrm{I}}$. Тогда,
выражая всё через $p_x$ и подставляя в закон
сохранения энергии, имеем:
\[
\frac{1}{m^*_x}+ \frac{\ctg^2 \alpha}{m^*_y}=
\frac{1}{m^*_y}+\frac{\ctg^2 \beta}{m^*_{x}}
.\] 
\[
\frac{\ctg^2 \alpha -1}{m^*_{y}}= \frac{\ctg ^2 \beta-1}{m^*_x}
.\] 
\[
\frac{1}{m^*_y} \frac{\cos 2\alpha}{\sin^2 \alpha}=
\frac{1}{m^*_x} \frac{\cos 2 \beta}{\sin^2 \beta }
.\] 
Откуда
\[
\frac{\sin \alpha}{\sin \beta}\sqrt{\frac{\cos 2\beta}{\cos 2\alpha}} =\sqrt{\frac{m_x^*}{m_y^*}} 
.\] 
Асимптотика при малых углах
\[
	\frac{\alpha}{\beta}= \sqrt{\frac{m^*_x}{
	m^*_y}} 
.\] 
\end{sol}
\begin{hiProb}[4.54]
\end{hiProb}
\begin{sol}
Основное свойство кристаллов --- их трансляционная
симметрия. Поэтому для описания кристалла можно
воспользоваться следующим приёмом: взять кристалл
конечных размеров и транслировать его для
получения кристаллов произвольных размеров. При
этом размер этого кристалла (т.\:н. основная
область) должен быть выбран так, чтобы сохранилась
трансляционная симметрия, т.\:е. на основных
направлениях трансляции укладывалось целое
число элементарных (примитивных) ячеек. При
этом набег фазы волновой функции электрона вдоль
соответствующего направления будет кратен
$2\pi$. Это приводит к дискретным значениям проекций
квазиимпульса в зоне Бриллюэна, определяемым
размерами основной области кристалла. Поскольку
размер зоны Бриллюэна однозначно определяется
размеров примитивной ячейки, число разрешённых
значений квазиимпульса в зоне Бриллюэна равно
числу примитивных ячеек. Если кристалл графена
состоит из $N$ атомов, а в примитивной ячейке
графена содержится два атома, то число разрешённых
значений будет равно $N /2$. Каждому значению
квазиимпульса, как следует из закона дисперсии,
соответствует два уровня энергии: один в валентной
зоне, другой --- в зоне проводимости. Поэтому число
мест для электронов с учётом спинового вырождения
будет равно $2\cdot 2\cdot N /2 =2N$. Из них
$N$ мест будет в валентной зоне и $N$ --- в зоне
проводимости. Т.\:к. на один атом приходится по
одному <<свободному>> электрону, то всего делокализованных
электронов будет $N$. При нулевой температуре электроны
занимают наинизшие энергетические состояния, поэтому
вследствие симметричного характера спектра
вся валентная зона будет заполнена, а зона проводимости
будет пустой. Таким образом для указанного вида
спектра химический потенциал графена, отсчитываемый
от уровня энергии электрона в атоме, равен нулю
и уравнение для определения формы ферми-поверхности
имеет вид
\[
1+4 \cos p_x a \cos \frac{p_x a}{\sqrt{3} }+
4 \cos ^2 \frac{p_y a}{\sqrt{3} }=0
.\] 
Т.\:к. $\displaystyle \cos \frac{p_y a}{\sqrt{3} }
 \neq 0$, то уравнение можно переписать так:
\[
-4 \cos  p_x a = 4 \cos \frac{p_y a}{\sqrt{3} }+
\frac{1}{\cos \frac{p_y a}{\sqrt{3} } }
.\] 
Согласно неравенству о среднем арифметическом и
среднем геометрическом, правая часть уравнения
$\ge 4$ при $\displaystyle \cos \frac{p_y a}{\sqrt{3} }>0$, либо $\le -4$ при $\displaystyle \cos \frac{p_y
a}{\sqrt{3} }<0$.

Поскольку левая часть уравнения ограничена $-4 \le 
4 \cos p_x a\le 4$, то равенство возможно
только если одновременно
\[
\left\{
\begin{aligned}
\cos p_x a &= 1 \\
\cos \frac{p_y a}{\sqrt{3} }&= -\frac{1}{2} \\
\end{aligned}
\right.
\text{ или }
\left\{
\begin{aligned}
\cos p_x a&= -1 \\
\cos \frac{p_y a}{\sqrt{3} }=\frac{1}{2},
\end{aligned}
\right.
\]
откуда
\[
\left\{
\begin{aligned}
p_x a&= 2 \pi m \\
\frac{p_y a}{\sqrt{3} }&= \pm \frac{2}{3}
\pi+2 \pi s\\
\end{aligned}
\right.
\text{ или }
\left\{
\begin{aligned}
	p_x a&= \pi +2\pi n\\
	\frac{p_y a}{\sqrt{3} }&=  \pm \frac{1}{3}
	\pi +2 \pi l.\\
\end{aligned}
\right.
\] 

Подставляя данные, получаем, что
\[
\left\{
\begin{aligned}
p_x &= \frac{4\pi \hbar }{3b}m \\
p_y &= \pm \frac{4\pi \hbar }{3 \sqrt{3} b}+
\frac{4\pi \hbar }{\sqrt{3} b}s\\
\end{aligned}
\right.
\text{ или}
\left\{
\begin{aligned}
	p_x&= \frac{2\pi \hbar }{3b}+ \frac{4\pi \hbar }{3b}
m\\
	p_y &= \pm  \frac{2\pi \hbar }{3\sqrt{3} b}+
\frac{4\pi \hbar }{\sqrt{3} b}l.
\end{aligned}
\right.
\] 

Отбирая только те значения квазиимпульса, которые
попадают в первую зону Бриллюэна, получаем уравнение
прямых линий в $\mathbf{p}$-пространстве:
 \[
\left\{
\begin{aligned}
p_x&= 0 \\
p_y &= \pm \frac{4\pi \hbar }{3\sqrt{3} b} \\
\end{aligned}
\right.
\text{ или }
\left\{
\begin{aligned}
p_x &= \pm  \frac{2\pi \hbar }{3b} \\
p_y &= \pm \frac{2\pi \hbar }{3\sqrt{3} b}. \\
\end{aligned}
\right.
\] 

Пересечение указанных линий даёт вершины шестиугольника.
Таким образом, поверхность Ферми графена состоит
из шести точек и её площадь равна нулю.

Как видно из закона дисперсии, для любого направления
в обратном пространстве кроме диагоналей шестиугольника, на границе зоны Бриллюэна возникает щель 
(запрещённая зона) в спектре электронов. Она
максимальна в середине сторон шестиугольника и
уменьшается в стороны вершин, где исчезает вовсе
(зона проводимости соприкасается с валентной
зоной). С точки зрения своих электронных свойств,
графен можно считать либо бесщелевым полупроводником,
либо полуметаллом с нулевым перекрытием зон.
\end{sol}
\begin{hiProb}[Т4-2]
\end{hiProb}
\begin{sol}
\begin{figure}[ht]
    \centering
    \incfig{6}
    \caption{Схема заполнения электронных состояний до (слева)
    и после (справа) димеризации цепочек}
    \label{fig:6}
\end{figure}
В исходной цепочке один электрон на примитивную ячейку и первая
зона Бриллюэна одномерной цепочки оказывается заполнена ровно на половину --- поэтому это и будет проводником (металлом). После димеризации объём, занимаемый
электронами в $k$-пространстве не меняется, а первая
зона Бриллюэна уменьшается вдвое. Изменение
периодичности кристалла приводит к появлению
дополнительного вклада в создаваемый им потенциал
с периодом $2a$, этот вклад мал в силу малости смещения ионов и может быть учтён в рамках приближения слабой
связи.

Точный расчёт задачей не требуется, но результат
применения приближения слабой связи очевиден ---
на границе новой зоны Бриллюэна образуется
разрыв спектра и запрещённая зона. При этом
вблизи границы энергия электронов из нижней
ветви станет ещё немного ниже. В результате
суммарная кинетическая энергия электронов понизится.
Таким образом, в этом фазовом переходе выигрыш
в энергии возникает за счёт уменьшения
суммарной кинетической энергии электронов и
оказывается, что в одномерном случае
он всегда больше проигрыша в упругой энергии при
деформации (проигрыш $\propto \delta^2$, выигрыш
$\propto \delta^2 \ln \delta$).

После димеризации нижняя энергетическая зона
окажется заполнена полностью и система станет
диэлектриком.
\end{sol}
\begin{hiProb}[Т4-3]
\end{hiProb}
\begin{sol}
Для простой кубической решётки обратная решётка
также простая кубическая с периодом $2\pi /a$.
Фермиевский волновой вектор $k_F = \sqrt[3]{3\pi^2 n} \approx 3,09 /a< \pi /a$, поэтому ферми-сфера
полностью умещается в первой зоне Бриллюэна.
В силу слабости взаимодействия (по условию)
искажением ферми-поверхности от идеальной
сферической формы пренебрегаем.

В силу слабости взаимодействия электронов с периодическим потенциалом кристалла изменение спектра можно
считать слабым, запрещённые зоны узкими и за
исключением непосредственной окрестности границ
зоны Бриллюэна можно считать спектр квадратичным
\[
	E(\mathbf{k})= \frac{\hbar ^2 k^2}{2m}
.\] 
Импульс кванта света в УФ-диапазоне много меньше
фермиевского, поэтому в рамках приведённой зонной
схемы переход происходит <<вертикально>> ---
без изменения импульса электрона. Мы можем добавить
вектор обратной решётки, чтобы перейти к расширенной
или периодической зонной схемам. Необходимо после
такого перехода <<вверх>> на $\hbar \omega$ и
<<вбок>> на $\mathbf{G}$ попасть на квадратичный
спектр с минимальным приростом энергии. Очевидно,
что минимальному изменению энергии будет
соответствовать смещение на вектор обратной
решётки минимальной длины, например, $\left( 2\pi /a;\, 0;\, 0 \right) $ из точки в $k$-пространстве
$\left( -k_F;\,0;\,0 \right) $. Так как $k_F$ 
всё же близко к границе зоны Бриллюэна, то и
оттранслированная точка оказывается близка к
границе:
\[
\frac{2\pi}{a} -k_F \approx \frac{3.19
}{a}
.\] 
Отсюда искомая энергия кванта
\begin{multline*}
	\hbar  \omega= \frac{\hbar ^2}{2m} \left( 
	\left( \frac{2\pi}{a}-k_F \right) ^2-
k_F^2\right) = \frac{\hbar ^2 k_F^2}{2m} \left( 
\frac{(2\pi)^2}{\left( 3\pi ^2 \right) ^{2 /3}}-
2 \frac{2\pi}{\left( 3\pi^2 \right) ^{1 /3}}\right) 
\approx \\
\approx 0,068 E_F= 0,2 \text{ эВ }
.\end{multline*} 
Это энергия кванта ИК-диапазона, она существенно
меньше типичных значений работы выхода для металла.
\end{sol}
\end{document}
