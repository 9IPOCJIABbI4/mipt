\documentclass[a4paper]{article}
% Этот шаблон документа разработан в 2014 году
% Данилом Фёдоровых (danil@fedorovykh.ru) 
% для использования в курсе 
% <<Документы и презентации в \LaTeX>>, записанном НИУ ВШЭ
% для Coursera.org: http://coursera.org/course/latex .
% Исходная версия шаблона --- 
% https://www.writelatex.com/coursera/latex/5.3

% В этом документе преамбула

\usepackage{siunitx}
%%% Работа с русским языком
%\usepackage{cmap}					% поиск в PDF
%\usepackage{mathtext} 				% русские буквы в формулах
%\usepackage[T2A]{fontenc}			% кодировка
%\usepackage[utf8]{inputenc}			% кодировка исходного текста
%\usepackage[english,russian]{babel}	% локализация и переносы
%\usepackage{indentfirst}
%\frenchspacing
%
%\renewcommand{\epsilon}{\ensuremath{\varepsilon}}
%\newcommand{\phibackup}{\ensuremath{\phi}}
%\renewcommand{\phi}{\ensuremath{\varphi}}
%\renewcommand{\varphi}{\ensuremath{\phibackup}}
%\renewcommand{\kappa}{\ensuremath{\varkappa}}
%\renewcommand{\le}{\ensuremath{\leqslant}}
%\renewcommand{\leq}{\ensuremath{\leqslant}}
%\renewcommand{\ge}{\ensuremath{\geqslant}}
%\renewcommand{\geq}{\ensuremath{\geqslant}}
%\renewcommand{\emptyset}{\varnothing}
%\renewcommand{\Im}{\operatorname{Im}}
%\renewcommand{\Re}{\operatorname{Re}}


%%% Дополнительная работа с математикой
\usepackage{amsmath,amsfonts,amssymb,amsthm,mathtools} % AMS
%\usepackage{icomma} % "Умная" запятая: $0,2$ --- число, $0, 2$ --- перечисление

%% Номера формул
%\mathtoolsset{showonlyrefs=true} % Показывать номера только у тех формул, на которые есть \eqref{} в тексте.
%\usepackage{leqno} % Нумереация формул слева

%% Свои команды
\DeclareMathOperator{\sgn}{\mathop{sgn}}
\DeclareMathOperator{\sign}{\mathop{sign}}
\DeclareMathOperator*{\res}{\mathop{res}}
\DeclareMathOperator*{\tr}{\mathop{tr}}
\DeclareMathOperator*{\rot}{\mathop{rot}}
\DeclareMathOperator*{\divop}{\mathop{div}}
\DeclareMathOperator*{\grad}{\mathop{grad}}

%% Перенос знаков в формулах (по Львовскому)
\newcommand*{\hm}[1]{#1\nobreak\discretionary{}
{\hbox{$\mathsurround=0pt #1$}}{}}

%%% Работа с картинками
\usepackage{graphicx}  % Для вставки рисунков
\graphicspath{{figures/}}  % папки с картинками
\setlength\fboxsep{3pt} % Отступ рамки \fbox{} от рисунка
\setlength\fboxrule{1pt} % Толщина линий рамки \fbox{}
\usepackage{wrapfig} % Обтекание рисунков текстом

%%% Работа с таблицами
\usepackage{array,tabularx,tabulary,booktabs} % Дополнительная работа с таблицами
\usepackage{longtable}  % Длинные таблицы
\usepackage{multirow} % Слияние строк в таблице

%%% Теоремы
\theoremstyle{plain} % Это стиль по умолчанию, его можно не переопределять.
\newtheorem{thm}{Теорема}
\newtheorem*{thm*}{Теорема}
\newtheorem{prop}{Предложение}
\newtheorem*{prop*}{Предложение}
 
\theoremstyle{definition} % "Определение"
%\newtheorem{corollary}{Следствие}[theorem]
\newtheorem{dfn}{Определение}
\newtheorem*{dfn*}{Определение}
\newtheorem{prob}{Задача}
\newtheorem*{prob*}{Задача}

 
\theoremstyle{remark} % "Примечание"
\newtheorem*{sol}{Решение}
\newtheorem*{rem}{Замечание}

%%% Программирование
\usepackage{etoolbox} % логические операторы

%%% Страница
%\usepackage{extsizes} % Возможность сделать 14-й шрифт
%\usepackage{geometry} % Простой способ задавать поля
%	\geometry{top=25mm}
%	\geometry{bottom=35mm}
%	\geometry{left=35mm}
%	\geometry{right=20mm}
 
\usepackage{fancyhdr} % Колонтитулы
%	\pagestyle{fancy}
 %	\renewcommand{\headrulewidth}{0pt}  % Толщина линейки, отчеркивающей верхний колонтитул
	%\lfoot{Нижний левый}
	%\rfoot{Нижний правый}
	%\rhead{Верхний правый}
	%\chead{Верхний в центре}
	%\lhead{Верхний левый}
	%\cfoot{Нижний в центре} % По умолчанию здесь номер страницы

\usepackage{setspace} % Интерлиньяж
%\onehalfspacing % Интерлиньяж 1.5
%\doublespacing % Интерлиньяж 2
%\singlespacing % Интерлиньяж 1

\usepackage{lastpage} % Узнать, сколько всего страниц в документе.

\usepackage{soul} % Модификаторы начертания

\usepackage{hyperref}
\usepackage[usenames,dvipsnames,svgnames,table,rgb]{xcolor}
\hypersetup{				% Гиперссылки
    unicode=true,           % русские буквы в раздела PDF
    pdftitle={Заголовок},   % Заголовок
    pdfauthor={Автор},      % Автор
    pdfsubject={Тема},      % Тема
    pdfcreator={Создатель}, % Создатель
    pdfproducer={Производитель}, % Производитель
    pdfkeywords={keyword1} {key2} {key3}, % Ключевые слова
%    colorlinks=true,       	% false: ссылки в рамках; true: цветные ссылки
    %linkcolor=red,          % внутренние ссылки
    %citecolor=black,        % на библиографию
    %filecolor=magenta,      % на файлы
    %urlcolor=cyan           % на URL
}

\usepackage{csquotes} % Еще инструменты для ссылок

%\usepackage[style=apa,maxcitenames=2,backend=biber,sorting=nty]{biblatex}

\usepackage{multicol} % Несколько колонок

\usepackage{tikz} % Работа с графикой
\usepackage{pgfplots}
\usepackage{pgfplotstable}
%\usepackage{coloremoji}
\usepackage{floatrow}
\usepackage{subcaption}
\graphicspath{{figures/}}

\renewcommand\thesubfigure{\asbuk{subfigure}}
%\addbibresource{master.bib}

\usepackage{import}
\usepackage{pdfpages}
\usepackage{transparent}
\usepackage{xcolor}
\usepackage{xifthen}

\newcommand{\incfig}[2][1]{%
    \def\svgwidth{#1\columnwidth}
    \import{./figures/}{#2.pdf_tex}
}
%\usepackage{titlesec}
%\titleformat{\section}{\normalfont\Large\bfseries}{}{0pt}{}
%----------------------STANDART:
%\titleformat{\chapter}[display]
%  {\normalfont\huge\bfseries}{\chaptertitlename\ \thechapter}{20pt}{\Huge}
%\titleformat{\section}{\normalfont\Large\bfseries}{\thesection}{1em}{}
%\titleformat{\subsection}
%  {\normalfont\large\bfseries}{\thesubsection}{1em}{}
%\titleformat{\subsubsection}
%  {\normalfont\normalsize\bfseries}{\thesubsubsection}{1em}{}
%\titleformat{\paragraph}[runin]
%  {\normalfont\normalsize\bfseries}{\theparagraph}{1em}{}
%\titleformat{\subparagraph}[runin]
%  {\normalfont\normalsize\bfseries}{\thesubparagraph}{1em}{}

\pdfsuppresswarningpagegroup=1
\pgfplotsset{compat=1.16}



%\setcounter{tocdepth}{1} % only parts,chapters,sections
%\titleformat{\subsection}{\normalfont\large\bfseries}{}{0em}{}
%\titleformat{\subsubsection}{\normalfont\normalsize\bfseries}{}{0em}{}

%\newcommand{\textover}[2]{\stackrel{\mathclap{\normalfont\mbox{#2}}}{#1}}

\author{Yaroslav Drachov\\
Moscow Institute of Physics and Technology}
%\author{Драчов Ярослав\\
%Факультет общей и прикладной физики МФТИ}
\newcommand{\veq}{\mathrel{\rotatebox{90}{$=$}}}
%\newcommand{\teto}[1]{\stackrel{\mathclap{\normalfont\tiny\mbox{#1}}}{\to}}
%\renewcommand{\thesubsection}{\arabic{subsection}}

%%\setcounter{secnumdepth}{0}

\definecolor{tabblue}{RGB}{30, 119, 180}
\definecolor{taborange}{RGB}{255, 127, 15}
\definecolor{tabgreen}{RGB}{45, 160, 43}
\definecolor{tabred}{RGB}{214, 38, 40}
\definecolor{tabpurple}{RGB}{148, 103, 189}
\definecolor{tabbrown}{RGB}{140, 86, 76}
\definecolor{tabpink}{RGB}{227, 119, 193}
\definecolor{tabgray}{RGB}{127, 127, 127}
\definecolor{tabolive}{RGB}{188, 189, 33}
\definecolor{tabcyan}{RGB}{22, 190, 207}
\pgfplotscreateplotcyclelist{colorbrewer-tab}{
{tabblue},
{taborange},
{tabgreen},
{tabred},
{tabpurple},
{tabbrown},
{tabpink},
{tabgray},
{tabolive},
{tabcyan},
}
\usepackage{csvsimple}
\usepackage{extarrows}
%\renewcommand{\labelenumii}{\asbuk{enumii})}
%\renewcommand{\labelenumiv}{\Asbuk{enumiv}}
%\newcommand{\prob}[1]{\subsubsection*{#1}}
\sisetup{output-decimal-marker = {,},separate-uncertainty = true,exponent-product = \cdot}

\usepackage{braket}
\usepackage{enumerate}
\usepackage{chngcntr}
%\counterwithin*{equation}{problem}
%\usepackage{bbold}

\newtheoremstyle{hiProb}% ⟨name ⟩ 
{3pt}% ⟨Space above ⟩1 
{3pt}% ⟨Space below ⟩1
{}% ⟨Body font ⟩
{}% ⟨Indent amount ⟩2
{\bfseries}% ⟨Theorem head font⟩
{.}% ⟨Punctuation after theorem head ⟩
{.5em}% ⟨Space after theorem head ⟩3
%{\thmname{#1} \thmnote{#3}}% ⟨Theorem head spec (can be left empty, meaning ‘normal’)⟩
{\thmnote{#3}}% ⟨Theorem head spec (can be left empty, meaning ‘normal’)⟩
\theoremstyle{hiProb} % "Определение"
%\newtheorem{hiProb}{Задача}
\newtheorem{hiProb}{}
%\usepackage{mmacells}
\newcommand{\textover}[2]{\stackrel{\mathclap{\normalfont\scriptsize\mbox{#2}}}{#1}}
\usepackage{units}
\usepackage[math]{cellspace}%
\setlength\cellspacetoplimit{2pt}
\setlength\cellspacebottomlimit{2pt}

\DeclareMathAlphabet{\mathbbold}{U}{bbold}{m}{n}

\newcommand{\normord}[1]{:\mathrel{#1}:}

\title{Семинар №2 по квантовой механике}
\begin{document}
	\maketitle
\begin{hiProb}[Задача 4]
Покажите,  что для оператора эволюции
\[
	\widehat{\operatorname{U}}(t_2,\,t_1)\equiv T e^{
	i \int\limits_{t_1}^{t_2} dt \widehat{\operatorname{V}}_0
(t)},
\]
верно равенство
\[
	\widehat{\operatorname{U}}(t_2,\,t_1)\widehat{\operatorname{U}}(t_1,\,t_0)= \widehat{\operatorname{U}}(t_2,\,t_0)
.\] 
\end{hiProb}
\begin{sol}
\begin{multline*}
	\widehat{\operatorname{U}}(t_1,\,t_0)=T e ^{
	- \frac{i}{\hbar }\int\limits_{t_0}^{t_1} d\tau
\widehat{\operatorname{H}}(\tau)}=
\sum_{n=0}^{\infty} \left( - \frac{i}{\hbar } \right) ^n
\int\limits_{t_0}^{t_1} d \tau_1 \ldots
\int\limits_{t_0}^{\tau_{n-1}} d \tau_n \widehat{\operatorname{H}}
(\tau_1) \ldots \widehat{\operatorname{H}}(\tau_n)=\\=
\widehat{\operatorname{1}}-\frac{i}{\hbar }
\int\limits_{t_0}^{t_1} d\tau \widehat{\operatorname{H}}(\tau)
-\frac{1}{\hbar^2 }\int\limits_{t_0}^{t_1} d \tau_1 \int\limits_{t_0}^{\tau_1} d\tau_2 \widehat{\operatorname{H}}  (\tau_1)
\widehat{\operatorname{H}}(\tau_2)+\\+\frac{i}{\hbar^3}\int\limits_{t_0}^{t_1} 
d\tau_1 \int\limits_{t_0}^{\tau_1} d\tau_2 \int\limits_{t_0}^{
\tau_2} d\tau_3 \widehat{\operatorname{H}}(\tau_1)\widehat{\operatorname{H}}(\tau_2)\widehat{\operatorname{H}}(\tau_3)+\ldots  
.\end{multline*} 
Проделать самостоятельно формулу 6 из лекций
\[
	\ket{\psi(t)}=\ket{\psi_0}- \frac{i}{\hbar }
	\int\limits_{t_0}^{t_1} d\tau \widehat{\operatorname{H}}
	(\tau)\ket{\psi(\tau}
.\] 
\[
	\widehat{\operatorname{U}}(t_2,\,t_1)=\widehat{\operatorname{1}}- \frac{i}{\hbar }\int\limits_{t_1}^{t_2} d\tau \widehat{\operatorname{H}}(\tau)- \frac{1}{\hbar ^2}\int\limits_{t_1}^{t_2} d \tau_1
	\int\limits_{t_1}^{\tau_1} d\tau_2 \widehat{\operatorname{H}}
	(\tau_1) \widehat{\operatorname{H}}(\tau_2)+\ldots
.\] 
\begin{multline*}
	\widehat{\operatorname{U}}(t_2,\,t_1)\widehat{\operatorname{U}}
	(t_1,\,t_0)=\\=
	\left( \widehat{\operatorname{1}}-\frac{i}{\hbar }
	\int\limits_{t_1}^{t_2} d\tau \widehat{\operatorname{H}}
(\tau)-\frac{1}{\hbar^2}\int\limits_{t_1}^{t_2} d\tau_2d\tau_1
\int\limits_{t_1}^{\tau_1} d\tau_2 \widehat{\operatorname{H}}
(\tau_1)\widehat{\operatorname{H}}(\tau_2)+
\frac{i}{\hbar^3 }\int\limits_{t_1}^{t_2} d\tau_1
\int\limits_{t_1}^{\tau_1} d\tau_2 \int\limits_{t_1}^{\tau_2} 
d\tau_3 \widehat{\operatorname{H}}(\tau_1)\widehat{\operatorname{H}}
(\tau_2)\widehat{\operatorname{H}}(\tau_3)+\ldots \right) \widehat{\operatorname{1}}+\\+
\frac{i}{\hbar^3} \int\limits_{t_1}^{t_2} d\tau_1 \int\limits_{t_1}^{
\tau_1} d \tau_2 d\tau_3 \widehat{\operatorname{H}}(\tau_1)
\widehat{\operatorname{H}}(\tau_2) \widehat{\operatorname{H}}(\tau_3)
\left( - \frac{i}{\hbar }\int\limits_{t_0}^{t_1} d \tau' \widehat{\operatorname{H}}(\tau')  \right) +
\left(\widehat{\operatorname{1}}\ldots\right)
\left( - \frac{1}{\hbar^2} \int\limits_{t_0}^{t_1} d \tau_1'
\int\limits_{t_0}^{\tau_1'} d \tau_2' \widehat{\operatorname{H}}(
\tau_1')\widehat{\operatorname{H}}(\tau_2')\right) +\\+
\left(\widehat{\operatorname{1}}\ldots\right)\left( 
\left( -\frac{i}{\hbar ^3} \right) \int\int\int\right) 
.\end{multline*} 
\begin{multline*}
\widehat{\operatorname{1}}- \frac{i}{\hbar }\int\limits_{t_1}^{t_2} 
d\tau \widehat{\operatorname{H}}(\tau) - \frac{i}{\hbar } \int\limits_{t_0}^{t_1} d \tau' \widehat{\operatorname{H}} (\tau')=
-\frac{1}{\hbar^2} \int\limits_{t_1}^{t_2} d\tau_1 \int\limits_{t_1}^{\tau_1} d\tau_2 \widehat{\operatorname{H}}(\tau_1) \widehat{\operatorname{H}}(\tau_2)+\\+
\left( - \frac{i}{\hbar } \right) ^2 \int\limits_{t_1}^{t_2} 
d\tau H(\tau) \int\limits_{t_0}^{t_1} d\tau' \widehat{\operatorname{H}}(\tau')+ \left( -\frac{i}{\hbar } \right) ^2 \int\limits_{t_0}^{t_1}  d\tau_1 \int\limits_{t_0}^{\tau_1} d\tau_2 \widehat{\operatorname{H}}(\tau_1) \widehat{\operatorname{H}} (\tau_2)=\\=
\left( -\frac{i}{\hbar } \right) ^2 \int\limits_{t_1}^{t_2} 
d\tau_1 \left[ \int\limits_{t_1}^{\tau_1} d\tau_2 \widehat{\operatorname{H}}(\tau_1) \widehat{\operatorname{H}}(\tau_2)+
\int\limits_{t_0}^{t_1} d\tau_2 \widehat{\operatorname{H}}(\tau_1)
H(\tau_2) \right] +\ldots=\\=-\left( \frac{i}{\hbar } \right) ^2
\int\limits_{t_1}^{t_2} d\tau_1 \int\limits_{t_0}^{\tau_1} 
\widehat{\operatorname{H}}(\tau_1) \widehat{\operatorname{H}}(\tau_2)
d\tau_2+ \left( - \frac{i}{\hbar } \right) ^2 \int\limits_{t_0}^{t_1} 
d\tau_1 \int\limits_{t_0}^{\tau_1} d\tau_2 \widehat{\operatorname{H}}
(\tau_1) \widehat{\operatorname{H}}(\tau_2)=\\=
\left( - \frac{i}{\hbar } \right) ^2 \int\limits_{t_0}^{t_2} 
d\tau_1 \int\limits_{t_0}^{\tau_1} d \tau_2 \widehat{\operatorname{H}}
(\tau_1) \widehat{\operatorname{H}}(\tau_2)
.\end{multline*} 
\end{sol}
\begin{hiProb}[Задача 2]
\[
	\Psi(x,\,t=0)= C e ^{-\frac{m\omega}{2\hbar }(x-x_0)^2}
.\] 
\end{hiProb}
\begin{sol}
\[
	K(t,\,t_0=0|x,\,y)= \sqrt{\frac{m\omega}{2\pi i \hbar
	\sin (\omega t)}} e^{\frac{i S[z_{\text{Cl}}(\cdot)]}{\hbar}}
.\] 
\[
	\Psi(t,\,x)= \int dy K(t|x,\,y)\Psi_{t=0}(y)=
	C \int dy \frac{m\omega}{2\pi i \hbar \sin \omega t}
	e^{\frac{i m \omega}{2\hbar \sin \omega t}\left( 
	(x^2+y^2)\cos\omega t-2xy\right) }
	e^{-\frac{m \omega (y-y_0)^2}{2\hbar}}
.\] 
\[
\lambda\to \lambda+i \epsilon
.\] 
\end{sol}
\begin{hiProb}[Задача 7а]
\end{hiProb}
\begin{sol}
\[
	\Psi_\text{SM}= \Phi \left( \underbrace{\mathbf{r}_1}_{1e^-},\,
\underbrace{	\mathbf{r}_2}_{2e^-}\right) \cdot \chi\left( S_{1z},\,S_{2z} \right) 
.\] 
\[
\widehat{\operatorname{S}}= \widehat{\operatorname{S}}_1+\widehat{\operatorname{S}}_2
.\] 
\[
	\left\{
	\begin{aligned}
		\widehat{\operatorname{S}}^2(1) \Ket{\frac{1}{2},m_1}&= \frac{3}{4}
\hbar^2\Ket{\frac{1}{2},m_1}\\
		S_z(1) \Ket{\frac{1}{2},\,m_1}&=\hbar m_1 \Ket{\frac{1}{2},\,m_1}
	\end{aligned}
	\right.
\] 
\[
\left\{
\begin{aligned}
	\widehat{\operatorname{S}}^2(2) \Ket{\frac{1}{2},\,m_2}&=
	\hbar^2 \frac{3}{4}\Ket{\frac{1}{2},\,m_2}\\
\widehat{\operatorname{S}}_z(2)\Ket{\frac{1}{2},\,m_2}&=
\hbar m_2 \Ket{\frac{1}{2},\,m_2}
\end{aligned}
\right.
\] 
\[
\left\{
\begin{aligned}
	\widehat{\operatorname{S}}^2 \Psi_\text{SM}&= \hbar^2 S
	(S+1)\Psi_\text{SM}\\
	\widehat{\operatorname{S}}_z \Psi_{\text{SM}}&=
	\hbar M \Psi_{\text{SM}}
\end{aligned}
\right.
\] 
\emph{Частица №1}:
\[
	\chi_{1 /2}(1)\to S_z = \frac{\hbar }{2}
.\] 
\[
	\chi_{- 1 /2}(1)\to S_z= - \frac{\hbar }{2}
.\] 
\emph{Частица №2}
\[
	\chi_{1 /2}(2)\to S_z = \frac{\hbar }{2}
.\] 
\[
	\chi_{- 1 /2}(2)\to  S_z =- \frac{\hbar}{2}
.\] 
\[
	\mathcal{H}(1,\,2)= \mathcal{H}(1) \otimes \mathcal{H}(2)
.\] 
\[
	\ket{\Psi_1}= \chi_{1 /2}(1) \otimes \chi_{1 /2}(2),\quad
	m_1=1 /2,\ m_2=1 /2
.\] 
\[
	\ket{\Psi_2}= \chi_{1 /2}(1) \otimes \chi_{-1 /2}(2)
	m_1=1 /2,\ m_2=-1 /2
.\] 
\[
	\ket{\Psi_3}= \chi_{-1 /2}(1) \otimes \chi_{1 /2}(2)
	m_1=-1 /2,\ m_2=1 /2
.\] 
\[
	\ket{\Psi_4}= \chi_{-1 /2}(1) \otimes \chi_{-1 /2}(2)
	m_1=-1 /2,\ m_2=-1 /2
.\] 
\[
	\widehat{\operatorname{S}}_z(1) \chi_1 =\hbar m_1
	\chi_1,\quad m_1= \pm \frac{1}{2}
.\] 
\[
	\sigma_3 = \begin{pmatrix} 1 & 0 \\ 0 & -1 \end{pmatrix} 
.\] 
\[
	\frac{1}{2} \begin{pmatrix} 1 & 0 \\ 0 & -1 \end{pmatrix} =
	\frac{1}{2}\begin{pmatrix} 1 \\0 \end{pmatrix} 
.\] 
\[
	\frac{1}{2}\begin{pmatrix} 0 &0 \\0 & -1 \end{pmatrix} = -\frac{1}{2}\begin{pmatrix} 0 \\1 \end{pmatrix} 
.\] 
Общие собственные функции: $\Psi_{1,\,1},\ \Psi_{1,\,-1}$ 
\renewcommand{\labelenumi}{\asbuk{enumi})}
\begin{enumerate}
\item 
	\[
	\left\{
	\begin{aligned}
		\widehat{\operatorname{S}}_z\Psi_{1,\,1}&= \hbar \cdot 1
	\Psi_{1,\,-1}\\
		\widehat{\operatorname{S}}^2 \Psi_{1,\,1}&=\hbar^2\cdot_2\Psi_{1,\,1}
	\end{aligned}
	\right.
	.\] 
	\[
		\Psi_{1,\,1}=\chi_{1 /2}(1) \chi_{ 1/2}(2)=
		\begin{pmatrix} 1 \\ 0 \\ 0 \\ 0 \end{pmatrix} ,\quad
		S=1,\ M=1
	.\] 
\item 
	\[
	\left\{
	\begin{aligned}
		\widehat{\operatorname{S}}_z\Psi_{1,\,-1}&=\hbar (-1)
		\Psi_{1,\,-1}\\
		\widehat{\operatorname{S}}^2_{1,\,-1}&= 
		\hbar^2 \cdot 2 \Psi_{1,\,-1}\\
	\end{aligned}
	\right.
	.\] 
	\[
		\Psi_{1,\,-1}=\chi_{- 1 /2}(1)\chi_{- 1/ 2}(2)=
		\begin{pmatrix} 0 \\ 0 \\0 \\ 1 \end{pmatrix},\quad
		M=-1,\ S=1
	.\] 
\end{enumerate}
\[
	\widehat{\operatorname{S}}_+ \Psi_{SM}= \hbar \sqrt{S(S+1)-
	M(M+1)} 
.\] 
\begin{multline*}
	\left( \widehat{\operatorname{S}}_+(1)+\widehat{\operatorname{S}}_+ (2)\chi_{1 /2}(1)\chi_{- 1 /2}(2) \right) =
	\widehat{\operatorname{S}}_+(1)\chi_{-1 /2}(1) \chi_{-1 /2}
	(2)+\chi_{- 1 /2}(1) \widehat{\operatorname{S}}_+(2)
	\chi_{-1 /2}(2)=\\=\hbar \left[ 1\cdot \chi_{1 /2}(1)
	\chi_{-1 /2}(2)+\chi_{- 1 /2}(1)\chi_{1 /2}(2)\right] 
.\end{multline*} 
\[
	\Psi_{0,\,0}=\frac{1}{\sqrt{2} }\left( \chi_{1 /2}(1)
	\chi_{- 1 /2}(2)-\chi_{-1 /2}(1)\chi_{1 /2}(2)\right) 
.\] 
\[
	\left\{
	\begin{aligned}
	\sum_{m_1,\,m_2}^{} C^{SM}_{s_1 m_1 s_2 m_2} C_{s_1 m_1 s_2
	m_2}^{S'M'}&=\delta_{SS'}\delta_{MM'}\\
	\sum_{SM} C^{SM}_{s_1 m_1 s_2 m_2} C^{SM}_{s_1 m_1' s_2 m_2'}
	&= \delta_{m_1 m_1'}\delta_{m_2 m_2'}
	\end{aligned}
	\right.
\] 
\[
C_{\frac{1}{2}\, \frac{1}{2}\, \frac{1}{2}\, \frac{1}{2}}^{1\, 1}=1
.\] 
\[
C_{\frac{1}{2}\, -\frac{1}{2}\, \frac{1}{2}\, -\frac{1}{2}}^{1\, -1}=
C_{\frac{1}{2}\, \frac{1}{2}\, \frac{1}{2}\, \frac{1}{2}}^{1\, 1}=1
.\] 
\[
C_{\frac{1}{2}\, -\frac{1}{2}\, \frac{1}{2}\, \frac{1}{2}}^{1\, 0}=
\frac{1}{\sqrt{2} }
.\] 
\[
C_{\frac{1}{2}\, \frac{1}{2}\, \frac{1}{2}\, -\frac{1}{2}}^{1\, 0}=
\frac{1}{\sqrt{2} }
.\] 
\[
C_{\frac{1}{2}\, -\frac{1}{2}\, \frac{1}{2}\, \frac{1}{2}}^{0\, 0}=
\frac{1}{\sqrt{2} }
.\] 
\[
C_{\frac{1}{2}\, \frac{1}{2}\, \frac{1}{2}\, -\frac{1}{2}}^{0\, 0}=
\frac{1}{\sqrt{2} }
.\]
\[
\widehat{\operatorname{S}}_+ \Psi_{0,\,0}= \widehat{\operatorname{S}}_- \Psi_{0,\,0}=0
.\] 
\end{sol}
\begin{hiProb}[Задача 7б]
\end{hiProb}
\begin{sol}
\[
	\widehat{\operatorname{\mathbf{j}}}=\widehat{\operatorname{\mathbf{l}}}+ \widehat{\operatorname{\mathbf{s}}}
.\] 
\[
	\widehat{\operatorname{j^2}},\ 
		\widehat{\operatorname{j_z}},\
		\widehat{\operatorname{l^2}},\
		\widehat{\operatorname{s^2}}
.\] 
\[
\widehat{\operatorname{j_z}}\Psi_{\text{об}}=m_j \Psi_\text{об}
.\] 
\[
\widehat{\operatorname{j_z}}=\widehat{\operatorname{l_z}}+\widehat{\operatorname{s_z}}=-i \widehat{\operatorname{1}}\frac{\partial }{\partial \phi} + \frac{1}{2}\widehat{\operatorname{\sigma_z}}=
\begin{pmatrix} -i \frac{\partial }{\partial \phi} +\frac{1}{2}&
0\\ 0& -i \frac{\partial }{\partial \phi} -\frac{1}{2}\end{pmatrix} 
.\] 
\end{sol}
\end{document}
