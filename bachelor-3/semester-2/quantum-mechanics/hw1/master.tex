\documentclass[a4paper]{article}
% Этот шаблон документа разработан в 2014 году
% Данилом Фёдоровых (danil@fedorovykh.ru) 
% для использования в курсе 
% <<Документы и презентации в \LaTeX>>, записанном НИУ ВШЭ
% для Coursera.org: http://coursera.org/course/latex .
% Исходная версия шаблона --- 
% https://www.writelatex.com/coursera/latex/5.3

% В этом документе преамбула

\usepackage{siunitx}
%%% Работа с русским языком
%\usepackage{cmap}					% поиск в PDF
%\usepackage{mathtext} 				% русские буквы в формулах
%\usepackage[T2A]{fontenc}			% кодировка
%\usepackage[utf8]{inputenc}			% кодировка исходного текста
%\usepackage[english,russian]{babel}	% локализация и переносы
%\usepackage{indentfirst}
%\frenchspacing
%
%\renewcommand{\epsilon}{\ensuremath{\varepsilon}}
%\newcommand{\phibackup}{\ensuremath{\phi}}
%\renewcommand{\phi}{\ensuremath{\varphi}}
%\renewcommand{\varphi}{\ensuremath{\phibackup}}
%\renewcommand{\kappa}{\ensuremath{\varkappa}}
%\renewcommand{\le}{\ensuremath{\leqslant}}
%\renewcommand{\leq}{\ensuremath{\leqslant}}
%\renewcommand{\ge}{\ensuremath{\geqslant}}
%\renewcommand{\geq}{\ensuremath{\geqslant}}
%\renewcommand{\emptyset}{\varnothing}
%\renewcommand{\Im}{\operatorname{Im}}
%\renewcommand{\Re}{\operatorname{Re}}


%%% Дополнительная работа с математикой
\usepackage{amsmath,amsfonts,amssymb,amsthm,mathtools} % AMS
%\usepackage{icomma} % "Умная" запятая: $0,2$ --- число, $0, 2$ --- перечисление

%% Номера формул
%\mathtoolsset{showonlyrefs=true} % Показывать номера только у тех формул, на которые есть \eqref{} в тексте.
%\usepackage{leqno} % Нумереация формул слева

%% Свои команды
\DeclareMathOperator{\sgn}{\mathop{sgn}}
\DeclareMathOperator{\sign}{\mathop{sign}}
\DeclareMathOperator*{\res}{\mathop{res}}
\DeclareMathOperator*{\tr}{\mathop{tr}}
\DeclareMathOperator*{\rot}{\mathop{rot}}
\DeclareMathOperator*{\divop}{\mathop{div}}
\DeclareMathOperator*{\grad}{\mathop{grad}}

%% Перенос знаков в формулах (по Львовскому)
\newcommand*{\hm}[1]{#1\nobreak\discretionary{}
{\hbox{$\mathsurround=0pt #1$}}{}}

%%% Работа с картинками
\usepackage{graphicx}  % Для вставки рисунков
\graphicspath{{figures/}}  % папки с картинками
\setlength\fboxsep{3pt} % Отступ рамки \fbox{} от рисунка
\setlength\fboxrule{1pt} % Толщина линий рамки \fbox{}
\usepackage{wrapfig} % Обтекание рисунков текстом

%%% Работа с таблицами
\usepackage{array,tabularx,tabulary,booktabs} % Дополнительная работа с таблицами
\usepackage{longtable}  % Длинные таблицы
\usepackage{multirow} % Слияние строк в таблице

%%% Теоремы
\theoremstyle{plain} % Это стиль по умолчанию, его можно не переопределять.
\newtheorem{thm}{Теорема}
\newtheorem*{thm*}{Теорема}
\newtheorem{prop}{Предложение}
\newtheorem*{prop*}{Предложение}
 
\theoremstyle{definition} % "Определение"
%\newtheorem{corollary}{Следствие}[theorem]
\newtheorem{dfn}{Определение}
\newtheorem*{dfn*}{Определение}
\newtheorem{prob}{Задача}
\newtheorem*{prob*}{Задача}

 
\theoremstyle{remark} % "Примечание"
\newtheorem*{sol}{Решение}
\newtheorem*{rem}{Замечание}

%%% Программирование
\usepackage{etoolbox} % логические операторы

%%% Страница
%\usepackage{extsizes} % Возможность сделать 14-й шрифт
%\usepackage{geometry} % Простой способ задавать поля
%	\geometry{top=25mm}
%	\geometry{bottom=35mm}
%	\geometry{left=35mm}
%	\geometry{right=20mm}
 
\usepackage{fancyhdr} % Колонтитулы
%	\pagestyle{fancy}
 %	\renewcommand{\headrulewidth}{0pt}  % Толщина линейки, отчеркивающей верхний колонтитул
	%\lfoot{Нижний левый}
	%\rfoot{Нижний правый}
	%\rhead{Верхний правый}
	%\chead{Верхний в центре}
	%\lhead{Верхний левый}
	%\cfoot{Нижний в центре} % По умолчанию здесь номер страницы

\usepackage{setspace} % Интерлиньяж
%\onehalfspacing % Интерлиньяж 1.5
%\doublespacing % Интерлиньяж 2
%\singlespacing % Интерлиньяж 1

\usepackage{lastpage} % Узнать, сколько всего страниц в документе.

\usepackage{soul} % Модификаторы начертания

\usepackage{hyperref}
\usepackage[usenames,dvipsnames,svgnames,table,rgb]{xcolor}
\hypersetup{				% Гиперссылки
    unicode=true,           % русские буквы в раздела PDF
    pdftitle={Заголовок},   % Заголовок
    pdfauthor={Автор},      % Автор
    pdfsubject={Тема},      % Тема
    pdfcreator={Создатель}, % Создатель
    pdfproducer={Производитель}, % Производитель
    pdfkeywords={keyword1} {key2} {key3}, % Ключевые слова
%    colorlinks=true,       	% false: ссылки в рамках; true: цветные ссылки
    %linkcolor=red,          % внутренние ссылки
    %citecolor=black,        % на библиографию
    %filecolor=magenta,      % на файлы
    %urlcolor=cyan           % на URL
}

\usepackage{csquotes} % Еще инструменты для ссылок

%\usepackage[style=apa,maxcitenames=2,backend=biber,sorting=nty]{biblatex}

\usepackage{multicol} % Несколько колонок

\usepackage{tikz} % Работа с графикой
\usepackage{pgfplots}
\usepackage{pgfplotstable}
%\usepackage{coloremoji}
\usepackage{floatrow}
\usepackage{subcaption}
\graphicspath{{figures/}}

\renewcommand\thesubfigure{\asbuk{subfigure}}
%\addbibresource{master.bib}

\usepackage{import}
\usepackage{pdfpages}
\usepackage{transparent}
\usepackage{xcolor}
\usepackage{xifthen}

\newcommand{\incfig}[2][1]{%
    \def\svgwidth{#1\columnwidth}
    \import{./figures/}{#2.pdf_tex}
}
%\usepackage{titlesec}
%\titleformat{\section}{\normalfont\Large\bfseries}{}{0pt}{}
%----------------------STANDART:
%\titleformat{\chapter}[display]
%  {\normalfont\huge\bfseries}{\chaptertitlename\ \thechapter}{20pt}{\Huge}
%\titleformat{\section}{\normalfont\Large\bfseries}{\thesection}{1em}{}
%\titleformat{\subsection}
%  {\normalfont\large\bfseries}{\thesubsection}{1em}{}
%\titleformat{\subsubsection}
%  {\normalfont\normalsize\bfseries}{\thesubsubsection}{1em}{}
%\titleformat{\paragraph}[runin]
%  {\normalfont\normalsize\bfseries}{\theparagraph}{1em}{}
%\titleformat{\subparagraph}[runin]
%  {\normalfont\normalsize\bfseries}{\thesubparagraph}{1em}{}

\pdfsuppresswarningpagegroup=1
\pgfplotsset{compat=1.16}



%\setcounter{tocdepth}{1} % only parts,chapters,sections
%\titleformat{\subsection}{\normalfont\large\bfseries}{}{0em}{}
%\titleformat{\subsubsection}{\normalfont\normalsize\bfseries}{}{0em}{}

%\newcommand{\textover}[2]{\stackrel{\mathclap{\normalfont\mbox{#2}}}{#1}}

\author{Yaroslav Drachov\\
Moscow Institute of Physics and Technology}
%\author{Драчов Ярослав\\
%Факультет общей и прикладной физики МФТИ}
\newcommand{\veq}{\mathrel{\rotatebox{90}{$=$}}}
%\newcommand{\teto}[1]{\stackrel{\mathclap{\normalfont\tiny\mbox{#1}}}{\to}}
%\renewcommand{\thesubsection}{\arabic{subsection}}

%%\setcounter{secnumdepth}{0}

\definecolor{tabblue}{RGB}{30, 119, 180}
\definecolor{taborange}{RGB}{255, 127, 15}
\definecolor{tabgreen}{RGB}{45, 160, 43}
\definecolor{tabred}{RGB}{214, 38, 40}
\definecolor{tabpurple}{RGB}{148, 103, 189}
\definecolor{tabbrown}{RGB}{140, 86, 76}
\definecolor{tabpink}{RGB}{227, 119, 193}
\definecolor{tabgray}{RGB}{127, 127, 127}
\definecolor{tabolive}{RGB}{188, 189, 33}
\definecolor{tabcyan}{RGB}{22, 190, 207}
\pgfplotscreateplotcyclelist{colorbrewer-tab}{
{tabblue},
{taborange},
{tabgreen},
{tabred},
{tabpurple},
{tabbrown},
{tabpink},
{tabgray},
{tabolive},
{tabcyan},
}
\usepackage{csvsimple}
\usepackage{extarrows}
%\renewcommand{\labelenumii}{\asbuk{enumii})}
%\renewcommand{\labelenumiv}{\Asbuk{enumiv}}
%\newcommand{\prob}[1]{\subsubsection*{#1}}
\sisetup{output-decimal-marker = {,},separate-uncertainty = true,exponent-product = \cdot}

\usepackage{braket}
\usepackage{enumerate}
\usepackage{chngcntr}
%\counterwithin*{equation}{problem}
%\usepackage{bbold}

\newtheoremstyle{hiProb}% ⟨name ⟩ 
{3pt}% ⟨Space above ⟩1 
{3pt}% ⟨Space below ⟩1
{}% ⟨Body font ⟩
{}% ⟨Indent amount ⟩2
{\bfseries}% ⟨Theorem head font⟩
{.}% ⟨Punctuation after theorem head ⟩
{.5em}% ⟨Space after theorem head ⟩3
%{\thmname{#1} \thmnote{#3}}% ⟨Theorem head spec (can be left empty, meaning ‘normal’)⟩
{\thmnote{#3}}% ⟨Theorem head spec (can be left empty, meaning ‘normal’)⟩
\theoremstyle{hiProb} % "Определение"
%\newtheorem{hiProb}{Задача}
\newtheorem{hiProb}{}
%\usepackage{mmacells}
\newcommand{\textover}[2]{\stackrel{\mathclap{\normalfont\scriptsize\mbox{#2}}}{#1}}
\usepackage{units}
\usepackage[math]{cellspace}%
\setlength\cellspacetoplimit{2pt}
\setlength\cellspacebottomlimit{2pt}

\DeclareMathAlphabet{\mathbbold}{U}{bbold}{m}{n}

\newcommand{\normord}[1]{:\mathrel{#1}:}

\makeatletter
\newcommand\incircbin
{%
  \mathpalette\@incircbin
}
\newcommand\@incircbin[2]
{%
  \mathbin%
  {%
    \ooalign{\hidewidth$#1#2$\hidewidth\crcr$#1\bigcirc$}%
  }%
}
\newcommand{\oeq}{\incircbin{=}}
\makeatother
\title{Домашняя работа по квантовой механике}
\begin{document}
	\maketitle
\section*{Первое задание}
\begin{hiProb}[1]
Вычислите интеграл по траекториям:
\[
	Z[\mathbf{J}(\cdot )]= \int\limits_{\mathbf{z}(0)=
	\mathbf{y},\ \mathbf{z}(t)=\mathbf{x}}^{} 
	\mathcal{D} \mathbf{z}(\tau) e^{
	i \int\limits_{0}^{t} d\tau \left[ 
\frac{m\dot{\mathbf{z}}^2(\tau)}{2}+\mathbf{J}(\tau)
\mathbf{z}(\tau)\right]  }
.\] 
\end{hiProb}
\begin{sol}
Разложим переменную интегрирования следующим образом:
\[
	\mathbf{z}(\tau)=\mathbf{z}_{cl}(\tau)+\mathbf{q}(\tau)
,\]
где
\[
	\mathbf{z}_{cl}(0)=\mathbf{y},\
	\mathbf{z}_{cl}(t)=\mathbf{x},\quad
	\mathbf{q}(0)=0,\
	\mathbf{q}(t)=0
.\] 
Т.\:к.
\[
	\mathcal{L}= \frac{m \dot{\mathbf{z}}_{cl}^2}{2}+\mathbf{J}(\tau)\mathbf{z}_{cl}
,\]
то
\[
	m\ddot{\mathbf{z}}_{cl}=\mathbf{J}(\tau)
 \] 
и
\[
S_{cl}= \int\limits_{0}^{t} d\tau
\left( \frac{m}{2} \dot{\mathbf{z}}^2_{cl}+
\mathbf{J}(\tau) \mathbf{z}_{cl}\right) 
.\] 
Тогда
\[
S= S_{cl}+ \int\limits_{0}^{t} d\tau
\left[ m \dot{\mathbf{z}}_{cl}\dot{\mathbf{q}}+
\mathbf{J}(\tau) \mathbf{q}\right] +\int\limits_{0}^{t}  d\tau \frac{m \dot{\mathbf{q}}^2}{2}
.\] 
\[
\int\limits_{0}^{t}d \tau \left[ 
m \dot{\mathbf{z}}_{cl} \dot{\mathbf{q}}\right] = 
\left. m \dot{\mathbf{z}}_{cl} \mathbf{q} \right|_{0}^t 
	- \int\limits_{0}^{t}  d\tau \ddot{\mathbf{z}} _{cl}\mathbf{q}m
	=- \int\limits_{0}^{t}  d\tau \ddot{\mathbf{z}} _{cl}\mathbf{q}m
.\] 
\[
\int\limits_{0}^{t}d\tau \left[ 
m \dot{\mathbf{z}}_{cl} \dot{\mathbf{q}}+
\mathbf{J} \mathbf{q}\right] =
\int\limits_{0}^{t}  d\tau \left[ 
\mathbf{J}(\tau) \mathbf{q}-
\ddot{\mathbf{z}}_{cl}\mathbf{q} m\right] =
\int\limits_{0}^{t} d\tau
\mathbf{q} \underbrace{\left[ -m \ddot{\mathbf{z}}_{cl}
+\mathbf{J}(\tau)\right] }_{=0}=0
.\] 
Следовательно
\[
	S=S_{cl}+ \int\limits_{0}^{t} d\tau \frac{m \dot{\mathbf{q}}^2}{2} 
.\] 
Значит
\[
	Z= \exp \left( 
		i \frac{S_{cl}(t,\,\mathbf{x},\,\mathbf{y})}{\hbar  }\right) \underbrace{\int\limits_{\mathbf{q}(0)=\mathbf{0},\ 
	\mathbf{q}(t)=\mathbf{0}}
	\mathcal{D} \mathbf{q}(\tau)
	\exp \left( \frac{i}{\hbar }
\int\limits_{0}^{t} d\tau \frac{m \dot{\mathbf{q}}^2}{2} \right) }_{K(t| \mathbf{0},\,\mathbf{0})}
.\] 
\begin{multline*}
	K(t|\mathbf{x},\,\mathbf{y})=
	\Bra{\mathbf{x}}e^{-\frac{i}{\hbar } \frac{
	\widehat{\operatorname{\mathbf{p}}}^2}{2m}t}\Ket{\mathbf{y}}=
	\bra{\mathbf{x}}e^{-\frac{i}{\hbar } \frac{
	\widehat{\operatorname{\mathbf{p}}}^2}{2m}t}
	\int\limits_{\mathbb{R}^3}^{} d^3 \mathbf{p} 
	\ket{\mathbf{p}}\braket{\mathbf{p|}\mathbf{y}}=\\=
	\int\limits_{\mathbb{R}^3}^{} d^3 
	\mathbf{p} e^{-\frac{i}{\hbar } \frac{
	\mathbf{p}^2}{2m}t}\frac{1}{(2\pi \hbar )^3}
	e^{-\frac{i \mathbf{p} (\mathbf{x}-\mathbf{y})}{\hbar }}
.\end{multline*} 
\[
	K(t|\mathbf{0},\,\mathbf{0})=
	\int\limits_{\mathbb{R}^3}^{} 
	d^3 \mathbf{p} e^{-\frac{i}{\hbar }
	\frac{\mathbf{p}^2}{2m}t} \frac{1}{
(2\pi \hbar )^3}=
\frac{1}{(2\pi \hbar )^3} \sqrt{\frac{\pi^3}{
i \left(\frac{t}{2m\hbar }\right)^3}}=
\left( \sqrt{ \frac{m}{2\pi \hbar  t i}}  \right) ^3
.\] 
Найдём теперь $S_{cl}(\mathbf{z}_{cl})$. $\mathbf{z}_{cl}(\tau)$ 
ищем в виде:
\[
	\mathbf{z}(\tau)= \mathbf{a} \tau +\mathbf{b} +
	\int\limits_{-\infty}^{\infty} G(\tau,\,\tau')
	\frac{\mathbf{J}(\tau')}{m}d\tau'
.\] 
Из уравнения движения $m \ddot{\mathbf{z}}_{cl}=\mathbf{J}(\tau)$ 
следует:
\[
	\frac{d^2 G(\tau,\,\tau')}{d\tau^2}=\delta(\tau-\tau')
.\] 
\[
	G(0,\,\tau')=G(t,\,\tau')=0
.\] 
Из условий $\mathbf{z}_{cl}(0)=\mathbf{y}$, $\mathbf{z}_{cl}
(t)=\mathbf{x}$ следует, что $\mathbf{b}=\mathbf{y}$, 
$\mathbf{a}=(\mathbf{x}-\mathbf{y}) /t$. Тогда
решение имеет вид:
\[
	G(\tau,\,\tau')=(\tau-\tau')\theta(\tau-\tau')+
	C_1 (\tau')\tau +C_2(\tau')
.\] 
\[
	G(0,\,\tau')=\tau' \theta(-\tau')+C_2(\tau')=0\implies
	C_2(\tau')=\tau' \theta(-\tau')
.\] 
\begin{multline*}
	G(t,\,\tau')=(t-\tau') \theta(t-\tau')+
	C_1(\tau') t+ \tau'\theta(-\tau')\implies\\\implies
	C_1(\tau')= \frac{-(t-\tau')\theta(t-\tau')-\tau' \theta(-\tau')}{t}
.\end{multline*} 
\[
	G(\tau,\,\tau')=(\tau-\tau') \theta(\tau-\tau')+
	\frac{\tau}{t} \left[ 
	-(t-\tau')\theta(t-\tau')-\tau' \theta(-\tau')\right] +
	\tau' \theta(-\tau')
.\] 
Вычислим $S_{cl}$:
\[
S_{cl}= \int d\tau \left[ 
\frac{m \dot{\mathbf{z}}_{cl}^2}{2}- \mathbf{J}(\tau)
\mathbf{z}_{cl}(\tau)\right] 
.\] 
\[
	\int\limits_{0}^{t} d\tau \frac{m}{2} \left( 
	\dot{\mathbf{z}}_{cl}\right) ^2 = \int\limits_{0}^{t} 
	d\tau \frac{d}{d\tau} \left( 
	\frac{m}{2} \dot{\mathbf{z}}_{cl}\mathbf{z}_{cl}\right) -
	\int\limits_{0}^{t}  d\tau \frac{m}{2}
	\mathbf{z}_{cl}\ddot{\mathbf{z}}_{cl}
.\] 
\[
	S_{cl}= \underbrace{\left. \frac{m}{2} \mathbf{z}_{cl} \dot{\mathbf{z}}_{cl}
\right|_{\tau=0}^{\tau=t}}_{S_{cl1}}+
\underbrace{\frac{1}{2} \int\limits_{0}^{t} 
d\tau \mathbf{J}(\tau) \mathbf{z}_{cl}(\tau)}_{S_{cl 2}}.\] 
\begin{multline*}
	S_{cl 1}= \frac{m}{2} \mathbf{z}_{cl}(t)
	\dot{\mathbf{z}}_{cl}(t)-\frac{m}{2}
	\mathbf{z}_{cl}(0)\dot{\mathbf{z}}_{cl}(0)=
	\frac{m}{2} \mathbf{x} \left[ 
	\frac{\mathbf{x}-\mathbf{y}}{t}+
 \frac{1}{2} \int\limits_{-\infty}^{\infty} \left.
\frac{dG(\tau,\,\tau')}{d\tau} \right|_{\tau=t}\cdot 
\mathbf{J}(\tau') d\tau\right] -\\-
\frac{m}{2} \mathbf{y} \left[ 
\frac{\mathbf{x}-\mathbf{y}}{t}+\frac{1}{2}
\int\limits_{-\infty}^{\infty} \left. \frac{dG(\tau,\,\tau')}{
d\tau} \right|_{\tau=0}\mathbf{J}(\tau') d\tau' \right] 
.\end{multline*} 
И, т.\:к. $G(\tau,\,\tau') \neq 0$ только при $\tau' \in 
\left[ 0,\,t \right] $, то
\[
	S_{cl 1}=\frac{m \left| \mathbf{x} -\mathbf{y} \right| ^2}{2t}+ \frac{1}{2t} \int\limits_{0}^{t} 
	\left( \tau' \mathbf{x} \mathbf{J}(\tau')-
	(t-\tau') \mathbf{y} \mathbf{J}(\tau')\right) d\tau'
.\] 
\[
S_{cl 2}= \frac{1}{2} \int\limits_{0}^{t} 
\left(  \frac{\mathbf{x} -\mathbf{y}}{t}\tau+\mathbf{y} \right) 
\mathbf{J}(\tau) d\tau + \frac{1}{2} \int\limits_{0}^{t} 
d\tau \int\limits_{-\infty}^{\infty} 
d\tau' \mathbf{J}(\tau) G(\tau,\,\tau')
\frac{\mathbf{J}(\tau')}{m}
.\] 
Подставляя полученное значение $S_{cl}$, получаем ответ
для $Z\left[ \mathbf{J}(\cdot ) \right] $.
%На лекциях мы нашли
%\[
%	K_\omega\left( t|x,\,y \right) =
%	\sqrt{ \frac{m\omega}{2\pi i \hbar \sin 
%	\omega t}} e^{ 
%\frac{i}{\hbar } S_{cl}}
%.\] 
%В нашем случае $\omega\to 0$, поэтому
%\[
%	\mathbf{z}_{cl}(\tau)=
%	\frac{1}{2} \mathbf{J} \tau (\tau-t )+\frac{\tau (\mathbf{x}-\mathbf{y})}{t }+\mathbf{y}
%.\] 
%Классическое действие тогда
%\begin{multline*}
%	S_{cl}= \int\limits_{0}^{t} \mathcal{L}dt=\\
%	=
%	\int\limits_{0}^{t}\left[  
%	\frac{m}{2}\left( 
%	\mathbf{J}\left( \tau-\frac{t}{2} \right) +\frac{\mathbf{x}-\mathbf{y}}{t}\right) ^2+
%	\frac{1}{2}\mathbf{J}^2 \tau (\tau-t)+
%	\frac{\tau}{t}\left( \mathbf{x}-\mathbf{y},\,
%	\mathbf{J}\right) +\left( 
%\mathbf{y},\,\mathbf{J}\right)\right] d\tau=
%\\=\frac{m}{2}\left( 
%\mathbf{J}^2 \frac{t^3}{12}+\frac{\left(\mathbf{x}-\mathbf{y}\right)^2}{t}\right) 
%-\mathbf{J}^2 \frac{t^3}{12}+
%(\mathbf{x}-\mathbf{y},\,\mathbf{J})\frac{t}{2}+
%(\mathbf{y},\,\mathbf{J})t
%.\end{multline*} 
%Далее, раскладывая отклонение от классического
%пути по Фурье, и в
\end{sol}
\begin{hiProb}[2]
Волновая функция гармонического осциллятора с массой
$m$ и частотой $\omega$ в начальный момент времени
имеет вид:
\[
	\Psi (x,\,t=0)=C e^{- \frac{m\omega}{2\hbar }(x-x_0)^2}
,\]
где $C$ и $x_0$ --- некоторые константы.
Используя ядро оператора эволюции
для осциллятора, найдите вид волновой функции
в произвольный момент времени.
\end{hiProb}
\begin{sol}
\[
	\Psi (t,\,x)=
	\int\limits_{-\infty}^{\infty} 
	dy K_\omega (t|x,\,y)\Psi(y|t=0)
.\] 
Ядро оператора эволюции для осциллятора:
\[
	K_\omega (t|x,\,y)=
	\sqrt{ \frac{m\omega}{2\pi i \hbar 
	\sin  \omega t}} 
	\exp  \left( 
	\frac{i}{\hbar } S_{cl}
(t,\,x,\,y)\right) 
.\] 
\[
	S_{cl} (t,\,x,\,y)=
	\frac{m\omega}{2 \sin \omega t}
	\left[ 
	(x^2+y^2) \cos \omega t -2xy\right] 
.\] 
\[
	\Psi(x,\,t=0)=
	C e^{-\frac{m\omega}{2\hbar }(x-x_0)^2}
.\] 
\begin{multline*}
	\Psi (t|x)=C \sqrt{
	\frac{m\omega}{ 2\pi i \hbar  \sin \omega
t}} \times \\\times\int \exp  \left[ 
\frac{i}{\hbar } \left( \frac{m\omega}{ 2 \sin \omega t} \left\{ (x^2+y^2) \cos  \omega t -2xy \right\} \right) - \frac{m\omega}{2\hbar } (y-x_0)^2\right] dy=\\=C \sqrt{ \frac{m\omega }{2 \pi i \hbar  \sin  \omega
t}}  \times\\\times\int \exp  \left[ 
\left( \frac{im\omega \cos \omega t}{2\hbar 
\sin \omega t}- \frac{m\omega}{2\hbar } \right) y^2-
\frac{im \omega}{\hbar \sin  \omega t }xy
+\frac{m\omega}{\hbar }x_0y
+ \frac{im\omega \cos \omega t}{2\hbar  \sin \omega t}x^2 - \frac{m\omega}{2\hbar } x_0^2\right] dy=\\=
C \sqrt{ \frac{m\omega }{2\pi i \hbar  \sin \omega
t}}  \exp  \left[ 
\frac{im \omega}{2\hbar } \ctg  \omega t\cdot x^2-
\frac{m\omega}{2\hbar }x_0^2\right] \times\\\times 
\int \exp  \left[ 
\underbrace{\left( \frac{im \omega \cos  \omega t}{2\hbar 
\sin \omega t}- \frac{m\omega}{2\hbar } \right)}_{A} 
y^2 +2 \underbrace{\left(  - \frac{i m \omega}{2 \hbar \sin  \omega t }x
+\frac{m\omega}{2\hbar }x_0\right)}_{B}y \right] dy=\\=
C_1 \sqrt{\frac{\pi}{-A}} \exp  \left[ 
-\frac{1}{4} \cdot 2B\cdot \frac{1}{A}\cdot 2B\right] =
C_1 \sqrt{\frac{\pi}{-A}} \exp \frac{B^2}{-A}
.\end{multline*} 
\[
	B^2= \left( \frac{m\omega}{2\hbar } \right) ^2
	\left( -\frac{x^2}{\sin ^2 \omega t}-
	\frac{2i x x_0}{\sin  \omega t}+x_0^2\right) 
.\] 
\begin{multline*}
	\Psi (t|x)= C_1 \sqrt{ \frac{2\pi \hbar }{- m\omega
	\left( \frac{i \cos  \omega t}{\sin \omega t}-1 \right) }}\exp  \left[ 
\frac{-m\omega}{2\hbar } \frac{\left( -\frac{x^2}{\sin \omega t}-2 i x x_0+x_0^2 \sin \omega t \right) }{\left( 
i \cos  \omega t -\sin  \omega t\right) }\right]  =\\=
C_1 \sqrt{\frac{-2\pi \hbar  \sin \omega t}{m\omega i}} 
e^{-\frac{i \omega t}{2}} \exp 
\left[ \frac{m\omega i e^{-i\omega t}\sin \omega t}{2\hbar }
\left( - \frac{ix}{\sin \omega t}+x_0 \right) ^2\right] =\\=
C e^{- \frac{i \omega t}{2}} \exp \left[ 
\frac{m\omega}{2\hbar }\left( 
\frac{i \cos \omega t}{\sin  \omega t}x^2-x_0^2
+i e^{-i\omega t}\left( 
-\frac{x^2}{\sin  \omega t}-2 i x x_0 +x_0^2 \sin \omega t\right) \right) \right]  =\\=
C e^{- \frac{i\omega t}{2}} \exp \left[ 
\frac{m\omega }{2\hbar }\left( -x_0^2+
2x x_0 \cos \omega t +i x_0^2 \sin \omega t \cos \omega t-
x^2 - 2 i x x_0 \sin \omega t+x_0^2 \sin ^2 \omega t\right) \right] =
\\=C e^{- \frac{i\omega t}{2}} \exp \left[ 
\frac{m\omega}{2\hbar } \left( 
-(x-x_0 \cos \omega t)^2+ i
\left( \frac{x_0^2}{2} \sin  \omega t- 2 x x_0 \sin \omega t \right) \right) \right] 
.\end{multline*} 
\end{sol}
%\begin{hiProb}[3]
%\end{hiProb}
%\begin{sol}
%\renewcommand{\labelenumi}{\asbuk{enumi})}
%\begin{enumerate}
%\item \[S= \int\limits_{0}^{t}  d \tau
%	\left[ \frac{m \dot{\mathbf{X}}^2}{2}+
%	\frac{eB}{2c} (x \dot{y}-y\dot{x})\right] =
%	\int\limits_{0}^{t} d \tau \left[ 
%	\frac{m \dot{x}^2}{2}+\frac{m\dot{y}^2}{2}+
%\frac{m\dot{z}^2}{2}+ \frac{eB}{2c}(x\dot{y}-y\dot{x})\right] \]
%Проварьировав действие, получим уравнения движения:
%\begin{multline*}
%0 = \delta S= \int\limits_{0}^{t} d \tau
%\left[ m \dot{x} \delta \dot{x}+m \dot{y}\delta
%\dot{y}+ m \dot{z}\delta\dot{z}+
%\frac{eB}{2c} \left( 
%x \delta \dot{y}+\dot{y} \delta x-
%y \delta \dot{x}-\dot{x} \delta y\right) \right] =\\=
%\int\limits_{0}^{t} d\tau
%\left[ \left( m\dot{x}- \frac{eB}{2c}y \right) \delta
%\dot{x}+\left( 
%m \dot{y}+\frac{eB}{2c}x\right) \delta\dot{y}\right] 
%.\end{multline*} 
%\end{enumerate}
%\end{sol}
\begin{hiProb}[4]
\end{hiProb}
\begin{sol}
\[
	\widehat{\operatorname{U}}(t_2,\,t_1)\equiv
	T e^{i \int\limits_{t_1}^{t_2} dt \widehat{\operatorname{V}}_0 (t)}
.\] 
\[
	\widehat{\operatorname{U}}(t_2,\,t_0)=
	\sum_{n=0}^{\infty} i^n \int\limits_{t_0}^{t_2} 
	d\tau_1 \int\limits_{t_0}^{\tau_1} d\tau_2\ldots
	\int\limits_{t_0}^{\tau_{n-1}} d\tau_n \widehat{\operatorname{V}}_0 (\tau_1) \ldots
	\widehat{\operatorname{V}}_0 (\tau_n)
.\] 
\begin{multline*}
	\widehat{\operatorname{U}}(t_2,\,t_1)\widehat{\operatorname{U}}(t_1,\,t_0)=
	\sum_{m=0}^{\infty} \sum_{n=0}^{\infty} 
	i^{n+m} \int\limits_{t_1}^{t_2} d\tau_1 \ldots
	\int\limits_{t_1}^{\tau_{n-1}} d \tau_n
	\int\limits_{t_0}^{t_1} d\tilde{\tau}_1
	\int\limits_{t_0}^{\tilde{\tau}_1} 
	d\tilde{\tau}_2 \ldots
	\int\limits_{t_0}^{\tilde{\tau}_{m-1}} d\tilde{\tau}_m\times\\\times
	\widehat{\operatorname{V}}_0(\tau_1)\ldots
	\widehat{\operatorname{V}}_0 (\tau_n)
	\widehat{\operatorname{V}}_0 (\tilde{\tau}_1)\ldots
	\widehat{\operatorname{V}}_0(\tilde{\tau}_m)
	\xlongequal[]{N=n+m}\\=
	\sum_{N=0}^{\infty} i^N \sum_{n=0}^{N} \int\limits_{t_1}^{t_2} d\tau_1 \int\limits_{t_1}^{\tau_1} 
	d\tau_2 \ldots \int\limits_{t_1}^{\tau_{n-1}} 
	d\tau_n \int\limits_{t_0}^{t_1} d\tau_{n+1}
	\int\limits_{t_0}^{\tau_{n+1}} d\tau_{n+2}
	\ldots \int\limits_{t_0}^{\tau_{N-1}} d\tau_{N} 
	\widehat{\operatorname{V}}_0(\tau_1) \ldots
	\widehat{\operatorname{V}}_0 (\tau_N)=\\=
	\sum_{N=0}^{\infty} i^N
	\int\limits_{t_0}^{t_2} d\tau_1 \int\limits_{t_0}^{\tau_1} d\tau_2\ldots
	\int\limits_{t_0}^{\tau_{n-1}} d\tau_n 
	\int\limits_{t_0}^{\tau_n} d\tau_{n+1}\ldots
	\int\limits_{t_0}^{\tau_{N-1}} d\tau_N
	\widehat{\operatorname{V}}_0(\tau_1)
	\ldots \widehat{\operatorname{V}}_0 (\tau_N)=
	\widehat{\operatorname{U}}(t_2,\,t_0)
\end{multline*} 
так как 
\begin{multline*}
\sum_{n=0}^{N} \int\limits_{t_1}^{t_2} 
d\tau_1 \int\limits_{t_1}^{\tau_1} d\tau_2\ldots
\int\limits_{t_1}^{t_{n-1}} d\tau_n \int\limits_{t_0}^{\tau_n} 
d\tau_{n+1}\ldots \int\limits_{t_0}^{\tau_{N-1}} d\tau_{N}=\\
= \int\limits_{t_0}^{t_2} d\tau_1 \int\limits_{t_0}^{t_1} 
d\tau_2 \ldots \int\limits_{t_0}^{\tau_{n-1}} d\tau_n
\int\limits_{t_0}^{t_1} d \tau_{n+1}
\int\limits_{t_0}^{\tau_{n+1}} d\tau_{n+2}\ldots
\int\limits_{t_0}^{\tau_{N-1}} d\tau_N 
.\end{multline*} 
Покажем это:
\[
N=1:\quad \int\limits_{t_0}^{t_1} d\tau_1+\int\limits_{t_1}^{t_2} d\tau_1=\int\limits_{t_0}^{t_2} d\tau_1   
.\] 
\begin{multline*}
N=2:\quad \int\limits_{t_0}^{t_1}  d\tau_1 \int\limits_{t_0}^{\tau_1} d\tau_2+
\int\limits_{t_1}^{t_2} d\tau_1 \int\limits_{t_0}^{t_1} 
d\tau_2+ \int\limits_{t_1}^{t_2} d\tau_1 \int\limits_{t_1}^{\tau_1} d\tau_2=\\=
\int\limits_{t_0}^{t_1} d\tau_1 \int\limits_{t_0}^{\tau_1} 
d\tau_2+ \int\limits_{t_1}^{t_2} d\tau_1 \int\limits_{t_0}^{\tau_1} d\tau_2= \int\limits_{t_0}^{t_2} 
d\tau_1 \int\limits_{t_0}^{\tau_1} d\tau_2 
.\end{multline*} 
Далее аналогично.
\end{sol}
\begin{hiProb}[5]
\end{hiProb}
\begin{sol}
Гамильтониан Дирака:
\[
	\widehat{\operatorname{H}}_D=c \left( 
	\widehat{\operatorname{\alpha}}_i \widehat{\operatorname{p}}_i\right) +\beta m c^2
,\]
где
\[
	\alpha_i= \begin{pmatrix} 0 & \sigma_i \\
	\sigma_i & 0\end{pmatrix} ,\quad
		\beta=\begin{pmatrix} 1&0&0&0\\0&1&0&0\\
		0&0&-1&0\\0&0&0&-1\end{pmatrix} 
.\] 
\begin{enumerate}
\item \[
\left[ \widehat{\operatorname{L}}_i,\,\widehat{\operatorname{H}}_D \right] =
\hbar \left[ \widehat{\operatorname{l}}_i,\,
\widehat{\operatorname{H}}_D\right] =
c \hbar  \widehat{\operatorname{\alpha}}_k\left[ \widehat{\operatorname{l}}_i,\,\widehat{\operatorname{p}}_k \right] 
.\] 
Т.\:к. $\left[ \widehat{\operatorname{l}}_i,\,\widehat{\operatorname{p}}_k \right] = i \epsilon _{ikl} \widehat{\operatorname{p}}_l$, то
\[
\left[ \widehat{\operatorname{L}}_i,\,\widehat{\operatorname{H}}
_D\right] = i \hbar  \epsilon _{ikl} \alpha_k 
\widehat{\operatorname{p}}_l c
.\] 
\item 
\[
\widehat{\operatorname{S}}_i= \hbar /2
\begin{pmatrix} \sigma_i & 0 \\ 0 & \sigma_i \end{pmatrix} =
\frac{\hbar}{2} \Sigma_i'
.\] 
\[
\left[ \widehat{\operatorname{S}}_i, \widehat{\operatorname{H}}_D \right] = \frac{\hbar}{2}
\left[ \widehat{\operatorname{\Sigma}}_i',\,\widehat{\operatorname{H}}_D \right] =
\frac{\hbar}{2} c \widehat{\operatorname{p}}_k \left[ 
\Sigma_i',\,\widehat{\operatorname{\alpha}}_k\right] +
\frac{\hbar }{2} m c^2 \left[ 
\Sigma_i',\,\beta\right] 
.\] 
\[
\left[ \Sigma_i',\,\beta \right] =
\begin{pmatrix} \sigma_i & 0 \\ 0 & \sigma_i \end{pmatrix} 
\begin{pmatrix} \widehat{\operatorname{1}} & 0 \\ 0&-\widehat{\operatorname{1}} \end{pmatrix} -
\begin{pmatrix} \widehat{\operatorname{1}}& 0\\ 0 &
-\widehat{\operatorname{1}}\end{pmatrix} 
	\begin{pmatrix} \sigma_i & 0\\ 0 & \sigma_i \end{pmatrix} =0
.\] 
\begin{multline*}
\left[ \Sigma_i',\,\alpha_k \right] =
\begin{pmatrix}\sigma_i & 0\\ 0 & \sigma_i  \end{pmatrix} 
\begin{pmatrix} 0 & \sigma_k\\ \sigma_k &0 \end{pmatrix} -
\begin{pmatrix} 0 & \sigma_k \\ \sigma_k & 0 \end{pmatrix} 
\begin{pmatrix} \sigma_i & 0 \\ 0&\sigma_i \end{pmatrix} =\\=
\begin{pmatrix} 0 & \sigma_i \sigma_k \\
\sigma_i \sigma_k &0\end{pmatrix} -
	\begin{pmatrix} 0 & \sigma_k \sigma_i\\
	\sigma_k \sigma_i &0\end{pmatrix} =
		\begin{pmatrix} 0 & \left[ \sigma_i,\,\sigma_k \right] \\ \left[ \sigma_i,\,\sigma_k \right] &0 \end{pmatrix} =\\=
		2i \epsilon _{ikl} \begin{pmatrix} 
		0 & \sigma_l \\ \sigma_l &0\end{pmatrix} 
		= 2i \epsilon _{ikl} \widehat{\operatorname{\alpha}}_l
.\end{multline*} 
Следовательно
\[
\left[ \widehat{\operatorname{S}}_i,\,\widehat{\operatorname{H}}_D \right] =
\frac{\hbar }{2}c \widehat{\operatorname{p}}_k \cdot i
\cdot 2 \cdot  \epsilon _{ikl} \widehat{\operatorname{\alpha}}_l=
-\hbar c \epsilon _{ikl} \widehat{\operatorname{\alpha}}_{k}
\widehat{\operatorname{p}}_l \cdot i
.\] 
Значит
\[
\left[ \widehat{\operatorname{J}}_i,\,\widehat{\operatorname{H}}_D \right] =
\left[ \widehat{\operatorname{L}}_i,\,\widehat{\operatorname{H}}_D \right] +
\left[ \widehat{\operatorname{S}}_i,\,\widehat{\operatorname{H}}_D \right] =0
.\] 
\end{enumerate}
\end{sol}
\begin{hiProb}[6]
\end{hiProb}
\begin{sol}
\[
	\mathbf{p}= \begin{pmatrix} 0 & 0 & p \end{pmatrix} 
.\] 
Уравнение Дирака:
\[
i \hbar \frac{\partial }{\partial t} \Psi= \widehat{\operatorname{H}}_D \Psi
.\] 
\[
\widehat{\operatorname{H}}_D= c \widehat{\operatorname{\boldsymbol{\alpha }}}
\cdot \widehat{\operatorname{\mathbf{p}}}+\beta mc^2
.\] 
\[
	\alpha_i = \begin{pmatrix} 0 & \sigma_i \\
	\sigma_i &0\end{pmatrix} ,\quad \beta=
		\begin{pmatrix} \widehat{\operatorname{1}}& 0\\
		0 & -\widehat{\operatorname{1}}\end{pmatrix} 
.\] 
Ищем решение в виде
\[
	\Psi_\alpha= U_\alpha (p) e^{\frac{i}{\hbar }
	\left( \mathbf{p} \mathbf{r} -\epsilon t \right) },\quad
	(\alpha=1,\,2,\,3,\,4)
.\] 
\[
	\epsilon  U= \begin{pmatrix} mc^2 & c
	\left( \boldsymbol{\rho},\,\boldsymbol{\sigma}\right)\\
c \left( \boldsymbol{\rho},\,\boldsymbol{\sigma} \right) &
-mc^2 \end{pmatrix} U
.\] 
\[
U= N \begin{pmatrix} \phi\\ \chi \end{pmatrix} ,\quad
U^\dagger U=1,\quad \phi^\dagger \phi=1
,\]
где $\phi,\ \chi$ --- биспиноры.
\[
	\begin{pmatrix} mc^2-\epsilon  & c \boldsymbol{\rho}
	\boldsymbol{\sigma}\\ c \boldsymbol{\rho}\boldsymbol{\sigma} & -mc^2 -\epsilon \end{pmatrix} \begin{pmatrix} \phi\\\chi \end{pmatrix} =0
,\]
откуда
\[
\left\{
\begin{aligned}
	(mc^2 -\epsilon ) \phi+ c \boldsymbol{\rho}\boldsymbol{\sigma}\chi&= 0, \\
	c \boldsymbol{\rho}\boldsymbol{\sigma} \phi-
	(mc^2 +\epsilon ) \chi&= 0. \\
\end{aligned}
\right.
\] 
Откуда
\[
\chi= \frac{c \boldsymbol{\rho \boldsymbol{\sigma}}}{\epsilon +mc^2}\phi
.\] 
\[
	(mc^2-\epsilon ) \phi + \frac{c^2 (\boldsymbol{\rho}\boldsymbol{\sigma})^2}{\epsilon +mc^2}=0
.\] 
Следовательно
\[
\epsilon ^2=p^2c^2+m^2 c^4
.\] 
\[
	1 = U^\dagger U=|N|^2 (\phi^\dagger \phi+
	\chi^\dagger \chi)= |N|^2 \left( 
	\phi^\dagger \phi+ \phi^\dagger
\frac{c \boldsymbol{\rho }\boldsymbol{\sigma}c \boldsymbol{\rho\sigma}}{(\epsilon +mc^2)^2}\phi\right) 
\xlongequal[]{*}
.\] 
Т.\:к.
\[
\sigma_i \sigma_k = i \epsilon _{ikl} \sigma_l +\delta_{ik},\quad
(\boldsymbol{\sigma\rho})^2= \sigma_i p_i \sigma_k p_k=
|\mathbf{p}|^2
,\]
то
\[
	\xlongequal[]{*}|N|^2 \left( 
	\phi^\dagger \phi +
\frac{c^2|\mathbf{p}|^2}{(\epsilon +mc^2)^2}\phi^\dagger \phi\right) =
|N|^2 \left( 1+ \frac{\epsilon -mc^2}{\epsilon +mc^2} \right) 
\implies N= \sqrt{\frac{\epsilon +mc^2}{2\epsilon }} 
.\] 
Пусть $\hbar =1$, тогда
\[
	\Sigma_i = \frac{1}{2} \begin{pmatrix}  \sigma_i & 0\\
	0 & \sigma_i\end{pmatrix} 
.\] 
\[
\left< \Sigma_i \right> = \frac{U^\dagger \Sigma_i U}{
U^\dagger U}= \frac{|N|^2}{2} \begin{pmatrix} \phi^\dagger
& \chi^\dagger\end{pmatrix} \begin{pmatrix} 
\sigma_i & 0\\ 0 & \sigma_i\end{pmatrix} \begin{pmatrix} 
\phi& \chi\end{pmatrix} =
\frac{|N|^2}{2} \left( \phi^\dagger \sigma_i \phi+
\chi^\dagger \sigma_i \chi\right) 
.\] 
$\mathbf{p}$ направлен вдоль оси $Z$, поэтому 
\[
	\chi= \frac{c |\mathbf{p}| \sigma_z \phi}{\epsilon +mc^2}\implies \left<\Sigma_i \right> =
	\frac{|N|^2}{2}\left( 
	\phi^\dagger \sigma_i \phi+  \phi^\dagger
	\frac{\sigma_z \sigma_i \sigma_z}{(\epsilon +mc^2)^2}
	\phi |\mathbf{p}|^2 c^2\right)
.\] 
Т.\:к. $\sigma_z \sigma_x \sigma_z=- \sigma_x$,
$\sigma_z \sigma_y \sigma_z=- \sigma_y$, то
\begin{multline*}
\left<\Sigma_{1,2} \right> =
\frac{|N|^2}{2} \left( 
\phi^\dagger \sigma_{1,2} \phi - \phi^\dagger
\sigma_{1,2} \phi \frac{|p|^2 c^2}{(\epsilon +mc^2)^2}\right) =\\=
\frac{\epsilon +mc^2}{4\epsilon } \left( 
1- \frac{\epsilon ^2 -m^2 c^4}{(\epsilon +mc^2)^2}\right) 
(\phi^\dagger \sigma_{1,2} \phi)=\\=
\frac{\epsilon +mc^2}{4\epsilon } \left( 
1- \frac{\epsilon -mc^2}{\epsilon +mc^2}\right) (\phi^\dagger
\sigma_{1,2} \phi) =\frac{mc^2}{2\epsilon }(\phi^\dagger\sigma_{1,2} \phi)
.\end{multline*} 
\begin{multline*}
	\left<\Sigma_3 \right> = \frac{|N|^2}{2} \left( 
		\phi^\dagger \sigma_z \phi+ \phi^\dagger
	\sigma_z \phi \frac{|\mathbf{p}|^2 c^2}{(\epsilon +mc^2)^2}\right) =\\=
	\frac{(\epsilon +mc^2)}{4\epsilon }
	(\phi^\dagger \sigma_z \phi) \left( 
	1+ \frac{\epsilon ^2 -m^2 c^4}{(\epsilon +mc^2)^2}\right) =\\=
	\frac{\epsilon +mc^2}{4\epsilon } \left( 
	\frac{\epsilon +mc^2 +\epsilon  -mc^2}{\epsilon +mc^2}\right) (\phi^\dagger \sigma_z \phi)=
	\frac{1}{2} (\phi^\dagger \sigma_z \phi)
.\end{multline*} 
Таким образом
\[
\left<\Sigma_{1,2} \right> = \frac{mc^2}{2\epsilon }
(\phi^\dagger \sigma_{1,2} \phi),\quad
\left<\Sigma_3 \right> =\frac{1}{2} (\phi^\dagger \sigma_z \phi)
.\] 
\end{sol}
\begin{hiProb}[7]
\end{hiProb}
\begin{sol}
\renewcommand{\labelenumi}{\asbuk{enumi})}
\begin{enumerate}
\item \[
\widehat{\operatorname{S}}_i= \widehat{\operatorname{S}}_{1i}
+ \widehat{\operatorname{S}}_{2i}= \frac{1}{2} (\sigma_1)_i
+\frac{1}{2} (\sigma_2)_i
.\] 
\[
	\widehat{\operatorname{\mathbf{S}}}^2= \widehat{\operatorname{\mathbf{S}}}_1^{2}+
	\widehat{\operatorname{\mathbf{S}}}_2^2 +2 \widehat{\operatorname{\mathbf{S}}}_1 \widehat{\operatorname{\mathbf{S}}}_2
.\] 
Так как $\left[ \widehat{\operatorname{\mathbf{S}}}^2,\,
\widehat{\operatorname{S}}_z\right] =0$, $\left[ 
\widehat{\operatorname{\mathbf{S}}}^2,\, \widehat{\operatorname{\mathbf{S}}}^2_1\right] =0$,
$\left[ \widehat{\operatorname{\mathbf{S}}}^2,\,
\widehat{\operatorname{\mathbf{S}}}_2^2\right] =0$,
$
	\left[ \widehat{\operatorname{\mathbf{S}_1^2}},\,
	\widehat{\operatorname{S}}_{1i}\right] =0
,$ 
$\left[ \widehat{\operatorname{\mathbf{S}_1^2}},\,
\widehat{\operatorname{\mathbf{S}_2^2}}\right] =0$,
то будем искать волновые функции, которые
являются собственными для следующего набора
операторов: $\widehat{\operatorname{\mathbf{S}^2}},\
\widehat{\operatorname{S}}_z,\ \widehat{\operatorname{\mathbf{S}_1^2}},\ \widehat{\operatorname{\mathbf{S}_2^2}}$.


Такие функции составляются из:
\[
	\widehat{\operatorname{\mathbf{S}_1^2}}
	\chi_{\pm  \frac{1}{2}}(1)=
	\frac{3}{4} \chi_{\pm \frac{1}{2}}(1)
.\] 
\[
	\widehat{\operatorname{\mathbf{S}_2^2}}
	\chi_{\pm  \frac{1}{2}}(2)=
	\frac{3}{4} \chi_{\pm \frac{1}{2}}(2)
.\]
\[
	\widehat{\operatorname{S}}_{1z}
	\chi_{\pm  \frac{1}{2}}(1)=
	\pm  \frac{1}{2} \chi_{\pm \frac{1}{2}}(1)
.\]
\[
	\widehat{\operatorname{S}}_{2z}
	\chi_{\pm  \frac{1}{2}}(2)=
	\pm  \frac{1}{2} \chi_{\pm \frac{1}{2}}(2)
.\]
Пусть $\Psi_{SM}$ --- собственные функции. Тогда
\[
\left\{
\begin{aligned}
	\widehat{\operatorname{\mathbf{S}^2}}
	\Psi_{SM}&= S(S+1)\Psi_{SM}\\
	\widehat{\operatorname{S}}_z\Psi_{SM}&=
	M \Psi_{SM}
\end{aligned}
\right.
.\] 
При $S=1$, $M=\pm 1$:
\[
	\Psi_{1,1}=\chi_{\frac{1}{2}}(1)\chi_{\frac{1}{2}}(2),\quad
	\Psi_{1,-1}=\chi_{-\frac{1}{2}}(1)
	\chi_{-\frac{1}{2}}(2)
.\] 
Понижающий оператор: $\widehat{\operatorname{S}}_-=
 \widehat{\operatorname{S}}_{1-}+ \widehat{\operatorname{S}}_{2-}$.
\[
\widehat{\operatorname{S}}_- \ket{1,\,1}=
\sqrt{2}  \ket{1,0}
,\]
т.\:к. $\widehat{\operatorname{j}}_{\pm }\ket{j,\,m}
=\sqrt{j(j+1)-m(m\pm 1)} \ket{j,\,m\pm 1}$.
Значит
\begin{multline*}
\Psi_{1,0}= \frac{1}{\sqrt{2} }
\widehat{\operatorname{S}}_- \Psi_{1,1}= \frac{1}{\sqrt{2} } \left( \widehat{\operatorname{S}}_{1-}+ \widehat{\operatorname{S}}_{2-} \right) 
\chi_{\frac{1}{2}}(1) \chi_{\frac{1}{2}}(2)=\\=
\frac{1}{\sqrt{2} }\left( \chi_{\frac{1}{2}}(1)
\chi_{-\frac{1}{2}}(2)+\chi_{-\frac{1}{2}}(1)
\chi_{\frac{1}{2}}(2)\right) 
.\end{multline*} 
Следовательно, спиновые функции для $S=1$ \emph{симметричны} по отношению к перестановкам
спиновых переменных.

$\Psi_{1,0}$ и $\Psi_{0,0}$ должны быть
ортогональны. Будем искать $\Psi_{0,0}$ в виде:
\[
	\Psi_{0,0}= A \chi_{\frac{1}{2}}(1)\chi_{-\frac{1}{2}}(2)+B\chi_{-\frac{1}{2}}(1)\chi_{\frac{1}{2}}(2)
.\] 
Из ортогональности $\Psi_{1,0}$, $\Psi_{0,0}$ 
следует, что $A+B=0$, значит
\[
	\Psi_{0,0}= \frac{1}{\sqrt{2} } \left( 
	\chi_{\frac{1}{2}}(1) \chi_{-\frac{1}{2}}(2) -\chi_{-\frac{1}{2}}(1) \chi_{\frac{1}{2}}(2)\right) 
.\] 
Волновые функции, отвечающие значению суммарного
спина $S=0$, \emph{антисимметричны} по отношению
к перестановкам  спиновых переменных.
\item Оператор углового момента:
	\[
	\widehat{\operatorname{J}}_i= \widehat{\operatorname{L}}_i +\widehat{\operatorname{S}}_i\implies
	\widehat{\operatorname{\mathbf{J}}}^2=
	\widehat{\operatorname{\mathbf{L}}}^2+
	\widehat{\operatorname{\mathbf{S}}}^2+
	2 \widehat{\operatorname{\mathbf{L}}}
	\widehat{\operatorname{\mathbf{S}}}
	.\] 
	Так как $\left[ \widehat{\operatorname{\mathbf{J}}}^2,\,J_z \right] =0$, $\left[ \widehat{\operatorname{\mathbf{J}J}}^2,
	\widehat{\operatorname{\mathbf{L}}}^2\right] =0$,
	$\left[ \widehat{\operatorname{\mathbf{J}}}^2,\,
	\widehat{\operatorname{\mathbf{S}}}^2\right] =0$,
	то будем искать волновую функцию, яввляющуюся
	собственной для $\widehat{\operatorname{\mathbf{J}}}^2$, $
	\widehat{\operatorname{J}}_z$, $
	\widehat{\operatorname{\mathbf{L}}}^2$,
	$\widehat{\operatorname{\mathbf{S}}}^2$.
Пусть $\hbar =1$. Тогда $\widehat{\operatorname{j}}_z= \widehat{\operatorname{l}}_z+ \widehat{\operatorname{s}}_z= - i \frac{\partial }{\partial \phi} + \frac{1}{2}\sigma_z$. Следовательно
\[
\left\{
\begin{aligned}
	\frac{\partial }{\partial \phi} \Psi_1&=i \left( 
	m-\frac{1}{2}\right) \Psi_1,\\
		\frac{\partial }{\partial \phi} \Psi_2&=
	i\left( m-\frac{1}{2} \right) \Psi_2,
\end{aligned}
\right.
\] 
где
\[
\Psi= \begin{pmatrix} \Psi_1\\\Psi_2 \end{pmatrix} ,\quad
J_z \Psi =m \Psi
.\] 
Откуда
\[
	\Psi_1= f(\theta) e^{i\left( m-\frac{1}{2} \right) \phi},\quad
	\Psi_2=f(\theta) e^{i \left( m+\frac{1}{2} \right) \phi}
.\] 
\[
	\widehat{\operatorname{\mathbf{L}}}^2 \Psi= l(l+1) \Psi
.\] 
Значит
\[
	\Psi_1= C_1 Y _l^{\left( m-\frac{1}{2} \right) },\quad
	\Psi_2=C_2 Y _l ^{\left( 
	m+\frac{1}{2}\right) },
\]
где $Y_l^{(m)}(\theta,\,\phi)$ --- сферические
функции.

По условию $\Psi_1$ и $\Psi_2$ должны удовлетворять
уравнению:
\[
	\widehat{\operatorname{\mathbf{J}}}^2\Psi=j(j+1)
	\Psi
.\] 
Матрицы Паули:
\[
	\sigma_1= \begin{pmatrix} 0 & 1\\ 1&0 \end{pmatrix} ,\quad \sigma_2
	\begin{pmatrix} 0 &-i\\ i & 0 \end{pmatrix} ,\quad
	\sigma_3 = \begin{pmatrix} 1 &0 \\ 0& -1 \end{pmatrix} 
.\] 
\[
	\widehat{\operatorname{\mathbf{J}}}^2=
	\widehat{\operatorname{\mathbf{L}}}^2+
	\widehat{\operatorname{\mathbf{S}}}^2+
	2 \widehat{\operatorname{\mathbf{L}}}
	\widehat{\operatorname{\mathbf{S}}}=
	\widehat{\operatorname{\mathbf{L}}}^2+
	\widehat{\operatorname{\mathbf{S}}}^2+
	l_i \sigma_i=
	\begin{pmatrix} \widehat{\operatorname{\mathbf{l}}}^2+ \frac{3}{4}+
	\widehat{\operatorname{l}}_z & \widehat{\operatorname{l}}_-\\
\widehat{\operatorname{l}}_+&
\widehat{\operatorname{\mathbf{l}}}^2+ \frac{3}{4}- \widehat{\operatorname{l}}_z\end{pmatrix} 
.\] 
\[
\widehat{\operatorname{j}}_{\pm }
\ket{j,\,m}= \sqrt{j(j+1)-m(m\pm 1)} 
\ket{j,\,m\pm 1}
.\] 
\begin{multline*}
	\widehat{\operatorname{\mathbf{J}}}^2
	\begin{pmatrix} \Psi_1\\ \Psi_2  \end{pmatrix} =
	\widehat{\operatorname{\mathbf{J}}}^2
	\begin{pmatrix} C_1 Y_l^{\left( m-\frac{1}{2} \right) }\\
	C_2 Y_l^{\left( m+\frac{1}{2} \right) }\end{pmatrix} =\\
	=\begin{pmatrix} \left( l(l+1)+\frac{3}{4}
	+m -\frac{1}{2}\right) C_1 Y_l^{\left( m-\frac{1}{2} \right) }+
\sqrt{l(l+1) -m^2 +\frac{1}{4}} C_2 Y_l^{\left( m-\frac{1}{2} \right) }\\
\sqrt{l(l+1)-m^2 +\frac{1}{4}} C_1 Y_{l}^{\left( m+\frac{1}{2} \right) }+\left( (l+1)l-m+\frac{1}{4} \right) C_2 Y_l^{\left( m+\frac{1}{2} \right) }\end{pmatrix} =\\=
j(j+1) \begin{pmatrix}  C_1 Y_l^{\left( m-\frac{1}{2} \right) }\\C_2 Y_l^{\left( m+\frac{1}{2} \right) }  
 \end{pmatrix} .\end{multline*} 
Следовательно
\[
\left\{
\begin{aligned}
	\left( l(l+1)+m+\frac{1}{4}-j(j+1) \right) C_1 +\sqrt{l(l+1)-m^2+\frac{1}{4}} C_2&= 0 ,\\
	\sqrt{l(l+1)-m^2+\frac{1}{4}} C_1+
	\left( l(l+1)-m+\frac{1}{4}-j(j+1) \right) C_2&= 0 .\\
\end{aligned}
\right.
\] 
Чтобы система имела нетривиальное решение:
\[
	\left( \left( l+\frac{1}{2} \right) ^2-
	j(j+1)\right) ^2-m^2 - \left( 
\left( l+\frac{1}{2} \right) ^2-m^2\right) =0
.\] 
\[
	j(j+1)= \left( l+\frac{1}{2} \right) ^2
	\pm  \left( l+\frac{1}{2} \right) =A
.\] 
\[
j^2\pm j-A=0
.\] 
Откуда
\[
j= \frac{-1+ \sqrt{1+4A} }{2}
\]
(т.\:к. $j>0$). Значит
\begin{align*}
	j_1 &= -\frac{1}{2}+ \sqrt{\frac{1}{4}+
	\left( l+\frac{1}{2} \right) \left( 
l+\frac{1}{2}+1\right) } =
l+1-\frac{1}{2}=l+\frac{1}{2},\\
		j_2&= -\frac{1}{2} + \sqrt{\frac{1}{4}+\left( 
		l+\frac{1}{2}\right) \left( 
l-\frac{1}{2}\right) } =l-\frac{1}{2}\\
.\end{align*}
Нормировка:
\[
	\int \left( |\Psi_1|^2+
	|\Psi_2|^2\right) d\Omega=|C_1|^2+|C_2|^2=1
.\] 
\renewcommand{\labelenumii}{\arabic{enumii})}
\begin{enumerate}
\item \[
j=l+\frac{1}{2}\implies l=j-\frac{1}{2}
.\] 
\[
	\left( j^2-\frac{1}{4}+m+\frac{1}{4}-j^2-j \right) C_1+\sqrt{jj^2-\frac{1}{4}-m^2+\frac{1}{4}}
	C_2=0
.\] 
\[
	C_2= \frac{(j-m)C_1}{\sqrt{j^2-m^2} }=
	\sqrt{\frac{j-m}{j+m}} C_1
.\] 
\[
C_1^2+\frac{j-m}{j+m}C_1^2=\frac{2j}{j+m}C_1^2=1
\implies \left\{
\begin{aligned}
C_1= \sqrt{\frac{j+m}{2j}} ,\\
C_2= \sqrt{\frac{j-m}{2j}}. 
\end{aligned}
\right. 
\] 
Следовательно
\[
	\Psi= Y_{lm}^{\left( l+\frac{1}{2} \right) }=
	\begin{pmatrix} \sqrt{\frac{j+1}{2j}} 
	Y_l^{\left( m-\frac{1}{2} \right)} (\theta,\,\phi)\\
\sqrt{\frac{j-1}{2j}} Y_l^{\left( m+\frac{1}{2} \right) }(\theta,\,\phi)\end{pmatrix} 
.\] 
\item \[
j=l-\frac{1}{2}\implies l=j+\frac{1}{2}
.\] 
\[
	\left( j^2+2j+\frac{3}{4}+m+\frac{1}{4}-j^2-j \right) C_1+ \sqrt{j^2+2j+\frac{3}{4}-m^2+\frac{1}{4}} C_2=0
.\] 
\[
	C_2= -\frac{(j+m+1)C_1}{\sqrt{j^2+2j+1-m^2} }
	=-\frac{(j+m+1)C_1}{\sqrt{(j+1)^2-m^2} }=
	- \sqrt{\frac{j+m+1}{j+1-m}} C_1
.\] 
\[
	C_1^2+\frac{j+m+1}{j+1-m}C_1^2=\frac{2(j+1)}{j+1-m}C_1^2=1\implies
	\left\{
	\begin{aligned}
		C_1&= \sqrt{\frac{j+1-m}{2(j+1)}} ,\\
		C_2&= - \sqrt{\frac{j+1+m}{2(j+1)}}. 
	\end{aligned}
	\right.
\] 
Откуда
\[
	\Psi=Y_{lm}^{\left( l-\frac{1}{2} \right) }=
	\begin{pmatrix} -\sqrt{\frac{j+1-m}{2(j+1)}} Y_l^{\left( m-\frac{1}{2} \right) }(\theta,\,\phi)\\
	-\sqrt{\frac{j+1+m}{2(j+1)}} Y_l^{\left( m+\frac{1}{2} \right) }(\theta,\,\phi)\end{pmatrix} 
.\] 
\end{enumerate}
\item
	\[
	\widehat{\operatorname{L}}_i= \widehat{\operatorname{L}}_{1i}+ \widehat{\operatorname{L}}_{2i}
	.\] 
	\[
	\widehat{\operatorname{L}}^2=
	\widehat{\operatorname{L}}_1^2+
	\widehat{\operatorname{L}}_2^2+
	2 \widehat{\operatorname{\mathbf{L}}}_1
	\widehat{\operatorname{\mathbf{L}}}_2
	.\] 
Т.\:к. $\left[ \widehat{\operatorname{L}}^2,\,\widehat{\operatorname{L}}_z \right]$,
$
\left[ \widehat{\operatorname{L}}^2,\,\widehat{\operatorname{L}}_1^2 \right] =0
,$ 
$
\left[ \widehat{\operatorname{L}}^2,\,\widehat{\operatorname{L}}_2^2 \right] =0
$,
то будем искать волновые функции, являющиеся
собственными для $\widehat{\operatorname{L}}^2,\ 
 \widehat{\operatorname{L}}_z,\ \widehat{\operatorname{L}}_1^2,\ \widehat{\operatorname{L}}_2^2$. Они составляются
 из:
\[
	\widehat{\operatorname{L}}_1^2 \phi_{l_1 m_1}(1)= l_1 (l_1+1) \phi_{l_1 m_1}(1)
.\] 
\[
	\widehat{\operatorname{L}}_{1z}\phi_{l_1 m_1}(1) = m_1 \phi_{l_1 m_1}(1) 
.\] 
\[
	\widehat{\operatorname{L}}_2^2 \phi_{l_2 m_2}(2)= l_2 (l_2+1) \phi_{l_2 m_2}(2)
.\] 
\[
	\widehat{\operatorname{L}}_{2z}\phi_{l_2 m_2}(2) = m_2 \phi_{l_2 m_2}(2) 
.\] 
Пусть $\Psi_{LM}$ --- искомые функции, тогда
\[
\widehat{\operatorname{L}}^2 \Psi_{LM}=
L(L+1) \Psi_{LM}
.\] 
\[
\widehat{\operatorname{L}}_z \Psi_{LM}=
M \Psi _{LM}
.\] 
\[
\Psi_{LM}= \sum_{}^{} C_{l_1 m_1 l_1 m_2}^{LM}
\phi_{l_1 m_1}\phi_{l_2 m_2}
.\] 
При $L=2$, $M=\pm 2$:
\[
	\Psi_{2,2}= \phi_{1,1}(1)\phi_{1,1}(2)
.\] 
\[
	\Psi_{2,-2}=\phi_{1,-1} (1)\phi_{1,-1}
	(2)
.\] 
Понижающий оператор: $\widehat{\operatorname{L}}_-
= \widehat{\operatorname{L}}_{1-} + \widehat{\operatorname{L}}_{2-}$,
\[
\widehat{\operatorname{L}}_- \Psi_{2,2}=
\sqrt{2\cdot 3-2} \Psi_{2,1}=2 \Psi_{2,1}
.\] 
\begin{multline*}
\Psi_{2,1}= \frac{1}{2} \widehat{\operatorname{L}}_-
\Psi_{2,2} = \frac{1}{2} \left( 
\widehat{\operatorname{L}}_{1-}+\widehat{\operatorname{L}}_{2-}\right) 
\phi_{1,1} (1) \phi_{1,1}(2)=\\=
\frac{1}{2} \left( \sqrt{2} \phi_{1,0}
(1) \phi_{1,1} (2) +\sqrt{2}  \phi_{1,1}(1)
\phi_{1,0} (2)\right) =\\=\frac{1}{\sqrt{2} }
\left( \phi_{1,0} (1) \phi_{1,1}(2)
+ \phi_{1,1}(1) \phi_{1,0} (2)\right) 
.\end{multline*} 
\[
\widehat{\operatorname{L}}_+ \Psi_{2,2}
= \sqrt{ 6+2\cdot (-1)} \Psi_{2,-1}=
2\cdot  \Psi_{2,-1}
.\] 
\[
\Psi_{2,-1}= \frac{1}{2} \widehat{\operatorname{L}}_+
\Psi_{2,-2}= \frac{1}{\sqrt{2} }
\left( \phi_{1,0}(1)\phi_{1,-1}(2)+
\phi_{1,-1}(1)\phi_{1,0}(2)\right) 
.\] 
\[
\widehat{\operatorname{L}}_- \Psi_{2,1}=
\sqrt{6} \Psi_{2,0}
.\] 
\begin{multline*}
\Psi_{2,0}= \frac{1}{\sqrt{6\cdot 2} }
\left( \widehat{\operatorname{L}}_{1-}+
\widehat{\operatorname{L}}_{2-}\right) 
\left( \phi_{1,0}(1)\phi_{1,1}(2)+
\phi_{1,1}(1) \phi_{1,0}(2)\right) =\\=
\frac{1}{\sqrt{12} } \left( 
\sqrt{2} \phi_{1,-1}(1) \phi_{1,1}(2)
+\sqrt{2}  \phi_{1,1} (1) \phi_{1,-1}(2)+
2\sqrt{2} \phi_{1,0}(1)\phi_{1,0}(2)\right) 
=\\= \frac{1}{\sqrt{6} } \left( 
\phi_{1,-1} (1)\phi_{1,1}(2)+\phi_{1,1}(1)
\phi_{1,-1}(2)+2\phi_{1,0}(1)\phi_{1,0}(2)\right) 
.\end{multline*} 
Коэффициенты Клебша-Гордана:
\[
C_{1,\pm 1,1,\pm 1}^{2,\pm 2}=1,\quad
C_{1,\pm 1,1,0}^{2,\pm 1}=C_{1,0,1,\pm 1}^{
2,\pm 1}= \frac{1}{\sqrt{2} },\quad
\] 
\[
C_{1,\pm 1,1,\mp 1}^{2,0}= \frac{1}{\sqrt{6} },\quad
C_{1,0,1,0}^{2,0}=\frac{2}{\sqrt{6} }=\sqrt{\frac{2}{3}} 
.\] 
Соотношение ортогональности:
\[
\sum_{m_1,m_2}^{} C_{j_1 m_1 j_2 m_2}^{
JM} C_{j_1 m_1 j_2 m_2}^{J' M'}=
\delta_{JJ'}\delta_{MM'}
.\] 
\[
\sum_{J,M}^{} C_{j_1 m_1 j_2 m_2}^{J M}
C_{j_1 m_1' j_2 m_2'}^{JM}=
\delta_{m_1 m_1'}\delta_{m_2 m_2'}
.\] 
Значит
\[
	\frac{1}{\sqrt{2} }\left( C_{1,1,1,0}^{1,1}+C_{1,0,1,1}^{1,1} \right) =0
.\] 
Из условия нормировки:
\[
\Psi_{1,1}= \frac{1}{\sqrt{2} }
\left( \phi_{1,0} (1) \phi_{1,1}(2) -\phi_{1,1}
(1) \phi_{1,0}(2)\right) 
.\] 
Аналогично:
\[
\Psi_{1,-1}= \frac{1}{\sqrt{2} }
\left( \phi_{1,0}(1) \phi_{1,1}(2)-\phi_{1,1}
(1) \phi_{1,-1}(2)\right) 
.\] 
\[
\widehat{\operatorname{L}}_{-} \Psi_{1,1}=
\sqrt{2}  \Psi_{1,0}
.\] 
Следовательно
\begin{multline*}
\Psi_{1,0}= \frac{1}{2}
\left( \widehat{\operatorname{L}}_{1-}+
\widehat{\operatorname{L}}_{2-}\right) 
\left( \phi_{1,0}(1)\phi_{1,1}(2)
-\phi_{1,1}(1)\phi_{1,0}(2)\right) =\\=
\frac{1}{\sqrt{2} }
\left( \phi_{1,-1}(1)\phi_{1,1}(2)-
\phi_{1,1}(1)\phi_{1,-1}(2)\right) 
.\end{multline*} 
Из соотношения ортогональности
при $M'=M=0,\ L'=2,\ L=0$:
\[
	\frac{1}{\sqrt{6} }\left( 
	C_{1,1,1,-1}^{0,0}+C_{1,-1,1,1}^{0,0}\right) +
	\frac{2}{\sqrt{6} }C_{1,0,1,0}^{0,0}
.\] 
И при $L'=1,\ L=0$:
\[
	\frac{1}{\sqrt{2} }\left( C_{1,1,1,-1}^{0,-}-C_{1,-1,1,1}^{0,0} \right) =0
.\] 
Поэтому
\[
C_{1,1,1,-1}^{0,0}=C_{1,-1,1,1}^{0,0}
.\] 
\[
C_{1,0,1,0}^{0,0}=-C_{1,1,1,-1}^{0,0}
.\] 
Используя условие нормировки получаем:
\[
	\Psi_{0,0}=\frac{1}{\sqrt{3} }\left( 
	\phi_{1,-1}(1)\phi_{1,1}(2)+
\phi_{1,1}(1)\phi_{1,-1}(2)-\phi_{1,0}(1)\phi_{1,0}(2)\right) 
.\] 
Функции $\Psi_{LM}$ при $L=0,\, 2$  \emph{симметричны}
по отношению к перестановке частиц, а при $L=1$ 
--- \emph{антисимметричны}.
\end{enumerate}
\end{sol}
\begin{hiProb}[8]
\end{hiProb}
\begin{sol}
Псевдовектор Паули-Любанского:
\[
W^\mu= -\frac{1}{2}\epsilon ^{\mu\nu\lambda\rho}
p_\nu S_{\lambda\rho}
,\]
где $S_{oa}=K^a$, $S_{ab}=\epsilon _{abc}J^c$,
$a,\,b=1,\,2,\,3$, $S_{\mu\nu}=-S_{\nu\mu}$.
$J^a$ и $K^a$ --- генераторы вращений и бустов.
 \[
W^2=W_\mu W^\mu= \frac{1}{4} \epsilon _{\mu\nu'\lambda'\rho'}p^{\nu'}S^{\lambda'\rho'}
\epsilon ^{\mu\nu\lambda\rho}p_\nu S_{\lambda\rho}
.\] 
\begin{multline*}
\epsilon _{\mu\nu'\lambda'\rho'}\epsilon ^{
\mu\nu\lambda\rho}=-
\begin{vmatrix} \delta_{\nu'}^\nu &
\delta_{\lambda'}^\nu & \delta_{\rho'}^\nu\\
\delta_{\nu'}^\lambda & \delta_{\lambda'}^\lambda & \delta_{\rho'}^\lambda\\
\delta_{\nu'}^\rho& \delta_{\lambda'}^\rho & \delta_{\rho'}^\rho\end{vmatrix} =\\=
-\left( \delta_{\nu'}^\nu \delta_{\lambda'}^\lambda
\delta_{\rho'}^\rho +\delta_{\lambda'}^\nu
\delta_{\rho'}^{\lambda}\delta_{\nu'}^\rho+
\delta_{\rho'}^\nu \delta_{\nu'}^\lambda
\delta_{\lambda'}^\rho\right. - \\ - \left. \delta_{\nu'}^\nu
\delta_{\nu'}^\nu\delta_{\rho'}^\lambda \delta_{\lambda'}^\rho-
\delta_{\lambda'}^\nu \delta_{\nu'}^\lambda
\delta_{\rho'}^\rho-
\delta_{\rho'}^\nu \delta_{\nu'}^\lambda
\delta_{\lambda'}^\rho\right) 
.\end{multline*} 
Значит
\begin{multline*}
	W_\mu W^\mu = -\frac{1}{4} \left( 
	p^2 S^2 + p^\rho p_\nu
S_{\lambda\rho}^{\nu\lambda}
+p^\lambda p_\nu S_{\lambda\rho}
S^{\rho \nu}\right. - \\ - \left.
p^2 S_{\lambda\rho} S^{\rho\lambda}
-p^\lambda p_\nu S_{\lambda \rho}
S^{\nu \rho}
-p^\lambda p_\nu S_{\lambda \rho
}S^{\rho \nu}\right) 
.\end{multline*} 
Следовательно
\[
	W^2= -\frac{1}{2} \left( 
	p^2 S^2-2p_\nu p^\lambda S_{\lambda\rho}
S^{\nu\rho}\right) 
.\] 
\end{sol}
\begin{hiProb}[9]
\end{hiProb}
\begin{sol}
\begin{enumerate}
\item Декартовы координаты.
\[
	\mathbf{H}=\rot \mathbf{A},\quad
	A_x=A_z=0,\quad
	A_y=\mathbf{H}\mathbf{x}\implies
	H_z=H,\quad H_y=H_x=0
.\] 
Гамильтониан системы:
\[
	\widehat{\operatorname{\mathcal{H}}}=
	\frac{1}{2m} \left[ \widehat{\operatorname{p}}_x +
	\left( \widehat{\operatorname{p}}_y-
\frac{e}{c} \mathbf{H}\mathbf{x}\right) ^2+
\widehat{\operatorname{p}}_z^2\right] 
.\] 
Будем искать стационарные состояния:
\[
	\widehat{\operatorname{\mathcal{H}}}\Psi
	=E \Psi
.\] 
Заметим, что:
\[
\left[ \widehat{\operatorname{p}}_z,\,
\widehat{\operatorname{\mathcal{H}}}\right] =0,\quad
\left[ \widehat{\operatorname{p}}_z,\,
\widehat{\operatorname{\mathcal{H}}}\right] =0
.\] 
Значит, будем искать волновые функции в виде
\[
\Psi_{E,p_y,p_z}= \frac{1}{2\pi \hbar }
\exp \left[ 
\frac{i}{\hbar } \left( p_y y+p_z z \right) \right] 
\tilde{\Psi}(x)
,\]
где коэффициент перед экспонент подобран из
условия нормировки. Поэтому
\[
	\tilde{\Psi}''(x)+\frac{2m}{\hbar ^2}
	\left[ \left( E-\frac{p_z^2}{2m}-
	\right)-\frac{m}{2}
\omega_H^2 \left( x-x_0 \right) ^2\right] 
\tilde{\Psi}(x)=0
,\]
где $\omega_H= \frac{eH}{mc}$, $x_0= \frac{cp_y}{eH}$.

Решение данного уравнения выражается
через решение уравнения Шрёдингера для 
гармонического осциллятора с частотой
$\omega_H$:
\[
\Psi_{n,p_y,p_z}= \frac{1}{2\pi \hbar }
\exp \left( \frac{i}{\hbar }
\left( p_y y+p_z z \right) \right) 
\Psi_n^{\text{осц}}(x-x_0)
.\] 
\[
	E_n= \hbar  \omega_H \left( n+\frac{1}{2} \right) + \frac{p_z^2}{2m}
\] 
--- уровни Ландау. Уровни энергии не зависят
от $p_y$, следовательно --- вырождены.
Ограничим область движения ящиком в виде
параллелепипеда со сторонами $L_x,\ L_y,\
L_z$ ($V=L_x L_y L_z$).
\begin{multline*}
	\Psi(x_i+L)=\Psi_{x_i}\implies
	\exp  \left( \frac{i}{\hbar }
	p_i x_i\right) = \exp \left( 
\frac{i}{\hbar } p_i (x_i+L_i)\right) \implies
\\ \implies
	p_i= \frac{2\pi \hbar }{L}
	k_i,\quad k_i= 0,\pm 1,\pm 2\ldots
\end{multline*} 
Следовательно число квантовых состояний
в объёме $V$ в расчёте на интервал $\Delta p_z$ 
и один дискретный уровень:
\[
	\Delta N= \frac{L_y L_z}{(2\pi \hbar )^2}
	\Delta p_y \Delta p_z
.\] 
\begin{multline*}
0<x_0<L_x \implies
\frac{\Delta p_y c}{eH}= Lx \implies
\Delta p_y= \frac{eH L_x}{c}\implies\\\implies
\Delta N= \frac{\Delta p_z}{(2\pi \hbar )^2}
\frac{eH}{c} L_x L_y L_z \implies
\Delta N= \frac{eH}{c} \frac{V}{(2\pi \hbar )^2}
\Delta p_z
\end{multline*} 
--- кратность вырождения.
\item В цилиндрических координатах.
	\[
		\widehat{\operatorname{\mathbf{A}}}=
		\frac{1}{2}
		\mathbf{H}\times
		\mathbf{r},\quad
		\mathbf{H} \parallel z \implies
		A_z=A_\rho=0,\ A_\phi= \frac{1}{2}
		H \rho
	.\] 
\[
A_x= -\frac{1}{2} Hy,\quad A_y= \frac{1}{2}
Hx
.\] 
\[
	\widehat{\operatorname{\mathcal{H}}}
	= \frac{1}{2m} \left( 
	\left( \widehat{\operatorname{p}}_x-
\frac{e}{c} \widehat{\operatorname{A}}_x\right) ^2 +\left( \widehat{\operatorname{p}}_y-
\frac{e}{c} \widehat{\operatorname{A}}_y\right) +\widehat{\operatorname{p}}_z^2\right) 
.\] 
\[
	\widehat{\operatorname{\mathcal{H}}}=
	\frac{1}{2m} \left( 
	\widehat{\operatorname{p}}_x^2 +
\widehat{\operatorname{p}}_y^2 +
\widehat{\operatorname{p}}_z^2 +
\frac{e^2}{c^2} \left( A_x^2 +A_y^2 \right) -
\frac{2e}{c}\left( A_x p_x +A_y p_y \right) \right) 
.\] 
\[
	\frac{e}{c}H\left( y\widehat{\operatorname{p}}_x -x \widehat{\operatorname{p}}_y \right) =
	-\frac{e}{c} H \widehat{\operatorname{L}}_z=
	i \hbar  \frac{e}{c} H \frac{\partial }{\partial \phi} 
.\] 
\[
	\widehat{\operatorname{\mathcal{H}}}=
	- \frac{\hbar ^2}{2m}\Delta
	+\frac{e^2}{c^2}
	\frac{H^2}{8m} \left( x^2+y^2 \right) 
	+ i \frac{e\hbar }{2mc}
	H \frac{\partial }{\partial \phi} 
.\] 
\[
	\widehat{\operatorname{\mathcal{H}}}
	= - \frac{\hbar ^2}{2m}
	\left( \frac{\partial ^2}{\partial z^2} +
	\frac{1}{\rho} \frac{\partial }{\partial \rho} \rho \frac{\partial }{\partial \rho} +
\frac{1}{\rho^2} \frac{\partial ^2}{\partial \phi^2} \right) +
\frac{i}{2} \hbar  \omega_H \frac{\partial }{\partial \phi} +
\frac{1}{8} m\omega_H^2 \rho^2
.\] 
Т.\:к.
\[
	\left[ \widehat{\operatorname{\mathcal{H}}},\,
	\widehat{\operatorname{p}}_z\right] =0,
	\quad \left[ 
	\widehat{\operatorname{\mathcal{H}}},\,
\widehat{\operatorname{L}}_z\right] =0
,\] 
то будем искать волновые функции, удовлетворяющие
следующим условиям:
\[
\widehat{\operatorname{p}}_z \Psi =p_z
\Psi;\quad
\widehat{\operatorname{\mathcal{H}}}\Psi
=E\Psi;\quad
\widehat{\operatorname{L}}_z\Psi=
\hbar  M\Psi
.\] 
\[
	\Psi(\rho,\,z,\,\phi)=
	\Psi_z (z) \Phi(\phi) R(\rho)
.\] 
\[
	\Psi_z (z)= \frac{1}{\sqrt{2 \pi \hbar } }
	e^{\frac{i p_z z}{\hbar }};\quad
	\Phi(\phi)= \frac{1}{\sqrt{2\pi} }
	e^{iM\phi}
.\] 
Значит при нормировке
\[
	\int\limits_{0}^{\infty} |R(\rho)|^2
	\rho d\rho=1
\] 
получаем
\[
	- \frac{\hbar ^2}{ 2m}\left( 
	\frac{1}{\rho} \frac{d}{d\rho}
\rho \frac{dR}{d\rho}
-\frac{M^2}{\rho^2} R\right) -
\frac{1}{2} \hbar  \omega_H M R+
\frac{1}{8} M \omega_H^2 \rho^2 R=
\underbrace{\left( E- \frac{p_z^2}{2M} \right) }_{
=E_H}R
\]
Пусть
\[
\epsilon= \frac{E_H}{\hbar  \omega_H},\quad
\xi= \sqrt{ \frac{2m \omega_H}{\hbar }}\rho= \frac{\sqrt{2} \rho}{\rho_H}
.\] 
Тогда
\[
R''+\frac{1}{\xi} R' +
\left( \epsilon - \frac{M^2}{\xi^2}+\frac{M}{2}
-\frac{\xi^2}{16}\right) R=0
.\] 
При $\xi\to 0$ получаем:
\[
R''+ \frac{1}{\xi}R'- \frac{M^2}{\xi^2}R=0,\quad
R \sim  \xi^s \implies
s(s-1)+s -M^2=0 \implies s=\pm M
.\] 
\[
R|_{\xi\to 0} \sim  \xi^{|M|}
.\] 
При $\xi \to \infty$:
\[
R''- \frac{1}{16}\xi^2 R=0,\quad
R \sim  \exp \left( -\alpha \xi^2 \right) 
.\] 
Значит
\[
	\left( -2\alpha \xi e^{-\alpha \xi^2} \right) '- \frac{1}{16} \xi^2 e^{-\alpha \xi^2}=0
.\] 
\[
-2\alpha e^{-\alpha \xi^2}+
4 \alpha^2 \xi^2 e^{-\alpha \xi^2}-\frac{1}{16}
\xi^2 e^{-\alpha \xi^2}=0
.\] 
\[
-\frac{2\alpha}{\xi^2}+4\alpha^2-\frac{1}{16}=0
.\] 
Поэтому
\[
\alpha=\frac{1}{8}
,\]
т.\:к. $\xi\to \infty$. Значит
\[
	R(\xi)= e^{-\frac{\xi^2}{8}}\xi^{|M|}
	W(\xi)
.\] 
Для $u= \xi^2 /4$ получим
\[
u \frac{d^2 W}{du^2}+
\left( |M|+1-u \right) \frac{dW}{du}
+\left( \epsilon - \frac{|M|-M+1}{2} \right) 
W=0
.\] 
Раскладывая $\displaystyle W= \sum_{k=0}^{\infty} a_k u^k$, получаем
\begin{multline*}
\sum_{k=0}^{\infty} \left[ \left( k(k+1)+
\left( |M|+1 \right) (k+1)\right) a_{k+1}+
\left( \epsilon -k - \frac{|M|-M+1}{2} \right) a_k\right] =0
.\end{multline*} 
Откуда
\[
	a_{k+1}= a_k \frac{2k-2\epsilon  +|M|-M+1}{2(k+1)(k+|M|+1)}\sim \frac{a_k}{k} \text{ при }
	k\gg 1
,\] 
т.\:е. если ряд не обрывается, то $W \sim 
e^u$ --- ненормируемое решение. Значит
\[
\epsilon = n_\rho + \frac{|M|-M+1}{2},\quad
n_\rho= 0,\,1,\,2,\ldots \text{ --- радиальное кв. число}
.\] 
Энергетический спектр:
\[
	E_{n \rho_z}= \hbar  \omega_H \left( 
	n+\frac{1}{2}\right) + \frac{p_z^2}{2m},\quad
	n= n_\rho+ \frac{|M|-M}{2}= 0,\,1,\,2
\]
($n$ --- главное квантовое число).

Уровни энергии вырождены по значениям проекции
момента на направление магнитного поля: каждому
фиксированному значению $n$ соответствуют состояния 
с $M=-n,-n+1,\ldots,\,0,\,1,\ldots,\,\infty$.
\[
	R_{n_\rho M}(\rho)=
	\frac{1}{\rho_H \sqrt{n_\rho r! \left( 
	|M|+n_\rho\right) !}} \exp 
	\left( - \frac{\rho^2}{4\rho_H^2} \right) 
	\left( \frac{\rho}{\sqrt{2} \rho_H} \right) 
	^{|M|} \underbrace{L_{n_\rho}^{|M|}\left( 
	\frac{\rho^2}{2\rho_H^2}\right) }_{\substack{\text{присоед. полин.}\\\text{Лагерра}}}
.\] 
\end{enumerate}
\end{sol}
\begin{hiProb}[12]
\end{hiProb}
\begin{sol}
Обобщая уравнения Клейна-Фока-Гордона
на случай присутствия ЭМ полей и учитывая
$\mathbf{A}=\mathbf{0}$ (поле --- электрическое)
получаем
\[
	\left[ \left( i \frac{\hbar}{c}
	\frac{\partial }{\partial t}- e \phi \right) ^2 + \left( \hbar  \frac{\partial }{\partial x_i}  \right) ^2-
(mc)^2\right] \Phi=0
.\] 
Ищем решение в виде:
\[
	\Phi= \Psi(x)e^{-\frac{iEt}{\hbar }}
.\] 
\[
	\left[ \left( E-e\phi \right) ^2-
	c^2 \widehat{\operatorname{p}}^2-
m^2 c^4 \right] \Psi=0
.\] 
\[
	\left( E^2- m^2 c^4 \right)\Psi
	=\left[ c^2 \widehat{\operatorname{p}}^2
	+2 E e \phi - e^2 \phi^2\right] \Psi
.\] 
\[
\frac{E^2 -m^2 c^4}{2mc^2}\Psi=
\left( \frac{\widehat{\operatorname{p}}^2}{2m}
+\underbrace{\frac{E}{mc^2}e \phi- \frac{e^2\phi^2}{2mc^2}}_{U_{\text{eff}}}\right) 
\Psi
.\] 
\[
	U_{\text{eff}}= \frac{E}{mc^2}
	e\phi- \frac{e^2 \phi^2}{2mc^2}
.\] 
При энергии частицы $E \simeq mc^2$, если
поле достаточно сильное (т.\:е. $e\phi >2mc^2$),
то эффективный потенциал $U_{\text{eff}}<0$ и 
частица испытывает притяжение.
\end{sol}
\section*{Второе задание}
\begin{hiProb}[13]
\end{hiProb}
\begin{sol}
Для расчётов используем выражение оператора
координаты $\widehat{\operatorname{x}}$ 
для гармонического осциллятора через операторы
рождения и уничтожения ($x_0$ --- осцилляторная
единица длины)
\[
\widehat{\operatorname{x}}= \frac{x_0}{\sqrt{2} }
\left( \widehat{\operatorname{a}}^+ +
\widehat{\operatorname{a}}\right) ,\quad
x_0= \sqrt{ \frac{\hbar }{m\omega}} 
.\] 
\renewcommand{\labelenumi}{\asbuk{enumi})}
\begin{enumerate}
\item Поправка первого приближения
	\[
	E_n^{(1)}= \bra{n}\widehat{\operatorname{V}}\ket{n}=
	\alpha\bra{n}\widehat{\operatorname{x}}\ket{n}
	=
\frac{\alpha x_0}{\sqrt{2} }
\bra{n} \widehat{\operatorname{a}}^+ + \widehat{\operatorname{a}}\ket{n}=0
	.\] 
Поправка второго приближения
\[
E_n^{(2)}= \sum_{k\neq n}^{} \frac{|V_{nk}|^2}{E_n^{(0)}-E_k^{(0)}}=
\frac{\alpha^2 x_0^2}{2}\left( 
\frac{n}{\hbar  \omega}- \frac{n+1}{\hbar \omega}\right) = -\frac{\alpha^2}{2m\omega^2}
,\] 
где $V_{nk}= \bra{n} \widehat{\operatorname{V}}\ket{k}$, $E_n^{(0)}=
 \hbar \omega (n+1 /2)$.
\item Можно показать, что
	\[
		\braket{\widehat{\operatorname{x}}^4} =
	\bra{n} \widehat{\operatorname{x}}^4\ket{n}=
	\frac{x_0^4}{4} \left( 6n^2+6n+3 \right) 
	,\]
\[
	\braket{\widehat{\operatorname{x}}^{2k+1}}=
	\bra{n}\widehat{\operatorname{x}}^{2k+1}
	\ket{n}=0,\quad k=0,\,1,\ldots
\]
Тогда знаем поправку первого приближения для слагаемого $Bx^4$, для $Ax^3$ необходимо
вычислять поправку второго приближения. Суммарная
поправка равна
\[
	\Delta E_n= \frac{3}{2} B \left( \frac{\hbar}{m\omega} \right) ^2 \left[ 
	n^2+n+\frac{1}{2}\right] -\frac{15}{4}
	\frac{A^2}{\hbar  \omega}
	\left( \frac{\hbar }{m\omega} \right) ^3
	\left[ n^2 +n+ \frac{11}{30} \right] 
.\] 
\end{enumerate}
\end{sol}
\begin{hiProb}[14]
\end{hiProb}
\begin{sol}
\begin{figure}[ht]
    \centering
    \incfig[0.5]{1}
    \caption{}
    \label{fig:1}
\end{figure}
\renewcommand{\labelenumi}{\asbuk{enumi})}
\begin{enumerate}
\item \[
		\boldsymbol{\mathcal{E}}=
		\frac{Ze}{R^2}\mathbf{e}_z
.\] 
\[
	\mathbf{d}= \sum_{i}^{} e_i
	\mathbf{r}_i= e \sum_{i}^{} \mathbf{r}_i
.\] 
\[
	\widehat{\operatorname{V}}= -\boldsymbol{\mathcal{E}} \widehat{\operatorname{\mathbf{d}}}=
	- \sum_{i}^{} \frac{Ze^2}{R^2} \widehat{\operatorname{z}}_i
.\] 
\[
E^{(1)}= \bra{e_1e_2\ldots e_n}\widehat{\operatorname{V}}\ket{e_i \ldots e_n}
=0
.\] 
\[
E^{(2)}= \sum_{k \neq n}^{}
\frac{\left| \bra{\mathbf{e}_k}\widehat{\operatorname{V}}
\ket{\mathbf{e}_n}\right| ^2}{E_n^{(0)}-E_k^{(0)}}=
\frac{Z^2 e^4}{R^4} \sum_{k\neq n}^{} 
\frac{\left\ket{ \bra{\mathbf{e}_k}\sum_{i}^{} \widehat{\operatorname{z}}_i|\mathbf{e}_n} \right| }{}
.\] 
\end{enumerate}
\end{sol}
\end{document}
