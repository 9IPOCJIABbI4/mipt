\documentclass[a4paper]{article}
% Этот шаблон документа разработан в 2014 году
% Данилом Фёдоровых (danil@fedorovykh.ru) 
% для использования в курсе 
% <<Документы и презентации в \LaTeX>>, записанном НИУ ВШЭ
% для Coursera.org: http://coursera.org/course/latex .
% Исходная версия шаблона --- 
% https://www.writelatex.com/coursera/latex/5.3

% В этом документе преамбула

\usepackage{siunitx}
%%% Работа с русским языком
\usepackage{cmap}					% поиск в PDF
\usepackage{mathtext} 				% русские буквы в формулах
\usepackage[T2A]{fontenc}			% кодировка
\usepackage[utf8]{inputenc}			% кодировка исходного текста
\usepackage[english,russian]{babel}	% локализация и переносы
\usepackage{indentfirst}
\frenchspacing

\renewcommand{\epsilon}{\ensuremath{\varepsilon}}
\renewcommand{\phi}{\ensuremath{\varphi}}
\renewcommand{\kappa}{\ensuremath{\varkappa}}
\renewcommand{\le}{\ensuremath{\leqslant}}
\renewcommand{\leq}{\ensuremath{\leqslant}}
\renewcommand{\ge}{\ensuremath{\geqslant}}
\renewcommand{\geq}{\ensuremath{\geqslant}}
\renewcommand{\emptyset}{\varnothing}
\renewcommand{\Im}{\operatorname{Im}}
\renewcommand{\Re}{\operatorname{Re}}


%%% Дополнительная работа с математикой
\usepackage{amsmath,amsfonts,amssymb,amsthm,mathtools} % AMS
\usepackage{icomma} % "Умная" запятая: $0,2$ --- число, $0, 2$ --- перечисление

%% Номера формул
%\mathtoolsset{showonlyrefs=true} % Показывать номера только у тех формул, на которые есть \eqref{} в тексте.
%\usepackage{leqno} % Нумереация формул слева

%% Свои команды
\DeclareMathOperator{\sgn}{\mathop{sgn}}
\DeclareMathOperator{\sign}{\mathop{sign}}
\DeclareMathOperator*{\res}{\mathop{res}}
\DeclareMathOperator*{\tr}{\mathop{tr}}

%% Перенос знаков в формулах (по Львовскому)
\newcommand*{\hm}[1]{#1\nobreak\discretionary{}
{\hbox{$\mathsurround=0pt #1$}}{}}

%%% Работа с картинками
\usepackage{graphicx}  % Для вставки рисунков
\graphicspath{{figures/}}  % папки с картинками
\setlength\fboxsep{3pt} % Отступ рамки \fbox{} от рисунка
\setlength\fboxrule{1pt} % Толщина линий рамки \fbox{}
\usepackage{wrapfig} % Обтекание рисунков текстом

%%% Работа с таблицами
\usepackage{array,tabularx,tabulary,booktabs} % Дополнительная работа с таблицами
\usepackage{longtable}  % Длинные таблицы
\usepackage{multirow} % Слияние строк в таблице

%%% Теоремы
\theoremstyle{plain} % Это стиль по умолчанию, его можно не переопределять.
\newtheorem{theorem}{Теорема}
\newtheorem*{thm}{Теорема}
\newtheorem{prop}{Утверждение}
 
\theoremstyle{definition} % "Определение"
%\newtheorem{corollary}{Следствие}[theorem]
\newtheorem*{dfn}{Определение}
\newtheorem{problem}{Задача}
\newtheorem*{problem*}{Задача}

 
\theoremstyle{remark} % "Примечание"
\newtheorem*{sol}{Решение}
\newtheorem*{rem}{Замечание}

%%% Программирование
\usepackage{etoolbox} % логические операторы

%%% Страница
%\usepackage{extsizes} % Возможность сделать 14-й шрифт
%\usepackage{geometry} % Простой способ задавать поля
%	\geometry{top=25mm}
%	\geometry{bottom=35mm}
%	\geometry{left=35mm}
%	\geometry{right=20mm}
 
\usepackage{fancyhdr} % Колонтитулы
%	\pagestyle{fancy}
 %	\renewcommand{\headrulewidth}{0pt}  % Толщина линейки, отчеркивающей верхний колонтитул
	%\lfoot{Нижний левый}
	%\rfoot{Нижний правый}
	%\rhead{Верхний правый}
	%\chead{Верхний в центре}
	%\lhead{Верхний левый}
	%\cfoot{Нижний в центре} % По умолчанию здесь номер страницы

\usepackage{setspace} % Интерлиньяж
%\onehalfspacing % Интерлиньяж 1.5
%\doublespacing % Интерлиньяж 2
%\singlespacing % Интерлиньяж 1

\usepackage{lastpage} % Узнать, сколько всего страниц в документе.

\usepackage{soul} % Модификаторы начертания

\usepackage{hyperref}
%\usepackage[usenames,dvipsnames,svgnames,table,rgb]{xcolor}
\hypersetup{				% Гиперссылки
    unicode=true,           % русские буквы в раздела PDF
    pdftitle={Заголовок},   % Заголовок
    pdfauthor={Автор},      % Автор
    pdfsubject={Тема},      % Тема
    pdfcreator={Создатель}, % Создатель
    pdfproducer={Производитель}, % Производитель
    pdfkeywords={keyword1} {key2} {key3}, % Ключевые слова
    colorlinks=true,       	% false: ссылки в рамках; true: цветные ссылки
    linkcolor=red,          % внутренние ссылки
    citecolor=black,        % на библиографию
    filecolor=magenta,      % на файлы
    urlcolor=cyan           % на URL
}

\usepackage{csquotes} % Еще инструменты для ссылок

%\usepackage[style=apa,maxcitenames=2,backend=biber,sorting=nty]{biblatex}

\usepackage{multicol} % Несколько колонок

\usepackage{tikz} % Работа с графикой
\usepackage{pgfplots}
\usepackage{pgfplotstable}
%\usepackage{coloremoji}
\usepackage{floatrow}
\usepackage{subcaption}
\newcommand*{\N}{\mathbb{N}}
\newcommand*{\R}{\mathbb{R}}
\newcommand*{\K}{\mathbb{K}}
\newcommand*{\V}{\mathcal{V}}
\newcommand*{\A}{\mathcal{A}}
\newcommand*{\ii}{\mathbf{1}}
\newcommand*{\oo}{\mathbf{0}}
\newcommand*{\ba}{\mathbf{a}}
\newcommand*{\bb}{\mathbf{b}}
\newcommand*{\Q}{\mathbb{Q}}
\graphicspath{{figures/}}
%\usepackage{breqn}

\renewcommand\thesubfigure{\asbuk{subfigure}}
%\addbibresource{master.bib}

\usepackage{import}
\usepackage{pdfpages}
\usepackage{transparent}
\usepackage{xcolor}
\usepackage{xifthen}

%\newcommand{\incfig}[1]{%
%    \def\svgwidth{\columnwidth}
%    \import{./figures/}{#1.pdf_tex}
%}


\newcommand{\incfig}[2][1]{%
    \def\svgwidth{#1\columnwidth}
    \import{./figures/}{#2.pdf_tex}
}
\usepackage{titlesec}
%\titleformat{\section}{\normalfont\Large\bfseries}{}{0pt}{}
%----------------------STANDART:
%\titleformat{\chapter}[display]
%  {\normalfont\huge\bfseries}{\chaptertitlename\ \thechapter}{20pt}{\Huge}
%\titleformat{\section}{\normalfont\Large\bfseries}{\thesection}{1em}{}
%\titleformat{\subsection}
%  {\normalfont\large\bfseries}{\thesubsection}{1em}{}
%\titleformat{\subsubsection}
%  {\normalfont\normalsize\bfseries}{\thesubsubsection}{1em}{}
%\titleformat{\paragraph}[runin]
%  {\normalfont\normalsize\bfseries}{\theparagraph}{1em}{}
%\titleformat{\subparagraph}[runin]
%  {\normalfont\normalsize\bfseries}{\thesubparagraph}{1em}{}

\pdfsuppresswarningpagegroup=1
\pgfplotsset{compat=1.16}

\usepackage{xifthen}
\makeatother
%\def\@lecture{}%
%\newcommand{\lecture}[3]{
%    \ifthenelse{\isempty{#3}}{%
%        \def\@lecture{Неделя #1}%
%    }{%
%        \def\@lecture{Неделя #1: #3}%
%    }%
%    \section*{\@lecture}
%    \marginpar{\small\textsf{\mbox{#2}}}
%}
\makeatletter

%\newcommand{\lec}{\subsection{Лекция}}
%\newcommand{\sem}{\subsection{Семинар}}
%\newcommand{\hw}{\subsection{Домашняя работа}}
%\newcommand{\prob}[1]{\textbf{#1}}
%\renewcommand{\thesubsection}{}
%\renewcommand{\thesubsubsection}{}

%\setcounter{tocdepth}{1} % only parts,chapters,sections
%\titleformat{\subsection}{\normalfont\large\bfseries}{}{0em}{}
%\titleformat{\subsubsection}{\normalfont\normalsize\bfseries}{}{0em}{}

%\newcommand{\textover}[2]{\stackrel{\mathclap{\normalfont\mbox{#2}}}{#1}}

\author{Драчов Ярослав\\
Факультет общей и прикладной физики МФТИ}
\newcommand{\veq}{\mathrel{\rotatebox{90}{$=$}}}
%\newcommand{\teto}[1]{\stackrel{\mathclap{\normalfont\tiny\mbox{#1}}}{\to}}
%\renewcommand{\thesubsection}{\arabic{subsection}}

%%\setcounter{secnumdepth}{0}

\definecolor{tabblue}{RGB}{30, 119, 180}
\definecolor{taborange}{RGB}{255, 127, 15}
\definecolor{tabgreen}{RGB}{45, 160, 43}
\definecolor{tabred}{RGB}{214, 38, 40}
\definecolor{tabpurple}{RGB}{148, 103, 189}
\definecolor{tabbrown}{RGB}{140, 86, 76}
\definecolor{tabpink}{RGB}{227, 119, 193}
\definecolor{tabgray}{RGB}{127, 127, 127}
\definecolor{tabolive}{RGB}{188, 189, 33}
\definecolor{tabcyan}{RGB}{22, 190, 207}
\pgfplotscreateplotcyclelist{colorbrewer-tab}{
{tabblue},
{taborange},
{tabgreen},
{tabred},
{tabpurple},
{tabbrown},
{tabpink},
{tabgray},
{tabolive},
{tabcyan},
}
\usepackage{csvsimple}
\usepackage{extarrows}
%\renewcommand{\labelenumii}{\asbuk{enumii})}
%\renewcommand{\labelenumiv}{\Asbuk{enumiv}}
\newcommand{\prob}[1]{\subsubsection*{#1}}
\sisetup{output-decimal-marker = {,},separate-uncertainty = true,exponent-product = \cdot}

\usepackage{braket}
\usepackage{enumerate}
\usepackage{chngcntr}
%\counterwithin*{equation}{problem}
%\usepackage{bbold}

\newtheoremstyle{hiProb}% ⟨name ⟩ 
{3pt}% ⟨Space above ⟩1 
{3pt}% ⟨Space below ⟩1
{}% ⟨Body font ⟩
{}% ⟨Indent amount ⟩2
{\bfseries}% ⟨Theorem head font⟩
{.}% ⟨Punctuation after theorem head ⟩
{.5em}% ⟨Space after theorem head ⟩3
%{\thmname{#1} \thmnote{#3}}% ⟨Theorem head spec (can be left empty, meaning ‘normal’)⟩
{\thmnote{#3}}% ⟨Theorem head spec (can be left empty, meaning ‘normal’)⟩
\theoremstyle{hiProb} % "Определение"
%\newtheorem{hiProb}{Задача}
\newtheorem{hiProb}{}
\usepackage{mmacells}
\newcommand{\textover}[2]{\stackrel{\mathclap{\normalfont\scriptsize\mbox{#2}}}{#1}}
\usepackage{units}
\usepackage[math]{cellspace}%
\setlength\cellspacetoplimit{2pt}
\setlength\cellspacebottomlimit{2pt}

\title{Домашняя работа по квантовой механике}
\begin{document}
	\maketitle
\section*{Первое задание}
\begin{hiProb}[1]
Вычислите интеграл по траекториям:
\[
	Z[\mathbf{J}(\cdot )]= \int\limits_{\mathbf{z}(0)=
	\mathbf{y},\ \mathbf{z}(t)=\mathbf{x}}^{} 
	\mathcal{D} \mathbf{z}(\tau) e^{
	i \int\limits_{0}^{t} d\tau \left[ 
\frac{m\dot{\mathbf{z}}^2(\tau)}{2}+\mathbf{J}(\tau)
\mathbf{z}(\tau)\right]  }
.\] 
\end{hiProb}
\begin{sol}
Разложим переменную интегрирования следующим образом:
\[
	\mathbf{z}(\tau)=\mathbf{z}_{cl}(\tau)+\mathbf{q}(\tau)
,\]
где
\[
	\mathbf{z}_{cl}(0)=\mathbf{y},\
	\mathbf{z}_{cl}(t)=\mathbf{x},\quad
	\mathbf{q}(0)=0,\
	\mathbf{q}(t)=0
.\] 
Т.\:к.
\[
	\mathcal{L}= \frac{m \dot{\mathbf{z}}_{cl}^2}{2}+\mathbf{J}(\tau)\mathbf{z}_{cl}
,\]
то
\[
	m\ddot{\mathbf{z}}_{cl}=\mathbf{J}(\tau)
 \] 
и
\[
S_{cl}= \int\limits_{0}^{t} d\tau
\left( \frac{m}{2} \dot{\mathbf{z}}^2_{cl}+
\mathbf{J}(\tau) \mathbf{z}_{cl}\right) 
.\] 
Тогда
\[
S= S_{cl}+ \int\limits_{0}^{t} d\tau
\left[ m \dot{\mathbf{z}}_{cl}\dot{\mathbf{q}}+
\mathbf{J}(\tau) \mathbf{q}\right] +\int\limits_{0}^{t}  d\tau \frac{m \dot{\mathbf{q}}^2}{2}
.\] 
\[
\int\limits_{0}^{t}d \tau \left[ 
m \dot{\mathbf{z}}_{cl} \dot{\mathbf{q}}\right] = 
\left. m \dot{\mathbf{z}}_{cl} \mathbf{q} \right|_{0}^t 
	- \int\limits_{0}^{t}  d\tau \ddot{\mathbf{z}} _{cl}\mathbf{q}m
	=- \int\limits_{0}^{t}  d\tau \ddot{\mathbf{z}} _{cl}\mathbf{q}m
.\] 
\[
\int\limits_{0}^{t}d\tau \left[ 
m \dot{\mathbf{z}}_{cl} \dot{\mathbf{q}}+
\mathbf{J} \mathbf{q}\right] =
\int\limits_{0}^{t}  d\tau \left[ 
\mathbf{J}(\tau) \mathbf{q}-
\ddot{\mathbf{z}}_{cl}\mathbf{q} m\right] =
\int\limits_{0}^{t} d\tau
\mathbf{q} \underbrace{\left[ -m \ddot{\mathbf{z}}_{cl}
+\mathbf{J}(\tau)\right] }_{=0}=0
.\] 
Следовательно
\[
	S=S_{cl}+ \int\limits_{0}^{t} d\tau \frac{m \dot{\mathbf{q}}^2}{2} 
.\] 
Значит
\[
	Z= \exp \left( 
		i \frac{S_{cl}(t,\,\mathbf{x},\,\mathbf{y})}{\hbar  }\right) \underbrace{\int\limits_{\mathbf{q}(0)=\mathbf{0},\ 
	\mathbf{q}(t)=\mathbf{0}}
	\mathcal{D} \mathbf{q}(\tau)
	\exp \left( \frac{i}{\hbar }
\int\limits_{0}^{t} d\tau \frac{m \dot{\mathbf{q}}^2}{2} \right) }_{K(t| \mathbf{0},\,\mathbf{0})}
.\] 
\begin{multline*}
	K(t|\mathbf{x},\,\mathbf{y})=
	\Bra{\mathbf{x}}e^{-\frac{i}{\hbar } \frac{
	\widehat{\operatorname{\mathbf{p}}}^2}{2m}t}\Ket{\mathbf{y}}=
	\bra{\mathbf{x}}e^{-\frac{i}{\hbar } \frac{
	\widehat{\operatorname{\mathbf{p}}}^2}{2m}t}
	\int\limits_{\mathbb{R}^3}^{} d^3 \mathbf{p} 
	\ket{\mathbf{p}}\braket{\mathbf{p|}\mathbf{y}}=\\=
	\int\limits_{\mathbb{R}^3}^{} d^3 
	\mathbf{p} e^{-\frac{i}{\hbar } \frac{
	\mathbf{p}^2}{2m}t}\frac{1}{(2\pi \hbar )^3}
	e^{-\frac{i \mathbf{p} (\mathbf{x}-\mathbf{y})}{\hbar }}
.\end{multline*} 
\[
	K(t|\mathbf{0},\,\mathbf{0})=
	\int\limits_{\mathbb{R}^3}^{} 
	d^3 \mathbf{p} e^{-\frac{i}{\hbar }
	\frac{\mathbf{p}^2}{2m}t} \frac{1}{
(2\pi \hbar )^3}=
\frac{1}{(2\pi \hbar )^3} \sqrt{\frac{\pi^3}{
i \left(\frac{t}{2m\hbar }\right)^3}}=
\left( \sqrt{ \frac{m}{2\pi \hbar  t i}}  \right) ^3
.\] 
%На лекциях мы нашли
%\[
%	K_\omega\left( t|x,\,y \right) =
%	\sqrt{ \frac{m\omega}{2\pi i \hbar \sin 
%	\omega t}} e^{ 
%\frac{i}{\hbar } S_{cl}}
%.\] 
%В нашем случае $\omega\to 0$, поэтому
%\[
%	\mathbf{z}_{cl}(\tau)=
%	\frac{1}{2} \mathbf{J} \tau (\tau-t )+\frac{\tau (\mathbf{x}-\mathbf{y})}{t }+\mathbf{y}
%.\] 
%Классическое действие тогда
%\begin{multline*}
%	S_{cl}= \int\limits_{0}^{t} \mathcal{L}dt=\\
%	=
%	\int\limits_{0}^{t}\left[  
%	\frac{m}{2}\left( 
%	\mathbf{J}\left( \tau-\frac{t}{2} \right) +\frac{\mathbf{x}-\mathbf{y}}{t}\right) ^2+
%	\frac{1}{2}\mathbf{J}^2 \tau (\tau-t)+
%	\frac{\tau}{t}\left( \mathbf{x}-\mathbf{y},\,
%	\mathbf{J}\right) +\left( 
%\mathbf{y},\,\mathbf{J}\right)\right] d\tau=
%\\=\frac{m}{2}\left( 
%\mathbf{J}^2 \frac{t^3}{12}+\frac{\left(\mathbf{x}-\mathbf{y}\right)^2}{t}\right) 
%-\mathbf{J}^2 \frac{t^3}{12}+
%(\mathbf{x}-\mathbf{y},\,\mathbf{J})\frac{t}{2}+
%(\mathbf{y},\,\mathbf{J})t
%.\end{multline*} 
%Далее, раскладывая отклонение от классического
%пути по Фурье, и в
\end{sol}
\begin{hiProb}[2]
Волновая функция гармонического осциллятора с массой
$m$ и частотой $\omega$ в начальный момент времени
имеет вид:
\[
	\Psi (x,\,t=0)=C e^{- \frac{m\omega}{2\hbar }(x-x_0)^2}
,\]
где $C$ и $x_0$ --- некоторые константы.
Используя ядро оператора эволюции
для осциллятора, найдите вид волновой функции
в произвольный момент времени.
\end{hiProb}
\begin{sol}
\[
	\Psi (t,\,x)=
	\int\limits_{-\infty}^{\infty} 
	dy K_\omega (t|x,\,y)\Psi(y|t=0)
.\] 
Ядро оператора эволюции для осциллятора:
\[
	K_\omega (t|x,\,y)=
	\sqrt{ \frac{m\omega}{2\pi i \hbar 
	\sin  \omega t}} 
	\exp  \left( 
	\frac{i}{\hbar } S_{cl}
(t,\,x,\,y)\right) 
.\] 
\[
	S_{cl} (t,\,x,\,y)=
	\frac{m\omega}{2 \sin \omega t}
	\left[ 
	(x^2+y^2) \cos \omega t -2xy\right] 
.\] 
\[
	\Psi(x,\,t=0)=
	C e^{-\frac{m\omega}{2\hbar }(x-x_0)^2}
.\] 
\begin{multline*}
	\Psi (t|x)=C \sqrt{
	\frac{m\omega}{ 2\pi i \hbar  \sin \omega
t}} \int \exp  \left[ 
\frac{i}{\hbar } \left( \frac{m\omega}{ 2 \sin \omegat} \left\{ (x^2+y^2) \cos  \omega t -2xy \right\} \right) - \frac{m\omega}{2\hbar } (y-x_0)^2\right] dy=\\=C \sqrt{ \frac{m\omega }{2 \pi i \hbar  \sin  \omega
t}}  \int \exp  \left[ 
\left( \frac{im\omega \cos \omega t}{2\hbar 
\sin \omega t}- \frac{m\omega}{2\hbar } \right) y^2-
\frac{im \omega}{\hbar \sin  \omega t }xy
+\frac{m\omega}{\hbar }x_0y
+ \frac{im\omega \cos \omega t}{2\hbar  \sin \omega t}x^2 - \frac{m\omega}{2\hbar } x_0^2\right] dy=\\=
C \sqrt{ \frac{m\omega }{2\pi i \hbar  \sin \omega
t}}  \exp  \left[ 
\frac{im \omega}{2\hbar } \ctg  \omega t\cdot x^2-
\frac{m\omega}{2\hbar }x_0^2\right] \cdot 
\int \exp  \left[ 
\left( \frac{im \omega \cos  \omega t}{2\hbar 
\sin \omega t}- \frac{m\omega}{2\hbar } \right) \right] 
.\end{multline*} 
\end{sol}
\end{document}
