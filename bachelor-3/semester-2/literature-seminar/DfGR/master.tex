\documentclass[a4paper]{article}
\usepackage{tikz}
\usetikzlibrary{graphs}
% Этот шаблон документа разработан в 2014 году
% Данилом Фёдоровых (danil@fedorovykh.ru) 
% для использования в курсе 
% <<Документы и презентации в \LaTeX>>, записанном НИУ ВШЭ
% для Coursera.org: http://coursera.org/course/latex .
% Исходная версия шаблона --- 
% https://www.writelatex.com/coursera/latex/5.3

% В этом документе преамбула

\usepackage{siunitx}
%%% Работа с русским языком
%\usepackage{cmap}					% поиск в PDF
%\usepackage{mathtext} 				% русские буквы в формулах
%\usepackage[T2A]{fontenc}			% кодировка
%\usepackage[utf8]{inputenc}			% кодировка исходного текста
%\usepackage[english,russian]{babel}	% локализация и переносы
%\usepackage{indentfirst}
%\frenchspacing
%
%\renewcommand{\epsilon}{\ensuremath{\varepsilon}}
%\newcommand{\phibackup}{\ensuremath{\phi}}
%\renewcommand{\phi}{\ensuremath{\varphi}}
%\renewcommand{\varphi}{\ensuremath{\phibackup}}
%\renewcommand{\kappa}{\ensuremath{\varkappa}}
%\renewcommand{\le}{\ensuremath{\leqslant}}
%\renewcommand{\leq}{\ensuremath{\leqslant}}
%\renewcommand{\ge}{\ensuremath{\geqslant}}
%\renewcommand{\geq}{\ensuremath{\geqslant}}
%\renewcommand{\emptyset}{\varnothing}
%\renewcommand{\Im}{\operatorname{Im}}
%\renewcommand{\Re}{\operatorname{Re}}


%%% Дополнительная работа с математикой
\usepackage{amsmath,amsfonts,amssymb,amsthm,mathtools} % AMS
%\usepackage{icomma} % "Умная" запятая: $0,2$ --- число, $0, 2$ --- перечисление

%% Номера формул
%\mathtoolsset{showonlyrefs=true} % Показывать номера только у тех формул, на которые есть \eqref{} в тексте.
%\usepackage{leqno} % Нумереация формул слева

%% Свои команды
\DeclareMathOperator{\sgn}{\mathop{sgn}}
\DeclareMathOperator{\sign}{\mathop{sign}}
\DeclareMathOperator*{\res}{\mathop{res}}
\DeclareMathOperator*{\tr}{\mathop{tr}}
\DeclareMathOperator*{\rot}{\mathop{rot}}
\DeclareMathOperator*{\divop}{\mathop{div}}
\DeclareMathOperator*{\grad}{\mathop{grad}}

%% Перенос знаков в формулах (по Львовскому)
\newcommand*{\hm}[1]{#1\nobreak\discretionary{}
{\hbox{$\mathsurround=0pt #1$}}{}}

%%% Работа с картинками
\usepackage{graphicx}  % Для вставки рисунков
\graphicspath{{figures/}}  % папки с картинками
\setlength\fboxsep{3pt} % Отступ рамки \fbox{} от рисунка
\setlength\fboxrule{1pt} % Толщина линий рамки \fbox{}
\usepackage{wrapfig} % Обтекание рисунков текстом

%%% Работа с таблицами
\usepackage{array,tabularx,tabulary,booktabs} % Дополнительная работа с таблицами
\usepackage{longtable}  % Длинные таблицы
\usepackage{multirow} % Слияние строк в таблице

%%% Теоремы
\theoremstyle{plain} % Это стиль по умолчанию, его можно не переопределять.
\newtheorem{thm}{Теорема}
\newtheorem*{thm*}{Теорема}
\newtheorem{prop}{Предложение}
\newtheorem*{prop*}{Предложение}
 
\theoremstyle{definition} % "Определение"
%\newtheorem{corollary}{Следствие}[theorem]
\newtheorem{dfn}{Определение}
\newtheorem*{dfn*}{Определение}
\newtheorem{prob}{Задача}
\newtheorem*{prob*}{Задача}

 
\theoremstyle{remark} % "Примечание"
\newtheorem*{sol}{Решение}
\newtheorem*{rem}{Замечание}

%%% Программирование
\usepackage{etoolbox} % логические операторы

%%% Страница
%\usepackage{extsizes} % Возможность сделать 14-й шрифт
%\usepackage{geometry} % Простой способ задавать поля
%	\geometry{top=25mm}
%	\geometry{bottom=35mm}
%	\geometry{left=35mm}
%	\geometry{right=20mm}
 
\usepackage{fancyhdr} % Колонтитулы
%	\pagestyle{fancy}
 %	\renewcommand{\headrulewidth}{0pt}  % Толщина линейки, отчеркивающей верхний колонтитул
	%\lfoot{Нижний левый}
	%\rfoot{Нижний правый}
	%\rhead{Верхний правый}
	%\chead{Верхний в центре}
	%\lhead{Верхний левый}
	%\cfoot{Нижний в центре} % По умолчанию здесь номер страницы

\usepackage{setspace} % Интерлиньяж
%\onehalfspacing % Интерлиньяж 1.5
%\doublespacing % Интерлиньяж 2
%\singlespacing % Интерлиньяж 1

\usepackage{lastpage} % Узнать, сколько всего страниц в документе.

\usepackage{soul} % Модификаторы начертания

\usepackage{hyperref}
\usepackage[usenames,dvipsnames,svgnames,table,rgb]{xcolor}
\hypersetup{				% Гиперссылки
    unicode=true,           % русские буквы в раздела PDF
    pdftitle={Заголовок},   % Заголовок
    pdfauthor={Автор},      % Автор
    pdfsubject={Тема},      % Тема
    pdfcreator={Создатель}, % Создатель
    pdfproducer={Производитель}, % Производитель
    pdfkeywords={keyword1} {key2} {key3}, % Ключевые слова
%    colorlinks=true,       	% false: ссылки в рамках; true: цветные ссылки
    %linkcolor=red,          % внутренние ссылки
    %citecolor=black,        % на библиографию
    %filecolor=magenta,      % на файлы
    %urlcolor=cyan           % на URL
}

\usepackage{csquotes} % Еще инструменты для ссылок

%\usepackage[style=apa,maxcitenames=2,backend=biber,sorting=nty]{biblatex}

\usepackage{multicol} % Несколько колонок

\usepackage{tikz} % Работа с графикой
\usepackage{pgfplots}
\usepackage{pgfplotstable}
%\usepackage{coloremoji}
\usepackage{floatrow}
\usepackage{subcaption}
\graphicspath{{figures/}}

\renewcommand\thesubfigure{\asbuk{subfigure}}
%\addbibresource{master.bib}

\usepackage{import}
\usepackage{pdfpages}
\usepackage{transparent}
\usepackage{xcolor}
\usepackage{xifthen}

\newcommand{\incfig}[2][1]{%
    \def\svgwidth{#1\columnwidth}
    \import{./figures/}{#2.pdf_tex}
}
%\usepackage{titlesec}
%\titleformat{\section}{\normalfont\Large\bfseries}{}{0pt}{}
%----------------------STANDART:
%\titleformat{\chapter}[display]
%  {\normalfont\huge\bfseries}{\chaptertitlename\ \thechapter}{20pt}{\Huge}
%\titleformat{\section}{\normalfont\Large\bfseries}{\thesection}{1em}{}
%\titleformat{\subsection}
%  {\normalfont\large\bfseries}{\thesubsection}{1em}{}
%\titleformat{\subsubsection}
%  {\normalfont\normalsize\bfseries}{\thesubsubsection}{1em}{}
%\titleformat{\paragraph}[runin]
%  {\normalfont\normalsize\bfseries}{\theparagraph}{1em}{}
%\titleformat{\subparagraph}[runin]
%  {\normalfont\normalsize\bfseries}{\thesubparagraph}{1em}{}

\pdfsuppresswarningpagegroup=1
\pgfplotsset{compat=1.16}



%\setcounter{tocdepth}{1} % only parts,chapters,sections
%\titleformat{\subsection}{\normalfont\large\bfseries}{}{0em}{}
%\titleformat{\subsubsection}{\normalfont\normalsize\bfseries}{}{0em}{}

%\newcommand{\textover}[2]{\stackrel{\mathclap{\normalfont\mbox{#2}}}{#1}}

\author{Yaroslav Drachov\\
Moscow Institute of Physics and Technology}
%\author{Драчов Ярослав\\
%Факультет общей и прикладной физики МФТИ}
\newcommand{\veq}{\mathrel{\rotatebox{90}{$=$}}}
%\newcommand{\teto}[1]{\stackrel{\mathclap{\normalfont\tiny\mbox{#1}}}{\to}}
%\renewcommand{\thesubsection}{\arabic{subsection}}

%%\setcounter{secnumdepth}{0}

\definecolor{tabblue}{RGB}{30, 119, 180}
\definecolor{taborange}{RGB}{255, 127, 15}
\definecolor{tabgreen}{RGB}{45, 160, 43}
\definecolor{tabred}{RGB}{214, 38, 40}
\definecolor{tabpurple}{RGB}{148, 103, 189}
\definecolor{tabbrown}{RGB}{140, 86, 76}
\definecolor{tabpink}{RGB}{227, 119, 193}
\definecolor{tabgray}{RGB}{127, 127, 127}
\definecolor{tabolive}{RGB}{188, 189, 33}
\definecolor{tabcyan}{RGB}{22, 190, 207}
\pgfplotscreateplotcyclelist{colorbrewer-tab}{
{tabblue},
{taborange},
{tabgreen},
{tabred},
{tabpurple},
{tabbrown},
{tabpink},
{tabgray},
{tabolive},
{tabcyan},
}
\usepackage{csvsimple}
\usepackage{extarrows}
%\renewcommand{\labelenumii}{\asbuk{enumii})}
%\renewcommand{\labelenumiv}{\Asbuk{enumiv}}
%\newcommand{\prob}[1]{\subsubsection*{#1}}
\sisetup{output-decimal-marker = {,},separate-uncertainty = true,exponent-product = \cdot}

\usepackage{braket}
\usepackage{enumerate}
\usepackage{chngcntr}
%\counterwithin*{equation}{problem}
%\usepackage{bbold}

\newtheoremstyle{hiProb}% ⟨name ⟩ 
{3pt}% ⟨Space above ⟩1 
{3pt}% ⟨Space below ⟩1
{}% ⟨Body font ⟩
{}% ⟨Indent amount ⟩2
{\bfseries}% ⟨Theorem head font⟩
{.}% ⟨Punctuation after theorem head ⟩
{.5em}% ⟨Space after theorem head ⟩3
%{\thmname{#1} \thmnote{#3}}% ⟨Theorem head spec (can be left empty, meaning ‘normal’)⟩
{\thmnote{#3}}% ⟨Theorem head spec (can be left empty, meaning ‘normal’)⟩
\theoremstyle{hiProb} % "Определение"
%\newtheorem{hiProb}{Задача}
\newtheorem{hiProb}{}
%\usepackage{mmacells}
\newcommand{\textover}[2]{\stackrel{\mathclap{\normalfont\scriptsize\mbox{#2}}}{#1}}
\usepackage{units}
\usepackage[math]{cellspace}%
\setlength\cellspacetoplimit{2pt}
\setlength\cellspacebottomlimit{2pt}

\DeclareMathAlphabet{\mathbbold}{U}{bbold}{m}{n}

\newcommand{\normord}[1]{:\mathrel{#1}:}

\titleformat{\section}%
  [hang]% <shape>
  {\normalfont\bfseries\Large}% <format>
  {}% <label>
  {0pt}% <sep>
  {}% <before code>
\renewcommand{\thesection}{}% Remove section references...
\renewcommand{\thesubsection}{\arabic{subsection}}%... from subsections
\title{Декогеренция в ОТО\\
2012.12903}
\begin{document}
	\maketitle
%	\section{Общие вводные слова (Резюме и введение статьи)}
%Декогеренция описывает тенденцию квантовых подсистем динамически терять свой квантовый характер. Это происходит, когда интересующая квантовая подсистема взаимодействует и запутывается с отслеживаемой средой.
%
%В одной из предшествующих работ авторы вычислили скорость декогеренции части материи, находящейся в состоянии суперпозиции, которая в основном взаимодействует только гравитационно, dark-matter-Schrödinger-cat-state (DMSCS), в нерелятивистском приближении. В данной
%же работе авторы обобщили данные результаты на случай ОТО. Сперва
%авторы получили релятивистское уравнение Шрёдингера для одной пробной цастицы, проходящей сквозь DMSCS; взаимодействие обеспечивается метрикой ОТО слабого поля от источника. Для статичной DMSCS
%авторы нашли отличное обобщение их предыдущих результатов.
%Далее они перешли к рассмотрению интересного нового случая
%нестационарной DMSCS, которая может быть получена когерентно
%колеблющемся аксионным полем, приводящим к суперпозиции нестационарных колебаний в метрике; по-настоящему квантово-гравитационному
%явлению. Авторы используют теорию рассеяния, чтобы получить
%скорость декогеренции во всех этих случаях.
%Когда DMSCS находится в суперпозиции различных профилей плотности, авторы обнаруживают, что скорость декогеренции может быть значительной.
%Затем авторы рассматривают новый частный случай, в котором в суперпозиции находится не плотность, а фаза колебаний ее поля; это свойство, которое нельзя декогерировать в рамках нерелятивистских СО.
%Авторы обнаружили, что если дисперсия скорости пробной частицы и / или DMSCS медленная, то скорость декогеренции фазы экспоненциально подавляется.
%Однако, если дисперсия скоростей и пробной частицы, и DMSCS являются релятивистскими, то фаза может декогерировать быстрее. В качестве приложений авторы обнаружили, что диффузные галактические аксионы с наложенными фазами устойчивы к декогерентности, в то время как плотные бозонные звезды и области вблизи горизонтов черных дыр --- нет, и авторы обсуждают их значение для эксперимента.
%
%Интересно исследовать новые явления, в которых центральную роль играют как гравитация, так и квантовая механика. Для большинства обычных веществ, несмотря на то, что их квантовый характер может проявляться, другие взаимодействия, такие как электромагнитные, часто играют центральную роль в их динамике. Чтобы исключить взаимосвязь между гравитационными и квантово-механическими явлениями, полезно рассмотреть материю, которая в основном имеет только гравитационные взаимодействия. Нечто такое неизвестно среди изученных частиц, но вполне может быть самой доминирующей формой материи во Вселенной (обзор см. В [2]). Далее, можно просто рассмотреть наличие такой экзотической материи, даже если бы она не была доминирующей составляющей. В любом случае, авторы называют такую материю, которая в первую очередь взаимодействует посредством гравитации, «темной материей» (ТМ) в этой работе.
%
%Текущее отсутствие прямого обнаружения ТМ, которая на самом деле составляет большую часть массы Вселенной, за исключением ее гравитационного воздействия на галактики и т.п., означает, что ТМ в лучшем случае очень слабо связана с обычными частицами Стандартной модели. Фактически возможно, что его единственная связь со Стандартной моделью является гравитационной (плюс другие Планковские или почти Планковские, подавленные операторы). С одной стороны, поскольку взаимодействие между частицами ТМ и частицами обычного вещества очень слабое и/или редкое, очень трудно обнаружить свойства ТМ. С другой стороны из-за отсутствия значимых взаимодействий ТМ может обладать долгоживущими экзотическими квантово-механическими явлениями. В частности, можно представить себе, что часть ТМ организована в макроскопическую суперпозицию состояний, которые иногда называют состояниями «кота Шредингера».
%
%Они обладают поистине квантовым поведением, закодированным в недиагональных членах матрицы плотности (в соответствующем базисе). Однако квантовость обычно недолговечна из-за взаимодействий с окружающей средой, что приводит к подавлению недиагональных элементов матрицы плотности; процесс, называемый \emph{декогеренцией}. Декогеренцию можно понять следующим образом: состояние кота Шрёдингера и его окружение взаимодействуют и неизбежно запутываются. Полная система остается в чистом состоянии, но наблюдатель обычно не отслеживает всю систему во всех ее деталях. Вместо этого принимается грубая точка зрения, в которой степени свободы окружающей среды (которых обычно много) игнорируются и отслеживаются. Это эффективно разрушает квантовую когерентность остаточной подсистемы (ранние работы, устанавливающие механизм декогеренции, см. [3–5], а о различных разработках см. [6–16]). Из-за декогеренции состояние кота Шрёдингера эволюционирует в смешанное состояние с по существу классическими вероятностями, а не в чисто квантово-механическую суперпозицию, и поэтому уникальные квантово-механические явления, такие как интерференция, больше не присутствуют в редуцированной системе. Этот процесс эффективен, когда взаимодействия велики, как в случае рассеяния воздуха на каком-то обычном материале. Однако эта декогеренция может быть неэффективной для ТМ, в которой отсутствуют эти взаимодействия.
%
%Было проделана большая работа по декогеренции в контексте гравитации и космологии; см. ссылки. [17–49]. Для ТМ вполне вероятно, что гравитационное или самовзаимодействие может привести к таким квантовым состояниям (КШ) из-за макроскопического расплывания волновой функции, особенно если система демонстрирует некоторую форму хаоса (например, см. [50]). В случае аксионной ТМ [51–58] часто изучаются частицы очень низкой массы, и поэтому они должны иметь очень высокую (числовую) плотность, чтобы быть всей ТМ. Поэтому они обычно находятся в состояниях с очень высокой степенью заполнения, и поэтому обычно считаются классическими (например, см. [59,60]), но любой хаос может привести к образованию квантовых состояний (КШ). Так что высокая заполняемость не является достаточным условием для классики. Тем не менее, могут быть формы ансамблевого усреднения классических траекторий, приближенно воспроизводящие некоторые квантовые корреляционные функции; см. [61]. Остаточную истинную квантовость таких состояний нетривиально исследовать экспериментально, но ее возможно и интересно рассмотреть. Таким образом, ключевой вопрос заключается в том, сохраняется ли его квантовый характер или декогерируется из-за какой-то астрофизической среды.
%
%Скорость декогеренции для состояния DMSCS локализованного распределения массы для нерелятивистской ТМ от нерелятивистских (пробных частиц?) была впервые вычислена авторами в работе [1]. Суперпозиция включала два различных распределения масс, а взаимодействие между окружающей средой и ТМ моделировалось ньютоновской гравитацией. Было обнаружено, что такое гравитационное взаимодействие фактически приводит к декогеренции DMSCS в зависимости от параметров. Для легких бозонных моделей ТМ, таких как аксион, характерные масштабы времени декогеренции оказались очень чувствительными к массе частицы ТМ, при этом более легкие частицы приводят к быстрой декогеренции, а более тяжелые частицы - к очень медленной декогеренции. Полная информация была предоставлена авторами в [1], включая зависимость от распределения масс, сравнение ТМ в гало с околоземной и так далее. Более того, учитывая  распространённость DMSCS, было обнаружено, что большинство конфигураций декогерируются во времена, меньшие, чем возраст Вселенной, за исключением более тяжелых аксионов.
%
%Можно задаться вопросом, могут ли общие релятивистские эффекты привести к новым интересным формам декогеренции. Предполагается, что любая DMSCS, которая подвергается декогеренции в рамках ньютоновского приближения, будет делать то же самое, если включить релятивистские поправки, даже если поправки заметны; это потому, что декогеренция - очень устойчивое явление. Однако можно рассмотреть DMSCS, в котором ньютоновская трактовка неспособна исследовать декогеренцию. В частности, рассмотрим суперпозицию двух состояний с идентичными (или почти идентичными) ньютоновскими взаимодействиями, но которые имеют разное поведение в рамках общей теории относительности. Конкретным примером является источник ТМ, созданный из когерентно колеблющегося скалярного поля (например, аксиона), которое находится в суперпозиции различных фаз колебаний, но в остальном имеет тот же пространственный профиль. В этом случае ньютоновское взаимодействие не приведет к декогеренции, потому что ньютоновская трактовка не различает подсостояния суперпозиции, в то время как общая релятивистская трактовка различает.
%
%В данной работе авторы рассматривают механизм декогеренции DMSCS в рамках полностью релятивистской трактовки, работающий
%в линейном порядке по возмущениям метрики. Авторы разрабатывают общий формализм для вычисления скорости декогеренции в этой релятивистской системе. Сначала авторы рассматривают случай, когда DMSCS генерирует статическую метрику пространства-времени, найдя много общего с ньютоновским формализмом в работе [1], но со всеми релятивистскими поправками. Затем авторы обобщают это на случай изменяющейся во времени метрики, мотивируемой моделями легкой бозонной ТМ (например, аксионами), которые обычно принимают форму регулярно осциллирующего поля, являющегося источником изменяющейся во времени метрики. Анализ изменяющегося во времени осциллирующего источника разбивается на два случая: первый, когда поле легкой бозонной ТМ имеет несколько отчетливую пространственную зависимость между двумя состояниями суперпозиции, и второй, когда подсостояния суперпозиции различаются только фазами поля. Этот второй случай отражает основную цель данной статьи, которая состоит в анализе ситуации, в которой ньютоновское гравитационное взаимодействие полностью закрыто для суперпозиции, что требует привлечения ОТО. В рамках этого второго случая авторы приводят характерный пример конкретной конфигурации аксионного поля и обсуждают суть такой DMSCS. Авторы применяют эти результаты в нескольких ситуациях, включая медленно движущуюся ТМ в галактике и более экзотических приложениях, таких как ТМ, которая является релятивистской в особых условиях, включая ближнюю область горизонта черных дыр или ТМ, которая формирует плотные бозонные звезды, и авторы обсуждают приложения данной теории.
%
%В данной статье авторы сперва выводят релятивистское уравнение Шрёдингера, необходимое для отслеживания эволюции частиц окружающей среды в пространстве-времени, порождаемом ТМ. Во-вторых излагается основная теория рассеяния и связь с декогеренцией, необходимые для последующих вопросов. В-третьих анализируется эволюция частиц окружающей среды в статическом пространственно-временном фоне и этот анализ применяется для вычисления скорости декогеренции для DMSCS. Потом предоставляется аналогичный анализ для случая изменяющейся во времени метрики, вычисляем частоту декогеренции, а также применяем эти результаты к конкретному профилю. Далее полученные результаты применяются к ТМ в галактике сегодня. И, наконец,  обсуждается распространение состояний, приложение к бозонным звездам, черным дырам и значение данной теории для эксперимента.
%\section{Гравитационная декогеренция ТМ (2005.12287v2)}
%\subsection*{Резюме}
%Декогеренция описывает тенденцию квантовых подсистем динамически терять свой квантовый характер. Это происходит, когда интересующая квантовая подсистема взаимодействует и запутывается с отслеживаемой средой. Для обычных макроскопических систем электромагнитные и другие взаимодействия вызывают быструю декогеренцию. Однако темная материя (ТМ) может иметь уникальную возможность проявлять естественные длительные макроскопические квантовые свойства из-за ее слабого взаимодействия с окружающей средой, особенно если она взаимодействует только гравитационно. В этой работе авторы вычисляют скорость декогеренции для легкой ТМ в галактике, где локальная плотность имеет свою массу, размер и положение в квантовой суперпозиции. Декогеренция происходит через гравитационное взаимодействие сверхплотности ТМ с окружающей средой, обеспечиваемой обычной материей.
%Авторы сосредотачиваются на относительно устойчивых конфигурациях: возмущениях ТМ, которые включают чрезмерную плотность, за которой следует пониженная плотность, без монополя, так что они наблюдаются только на относительно близких расстояниях. Авторы используют нерелятивистскую теорию рассеяния с ньютоновским потенциалом, порожденным сверхплотностью, чтобы определить, как пробная частица рассеивается от нее и тем самым запутывается. В качестве приложения мы рассматриваем легкую скалярную ТМ, включая аксионы. В галактическом гало авторы используют диффузный водород в качестве окружающей среды, в то время как вблизи Земли авторы используют воздух в качестве окружающей среды.
%Для сверхплотности, размер которой является типичной длиной волны де Бройля ТМ, авторы обнаруживают, что скорость декогеренции в гало выше, чем нынешняя скорость Хаббла для масс ТМ $m_a \lesssim 5\times 10^{-7}$ эВ, а в наземных экспериментах она выше, чем классическая скорость когерентности поля при $m_a \lesssim 5\times 10^{-6}$ эВ. Когда происходит распространение состояний, темпы могут стать намного выше, согласно оценкам авторов. Кроме того, авторы установили, что Бозе-Эйнштейновский конденсат (БЭК) ТМ декогерируются очень быстро и поэтому очень хорошо описываются классической теорией поля.
%\subsection{Введение}
%Квантовая механика позволяет непрерывно генерировать макроскопические суперпозиции состояний. Иногда их называют \emph{состояниями «кота Шрёдингера»}. Однако в обычных условиях повседневного мира таких макроскопических суперпозиций обычно не наблюдается. Причина этого, как было хорошо установлено, связана с (i) запутанностью и (ii) зернистостью, а именно: запутанность неизбежно возникает, когда частицы в окружающей среде взаимодействуют с состоянием кота Шрёдингера; затем, если не отслеживать окружающую среду внимательно и сосредотачиваться только на интересующей подсистеме (то есть на грубой точке зрения), квантовая когерентность эффективно разрушается (ранние работы см. в [1–3]). Это хорошо известное явление \emph{декогеренции}. Здесь мы используем слово «когерентный» для обозначения полного чистого квантово-механического состояния, которое унитарно развивается через уравнение Шрёдингера. Таким образом, декогеренция - это процесс, который переводит интересующую подсистему в эффективное «смешанное состояние». В этом случае различные наблюдаемые состояния, которые составляют редуцированную суперпозицию, связаны только классическими вероятностями, и, таким образом, истинно квантово-механические явления, такие как интерференция, напрямую не наблюдаются.
%
%Для большинства известных макроскопических систем декогеренция происходит очень быстро за счет обычных взаимодействий, таких как электромагнитные взаимодействия. Типичные макроскопические квантовые системы легко взаимодействуют с окружающей средой, такой как воздух, излучение и т.д., которые эффективно производят «измерения» в интересующей подсистеме. Поскольку роль среды часто играет огромное количество окружающих степеней свободы, последующая декогеренция обычно происходит чрезвычайно быстро, часто за крошечные доли секунды.
%
%Учитывая эту ситуацию, можно задаться вопросом, есть ли какие-нибудь интересные системы, которые могут быть устойчивы к декогеренции и, таким образом, сохранять свои квантовые свойства в течение длительного времени. В этой работе авторы будут рассматривать интересующую нас систему как темную материю (ТМ), которая составляет большую часть массы Вселенной. ТМ, скорее всего, построена из частиц, выходящих за рамки Стандартной модели (СМ). Пока что её прямое обнаружение ускользало от всех текущих экспериментов. Следовательно, мы знаем, что ТМ (если только она не состоит из чрезвычайно тяжелых частиц) имеет самое слабое взаимодействие с известными частицами СМ. Фактически, вполне возможно, что ТМ вообще не взаимодействует с СМ (кроме как через сильно подавленные операторы более высокой размерности), и, кроме того, она может иметь чрезвычайно подавленные взаимодействия или вообще не взаимодействовать с собой и любыми другими частицами (за пределами СМ). В этой ситуации состояние ТМ, подобное коту Шредингера, казалось бы, полностью устойчиво к декогеренции из окружающей среды, если не с чем заметно взаимодействовать.
%
%Очевидным исключением является гравитация, которая связана с любым источником энергии и импульса и, таким образом, связана с ТМ с той же силой $G$, что и все частицы. В результате макроскопическая квантовая суперпозиция ТМ может в первую очередь испытывать декогеренцию через гравитационные взаимодействия. Причина в том, что макроскопическая суперпозиция создает гравитационное поле и в макроскопической суперпозиции. (Некоторые авторы пропагандируют <<полуклассическую>> гравитацию, предполагающую, что гравитационное поле возникает из математического ожидания распределения масс 
%$G_{\mu\nu}=8\pi G \left<\hat{T}_{\mu\nu} \right>$, но
%эта крайняя точка зрения здесь будет проигнорирована). Но что важно, поскольку гравитация обычно слишком слаба, скорость декогеренции может быть крайне мала, так что ТМ может сохранять свою квантовую когерентность в течение очень долгого времени. Было проделано много работы по декогеренции в контексте гравитации и космологии; см. ссылки. [4–36]. Цель этой работы - исследовать скорость декогеренции избыточных/недостаточных плотностей ТМ в галактике, которые организованы в макроскопические суперпозиции с различными распределениями плотности массы. Поскольку гравитация сама по себе является нелинейным взаимодействием, вполне вероятно, что такие квантовые состояния возникают, поскольку волновая функция имеет тенденцию расширяться от своего начального состояния. Это особенно заметно, когда система демонстрирует некоторую форму хаоса (например, см. [37]).
%
%Основной интерес для авторов статьи представляет очень легкая бозонная ТМ, особенно аксионы. Поскольку ТМ является нерелятивистской, такие частицы имеют очень большую длину волны де Бройля $\lambda=2\pi \hbar /(m_a v_a)$.
%Правдоподобные значения масс аксионов, включая аксионы КХД, аксионы теории струн и аксионоподобные частицы, простираются от сверхлегких аксионов $m_a \sim 10^{-21}\text{эВ} /c^2$ до $m_a \sim  10^{-3}\text{эВ} /c^2$. Это соответствует
%волнам де Бройля, простирающимся от $\sim 100$ пк
%до $\sim $ метров.
%Таким образом, в отличие от типичных длин волн де Бройля, связанных со знакомыми частицами, такими как электроны или протоны, соответствующие масштабы для легкой ТМ могут быть макроскопически большими. Более того, ТМ в галактике будет вириализоваться, что приведет к большому разбросу скоростей $v_a$. Это означает, что регулярная диффузная ТМ в галактике, как ожидается, будет иметь $\mathcal{O}(1)$ флуктуации плотности массы в масштабе этих больших длин волн де Бройля. Кроме того, авторы также обсуждают более компактные структуры ТМ, а именно бозонные звезды, которые возникают в некоторых контекстах и представляют собой очень массивные конфигурации конденсированных бозонов.
%
%Это само по себе не делает его состоянием кота Шредингера. Фактически, эти состояния часто считаются хорошо описываемыми классической теорией поля, поскольку они соответствуют состояниям с чрезвычайно высокими числами заполнения (например, см. [38,39]). Тем не менее интересно то, что структуры макроскопически большие. Таким образом, если они эволюционируют в квантовые суперпозиции различных классических конфигураций поля (и, как мы упоминали выше, хаотические системы легко к этому стремятся), то в галактике мы будем иметь макроскопически большие состояния кота Шредингера. Наблюдательные последствия неясны, так как даже некоторые состояния кота Шредингера могут быть имитированы классическим усреднением по ансамблю [40]. Но само существование таких состояний само по себе интересно и является нашим объектом изучения здесь.
%
%Мы будем называть эту конфигурацию dark-matter-Schrödinger-cat-state(DMSCS). Наша основная цель - определить временной порядок декогеренции DMSCS из среды обычной материи. Мы применяем эти результаты как к ТМ в галактическом гало, так и к ТМ у поверхности Земли, где проводится множество текущих экспериментов.
%
%Авторы во-первых вводят базовый формализм для анализа декогеренции. Во-вторых осуществляют расчёт нерелятивистского
%квантового рассеяния и получают общие формулы для скорости
%декогеренции. В-третьих, авторы применяют данную теорию
%к аксионной ТМ и получают количественные результаты для
%скорости декогеренции.
%
%\subsection{Формализм декогеренции}
%В этой работе авторов интересуют квантовые системы, находящиеся в макроскопических квантовых суперпозициях. Мы представим себе, что ТМ находится в таком состоянии. Фактически, (существенное расплывание?) волновой функции настолько повсеместно, что во многих случаях это неизбежно, учитывая, что ТМ существует уже миллиарды лет. Мы не будем углубляться в детали формирования такого состояния, но отметим, что вполне вероятно, что оно возникнет в некоторых условиях. Интересный вопрос состоит в том, может ли такое состояние, подобное коту Шредингера, продолжать сохраняться, когда неизбежно существует среда той или иной формы, с которой оно будет взаимодействовать. В этом разделе авторы излагают базовый формализм для изучения квантовой декогеренции из-за гравитации, а в следующих разделах применят его к ТМ и количественно определят частоту декогеренции.
%\subsubsection{Запутанность}
%Мы будем рассматривать DMSCS, то естьсуперпозицию двух различных наблюдаемых макроскопических состояний. Хотя мы сосредоточимся только на двух состояниях, расширение до произвольного числа различных состояний несложно. Предполагается, что эти состояния представляют собой различные распределения масс. Обозначим для этого начальное состояние кет как $\ket{\text{DM}}$. Два наблюдаемых состояния будем обозначать $\ket{\text{DM}_1}$ и $\ket{\text{DM}_2}$, а их начальное состояние --- это суперпозиция
%\[
%	\ket{\text{DM}}=\ket{\text{DM}_1}+
%	\ket{\text{DM}_2}
%.\] 
%\section{Об интерпретации измерения в квантовой теории}
%\subsection*{Резюме}
%Показано, что ни доводы, приводящие к противоречиям в описании квантово-механических измерений, ни <<объясняющие>> процесс измерения с помощью термодинамической статистики, не являются вескими. Напротив, показано, что вероятностная
%интерпретация совместима с (объективной?) интерпретацией
%волновой функции.
%\subsection{Введение}
%Задача измерения в квантовой теории и, связанная с ней, задача описания классического явления в рамках квантовой теории
%получили повышенное внимание в течение последних лет.
%Различие статьи выражают различные точки зрения и могут быть
%классифицированы следующим образом:
%\begin{enumerate}
%\item Те, кто подчёркивают противоречия, возникающие
%	при самом процессе измерения в терминах квантовой
%	теории
%\item Те, кто утверждают, что измерение может быть
%	хорошо объяснено в квантовой теории в смысле, что
%	<<квантово-механическая беспричинность>> может
%	быть получена из статистических неопределённостей,
%	присущих обязательно макроскопическому
%	аппарату измерения.
%\item Те, кто вводят новые физические концепции, такие как
%	скрытые переменные.
%\end{enumerate}
%
%Предложения третьей группы обычно основаны на первой
%точке зрения и имеют смысл только в том случае, если
%они экспериментально подтверждаются. Пока такого не наблюдалось.
%
%Измерение в квантовой теории аксиоматически описывается
%с помощью Эрмитового оператора. Если собственные состояния
%этого оператора --- $\varphi_n$, и состояние измеряемой
%системы $S$ --- это $\varphi= \sum_{}^{} c_n \varphi_n$, тогда,
%следуя аксиоме, результат измерения будет с вероятностью
%$|c_n|^2$ соответствующим собственным значением $a_n$,
%физически представленным подходящим состоянием измеряемого
%устройства $M$. Для самого распостраннённого класса измерений, более того, прогнозируется, что каждое последующее измерение может быть описано, полагая $S$ в состоянии $\varphi_n$ 
%после измерения.
%
%При описании процесса в целом в рамках квантовой теории,
%полагается также что аппарат $M$ может быть описан
%волновой функцией $\phi_\alpha$, состояние полной системы
% $M+S$ подчиняющейся уравнению Шрёдингера,
% \[
%\psi(t)= e^{iHt} \phi_\alpha \sum_{n}^{} c_n \varphi_n=
%\sum_{n,m,\beta}^{} c_n U_{\alpha\beta}^{nm}(t)\phi_\beta
%\varphi_m
%,\] 
%где $U_{\alpha \beta}^{nm}(0)=\delta_{nm} \delta_{\alpha\beta}$.
%Поскольку состояние макроскопического аппарата может быть
%определено лишь частично, должен быть большой набор
%состояний $\left\{ \phi \right\}_0 $, совместимых со знанием
%о $M$. Если этот набор предполагается независимым от
%состояния $S$ перед измерением, условие на коэффициенты
%$U_{\alpha\beta}^{nm}$ может быть получено из требования
%выполнения аксиомы измерения в случае $c_n=\delta_{n n_0}$,
%т.\:е. $\varphi=\varphi_{n_0}$.
%\section{Базис указателя квантового аппарата: во что
%смешивается волновой пакет?}
%\subsection*{Резюме}
%Формы гамильтониана взаимодействия между прибором и его окружением достаточно, чтобы определить, какая наблюдаемая из измеряемой квантовой системы может считаться "записанной" прибором. Базис, содержащий эту запись ---
%\emph{базис указателя} аппарата --- состоит из собственных векторов оператора, который коммутирует с гамильтонианом взаимодействия аппарат-среда. Таким образом, можно сказать, что среда выполняет неразрушающее измерение наблюдаемой диагонали в базисе указателя.
%\subsection{Что измеряется в квантовом измерении?}
%Фон Нейман показал, что одной унитарной эволюции достаточно,
%чтобы установить \emph{неразрывную корреляцию} между
%вектором состояния $\ket{A}$ квантового аппарата $\mathcal{A}$ и вектором состояния $\ket{\psi}$ квантовой системы
%$\mathcal{S}$, который должен быть измерен:
% \[
%\ket{A_0}\otimes \ket{\psi}= 
%\left\{ \sum_{s}^{} a_s \ket{A_s} \right\} \otimes 
%\left\{ \sum_{s}^{} c_s\ket{s} \right\} \to 
%\sum_{s}^{} c_s \ket{A_s} \otimes  \ket{s}
%.\] 



\emph{Декогеренция} описывает тенденцию квантовых подсистем динамически терять свой квантовый характер. Это происходит, когда интересующая квантовая подсистема взаимодействует и запутывается с отслеживаемой средой.

Квантовая механика позволяет непрерывно генерировать макроскопические суперпозиции состояний. Иногда их называют \emph{состояниями «кота Шрёдингера»}. Однако в обычных условиях повседневного мира таких макроскопических суперпозиций обычно не наблюдается. Причина этого, как было хорошо установлено, связана с (i) запутанностью и (ii) зернистостью, а именно: запутанность неизбежно возникает, когда частицы в окружающей среде взаимодействуют с состоянием кота Шрёдингера; затем, если не отслеживать окружающую среду внимательно и сосредотачиваться только на интересующей подсистеме (то есть на грубой точке зрения), квантовая когерентность эффективно разрушается.

При описании процесса в целом в рамках квантовой теории,
полагается также что аппарат $M$ может быть описан
волновой функцией $\phi_\alpha$, состояние полной системы
 $M+S$ подчиняющейся уравнению Шрёдингера,
 \[
\psi(t)= e^{iHt} \phi_\alpha \sum_{n}^{} c_n \varphi_n=
\sum_{n,m,\beta}^{} c_n U_{\alpha\beta}^{nm}(t)\phi_\beta
\varphi_m
,\] 
где $U_{\alpha \beta}^{nm}(0)=\delta_{nm} \delta_{\alpha\beta}$.

Фон Нейман показал, что одной унитарной эволюции достаточно,
чтобы установить \emph{неразрывную корреляцию} между
вектором состояния $\ket{A}$ квантового аппарата $\mathcal{A}$ и вектором состояния $\ket{\psi}$ квантовой системы
$\mathcal{S}$, который должен быть измерен:
 \[
\ket{A_0}\otimes \ket{\psi}= 
\left\{ \sum_{s}^{} a_s \ket{A_s} \right\} \otimes 
\left\{ \sum_{s}^{} c_s\ket{s} \right\} \to 
\sum_{s}^{} c_s \ket{A_s} \otimes  \ket{s}
.\] 
\end{document}
