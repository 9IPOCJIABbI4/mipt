\documentclass[a4paper]{article}
% Этот шаблон документа разработан в 2014 году
% Данилом Фёдоровых (danil@fedorovykh.ru) 
% для использования в курсе 
% <<Документы и презентации в \LaTeX>>, записанном НИУ ВШЭ
% для Coursera.org: http://coursera.org/course/latex .
% Исходная версия шаблона --- 
% https://www.writelatex.com/coursera/latex/5.3

% В этом документе преамбула

\usepackage{siunitx}
%%% Работа с русским языком
\usepackage{cmap}					% поиск в PDF
\usepackage{mathtext} 				% русские буквы в формулах
\usepackage[T2A]{fontenc}			% кодировка
\usepackage[utf8]{inputenc}			% кодировка исходного текста
\usepackage[english,russian]{babel}	% локализация и переносы
\usepackage{indentfirst}
\frenchspacing

\renewcommand{\epsilon}{\ensuremath{\varepsilon}}
\renewcommand{\phi}{\ensuremath{\varphi}}
\renewcommand{\kappa}{\ensuremath{\varkappa}}
\renewcommand{\le}{\ensuremath{\leqslant}}
\renewcommand{\leq}{\ensuremath{\leqslant}}
\renewcommand{\ge}{\ensuremath{\geqslant}}
\renewcommand{\geq}{\ensuremath{\geqslant}}
\renewcommand{\emptyset}{\varnothing}
\renewcommand{\Im}{\operatorname{Im}}
\renewcommand{\Re}{\operatorname{Re}}


%%% Дополнительная работа с математикой
\usepackage{amsmath,amsfonts,amssymb,amsthm,mathtools} % AMS
\usepackage{icomma} % "Умная" запятая: $0,2$ --- число, $0, 2$ --- перечисление

%% Номера формул
%\mathtoolsset{showonlyrefs=true} % Показывать номера только у тех формул, на которые есть \eqref{} в тексте.
%\usepackage{leqno} % Нумереация формул слева

%% Свои команды
\DeclareMathOperator{\sgn}{\mathop{sgn}}
\DeclareMathOperator{\sign}{\mathop{sign}}
\DeclareMathOperator*{\res}{\mathop{res}}
\DeclareMathOperator*{\tr}{\mathop{tr}}

%% Перенос знаков в формулах (по Львовскому)
\newcommand*{\hm}[1]{#1\nobreak\discretionary{}
{\hbox{$\mathsurround=0pt #1$}}{}}

%%% Работа с картинками
\usepackage{graphicx}  % Для вставки рисунков
\graphicspath{{figures/}}  % папки с картинками
\setlength\fboxsep{3pt} % Отступ рамки \fbox{} от рисунка
\setlength\fboxrule{1pt} % Толщина линий рамки \fbox{}
\usepackage{wrapfig} % Обтекание рисунков текстом

%%% Работа с таблицами
\usepackage{array,tabularx,tabulary,booktabs} % Дополнительная работа с таблицами
\usepackage{longtable}  % Длинные таблицы
\usepackage{multirow} % Слияние строк в таблице

%%% Теоремы
\theoremstyle{plain} % Это стиль по умолчанию, его можно не переопределять.
\newtheorem{theorem}{Теорема}
\newtheorem*{thm}{Теорема}
\newtheorem{prop}{Утверждение}
 
\theoremstyle{definition} % "Определение"
%\newtheorem{corollary}{Следствие}[theorem]
\newtheorem*{dfn}{Определение}
\newtheorem{problem}{Задача}
\newtheorem*{problem*}{Задача}

 
\theoremstyle{remark} % "Примечание"
\newtheorem*{sol}{Решение}
\newtheorem*{rem}{Замечание}

%%% Программирование
\usepackage{etoolbox} % логические операторы

%%% Страница
%\usepackage{extsizes} % Возможность сделать 14-й шрифт
%\usepackage{geometry} % Простой способ задавать поля
%	\geometry{top=25mm}
%	\geometry{bottom=35mm}
%	\geometry{left=35mm}
%	\geometry{right=20mm}
 
\usepackage{fancyhdr} % Колонтитулы
%	\pagestyle{fancy}
 %	\renewcommand{\headrulewidth}{0pt}  % Толщина линейки, отчеркивающей верхний колонтитул
	%\lfoot{Нижний левый}
	%\rfoot{Нижний правый}
	%\rhead{Верхний правый}
	%\chead{Верхний в центре}
	%\lhead{Верхний левый}
	%\cfoot{Нижний в центре} % По умолчанию здесь номер страницы

\usepackage{setspace} % Интерлиньяж
%\onehalfspacing % Интерлиньяж 1.5
%\doublespacing % Интерлиньяж 2
%\singlespacing % Интерлиньяж 1

\usepackage{lastpage} % Узнать, сколько всего страниц в документе.

\usepackage{soul} % Модификаторы начертания

\usepackage{hyperref}
%\usepackage[usenames,dvipsnames,svgnames,table,rgb]{xcolor}
\hypersetup{				% Гиперссылки
    unicode=true,           % русские буквы в раздела PDF
    pdftitle={Заголовок},   % Заголовок
    pdfauthor={Автор},      % Автор
    pdfsubject={Тема},      % Тема
    pdfcreator={Создатель}, % Создатель
    pdfproducer={Производитель}, % Производитель
    pdfkeywords={keyword1} {key2} {key3}, % Ключевые слова
    colorlinks=true,       	% false: ссылки в рамках; true: цветные ссылки
    linkcolor=red,          % внутренние ссылки
    citecolor=black,        % на библиографию
    filecolor=magenta,      % на файлы
    urlcolor=cyan           % на URL
}

\usepackage{csquotes} % Еще инструменты для ссылок

%\usepackage[style=apa,maxcitenames=2,backend=biber,sorting=nty]{biblatex}

\usepackage{multicol} % Несколько колонок

\usepackage{tikz} % Работа с графикой
\usepackage{pgfplots}
\usepackage{pgfplotstable}
%\usepackage{coloremoji}
\usepackage{floatrow}
\usepackage{subcaption}
\newcommand*{\N}{\mathbb{N}}
\newcommand*{\R}{\mathbb{R}}
\newcommand*{\K}{\mathbb{K}}
\newcommand*{\V}{\mathcal{V}}
\newcommand*{\A}{\mathcal{A}}
\newcommand*{\ii}{\mathbf{1}}
\newcommand*{\oo}{\mathbf{0}}
\newcommand*{\ba}{\mathbf{a}}
\newcommand*{\bb}{\mathbf{b}}
\newcommand*{\Q}{\mathbb{Q}}
\graphicspath{{figures/}}
%\usepackage{breqn}

\renewcommand\thesubfigure{\asbuk{subfigure}}
%\addbibresource{master.bib}

\usepackage{import}
\usepackage{pdfpages}
\usepackage{transparent}
\usepackage{xcolor}
\usepackage{xifthen}

%\newcommand{\incfig}[1]{%
%    \def\svgwidth{\columnwidth}
%    \import{./figures/}{#1.pdf_tex}
%}


\newcommand{\incfig}[2][1]{%
    \def\svgwidth{#1\columnwidth}
    \import{./figures/}{#2.pdf_tex}
}
\usepackage{titlesec}
%\titleformat{\section}{\normalfont\Large\bfseries}{}{0pt}{}
%----------------------STANDART:
%\titleformat{\chapter}[display]
%  {\normalfont\huge\bfseries}{\chaptertitlename\ \thechapter}{20pt}{\Huge}
%\titleformat{\section}{\normalfont\Large\bfseries}{\thesection}{1em}{}
%\titleformat{\subsection}
%  {\normalfont\large\bfseries}{\thesubsection}{1em}{}
%\titleformat{\subsubsection}
%  {\normalfont\normalsize\bfseries}{\thesubsubsection}{1em}{}
%\titleformat{\paragraph}[runin]
%  {\normalfont\normalsize\bfseries}{\theparagraph}{1em}{}
%\titleformat{\subparagraph}[runin]
%  {\normalfont\normalsize\bfseries}{\thesubparagraph}{1em}{}

\pdfsuppresswarningpagegroup=1
\pgfplotsset{compat=1.16}

\usepackage{xifthen}
\makeatother
%\def\@lecture{}%
%\newcommand{\lecture}[3]{
%    \ifthenelse{\isempty{#3}}{%
%        \def\@lecture{Неделя #1}%
%    }{%
%        \def\@lecture{Неделя #1: #3}%
%    }%
%    \section*{\@lecture}
%    \marginpar{\small\textsf{\mbox{#2}}}
%}
\makeatletter

%\newcommand{\lec}{\subsection{Лекция}}
%\newcommand{\sem}{\subsection{Семинар}}
%\newcommand{\hw}{\subsection{Домашняя работа}}
%\newcommand{\prob}[1]{\textbf{#1}}
%\renewcommand{\thesubsection}{}
%\renewcommand{\thesubsubsection}{}

%\setcounter{tocdepth}{1} % only parts,chapters,sections
%\titleformat{\subsection}{\normalfont\large\bfseries}{}{0em}{}
%\titleformat{\subsubsection}{\normalfont\normalsize\bfseries}{}{0em}{}

%\newcommand{\textover}[2]{\stackrel{\mathclap{\normalfont\mbox{#2}}}{#1}}

\author{Драчов Ярослав\\
Факультет общей и прикладной физики МФТИ}
\newcommand{\veq}{\mathrel{\rotatebox{90}{$=$}}}
%\newcommand{\teto}[1]{\stackrel{\mathclap{\normalfont\tiny\mbox{#1}}}{\to}}
%\renewcommand{\thesubsection}{\arabic{subsection}}

%%\setcounter{secnumdepth}{0}

\definecolor{tabblue}{RGB}{30, 119, 180}
\definecolor{taborange}{RGB}{255, 127, 15}
\definecolor{tabgreen}{RGB}{45, 160, 43}
\definecolor{tabred}{RGB}{214, 38, 40}
\definecolor{tabpurple}{RGB}{148, 103, 189}
\definecolor{tabbrown}{RGB}{140, 86, 76}
\definecolor{tabpink}{RGB}{227, 119, 193}
\definecolor{tabgray}{RGB}{127, 127, 127}
\definecolor{tabolive}{RGB}{188, 189, 33}
\definecolor{tabcyan}{RGB}{22, 190, 207}
\pgfplotscreateplotcyclelist{colorbrewer-tab}{
{tabblue},
{taborange},
{tabgreen},
{tabred},
{tabpurple},
{tabbrown},
{tabpink},
{tabgray},
{tabolive},
{tabcyan},
}
\usepackage{csvsimple}
\usepackage{extarrows}
%\renewcommand{\labelenumii}{\asbuk{enumii})}
%\renewcommand{\labelenumiv}{\Asbuk{enumiv}}
\newcommand{\prob}[1]{\subsubsection*{#1}}
\sisetup{output-decimal-marker = {,},separate-uncertainty = true,exponent-product = \cdot}

\usepackage{braket}
\usepackage{enumerate}
\usepackage{chngcntr}
%\counterwithin*{equation}{problem}
%\usepackage{bbold}

\newtheoremstyle{hiProb}% ⟨name ⟩ 
{3pt}% ⟨Space above ⟩1 
{3pt}% ⟨Space below ⟩1
{}% ⟨Body font ⟩
{}% ⟨Indent amount ⟩2
{\bfseries}% ⟨Theorem head font⟩
{.}% ⟨Punctuation after theorem head ⟩
{.5em}% ⟨Space after theorem head ⟩3
%{\thmname{#1} \thmnote{#3}}% ⟨Theorem head spec (can be left empty, meaning ‘normal’)⟩
{\thmnote{#3}}% ⟨Theorem head spec (can be left empty, meaning ‘normal’)⟩
\theoremstyle{hiProb} % "Определение"
%\newtheorem{hiProb}{Задача}
\newtheorem{hiProb}{}
\usepackage{mmacells}
\newcommand{\textover}[2]{\stackrel{\mathclap{\normalfont\scriptsize\mbox{#2}}}{#1}}
\usepackage{units}
\usepackage[math]{cellspace}%
\setlength\cellspacetoplimit{2pt}
\setlength\cellspacebottomlimit{2pt}

\title{Семинар №7}
\begin{document}
	\maketitle
	Из теоремы Нётер
	\[
	\phi \to  \phi + \epsilon_i \delta \phi_i,\quad
	x^\mu \to  x^\mu+ \epsilon_i \delta x_i^\mu
	.\] 
	\[
		J^\mu= \left( \mathcal{L} \delta_\nu^\mu -
		\frac{\partial \mathcal{L}}{\partial \partial_\mu \phi} \partial_\nu \phi\right) 
		\delta x^\nu+ \frac{\partial \mathcal{L}}{\partial \partial_\mu \phi} \delta \phi
	.\] 
	\begin{table}[htpb]
	\centering
	\caption{}
	\label{tab:1}
	\begin{tabular}{|c|c|c|c|}
		\hline  & $\delta x^\mu$ & $\epsilon_i$ & $\epsilon_i \delta \phi_i$ \\
		\hline $P_\mu$ & $a^\mu$ & $a^\mu$ & 0\\
		\hline $D$ & $\lambda x^\mu$ & $\lambda$ &
		$-\lambda \cdot \Delta \cdot \phi$\\
		\hline $L_{\mu \nu}$ & $\omega^\mu_\nu x^\nu$ & $\omega^{\mu\nu}$ & $-i S_{\mu\nu} \phi$\\
		\hline $K_\mu$ & $2(b\cdot x) x^\mu - b^\mu x^2$ & $b^\mu$ &  ?\\
		\hline
	\end{tabular}
	\end{table}

Токи:
\begin{enumerate}
\item Трансляция:
	\[
		T_\text{с}^\mu {}_\nu = \mathcal{L} \delta^\mu_\nu-
		\frac{\partial \mathcal{L}}{\partial (\partial \phi)} \partial_\nu \phi,\quad \partial_\mu T_\text{с}^\mu {}_\nu=0 
	.\] 
\item Поворот:
	\[
		M^\mu {}_{\rho \nu}=
		T_\text{с}^\mu {}_\nu x_\rho - T_\text{с}^\mu {}_\rho x_\nu-
		i \frac{\partial \mathcal{L}}{\partial \left( \partial
		_\mu \phi\right) } S_{\nu \rho}\phi,\quad
		\partial_\mu M^\mu {}_{\nu\rho}=0
	.\] 
\item Дилотация
	\[
		J_{\text{D}}^\mu= T_\text{с}^\mu {}_\nu x^\nu-
		\Delta \cdot \frac{\partial \mathcal{L}}{\partial \partial _\mu \phi} \phi
	.\] 
\end{enumerate}
\[
	\delta S = \int\limits_{V}^{} d^d x' \mathcal{L}
	\left( \phi' ,\,\partial ' \phi\right) -S
	= \int d^d x \partial_\mu \mathcal{K}^\mu (\phi)=
	\int\limits_{\partial V}^{} d^d  S_\mu \mathcal{K}^\mu(\phi) 
.\] 
Пусть ток
\[
\partial_\mu J^\mu=0
\]
сохраняется. Если добавить
\[
	\partial_\mu \left( J^\mu +\partial_\nu
	B^{\mu \nu}\right)=0,\quad B^{\mu\nu}=- B^{\nu \mu}
\]
то это выражение также будет сохраняться.

У $T$ есть антисимметричная часть, т.\:к.
\[
0= \partial_\mu M^{\mu\nu \rho}=T ^{\rho \nu}-
T^{\nu \rho}- i \partial_\mu \left( 
\frac{\partial \mathcal{L}}{\partial \left( \partial_\mu
\phi\right) } S_{\nu\rho}\phi\right) 
.\] 
\[
	T^{\rho \nu}- T ^{\nu \rho}= i \partial_\mu \left( 
	\frac{\partial \mathcal{L}}{\partial \left( 
\partial_\mu \phi\right) } S_{\nu\rho} \phi\right) 
.\] 
\[
	T^{\mu\nu} \to  T_\text{с}^{\mu\nu} + \partial_\rho B^{
	\rho \mu \nu},\quad B^{\rho \mu}= -B ^{\mu \rho}
.\] 
\[
T^{\mu\nu}-T^{\nu\mu}= 2\partial_\rho B^{\rho \mu \nu}=
i \partial_\rho\underbrace{ \left( 
\frac{\partial \mathcal{L}}{\partial \left( \partial_\rho \phi \right) } S^{\nu\mu}\right) }_{b^{\rho \mu\nu}}
.\] 
\[
	B^{\rho \mu\nu}= \frac{i}{4} \left( 
	b^{\rho \mu\nu}+b ^{\nu \rho \mu}+ b^{\mu \nu \rho}\right) 
.\] 
\[
M^{\mu \nu \rho}=T^{\mu\nu}x^{\rho}-
T^{\mu \rho}x^\nu
.\] 
Тождество Уорда:
\begin{multline}
\partial_\mu \left< T^{\mu\nu} X \right> =
- i \sum_{i=1}^{n} \delta(x-x_i) \left<
\phi_1 (x_1) \ldots \underbrace{\widehat{\operatorname{P}}}_{
=-i \frac{\partial }{\partial x_\nu} }\phi_i (x_i)\ldots
\phi_n (x_n)\right>=\\=
- \sum_{i=1}^{n} \delta(x-x_i) \frac{\partial }{\partial x_i} 
\left<X \right>
\label{eq:1}
.\end{multline} 
\[
	L^{\nu \rho}= -i \left( x^\rho \partial^\nu
	-x^\nu \partial^\rho\right) +S^{\nu \rho}
.\] 
\begin{multline*}
	\partial_\mu \left< \left( T^{\mu\nu}x^\rho
	-T^{\mu \rho}x^\nu\right) X \right> =
	- \sum_{i=1}^{n} \delta (x-x_i) 
	\left< \phi_i (x_i) \ldots \widehat{\operatorname{L}}^{\nu\rho} \phi(x_i) \ldots \phi_n(x_n) \right> =\\=
	\sum_{i=1}^{n} \delta(x-x_i) \left( 
	\left( x_i^\nu \partial_i^\rho-x_i^\rho \partial_i^\nu \right) -i S_i^{\nu\rho}\right) \left<X \right>
.\end{multline*} 
Используя \ref{eq:1}, получаем
\[
x^\rho \partial_\mu \left< T^{\mu\nu} X \right>-
x^\nu \partial_\mu \left< T^{\mu\rho} X \right>+
\left<\left( T^{\rho\nu}-T^{\nu\rho} \right) X \right>
.\] 
\[
	\left<\left( T^{\rho\nu}-T^{\nu\rho} \right) X \right>
	=-i \sum_{i}^{} \delta(x-x_i) S^{\nu \rho}_i \left<X \right>
.\] 
\[
	\partial_\mu J^\mu _\text{D}=T^\mu_\text{с}?^\mu {}_\mu
	-\Delta \partial_\mu \left( \frac{\partial \mathcal{L}}{\partial (\partial_\mu \phi)} \phi \right) =0
.\] 
\begin{multline*}
	\partial_\mu \left< J^\mu_\text{D} X \right> = x^\nu \partial_\mu \left<T^\mu {}_\nu X \right>+\left<
	T^\mu{}_\mu X\right>=\\=
	-i\sum_{}^{} \delta(x-x_i) \left<
	\phi_1(x_1)\ldots \left(-i x_i^\rho \partial_\rho^i -i \Delta\right) \phi_i (x_i) \ldots \phi_n(x_n)\right>
.\end{multline*} 
\[
	\partial_\mu \left<T^{\mu\nu} X\right> =  - \sum_{i=1}^{n} \delta(x-x_i) \frac{\partial }{\partial x^i_\nu} \left<X \right>
.\] 
\[
	\left<T^\mu{}_\mu X \right> =
	-\sum_{i=1}^{n} \delta(x-x_i)\Delta_i
	\left<X \right>
.\] 
\begin{multline*}
	\phi'\left(w,\overline{w}\right)=
	\left( \frac{dw}{dz} \right) ^{-h}
	\left(\frac{d \overline{w}}{d \overline{z}}\right)^{-\overline{h}} \phi \left( z,\,\overline{z} \right) =
	e^{- i \phi S}\phi\left( z,\,\overline{z} \right) 
	\approx \\ \approx(1- i \phi s) \phi \left( z,\,\overline{z} \right) = \left(1- \frac{i}{2} \omega^{\mu \nu} S_{\mu\nu}\right)
	\phi\left( z,\,\overline{z} \right) 
.\end{multline*} 
\[
	h=\frac{1}{2} (\Delta+s),\quad \overline{h}
	=\frac{1}{2} (\Delta-s)
.\] 
\[
w=e^i\phi z,\quad \overline{w} =e^{- i \phi} \overline{z}
.\] 
\[
	\omega_{\mu\nu}= \begin{pmatrix} 0 & \phi \\ -\phi &0 \end{pmatrix} =\phi \cdot \epsilon _{\mu\nu},\quad
	S_{\mu\nu}= \epsilon_{\mu\nu}s
.\] 
\[
\epsilon_{\mu\nu}\epsilon^{\mu\nu}=2
.\] 
\[
S_{\mu\nu}=\epsilon_{\mu\nu}s
.\] 
\[
T^{\rho\sigma}- T^{\sigma\rho}=\epsilon^{\rho\sigma}
\epsilon_{\mu\nu} T^{\mu\nu}
.\] 
\[
\left<\epsilon^{\mu\nu} T_{\mu\nu}X \right> =
- i \sum_{i=1}^{n} \delta (x-x_i) S_i \left<X \right>
.\] 
\[
	z'^\mu =z^\mu +\epsilon ^\mu (z)
.\] 
\[
	\partial_\mu \left( \epsilon\nu T^{\mu\nu} \right) =
	\partial_{\mu} \epsilon_{\nu} T^{\mu\nu}+
	\epsilon_\nu \partial_{\mu} T^{\mu\nu}=
	\epsilon_\nu \partial_\mu T^{\mu\nu}+\frac{1}{2}
	\left( \partial_\rho \epsilon^\rho \right) 
	T^\mu {}_\mu + \frac{1}{2} e_{\rho \sigma}
	\partial^\rho \epsilon^\sigma e_{\mu\nu}
	T^{\mu\nu}
.\] 
\[
\partial_\mu \epsilon_\nu +\partial_\nu \epsilon_\mu=
\frac{1}{2} \left( \partial_\rho \epsilon^\rho \right) \underbrace{g_{\mu\nu}}_{\delta_{\mu\nu}}
.\] 
\[
\partial_\mu \epsilon_\nu -\partial_\nu \epsilon_\mu=
e_\mu\nu \left( \epsilon^{\rho\sigma}\partial_\rho
\epsilon_\sigma \right) 
.\] 
\[
\delta_\epsilon \left< X \right> =
\int d^2 x \partial_\mu \left<\epsilon_\nu T^{\mu\nu} X \right>
.\] 
\section*{Д/з}
\begin{hiProb}[Задача 1]
Проверить что это действительно дельта-функция, подставив
в формулы
\[
	\delta ^{(2)} (x)= \frac{1}{\pi} \partial_{\overline{z}}
	\frac{1}{z}= \frac{1}{\pi} \partial_z \frac{1}{\overline{z}}
.\] 
\[
	\int d^2 f(z) \delta ^{(2)}\left( x \right) =f(z=0)
.\] 
\[
	d^2 x f\left( \overline{z} \right) \delta ^{(2)}(x)
	=f(\overline{z}=0)
.\] 

\end{hiProb}
\begin{hiProb}[Задача 2 ]
Проверить 3 тождества Уорда для
\[
	\partial_\mu = \begin{pmatrix} \partial_\mu &
	\partial_{\overline{z}}\end{pmatrix},\quad
	T^{\mu\nu}=
		\begin{pmatrix} T^{zz} & T^{z \overline{z}}\\
		T^{\overline{z} z}& T^{\overline{z} \overline{z}}\end{pmatrix} 
.\] 
\end{hiProb}
\begin{hiProb}[Задача 3]
Вычислить кореллятор
\[
\left<T^{zz} X \right>
.\] 
\end{hiProb}
\end{document}
