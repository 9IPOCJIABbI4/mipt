\documentclass[a4paper]{article}
% Этот шаблон документа разработан в 2014 году
% Данилом Фёдоровых (danil@fedorovykh.ru) 
% для использования в курсе 
% <<Документы и презентации в \LaTeX>>, записанном НИУ ВШЭ
% для Coursera.org: http://coursera.org/course/latex .
% Исходная версия шаблона --- 
% https://www.writelatex.com/coursera/latex/5.3

% В этом документе преамбула

\usepackage{siunitx}
%%% Работа с русским языком
%\usepackage{cmap}					% поиск в PDF
%\usepackage{mathtext} 				% русские буквы в формулах
%\usepackage[T2A]{fontenc}			% кодировка
%\usepackage[utf8]{inputenc}			% кодировка исходного текста
%\usepackage[english,russian]{babel}	% локализация и переносы
%\usepackage{indentfirst}
%\frenchspacing
%
%\renewcommand{\epsilon}{\ensuremath{\varepsilon}}
%\newcommand{\phibackup}{\ensuremath{\phi}}
%\renewcommand{\phi}{\ensuremath{\varphi}}
%\renewcommand{\varphi}{\ensuremath{\phibackup}}
%\renewcommand{\kappa}{\ensuremath{\varkappa}}
%\renewcommand{\le}{\ensuremath{\leqslant}}
%\renewcommand{\leq}{\ensuremath{\leqslant}}
%\renewcommand{\ge}{\ensuremath{\geqslant}}
%\renewcommand{\geq}{\ensuremath{\geqslant}}
%\renewcommand{\emptyset}{\varnothing}
%\renewcommand{\Im}{\operatorname{Im}}
%\renewcommand{\Re}{\operatorname{Re}}


%%% Дополнительная работа с математикой
\usepackage{amsmath,amsfonts,amssymb,amsthm,mathtools} % AMS
%\usepackage{icomma} % "Умная" запятая: $0,2$ --- число, $0, 2$ --- перечисление

%% Номера формул
%\mathtoolsset{showonlyrefs=true} % Показывать номера только у тех формул, на которые есть \eqref{} в тексте.
%\usepackage{leqno} % Нумереация формул слева

%% Свои команды
\DeclareMathOperator{\sgn}{\mathop{sgn}}
\DeclareMathOperator{\sign}{\mathop{sign}}
\DeclareMathOperator*{\res}{\mathop{res}}
\DeclareMathOperator*{\tr}{\mathop{tr}}
\DeclareMathOperator*{\rot}{\mathop{rot}}
\DeclareMathOperator*{\divop}{\mathop{div}}
\DeclareMathOperator*{\grad}{\mathop{grad}}

%% Перенос знаков в формулах (по Львовскому)
\newcommand*{\hm}[1]{#1\nobreak\discretionary{}
{\hbox{$\mathsurround=0pt #1$}}{}}

%%% Работа с картинками
\usepackage{graphicx}  % Для вставки рисунков
\graphicspath{{figures/}}  % папки с картинками
\setlength\fboxsep{3pt} % Отступ рамки \fbox{} от рисунка
\setlength\fboxrule{1pt} % Толщина линий рамки \fbox{}
\usepackage{wrapfig} % Обтекание рисунков текстом

%%% Работа с таблицами
\usepackage{array,tabularx,tabulary,booktabs} % Дополнительная работа с таблицами
\usepackage{longtable}  % Длинные таблицы
\usepackage{multirow} % Слияние строк в таблице

%%% Теоремы
\theoremstyle{plain} % Это стиль по умолчанию, его можно не переопределять.
\newtheorem{thm}{Теорема}
\newtheorem*{thm*}{Теорема}
\newtheorem{prop}{Предложение}
\newtheorem*{prop*}{Предложение}
 
\theoremstyle{definition} % "Определение"
%\newtheorem{corollary}{Следствие}[theorem]
\newtheorem{dfn}{Определение}
\newtheorem*{dfn*}{Определение}
\newtheorem{prob}{Задача}
\newtheorem*{prob*}{Задача}

 
\theoremstyle{remark} % "Примечание"
\newtheorem*{sol}{Решение}
\newtheorem*{rem}{Замечание}

%%% Программирование
\usepackage{etoolbox} % логические операторы

%%% Страница
%\usepackage{extsizes} % Возможность сделать 14-й шрифт
%\usepackage{geometry} % Простой способ задавать поля
%	\geometry{top=25mm}
%	\geometry{bottom=35mm}
%	\geometry{left=35mm}
%	\geometry{right=20mm}
 
\usepackage{fancyhdr} % Колонтитулы
%	\pagestyle{fancy}
 %	\renewcommand{\headrulewidth}{0pt}  % Толщина линейки, отчеркивающей верхний колонтитул
	%\lfoot{Нижний левый}
	%\rfoot{Нижний правый}
	%\rhead{Верхний правый}
	%\chead{Верхний в центре}
	%\lhead{Верхний левый}
	%\cfoot{Нижний в центре} % По умолчанию здесь номер страницы

\usepackage{setspace} % Интерлиньяж
%\onehalfspacing % Интерлиньяж 1.5
%\doublespacing % Интерлиньяж 2
%\singlespacing % Интерлиньяж 1

\usepackage{lastpage} % Узнать, сколько всего страниц в документе.

\usepackage{soul} % Модификаторы начертания

\usepackage{hyperref}
\usepackage[usenames,dvipsnames,svgnames,table,rgb]{xcolor}
\hypersetup{				% Гиперссылки
    unicode=true,           % русские буквы в раздела PDF
    pdftitle={Заголовок},   % Заголовок
    pdfauthor={Автор},      % Автор
    pdfsubject={Тема},      % Тема
    pdfcreator={Создатель}, % Создатель
    pdfproducer={Производитель}, % Производитель
    pdfkeywords={keyword1} {key2} {key3}, % Ключевые слова
%    colorlinks=true,       	% false: ссылки в рамках; true: цветные ссылки
    %linkcolor=red,          % внутренние ссылки
    %citecolor=black,        % на библиографию
    %filecolor=magenta,      % на файлы
    %urlcolor=cyan           % на URL
}

\usepackage{csquotes} % Еще инструменты для ссылок

%\usepackage[style=apa,maxcitenames=2,backend=biber,sorting=nty]{biblatex}

\usepackage{multicol} % Несколько колонок

\usepackage{tikz} % Работа с графикой
\usepackage{pgfplots}
\usepackage{pgfplotstable}
%\usepackage{coloremoji}
\usepackage{floatrow}
\usepackage{subcaption}
\graphicspath{{figures/}}

\renewcommand\thesubfigure{\asbuk{subfigure}}
%\addbibresource{master.bib}

\usepackage{import}
\usepackage{pdfpages}
\usepackage{transparent}
\usepackage{xcolor}
\usepackage{xifthen}

\newcommand{\incfig}[2][1]{%
    \def\svgwidth{#1\columnwidth}
    \import{./figures/}{#2.pdf_tex}
}
%\usepackage{titlesec}
%\titleformat{\section}{\normalfont\Large\bfseries}{}{0pt}{}
%----------------------STANDART:
%\titleformat{\chapter}[display]
%  {\normalfont\huge\bfseries}{\chaptertitlename\ \thechapter}{20pt}{\Huge}
%\titleformat{\section}{\normalfont\Large\bfseries}{\thesection}{1em}{}
%\titleformat{\subsection}
%  {\normalfont\large\bfseries}{\thesubsection}{1em}{}
%\titleformat{\subsubsection}
%  {\normalfont\normalsize\bfseries}{\thesubsubsection}{1em}{}
%\titleformat{\paragraph}[runin]
%  {\normalfont\normalsize\bfseries}{\theparagraph}{1em}{}
%\titleformat{\subparagraph}[runin]
%  {\normalfont\normalsize\bfseries}{\thesubparagraph}{1em}{}

\pdfsuppresswarningpagegroup=1
\pgfplotsset{compat=1.16}



%\setcounter{tocdepth}{1} % only parts,chapters,sections
%\titleformat{\subsection}{\normalfont\large\bfseries}{}{0em}{}
%\titleformat{\subsubsection}{\normalfont\normalsize\bfseries}{}{0em}{}

%\newcommand{\textover}[2]{\stackrel{\mathclap{\normalfont\mbox{#2}}}{#1}}

\author{Yaroslav Drachov\\
Moscow Institute of Physics and Technology}
%\author{Драчов Ярослав\\
%Факультет общей и прикладной физики МФТИ}
\newcommand{\veq}{\mathrel{\rotatebox{90}{$=$}}}
%\newcommand{\teto}[1]{\stackrel{\mathclap{\normalfont\tiny\mbox{#1}}}{\to}}
%\renewcommand{\thesubsection}{\arabic{subsection}}

%%\setcounter{secnumdepth}{0}

\definecolor{tabblue}{RGB}{30, 119, 180}
\definecolor{taborange}{RGB}{255, 127, 15}
\definecolor{tabgreen}{RGB}{45, 160, 43}
\definecolor{tabred}{RGB}{214, 38, 40}
\definecolor{tabpurple}{RGB}{148, 103, 189}
\definecolor{tabbrown}{RGB}{140, 86, 76}
\definecolor{tabpink}{RGB}{227, 119, 193}
\definecolor{tabgray}{RGB}{127, 127, 127}
\definecolor{tabolive}{RGB}{188, 189, 33}
\definecolor{tabcyan}{RGB}{22, 190, 207}
\pgfplotscreateplotcyclelist{colorbrewer-tab}{
{tabblue},
{taborange},
{tabgreen},
{tabred},
{tabpurple},
{tabbrown},
{tabpink},
{tabgray},
{tabolive},
{tabcyan},
}
\usepackage{csvsimple}
\usepackage{extarrows}
%\renewcommand{\labelenumii}{\asbuk{enumii})}
%\renewcommand{\labelenumiv}{\Asbuk{enumiv}}
%\newcommand{\prob}[1]{\subsubsection*{#1}}
\sisetup{output-decimal-marker = {,},separate-uncertainty = true,exponent-product = \cdot}

\usepackage{braket}
\usepackage{enumerate}
\usepackage{chngcntr}
%\counterwithin*{equation}{problem}
%\usepackage{bbold}

\newtheoremstyle{hiProb}% ⟨name ⟩ 
{3pt}% ⟨Space above ⟩1 
{3pt}% ⟨Space below ⟩1
{}% ⟨Body font ⟩
{}% ⟨Indent amount ⟩2
{\bfseries}% ⟨Theorem head font⟩
{.}% ⟨Punctuation after theorem head ⟩
{.5em}% ⟨Space after theorem head ⟩3
%{\thmname{#1} \thmnote{#3}}% ⟨Theorem head spec (can be left empty, meaning ‘normal’)⟩
{\thmnote{#3}}% ⟨Theorem head spec (can be left empty, meaning ‘normal’)⟩
\theoremstyle{hiProb} % "Определение"
%\newtheorem{hiProb}{Задача}
\newtheorem{hiProb}{}
%\usepackage{mmacells}
\newcommand{\textover}[2]{\stackrel{\mathclap{\normalfont\scriptsize\mbox{#2}}}{#1}}
\usepackage{units}
\usepackage[math]{cellspace}%
\setlength\cellspacetoplimit{2pt}
\setlength\cellspacebottomlimit{2pt}

\DeclareMathAlphabet{\mathbbold}{U}{bbold}{m}{n}

\newcommand{\normord}[1]{:\mathrel{#1}:}

\title{Семинар №6}
\begin{document}
	\maketitle
\[
S= d^d x \frac{1}{2} \partial_\mu \phi \partial^\mu \phi,\quad \Delta= \frac{d}{2}-1
.\] 
Кл. симм.: токи
\[
\partial_\mu J^\mu=0,\quad J^\mu =\begin{pmatrix} 
T_\text{с}^{\mu}{}_{\nu} & j^\mu \end{pmatrix} \ldots
.\] 
\[
	S= \int d^d x \left( \frac{1}{2} \partial_\mu \phi \partial^\mu \phi -V(\phi) \right) = \int dt d^{d-1} x \left( \frac{1}{2}
	\dot{\phi}^2 -\frac{1}{2} (\nabla \phi)^2-V(\phi)\right) \xlongequal[]{*}
.\] 
\[
	Z= \int D\phi(x) e^{i S[\phi]}=
	\int D \phi e^{- S_E [\phi]}
.\] 
Виковский поворот
\[
t=-i \tau
.\]
\[
	\xlongequal[]{*} i \underbrace{\int d\tau d^{d-1} x \left( 
	\frac{1}{2} (\partial_\tau \phi)^2 +\frac{1}{2} (\nabla \phi)^2 +V (\phi)\right) }_{S_E}=
	i \int d^d x_E \left( \frac{1}{2} \partial_\mu \phi \partial^\mu \phi) +V(\phi) \right) 
.\] 
\[
\eta_{MN}=\delta_{MN} \quad d+1\text{-мерная теория}
.\] 
\[
z^0,\,z^1 \to  w^0,\,w^1
.\] 
\[
	g'^{\mu \nu} (\omega)= \frac{\partial w^\mu}{\partial z^\rho}
	\frac{\partial w^\nu}{\partial z^\sigma}  g ^{\rho\sigma}=\Lambda^{-1} \begin{pmatrix} 1 & 0\\ 0 &1 \end{pmatrix} ,\quad g^{\mu \nu}= \delta ^{\mu \nu}
.\] 
\[
0= g'^{01}= \frac{\partial w^0}{\partial z^\rho} \frac{\partial w^1}{\partial z^\sigma}g^{\rho\sigma}=
\frac{\partial w^0}{\partial z^0}  \frac{\partial w^1}{\partial z^0} +\frac{\partial w^0}{\partial z^1} \frac{\partial w^1}{\partial z^1}=0  
.\] 
\[
	g'^{00}= g'^{11},\quad \left( \frac{\partial w^0}{\partial z^0}  \right) ^2 + \left( \frac{\partial w^0}{\partial z^1}  \right) ^2=
	\left( \frac{\partial w^1}{\partial z^0}  \right) ^2+
	\left( \frac{\partial w^1}{\partial z^1}  \right) ^2
.\] 
Имеем систему
\[
\left\{
\begin{aligned}
ab+cd&= 0 \\
a^2+c^2&=b^2+d^2
\end{aligned}
\right.
.\] 
Её решениями являются
\[
\left\{
\begin{aligned}
b&= c \\
a&= -d \\
\end{aligned}
\right.,\quad
\left\{
\begin{aligned}
a&= -d \\
a&=d
\end{aligned}
\right.
.\] 
То есть
\[
\left\{
\begin{aligned}
\frac{\partial w^1}{\partial z^0} &= \frac{\partial w^0}{\partial z^1}  \\
\frac{\partial w^0}{\partial ^0} &= -\frac{\partial w^1}{\partial z^1}  \\
\end{aligned}
\right.
,\quad
\left\{
\begin{aligned}
\frac{\partial w^1}{\partial z^0} &= - \frac{\partial w^0}{\partial z^1}  \\
\frac{\partial w^0}{\partial z^0}&= \frac{\partial w^1}{\partial z^1}  \\ 
\end{aligned}
\right.
.\] 
Это условия голоморфности и антиголоморфности соответственно.
\[
	g_{\mu \nu} dz^\mu dz^\nu= (dz^0)^2+(dz^1)^2=
	\frac{1}{4} \begin{pmatrix} dz^2 d \overline{z}^2 +2 dz d \overline{z} \\ dz^2 d \overline{z}^2+2 dz d \overline{z} \end{pmatrix} =dz d \overline{z}
.\] 
\[
z=z^0+i z^1,\quad \overline{z}= z^0- i z^1
.\] 
\[
	g_{\mu\nu}=  \begin{pmatrix} 0 & \frac{1}{2}\\ \frac{1}{2} &0 \end{pmatrix} 
.\] 
\[
	w=w(z) \text{ либо } w=w(\overline{z})
.\] 
\[
g_{\mu\nu} dz^\mu dz^\nu= dz d \overline{z} = 
\left( \frac{\partial z}{\partial w} \frac{\partial \overline{z}}{\partial \overline{w}}  \right) dw d \overline{w}=
\left( \frac{d \overline{z}}{dw(\overline{z})} \frac{\partial z}{\partial \overline{w (\overline{z})}} \right) dw d \overline{w}
.\] 
\begin{table}[htpb]
	\centering
	\caption{}
	\label{tab:1}
	\begin{tabular}{|Cc|Cc|Cc|}
		\hline
	 & лок. & глоб \\\hline
		инф. & & \\\hline
		конечн. & $\emptyset$ & $w= \frac{az+b}{cz+d}$ \\\hline
	\end{tabular}
\end{table}
Квазипримарное поле с $\Delta=0$
\[
	\phi'\left(w,\,\overline{w}\right)= \phi \left( z ,\, \overline{z} \right) ,\quad w(z)= z + \epsilon (z),\quad \epsilon\to 0
.\] 
\[
	\phi'\left(z,\,\overline{z}\right)= \phi\left(z,\,\overline{z}\right)- \epsilon \partial_z \phi - \overline{\epsilon } \partial_{\overline{z}} \phi
.\] 
\[
	\epsilon(z)= \sum_{n=-\infty}^{\infty} \epsilon_n z^{n+1},
\]
где $\epsilon_n$ --- счётное число параметров преобразования.
\[
l_n= - z^{n+1}\partial_{z}\text{ --- голом.},\quad \overline{l}_n=- \overline{z}^{n+1}
\partial_{\overline{z}}\text{ --- антиголом.}
\] 
\[
	\left( \delta_{\epsilon_1\overline{\epsilon }_1}
		\delta_{\epsilon_2 \overline{\epsilon }_2}-
		\delta_{\epsilon_2 \overline{\epsilon }_2}
\delta_{\epsilon_1\overline{\epsilon }_1}
	\right) \phi=
	\left[ \delta_{\epsilon_1\overline{\epsilon }_1},\,
	\delta_{\epsilon_2\overline{\epsilon }_2}\right] \phi
.\] 
\[
	\left[ l_n,\,l_m \right] =(m+1)z^{n+m+1}\partial_z-(n+1)
	z^{n+m +1}\partial_z =(m-n) l_{n+m}
.\] 
Алгебра Витта:
\[
	\left[ \overline{l}_n,\,\overline{l}_m \right] =(m-n) \overline{l}_{n+m}
.\] 
\[
	\begin{pmatrix} a & b \\ c & d \end{pmatrix} \in 
	\text{SL}(2,\,\mathbb{C}),\quad
	\begin{pmatrix} a & b \\ c & d \end{pmatrix} =
	\begin{pmatrix} 1 & 0 \\ 0 & 1 \end{pmatrix} +
	\begin{pmatrix} \alpha & \beta \\ \gamma & \delta=-\alpha \end{pmatrix} 
.\] 
\[
	w= \frac{az+b}{cz+d}= \frac{(1+\alpha) z + \beta}{\gamma z+1 -\alpha}= z+ \beta +2 \alpha z-\gamma z^2
.\] 
\[
\beta=\epsilon_{-1},\quad \alpha= \frac{\epsilon_0}{2},\quad
\gamma=-\epsilon_1
.\] 
\[
l_{-1},\ l_0,\ l_1
.\] 
\[
\left[ l_{-1},\,l_0 \right] =l_{-1},\quad \left[ l_0,\,l_1 \right] =
l_1,\quad \left[ l_{-1},\,l_1 \right] =2 l_0
.\] 
\[
\delta \phi = -\epsilon \partial_z \phi - \overline{\epsilon }
\partial_{\overline{z}}\phi= \sum_{n=-\infty}^{\infty} \left( 
\epsilon_n l_n \phi + \overline{\epsilon }_n \overline{l}_n \phi\right) 
.\] 
\[
	w(z)= z+\epsilon(z)
.\] 
Квазипримарные поля
\[
	\phi'\left( w,\,\overline{w} \right) =
	\left( \frac{\partial w}{\partial z} \right)  ^{-h} 
	\left( \frac{\partial \overline{w}}{\partial \overline{z}}  \right) ^{-\overline{h}}\phi \left(z,\,\overline{z}\right)
.\] 
\[
	h=\frac{1}{2} (\Delta+S),\quad \overline{h}=\frac{1}{2}
	(\Delta -S)
.\] 
\[
	\phi_{z_1,\ldots,\,z_h,\,\overline{z}_1,\ldots,\,\overline{z}_{\overline{h}}}= \left( \frac{\partial z}{\partial w}  \right) ^h
	\left( \frac{\partial \overline{z}}{\partial \overline{w}}  \right) ^{\overline{h}} \phi'_{w_1,\ldots,\,w_h,\, \overline{w}_1,\ldots,\,\overline{w}_{\overline{h}}}
.\] 

\end{document}
